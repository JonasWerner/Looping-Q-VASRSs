
\documentclass[%
10pt,
dvipsnames,
]{beamer}

\usepackage[utf8]{inputenc}
\usepackage{xspace}
\usepackage{tabularx}
\usepackage{hyperref}
\usepackage{algorithm}
\usepackage[noend]{algpseudocode}
\usepackage{cite}
\usepackage{booktabs}
\usepackage{url}
\usepackage{listings}
\usepackage{amsthm}
\usepackage{amsmath}
\usepackage{tikz}
\usetikzlibrary{positioning,shapes.geometric, arrows.meta ,automata, decorations.pathreplacing, calc, fit, backgrounds, quotes, tikzmark}
\usepackage{pgf}
\usepackage{slantsc}
\usepackage{geometry}
\usepackage{amssymb}
\usepackage{subcaption}
\usepackage{float}
\usepackage{pgf}
\usepackage{slashbox}
\usepackage{pgfgantt}
\usepackage{wrapfig}
\usepackage{pdflscape}
\usepackage{xcolor}
\usepackage{xparse}
\usepackage[%disable,%
colorinlistoftodos,%
color=cyan!50!white,%
bordercolor=cyan!50!black]{todonotes}

\usepackage{varwidth}
\usepackage[most]{tcolorbox}% http://ctan.org/pkg/tcolorbox
\tcbuselibrary{skins,breakable}
\usepackage{comment}
\usepackage{makecell}
\usepackage{stmaryrd}

\DeclareSymbolFont{matha}{OML}{txmi}{m}{it}% txfonts
\DeclareMathSymbol{\varv}{\mathord}{matha}{118}

\lstset{ %
	backgroundcolor=\color{white}, 
	basicstyle=\footnotesize,       
	breakatwhitespace=false,        
	breaklines=true,                 
	captionpos=b,                    
	commentstyle=\color{mygreen},   
	escapeinside={\%*}{*)},        
	extendedchars=true,              
	frame=single,                  
	keywordstyle=\color{blue},       
	language=Prolog,                
	numbers=left,                    
	numbersep=5pt,                   
	rulecolor=\color{black},        
	showspaces=false,               
	showstringspaces=false,          
	showtabs=false,                  
	stringstyle=\color{mymauve},   
	tabsize=2,                      
	title=\lstname,                  
	morekeywords={not,\},\{,preconditions,effects },            
	deletekeywords={time}            
}

%%%%%%%%%%%% Colors
%% a somewhat friendly scheme for 5 different colors
\definecolor{g1}		{RGB}{215,25,28} % a kind of red
\definecolor{g2}		{RGB}{253,174,97} % a kind of orange
\definecolor{g3}		{RGB}{255,255,191} % a kind of yellow
\definecolor{g4}		{RGB}{171,217,233} % a kind of light blue
\definecolor{g5}		{RGB}{44,123,182} % a kind of dark blue

\definecolor{gr1}		{RGB}{250, 250, 250}
\definecolor{gr2}		{RGB}{229, 229, 229} % some grey

% color of interpolants
\definecolor{itpGreen}  {rgb}{0,1,0}
\definecolor{grey}      {RGB}{200,200,200}


%color for pictures
\colorlet{outlineblue}		{g5}
\colorlet{fillblue}			{g4}
\colorlet{darkback}			{gr2}
\colorlet{lightback}		{gr1}
\colorlet{stmtcolor}		{gr2} %default statement color
\colorlet{subgraphcolor}	{g3} %default statement color
\colorlet{colexamtitle}     {black} % Example block title color
\colorlet{colexamline}      {g1} % Example block sideline color
\colorlet{itp}              {itpGreen}



%%%%%%%%%%%%% Statements and labels Trace Abstraction Style
\tikzstyle{st} = [%
font=\ttfamily,%
shape=rectangle,%
rounded corners=.5em,%
fill=stmtcolor,%
inner xsep=.3em,%
inner ysep=0em, %
text height=2ex, %
text depth=.6ex,
]


\newcommand{\tikzstmt}[3]{{%
		\tikz[baseline]{%
			\node[st,fill=#2] at (0,.64ex){%
				\hspace{.3em}\texttt{\strut#3#1}\hspace{.3em}\strut};}
}}

\newcommand{\stcol}[2]{\tikzstmt{#1}{#2}{}}
\newcommand{\stsmcol}[2]{\tikzstmt{#1}{#2}{\small}}
\newcommand{\stfootcol}[2]{\tikzstmt{#1}{#2}{\footnotesize}}

\newcommand{\stnorm}[1]{\stcol{#1}{stmtcolor}}
\newcommand{\stsm}[1]{\stsmcol{#1}{stmtcolor}}
\newcommand{\stfoot}[1]{\stfootcol{#1}{stmtcolor}}

\newcommand{\st}[1]{\stfoot{#1}}
\newcommand{\lab}[1]{\stfoot{\ensuremath{#1}}}
\newcommand{\lan}[1]{\stnorm{\ensuremath{#1}}}
\newcommand{\stn}[1]{\stnorm{#1}}

\newcommand{\formula}[2]{\tikz[baseline]{\node[shape=rectangle,line width=1pt,draw=#2,fill=#2!30,inner sep=1pt, align=center] at (0,.64ex){\hspace{.2em}\texttt{\strut#1}\hspace{.1em}\strut};}}
\newcommand{\itp}[1]{\formula{\ensuremath{#1}}{itp}}

\newcommand{\tf}{\ensuremath{\varphi}}
\newcommand{\ctf}{\ensuremath{\widehat{\varphi}}}
\newcommand{\invars}{\ensuremath{In}}
\newcommand{\outvars}{\ensuremath{Out}}
\newcommand{\auxvars}{\ensuremath{Aux}}

\newcommand{\Var}{\ensuremath{\mathit{Var}}}
\newcommand{\stmt}{\ensuremath{\mathit{Stmt}}}
\newcommand{\Loc}{\ensuremath{\mathit{Loc}}}
\newcommand{\err}{\ensuremath{\mathit{err}}}
\newcommand{\init}{\ensuremath{\mathit{init}}}


\newcommand{\pq}[1]{\ensuremath{#1_{p, q}}}
\newcommand{\pqvasr}{\ensuremath{(\pq{S}, \pq{V})}}
\newcommand{\pqformula}{\ensuremath{\pq{\Gamma}}}

%%%%%%%%%%%% Numbered example environment
% \newcounter{example}[section]
% \newenvironment{example}[1][]{\refstepcounter{example}\par\medskip
%    \noindent \textbf{Example~\theexample. #1} \rmfamily}{\medskip}


\usepackage{regexpatch}

\makeatletter
\xpatchcmd{\algorithmic}{\labelsep 0.5em}{\labelsep 1.5em}{\typeout{Success!}}{\typeout{Oh dear!}}
\makeatother

%%%%%%%%%%%% Comments
\newif\iffinal
%\finaltrue % comment out to remove comments

\iffinal
\newcommand\mycom[1]{}
\else
\newcommand\mycom[1]{#1}
\overfullrule=1mm
\fi
\setlength\parindent{0pt}

\newcommand{\jw}[1]{\mycom{\todo[color=blue!40,inline]{\small JW: #1}}}
\newcommand{\dd}[1]{\mycom{\todo[color=orange!40,inline]{\small DD: #1}}}
\newcommand{\ts}[1]{\mycom{\todo[color=green!40,inline]{\small TS: #1}}}


\newcommand{\all}[1]{\mycom{\todo[color=green!40,inline]{\small #1}}}
\newcommand{\meta}[1]{\mycom{\todo[color=blue!10,inline,caption={Beschreibung},nolist]{\setlist{nolistsep}\small #1}}}
\newcommand{\xxx}{\mycom{\stfootcol{Placeholder}{blue!20}}}
\newcommand{\cn}{\mycom{\stfootcol{Cite}{blue!20}}}

\newcommand{\Q}{\ensuremath{\mathbb{Q}}}
\newcommand{\entails}[1]{\vdash_{#1}}

\newcommand{\qvasr}{\ensuremath{\mathit{\mathbb{Q} \text{-}VASR}}}
\newcommand{\qvasrs}{\ensuremath{\mathit{\mathbb{Q} \text{-}VASRS}}}

\newcommand{\tranFormula}{\ensuremath{F(\vec{x}, \vec{x}')}}
\newcommand{\coherent}{\ensuremath{\equiv_{\mathit{V}}}}
\newcommand{\conjunctTF}{\ensuremath{C(\vec{x}, \vec{x}')}}
\newcommand{\GammaTF}{\ensuremath{\Gamma(\vec{x}, \vec{x}')}}


\usetheme{Madrid}
\usecolortheme{dolphin}

\title[\qvasr]{Rational Vector Addition Systems \\ with Resets and States}

% For short you could for example use the last name only, it's optional as is
% the short title
\author[Jonas Werner]{Jonas Werner}

% Can be set, for students usually not required
%\institute{}

\date{\today}

% removes the navigation symbols
% often they don't work and I find them rather annoying
\beamertemplatenavigationsymbolsempty{}

% You can have a logo to appear on all slides
% \logo{\includegraphics[height=1.5cm]{logo}}

% If you want that at the beginning of each section the table of contents
% is shown, I don't like it for short presentations
\AtBeginSection[]{
	\begin{frame}
		\vfill
		\centering
		\begin{beamercolorbox}[sep=8pt,center,shadow=true,rounded=true]{title}
			\usebeamerfont{title}\insertsectionhead\par%
		\end{beamercolorbox}
		\vfill
	\end{frame}
}


\begin{document}

\begin{frame}
  \titlepage
\end{frame}

%%%%%%%%%%%%%%%%%%%%%%%%%%%%%%%%%%%%%%%%%%%%%%%%%%%%%%%%%%%%%%%%%%%%%%%%%%%%%% new frame
\begin{frame}[t]
	\frametitle{Motivation}
	\begin{itemize}
		\item Loop analysis is one of the most challenging parts of software verification \\
		$\rightarrow$ "Archilles Heel of program verification" \cite{DBLP:journals/fmsd/KroeningSTTW13} \pause
		\begin{columns}
			\begin{column}{0.4\textwidth}
				\begin{figure}[h]
					\vspace*{0.5cm}
					\resizebox{0.6\textwidth}{!}{\begin{lstlisting}[language=C++, style=withAssert]  % Start your code-block
	
	int x := 0;
	int y := 2;
	int z := 3;
	while x <= 20:
		if x <= 10:
			z := x;
			x := x + y;
			y := y + 1;
		else:
			x := x + 2;
			y := y - 3;
	assert x == 21;
	\end{lstlisting}}
				\end{figure}
			\end{column}
			\begin{column}{0.6\textwidth}
				\pause
				\only<3> {
				Loop Unwinding:
				\begin{itemize}
					\item Quite simple
					\item {\color{red}{But:}} \\
					Cannot reason about loop behavior beyond unwinding bound
				\end{itemize} }
				\only<4> {
					Loop Invariants:
					\begin{itemize}
						\item Loop invariants hold before and after the loop
						\item {\color{red}{But:}} \\
						Usefulness is tied to how strong they are \\
						$\rightarrow$ Finding strong loop invariants "is an art" \cite{DBLP:journals/fmsd/KroeningSTTW13}
				\end{itemize} }				
				\only<5> {
					Loop Summarization:
					\begin{itemize}
						\item Models relation between input and output as a set of constrains
						\item {\color{red}{But:}} \\
						Not guaranteed to be precise \\
						$\rightarrow$ Over- or underapproximation of loop behavior
				\end{itemize} }	
				\only<6> { }	
			\end{column}
		\end{columns}
	\only<6>{
	\item Solution: \\
	Overapproximative loop summary using rational vector addition systems (\qvasr) \cite{DBLP:conf/cav/SilvermanK19} \\
	$\rightarrow$ Overapproximation with a guaranteed degree of precision
	}
	\end{itemize}
\end{frame}
%%%%%%%%%%%%%%%%%%%%%%%%%%%%%%%%%%%%%%%%%%%%%%%%%%%%%%%%%%%%%%%%%%%%%%%%%%%%%% new frame

\begin{frame}[t]
	\frametitle{\qvasr}
	\begin{definition}
			A \qvasr\ $V$ is a transition system, consisting of a set of transformer: \\
			\begin{center}
			 $ V = 
			\begin{Bmatrix}
				\begin{pmatrix}
					\underbrace{
					\begin{bmatrix}
						r_1 \\
						\vdots \\
						r_n
					\end{bmatrix}}_{reset \in \{0,1\}^n},
					\underbrace{
					\begin{bmatrix}
						a_1 \\
						\vdots \\
						a_n
					\end{bmatrix}}_{addition \in \mathbb{Q}^n}
				\end{pmatrix}
			\end{Bmatrix}
		$
		\end{center}
	\end{definition}
	\begin{definition}
	Given a \qvasr $V$ and a transformer $(\vec{r}, \vec{a}) \in V$, transitions between two states $\vec{x} \rightarrow_V \vec{x}'$ are defined as: \\
	\begin{equation*}
			\begin{bmatrix}
				x_1' \\
				\vdots \\
				x_n'
			\end{bmatrix}
		=
			\begin{bmatrix}
				x_1 \\
				\vdots \\
				x_n
			\end{bmatrix}
		*
			\begin{bmatrix}
			r_1 \\
			\vdots \\
			r_n
			\end{bmatrix}
		+
			\begin{bmatrix}
			a_1 \\
			\vdots \\
			a_n
		\end{bmatrix}		
	\end{equation*}
	\end{definition}
\end{frame}

%%%%%%%%%%%%%%%%%%%%%%%%%%%%%%%%%%%%%%%%%%%%%%%%%%%%%%%%%%%%%%%%%%%%%%%%%%%%%% new frame
\begin{frame}[t]
	\frametitle{\qvasr\ Example}
	\begin{columns}
		\begin{column}{0.4\textwidth}
			\begin{figure}[h]
				\vspace*{0.5cm}
				\resizebox{0.6\textwidth}{!}{\begin{lstlisting}[language=C++, style=withAssert]  % Start your code-block
	
	int x := 0;
	int y := 2;
	int z := 3;
	while x <= 20:
		if x <= 10:
			z := x;
			x := x + y;
			y := y + 1;
		else:
			x := x + 2;
			y := y - 3;
	assert x == 21;
	\end{lstlisting}}
				\vspace{-0.5cm}
				\caption*{Program containing a loop.}
			\end{figure}
		\end{column} \pause
			\begin{column}{0.6\textwidth}
			\begin{figure}[h]
				\begin{equation*}
					G: x \leq 20 \land x > 10 \land x' = {\color<4>{red}{x}} + {\color<5>{red}{2}} \land y' = {\color<4>{red}{y}} {\color<5>{red}{-3}}
				\end{equation*}
			\caption*{Transition formula $G$ of the else branch.}
			\end{figure}
		\end{column}
	\end{columns}
\end{frame}

%%%%%%%%%%%%%%%%%%%%%%%%%%%%%%%%%%%%%%%%%%%%%%%%%%%%%%%%%%%%%%%%%%%%%%%%%%%%%% new frame
\begin{frame}[t]
	\frametitle{\qvasr\ Example}
	\begin{columns}
		\begin{column}{0.4\textwidth}
			\begin{figure}[h]
				\vspace*{0.5cm}
				\resizebox{0.6\textwidth}{!}{\begin{lstlisting}[language=C++, style=withAssert]  % Start your code-block
	
	int x := 0;
	int y := 2;
	int z := 3;
	while x <= 20:
		if x <= 10:
			z := x;
			x := x + y;
			y := y + 1;
		else:
			x := x + 2;
			y := y - 3;
	assert x == 21;
	\end{lstlisting}}
				\vspace{-0.5cm}
				\caption*{Program containing a loop.}
			\end{figure}
		\end{column} \pause
		\begin{column}{0.6\textwidth}
			\begin{figure}[h]
				\begin{equation*}
					H: x \leq 10 \land x' = x + {\color<3->{red}{y}} \land y' = y + 1 \land z' = {\color<3->{red}{x}}
				\end{equation*}
				\caption*{Transition formula $H$ of the if branch.}
			\end{figure}
		\end{column}
	\end{columns}
	\pause
	\vspace*{1cm}
	\only<4->{
	\begin{itemize}
		\item \alert{\qvasr\ can only model constant additions!} \pause \pause
		\item But, we can overapproximate $H$ by computing a \qvasr-abstraction
	\end{itemize}
	}
\end{frame}


%%%%%%%%%%%%%%%%%%%%%%%%%%%%%%%%%%%%%%%%%%%%%%%%%%%%%%%%%%%%%%%%%%%%%%%%%%%%%% new frame
\begin{frame}
	\frametitle{\qvasr-Abstraction}
	\newcommand{\s}{\ensuremath{\begin{bmatrix} s_x & s_y & s_z \end{bmatrix}}}
	\newcommand{\p}{\ensuremath{\begin{bmatrix} x' \\ y' \\ z' \end{bmatrix}}}
	\newcommand{\up}{\ensuremath{\begin{bmatrix} x \\ y \\ z \end{bmatrix}}}
	\begin{definition}
		Given a transition formula $F$, a \qvasr-abstraction for $F$ is a tuple $(S_F, V_F)$ consisting of a matrix $S_F$ and a \qvasr\ $V_F$. The matrix $S_F$ acts as a linear transformation of $F$ to $V_F$. \\ Meaning, for every transition \\
			$
			\begin{bmatrix}
				x_1 \\
				\vdots \\
				x_n
			\end{bmatrix}
			\rightarrow_F
			\begin{bmatrix}
				x_1' \\
				\vdots \\
				x_n'
			\end{bmatrix}
			$
			in $F$, there is: \ \
			$ S_F \cdot
				\begin{bmatrix}
					x_1' \\
					\vdots \\
					x_n'
				\end{bmatrix}
				=
				S_F \cdot
				\begin{bmatrix}
					x_1 \\
					\vdots \\
					x_n
				\end{bmatrix}
				*
				%%%%%%%%%%%%%%%%%%%% RESET
				\only<3-5>{
				\begin{bmatrix}
					\color{red}{0} \\
					\color{red}{\vdots} \\
				\color{red}{0}
				\end{bmatrix}	
				}
				%%%%%%%%%%%%%%%%%%%%% VANILLA
				\only<1, 4->{
				\begin{bmatrix}
					r_1 \\
					\vdots \\
					r_n
				\end{bmatrix}
				}
				%%%%%%%%%%%%%%%%%%%%%% NOT RESET
				\only<6>{
				\begin{bmatrix}
					\color{red}{1} \\
					\color{red}{\vdots} \\
					\color{red}{1}
				\end{bmatrix}
				}
				+
				\begin{bmatrix}
					a_1 \\
					\vdots \\
					a_n
				\end{bmatrix}	
			$
			in $V_F$
	\end{definition}
	\pause
	\begin{center}
			$H: x \leq 10 \land x' = x + y \land y' = y + 1 \land z' = x$ \pause
			\only<2-5>{
			\begin{equation*}
				Res_H = \left\{ (\s, a) : H \models \underbrace{\s}_{solve} \cdot \p = \underbrace{a}_{solve} \right\} 	
			\end{equation*} \pause
			$Res_H = \left\{ (\begin{bmatrix} -a & a & a \end{bmatrix}, a) \right\}\ \pause \xrightarrow{\text{Base}}\ Res_H = \left\{ (\begin{bmatrix} -1 & 1 & 1 \end{bmatrix}, 1) \right\}\ $
		}
		\only<6->{
			\begin{equation*}
				Inc_H = \left\{(\s, a) : H \models \s \cdot \p = \s \cdot \up + a\right\}	
			\end{equation*}
			$Inc_H = \left\{ (\begin{bmatrix} 0 & a & 0 \end{bmatrix}, a) \right\}\ \pause \xrightarrow{\text{Base}}\ Inc_H = \left\{ (\begin{bmatrix} 0 & 1 & 0 \end{bmatrix}, 1) \right\}\ $
		}
	\end{center}
\end{frame}

%%%%%%%%%%%%%%%%%%%%%%%%%%%%%%%%%%%%%%%%%%%%%%%%%%%%%%%%%%%%%%%%%%%%%%%%%%%%%% new frame
\begin{frame}
	\frametitle{\qvasr-Abstraction Partial Order}
	\resizebox{0.4\textwidth}{!}{% Define block styles
\begin{tikzpicture}[%
	->,
	>=stealth',
	shorten >=1pt,
	auto,
	node distance=2cm,
	scale=0.9,
	transform shape,
	align=center,
	smallnode/.style={inner sep=1.4},
	initial text =,
	anchor=center]
	% Place nodes
	\node [draw, ellipse](tf) {F};
	\node [draw, ellipse](best)[right of=tf, xshift=1cm]  {$\tilde{V}$};
	\node [draw, ellipse](vas1)[above right of=best] {$V_G$};
	\node [draw, ellipse](vas2)[below right of=best] {$V_H$};

	\draw (tf) to node[above] {} node[above] {$\tilde{S}$} (best);
	\draw (best) to node[above] {} node[left, xshift=-0.25cm] {$\tilde{S}$} (vas1);
	\draw (best) to node[above] {} node[left, xshift=-0.25cm] {$\tilde{S}$} (vas2);
	
	\draw[bend left] (tf) to node[above] {} node[left, xshift=-0.5cm] {$S_H$} (vas1);
	\draw[bend right] (tf) to node[above] {} node[left, xshift=-0.5cm] {$S_G$} (vas2);
	
\end{tikzpicture}}
\end{frame}

%%%%%%%%%%%%%%%%%%%%%%%%%%%%%%%%%%%%%%%%%%%%%%%%%%%%%%%%%%%%%%%%%%%%%%%%%%%%%% new frame

\begin{frame}[t]
	\frametitle{\qvasrs}
\end{frame}
%%%%%%%%%%%%%%%%%%%%%%%%%%%%%%%%%%%%%%%%%%%%%%%%%%%%%%%%%%%%%%%%%%%%%%%%%%%%%% new frame

\begin{frame}[t]
	\frametitle{Approach}
		\nocite{DBLP:conf/cav/SilvermanK19}
	\resizebox{11cm}{!}{% Define block styles
\tikzstyle{block} = [rectangle, draw, rounded corners, minimum height=5em, minimum width={width("predicate transformer")+15pt},
]
\tikzstyle{group} = [rectangle, draw, rounded corners, minimum height=5em
]
\begin{tikzpicture}[%
    ->,
	>=stealth',
	shorten >=1pt,
	auto,
	node distance=3.25cm and 4.5cm,
	scale=0.9,
	transform shape,
	align=center,
	smallnode/.style={inner sep=1.4},
	initial text =,
	anchor=center]
	% Place nodes
	\node [align=left](input) {\textbf{Input}};
	\node [block, align=left, below=of input](predtransformer) {\texttt{predicate transformer}};
	\node [block, align=center, right=of predtransformer](vasrs) {Are there $p, q \in P$ with no \\ \qvasr-abstraction \\ \pqvasr $\in \mathcal{A}$?};
	\node [block, align=left, right=of vasrs](reach) {\texttt{reach}};
	\node [block, align=left, below=of vasrs, yshift=1cm](H) {\pqformula satisfiable?};
	\node [block, align=center, right=of H](abstr) {\texttt{\qvasr-abstractor}};
	\node [block, align=left, below=of abstr, fill=white, yshift=1cm](pushout) {\texttt{pushout}};
	\node [block, align=left, below=of H, fill=white, yshift=1cm](image) {\texttt{image-builder}};
	\node [align=left, above= of reach](output) {\textbf{Output}};
	\begin{scope}[on background layer]
		\node[group, draw=black,fill=stmtcolor,fit=(pushout) (image), label=below :{\texttt{\qvasr-join}}](FIt1) {};
	\end{scope}
	
	% Draw edges
	\draw (input) to node[above] {} node[above right] {$F :=$ loop's transition formula \\ with $n$ variables} (predtransformer);
	%\draw (predtransformer) to node[right] {} node[right, align=left] {$F$\\ $P$ := set of predicates \\ $\mathcal{V} := \emptyset$} (vasrs);
	\draw (reach) to node[above] {} node[left] {\texttt{reach($\mathcal{V}$)} is \\ loop summary of $F$} (output);
	\draw [bend right]  (vasrs) to node[above left, align=right, yshift=-0.5cm] {\textbf{yes} \\ $\pqformula := p \land F \land q$ \\ $\pqvasr := (\mathit{I_n},\emptyset)$} node[right, align=left] {} (H);
	\draw (H) to node[above, xshift=-1.5cm] {\textbf{yes}} node[below] {$c :=$ cube in $\mathit{DNF(\pqformula)}$} (abstr);
	\draw (abstr) to node[above] {} node[left] {$(S_c, V_c) :=$ \\ \texttt{abstraction(c)}} (pushout);
	\draw (pushout) to node[below] {$\pq{S}$ simulates $(S_c, V_c)$} node[above] {$\pq{S}$ := \texttt{pushout($\pq{S}, S_c$)} } (image);
	
	%\draw (image) to node[above] {} node[left, align=left] {$\pqformula :=$ refine$(\pqformula)$ \\ $\pq{V} :=$ \texttt{image(\pq{V})} \\ $\cup$ \texttt{image$(V_c)$}} (H);
	
		\draw (image) to node[above] {} node[right, align=left] {%
		{\begin{varwidth}{3.5cm}
				\vspace*{-0.5cm}
				\begin{align*}
					\pqformula :=&\ \texttt{refine}(\pqformula) \\ \bigskip
					\pq{V} :=&\ \texttt{image}(\pq{V}) \\
					&\cup \texttt{image}(V_c)
		\end{align*}\end{varwidth}}
	} (H);
	
	
	\draw (predtransformer) to node[above] {} node[above, align=left] {%
		{\begin{varwidth}{3.75cm}
			\begin{align*}
				F \\
				P &:= \text{set of predicates} \\
				\mathcal{A} &:= \emptyset
			\end{align*}\end{varwidth}}
		} (vasrs);

	\draw [bend right]  (H) to node[above] {} node[right, align=left, yshift=-0.25cm] {$\mathcal{A} := \mathcal{A} \cup \pqvasr$ \vspace{0.5cm} \\ \pqvasr is best abstraction\\ \textbf{no}} (vasrs);
	\draw (vasrs) to node[below] {$\mathcal{V} := \qvasrs(\mathcal{A}, P)$} node[above, xshift=-1.5cm] {\textbf{no}} (reach);
\end{tikzpicture}}
\end{frame}
%%%%%%%%%%%%%%%%%%%%%%%%%%%%%%%%%%%%%%%%%%%%%%%%%%%%%%%%%%%%%%%%%%%%%%%%%%%%%% new frame

\begin{frame}
	\label{slide:bibliography}
	\frametitle{Bibliography}
	\fontsize{10}{10}\selectfont
	\bibliographystyle{apalike}
	\nocite{DBLP:conf/cav/SilvermanK19} \nocite{DBLP:conf/cav/SilvermanK19}
	\bibliography{bib/bib}
\end{frame}

\begin{comment}
\begin{frame}{Bibliography}
	\frametitle{References}
	\begin{thebibliography}{99} % Beamer does not support BibTeX so references must be inserted manually as below
	\bibitem[DBLP:conf/rp/HaaseH14]{p1} Christoph Haase and	Simon Halfon
	\newblock Integer Vector Addition Systems with States
	\newblock \href{https://doi.org/10.1007/978-3-319-11439-2\_9}{Lecture Notes in Computer Science, 2014}
	
	\bibitem[DBLP:journals/fmsd/KroeningLW15]{p2} Daniel Kroening and
	Matt Lewis and Georg Weissenbacher
	\newblock Under-approximating Loops in C Programs for Fast Counterexample Detection
	\newblock \href{https://doi.org/10.1007/s10703-015-0228-1}{Reachability Problems - 8th International Workshop, 2014}
	
	\bibitem[DBLP:journals/fmsd/KroeningSTTW13]{p3} Daniel Kroening and
	Natasha Sharygina and Stefano Tonetta and Aliaksei Tsitovich and Christoph M. Wintersteiger
	\newblock Loop Summarization using State and Transition Invariants
	\newblock \href{https://doi.org/10.1007/s10703-012-0176-y}{Formal Methods Syst. Des., 2013}
	
	\bibitem[DBLP:conf/cav/SilvermanK19]{p4} Jake Silverman and Zachary Kincaid
	\newblock Loop Summarization with Rational Vector Addition Systems
	\newblock \href{https://doi.org/10.1007/978-3-030-25543-5\_7}{CAV, 2019}
	
	\bibitem[DBLP:journals/tse/XieCZLLL19]{p5} Xiaofei Xie and
	Bihuan Chen and	Liang Zou and Yang Liu and Wei Le and Xiaohong Li
	\newblock Automatic Loop Summarization via Path Dependency Analysis
	\newblock \href{https://doi.org/10.1109/TSE.2017.2788018}{{IEEE} Trans. Software Eng., 2019}

	\end{thebibliography}

\end{frame}
\end{comment}

\end{document}
