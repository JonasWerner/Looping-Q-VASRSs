\usepackage[utf8]{inputenc}
\usepackage{xspace}
\usepackage{tabularx}
\usepackage{hyperref}
\usepackage{algorithm}
\usepackage[noend]{algpseudocode}
\usepackage{cite}
\usepackage{booktabs}
\usepackage{url}
\usepackage{listings}
\usepackage{amsthm}
\usepackage{amsmath}
\usepackage{tikz}
\usetikzlibrary{positioning,shapes.geometric, arrows.meta ,automata, decorations.pathreplacing, calc, fit, backgrounds, quotes, tikzmark, calc}
\usepackage{pgf}
\usepackage{slantsc}
\usepackage{geometry}
\usepackage{amssymb}
\usepackage{subcaption}
\usepackage{float}
\usepackage{pgf}
\usepackage{slashbox}
\usepackage{pgfgantt}
\usepackage{wrapfig}
\usepackage{pdflscape}
\usepackage{xcolor}
\usepackage{colortbl}
\usepackage{xparse}
\usepackage[%disable,%
colorinlistoftodos,%
color=cyan!50!white,%
bordercolor=cyan!50!black]{todonotes}

\usepackage{varwidth}
\usepackage[most]{tcolorbox}% http://ctan.org/pkg/tcolorbox
\tcbuselibrary{skins,breakable}
\usepackage{comment}
\usepackage{makecell}
\usepackage{stmaryrd}
\usepackage{nicematrix}
\usepackage{mathtools}


\DeclareSymbolFont{matha}{OML}{txmi}{m}{it}% txfonts
\DeclareMathSymbol{\varv}{\mathord}{matha}{118}

\lstset{ %
	backgroundcolor=\color{white}, 
	basicstyle=\footnotesize,       
	breakatwhitespace=false,        
	breaklines=true,                 
	captionpos=b,                    
	commentstyle=\color{mygreen},   
	escapeinside={\%*}{*)},        
	extendedchars=true,              
	frame=single,                  
	keywordstyle=\color{blue},       
	language=Prolog,                
	numbers=left,                    
	numbersep=5pt,                   
	rulecolor=\color{black},        
	showspaces=false,               
	showstringspaces=false,          
	showtabs=false,                  
	stringstyle=\color{mymauve},   
	tabsize=2,                      
	title=\lstname,                  
	morekeywords={not,\},\{,preconditions,effects },            
	deletekeywords={time}            
}

\usetheme{Madrid}
\usecolortheme{freiburg}

%%%%%%%%%%%% Colors
%generic colors
\definecolor{myyellow}		{RGB}{242,226,149} 
\definecolor{orange}   		{RGB}{255,128,0}
\definecolor{lightorange}   {RGB}{255,183,112}
\definecolor{darkred}  		{RGB}{128,0,0}
\definecolor{darkgreen}		{RGB}{0,128,0}
\definecolor{dirtgreen}		{RGB}{180,210,180}
\definecolor{mixgreen}		{RGB}{34,139,34}
\definecolor{darkblue} 		{RGB}{0,0,128}
\definecolor{darkpurple}	{RGB}{160,32,240}
\definecolor{lightpurple}	{RGB}{180,180,210}
\definecolor{bl1}   		{RGB}{204,204,255}
\definecolor{bl2}   		{RGB}{128,128,255}
\definecolor{bl3}   		{RGB}{140,160,255}


%generic grays
\definecolor{gr1}		{RGB}{250, 250, 250}
\definecolor{gr2}		{RGB}{229, 229, 229}
\definecolor{gr3}		{RGB}{212, 212, 212}
\definecolor{gr4}		{RGB}{204, 204, 204}

%% a somewhat friendly scheme for 5 different colors 
\definecolor{g1}		{RGB}{215,25,28} % a kind of red
\definecolor{g2}		{RGB}{253,174,97} % a kind of orange
\definecolor{g3}		{RGB}{255,255,191} % a kind of yellow
\definecolor{g4}		{RGB}{171,217,233} % a kind of light blue 
\definecolor{g5}		{RGB}{44,123,182} % a kind of dark blue 

%% possibly the closest color scheme for 9 different colors -- use it with extreme caution 
\definecolor{s1}		{RGB}{228,26,28}
\definecolor{s2}		{RGB}{55,126,184}
\definecolor{s3}		{RGB}{77,175,74}
\definecolor{s4}		{RGB}{152,78,163}
\definecolor{s5}		{RGB}{255,127,0}
\definecolor{s6}		{RGB}{255,255,51}
\definecolor{s7}		{RGB}{166,86,40}
\definecolor{s8}		{RGB}{247,129,191}
\definecolor{s9}		{RGB}{153,153,153}

% colors for boxes
\colorlet{workflowbg}		{lightorange}

% colors for syntax highlighting in listings 
\colorlet{codegreen}		{mixgreen}
\colorlet{codepurple}		{darkpurple}
\colorlet{codehighlight}	{gr2}

% color for statements throughout the thesis 
\colorlet{stmtcolor}		{gr2} %default statement color
\colorlet{prg}				{gr2} %bp: from program
\colorlet{ltl}				{g4} %bp: from ltl
\colorlet{notIC}			{bl3} %tip: not in unsat core
\colorlet{stateass}			{g2}
\colorlet{stateassH}		{ALUblue}

%color for emphasis 
\colorlet{emred}			{ALUred}
\colorlet{emgreen}			{darkgreen}
\colorlet{emblue}			{ALUblue}
\colorlet{highlight}		{g3}

%colors for matthias-style 
\colorlet{memsafety}		{darkpurple}
\colorlet{reachability}		{ALUred}
\colorlet{termination}		{darkgreen}
\colorlet{overflows}		{g2}
\colorlet{concurrency}		{g3!80!black}


\colorlet{colProg}			{darkpurple}
\colorlet{colInv}			{stateass}
\colorlet{colCov}			{g4}
\colorlet{colFH}			{ALUred!80!white}
\colorlet{colTr}			{g4}


% \definecolor{colProg}{rgb}{1,0,1}
% \definecolor{colInv}{rgb}{0,1,0}
% \definecolor{colCov}{rgb}{1,0,0}
% \definecolor{colFH}{rgb}{0,0.7,0}
% \definecolor{colTr}{rgb}{1,0.2,0}


\colorlet{v1fill}{white}
\colorlet{v2fill}{white}
\colorlet{v3fill}{white}



%color for pictures
\colorlet{outlineblue}		{g5}
\colorlet{fillblue}			{g4}
\colorlet{darkback}			{gr2}
\colorlet{lightback}		{gr1}
\colorlet{blockback}		{ALUblue!10!bg}


% colors for different ultimate plugins
\colorlet{source}			{gr1}
\colorlet{controller}		{gr1}
\colorlet{thirdparty}		{gr1} 
\colorlet{core}				{gr1}
\colorlet{libs}				{gr1}
\colorlet{source}			{gr1}
\colorlet{analyzer}			{gr1}
\colorlet{generator}		{gr1}
\colorlet{output}			{gr1}




%%%%%%%%%%%%% Statements and labels Trace Abstraction Style
\tikzstyle{st} = [%
font=\ttfamily,%
shape=rectangle,%
rounded corners=.5em,%
fill=stmtcolor,%
inner xsep=.3em,%
inner ysep=0em, %
text height=2ex, %
text depth=.6ex,
]


\newcommand{\tikzstmt}[3]{{%
		\tikz[baseline]{%
			\node[st,fill=#2] at (0,.64ex){%
				\hspace{.3em}\texttt{\strut#3#1}\hspace{.3em}\strut};}
}}

\newcommand{\stcol}[2]{\tikzstmt{#1}{#2}{}}
\newcommand{\stsmcol}[2]{\tikzstmt{#1}{#2}{\small}}
\newcommand{\stfootcol}[2]{\tikzstmt{#1}{#2}{\footnotesize}}

\newcommand{\stnorm}[1]{\stcol{#1}{stmtcolor}}
\newcommand{\stsm}[1]{\stsmcol{#1}{stmtcolor}}
\newcommand{\stfoot}[1]{\stfootcol{#1}{stmtcolor}}

\newcommand{\st}[1]{\stfoot{#1}}
\newcommand{\lab}[1]{\stfoot{\ensuremath{#1}}}
\newcommand{\lan}[1]{\stnorm{\ensuremath{#1}}}
\newcommand{\stn}[1]{\stnorm{#1}}

\newcommand{\formula}[2]{\tikz[baseline]{\node[shape=rectangle,line width=1pt,draw=#2,fill=#2!30,inner sep=1pt, align=center] at (0,.64ex){\hspace{.2em}\texttt{\strut#1}\hspace{.1em}\strut};}}
\newcommand{\itp}[1]{\formula{\ensuremath{#1}}{itp}}

\newcommand{\tf}{\ensuremath{\varphi}}
\newcommand{\ctf}{\ensuremath{\widehat{\varphi}}}
\newcommand{\invars}{\ensuremath{In}}
\newcommand{\outvars}{\ensuremath{Out}}
\newcommand{\auxvars}{\ensuremath{Aux}}

\newcommand{\Var}{\ensuremath{\mathit{Var}}}
\newcommand{\stmt}{\ensuremath{\mathit{Stmt}}}
\newcommand{\Loc}{\ensuremath{\mathit{Loc}}}
\newcommand{\err}{\ensuremath{\mathit{err}}}
\newcommand{\init}{\ensuremath{\mathit{init}}}

\newcommand{\abst}[1]{\ensuremath{\mathit{(S^{#1}, V^{#1})}}}
\algnewcommand\algorithmicforeach{\textbf{for each}}
\algdef{S}[FOR]{ForEach}[1]{\algorithmicforeach\ #1\ \algorithmicdo}

\newcommand{\pq}[1]{\ensuremath{#1_{p, q}}}
\newcommand{\pqvasr}{\ensuremath{(\pq{S}, \pq{V})}}
\newcommand{\pqformula}{\ensuremath{\pq{\Gamma}}}

%%%%%%%%%%%% Numbered example environment
% \newcounter{example}[section]
% \newenvironment{example}[1][]{\refstepcounter{example}\par\medskip
%    \noindent \textbf{Example~\theexample. #1} \rmfamily}{\medskip}


\usepackage{regexpatch}

\makeatletter
\xpatchcmd{\algorithmic}{\labelsep 0.5em}{\labelsep 1.5em}{\typeout{Success!}}{\typeout{Oh dear!}}
\makeatother

%%%%%%%%%%%% Comments
\newif\iffinal
%\finaltrue % comment out to remove comments

\iffinal
\newcommand\mycom[1]{}
\else
\newcommand\mycom[1]{#1}
\overfullrule=1mm
\fi
\setlength\parindent{0pt}

\newcommand{\jw}[1]{\mycom{\todo[color=blue!40,inline]{\small JW: #1}}}
\newcommand{\dd}[1]{\mycom{\todo[color=orange!40,inline]{\small DD: #1}}}
\newcommand{\ts}[1]{\mycom{\todo[color=green!40,inline]{\small TS: #1}}}


\newcommand{\all}[1]{\mycom{\todo[color=green!40,inline]{\small #1}}}
\newcommand{\meta}[1]{\mycom{\todo[color=blue!10,inline,caption={Beschreibung},nolist]{\setlist{nolistsep}\small #1}}}
\newcommand{\xxx}{\mycom{\stfootcol{Placeholder}{blue!20}}}
\newcommand{\cn}{\mycom{\stfootcol{Cite}{blue!20}}}

\newcommand{\Q}{\ensuremath{\mathbb{Q}}}
\newcommand{\entails}[1]{\vdash_{#1}}

\newcommand{\qvasr}{\ensuremath{\mathit{\mathbb{Q} \text{-}VASR}}}
\newcommand{\qvasrs}{\ensuremath{\mathit{\mathbb{Q} \text{-}VASRS}}}

\newcommand{\tranFormula}{\ensuremath{F(\vec{x}, \vec{x}')}}
\newcommand{\coherent}{\ensuremath{\equiv_{\mathit{V}}}}
\newcommand{\conjunctTF}{\ensuremath{C(\vec{x}, \vec{x}')}}
\newcommand{\GammaTF}{\ensuremath{\Gamma(\vec{x}, \vec{x}')}}
