\begin{frame}[t]
	\frametitle{Motivation}
	\begin{center}
		Loop analysis is one of the most challenging parts of software verification \\
		"Archilles Heel of program verification" \cite{DBLP:journals/fmsd/KroeningSTTW13} \pause
			\begin{minipage}{0.35\textwidth}
				\vspace*{0.5cm}
				\resizebox{0.7\textwidth}{!}{\begin{lstlisting}[language=C++,basicstyle=\ttfamily,keywordstyle=\color{blue}]  % Start your code-block
	
	int x := 0;
	int y := 2;
	int z := 3;
	while x <= 21:
		if x <= 10:
			z := x;
			x := x + y;
			y := y + 1;
		else:
			x := x + 2;
			y := y - 3;
	assert x == 22;
	\end{lstlisting}}
			\end{minipage}
		\pause
		\begin{tabular}{@{}l@{}}
			\tabitem Loop Unwinding \\
			\tabitem Loop Invariants \\
			\tabitem Loop Summarization
		\end{tabular}
	\end{center} 
\end{frame}

\begin{frame}[t]
	\frametitle{Loop Summarization}
	\begin{center}
		  \begin{tikzpicture}
		  \node[] at (2.7,2) {\color{emblue}{Actual Loop Behaviour}};
		  \draw [rounded corners,red,thick,dashed] (0,0) -- (3, -1) -- (5.5,0.5) -- (6, 3.5) -- (4, 5)-- (0.25, 4) node [midway, sloped, above] {Overapproximation} -- cycle;
		  \draw[emblue, fill=emblue!20, fill opacity=0.2] (2.7,2) circle (2.5);
		  \node at (7, 5) (head) []{Spurious Counterexample};
		  \node at (5.5, 3.25) (dot) [circle,fill,inner sep=1.5pt]{};
		  \path (head) edge node []{} (dot);
		\end{tikzpicture}
	\end{center}
\end{frame}

\begin{frame}[t]
	\frametitle{Loop Summarization}
		\begin{center}
		\begin{tikzpicture}
		 	 \node[] at (0 ,3.25) {\color{emblue}{Actual Loop Behaviour}};
			\draw[emblue, fill=emblue!20, fill opacity=0.2] (0,0) circle (3);
			\node[] at (0,0) {\color{green}{Underapproximation}};
			\draw [rounded corners, green,thick,dashed] (-2,-2) -- (0, -2.75) -- (1.25, -2.5) -- (2.5, -0.5) -- (1.25, 1.5) -- (0.25, 2.8) -- (-2, 2) -- (-2.8, 0) -- cycle;
			\node at (5, 3) (head) []{Uncovered Counterexample};
			\node at (2, 1.5) (dot) [circle,fill,inner sep=1.5pt]{};
			\path (head) edge node []{} (dot);
		\end{tikzpicture}
	\end{center}
\end{frame}