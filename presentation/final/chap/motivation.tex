\begin{frame}[t]
	\frametitle{Motivation}
	\begin{center}
		Loop analysis is one of the most challenging parts of software verification \\
		"\color{emblue}Archilles Heel of program verification \color{black}" \cite{DBLP:journals/fmsd/KroeningSTTW13} \pause
			\begin{minipage}{0.35\textwidth}
				\vspace*{0.5cm}
				\resizebox{0.7\textwidth}{!}{\begin{lstlisting}[language=C++, style=withAssert]  % Start your code-block
	
	int x := 0;
	int y := 2;
	int z := 3;
	while x <= 20:
		if x <= 10:
			z := x;
			x := x + y;
			y := y + 1;
		else:
			x := x + 2;
			y := y - 3;
	assert x == 21;
	\end{lstlisting}}
			\end{minipage}
		\pause
		\begin{tabular}{@{}l@{}}
			\onslide<3->\tabitem Loop Unwinding \\
			\onslide<4->\tabitem Loop Invariants \\
			\onslide<5->\tabitem Loop Summarization
		\end{tabular}
	\end{center} 
\end{frame}

\begin{frame}[t]
	\frametitle{Loop Unwinding}
	\begin{center}
		Unrolling a loop until a certain bound\\
		\begin{minipage}{0.35\textwidth}
			\vspace*{0.5cm}
			\resizebox{0.7\textwidth}{!}{\begin{lstlisting}[language=C++, style=withAssert]  % Start your code-block
	
	int x := 0;
	int y := 2;
	int z := 3;
	while x <= 20:
		if x <= 10:
			z := x;
			x := x + y;
			y := y + 1;
		else:
			x := x + 2;
			y := y - 3;
	assert x == 21;
	\end{lstlisting}}
		\end{minipage}
		\pause
		\begin{tabular}{@{}l@{}}
			\onslide<3->\tabitem  Simplest Approach \\
			\onslide<4->\tabitem \color{red} Cannot argue beyond bound!
		\end{tabular}
	\end{center} 
\end{frame}

\begin{frame}[t]
	\frametitle{Loop Invariants}
	\begin{center}
		\onslide<+->
		Conditions that hold before and after each loop iteration		
		\begin{columns}[c]
		\begin{column}{0.38\textwidth}
			\onslide<+->
			\begin{figure}[h]
				\vspace*{0.5cm}
				\resizebox{0.65\textwidth}{!}{\begin{lstlisting}[language=C++,basicstyle=\ttfamily,keywordstyle=\color{blue}, escapechar=\%]  % Start your code-block
	
	int x := 0;
	int y := 2;
	int z := 3;
%\itp{(x \leq 22 \land x > 11) \lor x \leq y + 9}%
	while x <= 20:
		if x <= 10:
			z := x;
			x := x + y;
			y := y + 1;
		else:
			x := x + 2;
			y := y - 3;
	assert x == 21;
	\end{lstlisting}}
			\end{figure}
		\end{column}
			\onslide<+->
		\begin{column}{0.05\textwidth}
				But:
		\end{column}
		\begin{column}{0.38\textwidth}
			\begin{figure}[h]
				\vspace*{0.5cm}
				\resizebox{0.65\textwidth}{!}{\begin{lstlisting}[language=C++,basicstyle=\ttfamily,keywordstyle=\color{blue}, escapechar=\%]  % Start your code-block
	
	int x := 0;
	int y := 2;
	int z := 3;
	%\itp{Useless Invariant}%
	while x <= 20:
		if x <= 10:
			z := x;
			x := x + y;
			y := y + 1;
		else:
			x := x + 2;
			y := y - 3;
	assert x == 22;
	\end{lstlisting}}
			\end{figure}
		\end{column}
	\end{columns}
	\onslide<+->
	Not all loop invariants are useful! \\
	\onslide<+->
	Finding useful loop invariants "\color{emblue}is an art in itself \color{black}" \cite{DBLP:journals/fmsd/KroeningSTTW13}
	\end{center}
\end{frame}

\begin{frame}[t]
	\frametitle{Loop Summarization}
	\begin{center}
		\onslide<+->
		 Represent relationships between inputs and outputs to model loop's effect on variables \\ \vspace*{0.25cm}
		There are two kinds of loop summarization: \\
		\onslide<+-> \color{emblue}Underapproximative \color{black} and \color{emblue}Overapproximative \color{black} Summarization
		\begin{columns}[t]
			\begin{column}{0.35\textwidth}
				\vspace*{0.5cm} \\
				\onslide<+-> \color{emblue} Underapproximation \color{black}:
				\begin{itemize}
					\onslide<+-> \item Summary forms a subset of actual loop behaviour
					\onslide<+-> \item Not every counterexample is covered!
				\end{itemize}
			\end{column}
			\begin{column}{0.5\textwidth}
				\vspace*{0.5cm}
				\resizebox{\textwidth}{!}{\begin{tikzpicture}
	\node[] at (0 ,3.25) {\color{emblue}{Actual Loop Behaviour}};
	\draw[emblue, fill=emblue!20, fill opacity=0.2] (0,0) circle (3);
	\node[] at (0,0) {\color{green}{Underapproximation}};
	\draw [rounded corners, green,thick,dashed] (-2,-2) -- (0, -2.75) -- (1.25, -2.5) -- (2.5, -0.5) -- (1.25, 1.5) -- (0.25, 2.8) -- (-2, 2) -- (-2.8, 0) -- cycle;
	\onslide<+-> \node at (3, 4) (head) []{Missed Counterexample};
	\node at (2, 1.5) (dot) [circle,fill,inner sep=1.5pt]{};
	\path (head) edge node []{} (dot);
\end{tikzpicture}}
			\end{column}
		\end{columns}
	\end{center}
\end{frame}

\begin{frame}[t]
	\frametitle{Loop Summarization}
	\begin{center}
		Represent relationships between inputs and outputs to model loop's effect on variables \\ \vspace*{0.25cm}
		There are two kinds of loop summarization: \\
		\onslide<+-> \color{emblue}Underapproximative \color{black} and \color{emblue}Overapproximative \color{black} Summarization
		\begin{columns}[t]
			\begin{column}{0.35\textwidth}
				\vspace*{0.5cm} \\
				\onslide<+-> \color{emblue} Overapproximation \color{black}:
				\begin{itemize}
					\onslide<+-> \item Summary forms a superset of actual loop behaviour
					\onslide<+-> \item Covers spurious counterexamples!
				\end{itemize}
			\end{column}
			\begin{column}{0.5\textwidth}
				\vspace*{0.5cm}
				\resizebox{\textwidth}{!}{ \begin{tikzpicture}
	\node[] at (2.7,2) {\color{emblue}{Actual Loop Behaviour}};
	\draw [rounded corners,red,thick,dashed] (0,0) -- (3, -1) -- (5.5,0.5) -- (6, 3.5) -- (4, 5)-- (0.25, 4) node [midway, sloped, above] {Overapproximation} -- cycle;
	\draw[emblue, fill=emblue!20, fill opacity=0.2] (2.7,2) circle (2.5);
	\node at (5, 5.75) (head) []{Spurious Counterexample};
	\node at (5.5, 3.25) (dot) [circle,fill,inner sep=1.5pt]{};
	\path (head) edge node []{} (dot);
\end{tikzpicture}}
			\end{column}
		\end{columns}
	\end{center}
\end{frame}

\begin{frame}[t]
	\frametitle{Thesis' Goal}
	\begin{center}
		Implement a new loop summarization library based on rational vector addition systems with resets (\qvasr) in Ultimate \\
		\onslide<+-> $\rightarrow$ Overapproximation with a \color{red} guaranteed \color{black} degree of precision! \\\vspace*{1cm}
		\onslide<+-> Adapt the \qvasr\ library to be used in the Trace Abstraction scheme
	\end{center}
\end{frame}