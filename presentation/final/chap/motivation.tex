\begin{frame}[t]
	\frametitle{Motivation}
	\begin{itemize}
		\item Loop analysis is one of the most challenging parts of software verification \\
		$\rightarrow$ "Archilles Heel of program verification" \cite{DBLP:journals/fmsd/KroeningSTTW13} \pause
		\vspace*{1cm}
		\begin{columns}[t]
			\begin{column}{0.4\textwidth}
				\begin{figure}[h]
					\vspace*{0.5cm}
					\resizebox{0.6\textwidth}{!}{\begin{lstlisting}[language=C++, style=withAssert]  % Start your code-block
	
	int x := 0;
	int y := 2;
	int z := 3;
	while x <= 20:
		if x <= 10:
			z := x;
			x := x + y;
			y := y + 1;
		else:
			x := x + 2;
			y := y - 3;
	assert x == 21;
	\end{lstlisting}}
				\end{figure}
			\end{column}
			\begin{column}{0.6\textwidth}
				\pause
				\only<3> {
					Loop Unwinding:
					\begin{itemize}
						\item Quite simple
						\item {\color{red}{But:}} \\
						Cannot reason about loop behavior beyond unwinding bound
				\end{itemize} }
				\only<4> {
					Loop Invariants:
					\begin{itemize}
						\item Loop invariants hold before and after the loop
						\item {\color{red}{But:}} \\
						Usefulness is tied to how strong they are \\
						$\rightarrow$ Finding strong loop invariants "is an art" \cite{DBLP:journals/fmsd/KroeningSTTW13}
				\end{itemize} }				
				\only<5> {
					Loop Summarization:
					\begin{itemize}
						\item Models relation between input and output as a set of constrains
						\item {\color{red}{But:}} \\
						Not guaranteed to be precise \\
						$\rightarrow$ Over- or underapproximation of loop behavior
				\end{itemize} }	
				\only<6> { }	
			\end{column}
		\end{columns}
		\only<6>{
			Overapproximative loop summary using rational vector addition systems (\qvasr) \cite{DBLP:conf/cav/SilvermanK19} \\
			$\rightarrow$ Overapproximation with a \textsl{guaranteed} degree of precision
		}
	\end{itemize}
\end{frame}