\begin{frame}[t]
	\frametitle{\qvasr-Abstraction}
	\newcommand{\s}{\ensuremath{\begin{bmatrix} s_x & s_y & s_z \end{bmatrix}}}
	\newcommand{\p}{\ensuremath{\begin{bmatrix} x' \\ y' \\ z' \end{bmatrix}}}
	\newcommand{\up}{\ensuremath{\begin{bmatrix} x \\ y \\ z \end{bmatrix}}}
	\begin{definition}
		A \qvasr-abstraction is a tuple $(S, V)$ consisting of a matrix $S$ and a \qvasr\ $V$ with $S$ being called a linear transformation.
	\end{definition}
	\begin{center}
		\onslide<+->
		Given \qvasr-abstraction\ $(S, V)$ with pair $(\vec{r}, \vec{a}) \in V$ of dimension $n$, $(S, V)$ defines a transition system $(Q_V, \rightarrow_V)$ with state space $Q_V =  \mathbb{Q}^n$  and transitions $\vec{x} \rightarrow_V \vec{x}'$ defined as follows: \\
		\begin{equation}
		S \cdot
		\begin{bmatrix}
			x_1' \\
			\vdots \\
			x_n'
		\end{bmatrix}
		=
		S \cdot
		\begin{bmatrix}
			x_1 \\
			\vdots \\
			x_n
		\end{bmatrix}
		*
			\begin{bmatrix}
				r_1 \\
				\vdots \\
				r_n
			\end{bmatrix}
		+
		\begin{bmatrix}
			a_1 \\
			\vdots \\
			a_n
		\end{bmatrix}
		\end{equation}	
	\end{center}
\end{frame}


\begin{frame}[t]
	\frametitle{\qvasr-Abstraction Example}
	\begin{center}
		$H: x \leq 10 \land x' = x + y \land y' = y + 1 \land z' = x$
	\end{center}
		\begin{figure}
			\begin{equation}
				S_H = \begin{bmatrix} \color<2->{red} -1 & \color<2->{red} 1 & \color<2->{red} 1 \\ 0 &1 &0 \end{bmatrix} \hspace*{0.5cm}
				V_H = 
				\begin{Bmatrix}
					\begin{pmatrix}
						\begin{bmatrix}
							0 \\
							1
						\end{bmatrix},
						\begin{bmatrix}
							1 \\
							1
						\end{bmatrix}
					\end{pmatrix}
				\end{Bmatrix}
			\end{equation}
			\caption*{\qvasr-abstraction of $H$}
		\end{figure}
		\begin{figure}
			\begin{equation}
				V_H = \begin{Bmatrix} \begin{bmatrix} \color<2->{red} -x' + y' + z' = 1 \\ y' = y + 1 \end{bmatrix} \end{Bmatrix}
			\end{equation}
			\caption*{Intuitive view of \qvasr-abstraction} 
		\end{figure}
\end{frame}