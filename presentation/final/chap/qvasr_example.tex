\begin{frame}[t]
	\frametitle{\qvasr\ Example}
	\begin{columns}
		\begin{column}{0.4\textwidth}
			\begin{figure}[h]
				\vspace*{0.5cm}
				\resizebox{0.6\textwidth}{!}{\begin{lstlisting}[language=C++, style=withAssert]  % Start your code-block
	
	int x := 0;
	int y := 2;
	int z := 3;
	while x <= 20:
		if x <= 10:
			z := x;
			x := x + y;
			y := y + 1;
		else:
			x := x + 2;
			y := y - 3;
	assert x == 21;
	\end{lstlisting}}
				\vspace{-0.5cm}
				\caption*{Program containing a loop.}
			\end{figure}
		\end{column} \pause
		\begin{column}{0.6\textwidth}
			\begin{figure}[h]
				\begin{equation*}
					G: x \leq 20 \land x > 10 \land x' = {\color<4>{red}{x}} + {\color<5>{red}{2}} \land y' = {\color<4>{red}{y}} {\color<5>{red}{-3}}
				\end{equation*}
				\caption*{Transition formula $G$ of the else branch.}
			\end{figure}
		\end{column}
	\end{columns}
	\begin{columns}
		\begin{column}{0.4\textwidth} \uncover<6>{
				\begin{figure}
					$V_G = 
					\begin{Bmatrix}
						\begin{bmatrix}
							x \\
							y
						\end{bmatrix} \rightarrow_{V_G}
						\begin{bmatrix}
							x + 2 \\
							y - 3
						\end{bmatrix}
					\end{Bmatrix}
					$
					\caption*{Intuitive notation of $V_G$.}
				\end{figure}
			}
		\end{column}
		\begin{column}{0.4\textwidth} \uncover<3->{
				\begin{figure}
					$V_G = 
					\begin{Bmatrix}
						\begin{pmatrix}
							\begin{bmatrix}
								{\color<4>{red}{1}} \\
								{\color<4>{red}{1}}
							\end{bmatrix},
							\begin{bmatrix}
								{\color<5>{red}{2}} \\
								{\color<5>{red}{-3}}
							\end{bmatrix}
						\end{pmatrix}
					\end{Bmatrix}
					$
					\caption*{\qvasr\ $V_G$.}
				\end{figure}
			}
		\end{column}
	\end{columns}
\end{frame}

%%%%%%%%%%%%%%%%%%%%%%%%%%%%%%%%%%%%%%%%%%%%%%%%%%%%%%%%%%%%%%%%%%%%%%%%%%%%%% new frame
\begin{frame}[t]
	\frametitle{\qvasr\ Example}
	\begin{columns}
		\begin{column}{0.4\textwidth}
			\begin{figure}[h]
				\vspace*{0.5cm}
				\resizebox{0.6\textwidth}{!}{\begin{lstlisting}[language=C++, style=withAssert]  % Start your code-block
	
	int x := 0;
	int y := 2;
	int z := 3;
	while x <= 20:
		if x <= 10:
			z := x;
			x := x + y;
			y := y + 1;
		else:
			x := x + 2;
			y := y - 3;
	assert x == 21;
	\end{lstlisting}}
				\vspace{-0.5cm}
				\caption*{Program containing a loop.}
			\end{figure}
		\end{column} \pause
		\begin{column}{0.6\textwidth}
			\begin{figure}[h]
				\begin{equation*}
					H: x \leq 10 \land x' = x + {\color<3->{red}{y}} \land y' = y + 1 \land z' = {\color<3->{red}{x}}
				\end{equation*}
				\caption*{Transition formula $H$ of the if branch.}
			\end{figure}
		\end{column}
	\end{columns}
	\pause
	\vspace*{1cm}
	\only<4->{
		\begin{center}
			\item \alert{\qvasr\ can only model constant additions!} \pause \pause
			\item We can overapproximate $H$ by computing a \qvasr-abstraction
		\end{center}		
	}
\end{frame}