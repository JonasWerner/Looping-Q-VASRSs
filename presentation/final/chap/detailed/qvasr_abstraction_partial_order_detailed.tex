\begin{frame}[t]
	\frametitle{\qvasr-Abstraction Partial Order}
	\begin{columns}
		\begin{column}{0.4\textwidth}
			\begin{figure}[h]
				\vspace*{0.5cm}
				\resizebox{0.6\textwidth}{!}{\begin{lstlisting}[language=C++, style=withAssert]  % Start your code-block
	
	int x := 0;
	int y := 2;
	int z := 3;
	while x <= 20:
		if x <= 10:
			z := x;
			x := x + y;
			y := y + 1;
		else:
			x := x + 2;
			y := y - 3;
	assert x == 21;
	\end{lstlisting}}
				\vspace{-0.5cm}
				\caption*{Program containing a loop.}
			\end{figure}
		\end{column} \pause
		\begin{column}{0.6\textwidth}
			\begin{itemize}
				\item The transition formula of the loop: 
				\begin{equation*}
					F = H \lor G
				\end{equation*}
				\pause
				\item We need a \qvasr-abstraction $(\tilde{S}, \tilde{V})$ that simulates both $(S_H, V_H)$ and $(S_G, V_G)$ 
				\pause
				\begin{equation*}
					\rightarrow T_HS_H = \tilde{S} = T_GS_G 
				\end{equation*}
				\pause
				\begin{equation*}
					\text{Solve:}\ \tilde{S} = \begin{bmatrix} t^{H}_1 & t^{H}_2  \end{bmatrix} S_H  = \begin{bmatrix} t^{G}_1 & t^{G}_2\end{bmatrix} S_G
				\end{equation*}
			\end{itemize}
		\end{column}
	\end{columns}
	\pause
	\vspace{0.5cm}
	\begin{itemize}
		\item Compute new changes on relations as $\tilde{V}$
		\begin{equation*}
			\tilde{V} = \{(T_H \times \vec{r}_H, T\vec{a}_H) \cup (T_G\times \vec{r}_G, T_G\vec{a}_G)\}
		\end{equation*}
		Where $(T \times \vec{r})_i = r_j$ is a translation of $r_i$ to a non-zero $r_j \in T$
	\end{itemize}
\end{frame}
