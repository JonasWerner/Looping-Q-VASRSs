\begin{frame}[t]
	\frametitle{\qvasrs}
	To improve precision, we constrain \qvasr\ transitions by using a set of predicates $P$ as states. We introduce a so called \qvasrs
	\begin{definition}
		Given a \qvasr-abstraction $(S,V)$, a \qvasrs\ $\mathcal{V}$ is a nondeterministic finite automaton $(Q, E)$ consisting of a set of states $Q$ and a set of edges $E \subseteq Q \times V \times Q$, \\
		where each transition is labeled by a transition in $V$ \\ \vspace*{0.25cm}
		$\mathcal{V}$ can transition $(q_1, \vec{x}) \rightarrow_\mathcal{V} (q_2, \vec{x}')$ if there is some edge $(q_1, (\vec{r}, \vec{a}), q_2)$
	\end{definition}
	\begin{center}
		We use $P_F = \{ x \leq 10, 10 < x \land x \leq 20\}:$
	\end{center} \vspace{0.5cm}
	
    \centering
    \begin{tikzpicture}[%
    ->,
    >=stealth',
    shorten >=1pt,
    auto,
    node distance=9cm,
    scale=0.9,
    transform shape,
    align=center,
    smallnode/.style={inner sep=1.2},
    initial text =,
    anchor=center]

    	\node[draw, ellipse, initial left, initial text =](1){$x \leq 10$};
    	\node[draw, ellipse](2) [right of=1] {$10 < x \land x \leq 20$};
    	% \node[draw, ellipse, accepting] (3) [right of=2] {$x > 20$};

    	\path (1) edge node [below]{ $
					\begin{bmatrix}
						-x' + y + z = - x + y + z - 5 \\
						y' = y - 3
					\end{bmatrix}
                     $ } (2)
	
	    	 (1) edge[loop above] node { $
					\begin{bmatrix}
						-x' + y' + z' = - x + y + z - 5 \\
					 	y' = y - 3
					\end{bmatrix}
	         $ } (1)
	
	    	 (2) edge[loop above] node { $
	               \begin{bmatrix}
						-x' + y' + z' = 1 \\
						y' = y + 1
					\end{bmatrix}
	                     $ } (2)
	    	;
    \end{tikzpicture}
    %\caption{\qvasrs abstraction $\mathcal{A}$ of program $P$.}
\end{frame}

\begin{frame}
	
    \centering
    \begin{tikzpicture}[%
    ->,
    >=stealth',
    shorten >=1pt,
    auto,
    node distance=9cm,
    scale=0.9,
    transform shape,
    align=center,
    smallnode/.style={inner sep=1.2},
    initial text =,
    anchor=center]

    	\node[draw, ellipse, initial left, initial text =](1){$x \leq 10$};
    	\node[draw, ellipse](2) [right of=1] {$10 < x \land x \leq 20$};
    	% \node[draw, ellipse, accepting] (3) [right of=2] {$x > 20$};

    	\path (1) edge node [below]{ $
					\begin{bmatrix}
						-x' + y + z = - x + y + z - 5 \\
						y' = y - 3
					\end{bmatrix}
                     $ } (2)
	
	    	 (1) edge[loop above] node { $
					\begin{bmatrix}
						-x' + y' + z' = - x + y + z - 5 \\
					 	y' = y - 3
					\end{bmatrix}
	         $ } (1)
	
	    	 (2) edge[loop above] node { $
	               \begin{bmatrix}
						-x' + y' + z' = 1 \\
						y' = y + 1
					\end{bmatrix}
	                     $ } (2)
	    	;
    \end{tikzpicture}
    %\caption{\qvasrs abstraction $\mathcal{A}$ of program $P$.} \\
	Using this \qvasrs, we can now compute a reachability relation $reach(S_F, P_F)$ in polytime \cite{DBLP:conf/rp/HaaseH14}
\end{frame}