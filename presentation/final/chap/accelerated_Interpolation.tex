\begin{frame}[t]
	\frametitle{Accelerated Interpolation using \qvasr}
	\resizebox{\textwidth}{!}{
\begin{tikzpicture}[%
	->,
	>=stealth',
	shorten >=1pt,
	auto,
	node distance=2.5cm,
	scale=0.9,
	transform shape,
	align=center,
	smallnode/.style={inner sep=1.4},
	initial text =,
	anchor=center]
	
	\tikzstyle{my node}=[draw,minimum height=1cm,minimum width=3cm]
	\begin{scope}
	\node [align=center](tainput) {\textbf{Input}: \\ A program $P$};
	\node[my node] (cfg) [below of=tainput, yshift=1cm] {Control-Flow Graph \\ Builder};
	\node[my node](ta1)[below of=cfg]{$\mathcal{L}(A_P) \subseteq \mathcal{L}(A_D)?$};
	\node[my node] (ta2) [below of=ta1, yshift=-1cm] {$\tau$ feasible?};
	\node[my node] (ta3) [below left of=ta1, xshift=-1cm]{\texttt{generalize($\tau$)}};
	\node [](corr)[right of=ta1, right=0.5cm] {\textbf{return}: $P$ is Safe};
	\node [](incorr)[right of=ta2, right=0.5cm] {\textbf{return}: $P$ is Unsafe};
	
	\path (ta1) edge[bend left] node[align=left, right=0.25cm]{no \\
		$\tau \in \mathcal{L}(A_P) \backslash \mathcal{L}(A_D)$} (ta2)
	(ta3.north) edge[bend left] node[align=right, left=0.25cm] {$A_D := A_D\ \cup$ $\texttt{generalize} (\tau)$} (ta1.west)
	(ta2.west) edge[bend left] node[left=0.25cm] {no} (ta3.south)
	(cfg) edge[] node[align=left, right=0.25cm] {$A_P := \text{control-flow graph as automaton}$ \\ 		$A_D = \emptyset$} (ta1)
	(ta1) edge[] node[above] {yes} (corr)
	(ta2) edge[] node[above] {yes} (incorr)
	(tainput) edge[] node[] {} (cfg)
	;
	\end{scope}


	\begin{scope}[xshift=10cm, node distance=2cm]
	\tikzstyle{in node}=[draw,minimum height=1cm,minimum width=5cm]
	\node [align=center](input) {\textbf{Input}: \\ Program trace $\tau$};
	\node[in node, fill=white](1)[below of=input, yshift=0.25cm]{Does $\tau$ contain a loop?};
	
	\node[my node,  fill=white](2)[below of=1, xshift=1cm]{Loop \\ Summarization};
	\node[my node, fill=white](3)[below of=2]{Meta-Trace \\ Construction};
	
	\node[my node, fill=white](4)[below of=3]{Meta-Interpolant \\ Computation};
	\node[my node, fill=white](interpolants)[below of=4, xshift=-1cm]{Inductive Interpolant  Computation};
	
	\node[](6)[below of=interpolants, yshift=0.25cm]{\textbf{Output}: \\ Inductive Sequence of Interpolants for $\tau$};
	
	\begin{scope}[on background layer]
		\node[group, line width=0.3mm, draw=emblue, fill=emblue!20, fill opacity=0.2, fit=(1) (interpolants) (3), yscale=1.025, xscale=1.025](FIt1) {};
	\end{scope}
	
	\path (input) edge[] node[]{} (1)
	([xshift=-1cm] 1.south) edge[] node[left=0.1cm]{\textbf{no}} ([xshift=-1cm]interpolants.north)
	;
	\draw [->] ([xshift=1cm]1.south) -- node[right] {\textbf{yes}} ([xshift=1cm]1.south |-2.north); 
	\draw [->] (2.south) -- (2.south |-3.north);
	\draw [->] (3.south) -- (3.south |-4.north); 
	\draw [->] (4.south) -- (4.south |-interpolants.north); 
	\draw [->] (interpolants.south) -- (6.north); 
	\end{scope}

	%\draw [Circle - Circle, color=emblue, line width=0.3mm] (ta3) -- (ta3 -| FIt1.west); 
	\draw [-, color=emblue, line width=0.3mm] (ta3.north east) -- ([xshift=0.15cm]FIt1.north west); 
	\draw [-, color=emblue, line width=0.3mm] (ta3.south east) -- ([xshift=0.15cm]FIt1.south west);
	%\draw [-, color=emblue, line width=0.3mm] (ta3.south east) -- (FIt1.south east); 
	%\draw [-, color=emblue, line width=0.3mm] (ta3.south west) -- (FIt1.south west);
\end{tikzpicture}}
\end{frame}

\begin{frame}[t]	\frametitle{Accelerated Interpolation using \qvasr}
	\resizebox{0.5\textwidth}{!}{
\begin{tikzpicture}[%
	->,
	>=stealth',
	shorten >=1pt,
	auto,
	node distance=2cm,
	scale=0.75,
	transform shape,
	align=center,
	smallnode/.style={inner sep=1.4},
	initial text =,
	anchor=center]
	
	\tikzstyle{my node}=[draw,minimum height=1cm,minimum width=2.5cm]
	\tikzstyle{in node}=[draw,minimum height=1cm,minimum width=5cm]
	\tikzstyle{dashed node}=[draw,minimum height=1cm,minimum width=3cm, dashed]
	
	\node [align=center](input) {\textbf{Input}: \\ Program trace $\tau$};
	\node[in node, fill=white](1)[below of=input, yshift=0.25cm]{Does $\tau$ contain a loop?};
	
	\node[my node,  fill=white](2)[below of=1, xshift=1cm]{Loop \\ summarizable?};
	\node[my node, fill=white](3)[below of=2]{Meta-Trace \\ Construction};
	
	\node[my node, fill=white](4)[below of=3]{Meta-Proof \\ Computation};
	
	\node[my node, fill=white](interpolants)[below of=4, xshift=-2.25cm]{Proof \\ Computation};
	\node[my node, fill=white](post)[right of=interpolants, xshift=0.25cm]{Post \\ Processing};
	
	\node[](6)[below of=interpolants, yshift=-0.2cm, xshift=1cm]{\textbf{Output}: \\ Inductive Sequence of \\ State Assertions for $\tau$};
	
	\node[dashed node](7)[right of=2, xshift=2.5cm]{Divide Loop};
	
	\begin{scope}[on background layer]
		\node[group, line width=0.3mm, draw=emblue, fill=emblue!20, fill opacity=0.2, fit=(1) (interpolants) (3), yscale=1.05, xscale=1.025](FIt1) {};
	\end{scope}
	
	\path (input) edge[] node[]{} (1)
	([xshift=-1.25cm] 1.south) edge[] node[left=0.1cm]{\textbf{no}} ([]interpolants.north)
	;
	\draw [->] ([xshift=1cm]1.south) -- node[right] {\textbf{yes}} ([xshift=1cm]1.south |-2.north); 
	\draw [->] (2.south) --  node[right] {\textbf{yes}} (2.south |-3.north);
	\draw [->] (3.south) -- (3.south |-4.north); 
	\draw [->] (4.south) -- (4.south |-interpolants.north); 
	\draw [->] ([xshift=1cm]interpolants.south) -- (6.north); 
	
	\draw[->, dashed] ([yshift=0.2cm]2.east) -- node[above] {\textbf{no}} ([yshift=0.2cm]7.west);
	\draw[->, dashed] ([yshift=-0.2cm]7.west) -- ([yshift=-0.2cm]2.east);
	;
\end{tikzpicture}}
\end{frame}

\begin{frame}[t]
	\frametitle{Accelerated Interpolation using \qvasr}
	\begin{columns}
		\begin{column}{0.3\textwidth}
			\resizebox{0.8\textwidth}{!}{
\begin{tikzpicture}[%
	->,
	>=stealth',
	shorten >=1pt,
	auto,
	node distance=2cm,
	scale=0.75,
	transform shape,
	align=center,
	smallnode/.style={inner sep=1.4},
	initial text =,
	anchor=center]
	
	\tikzstyle{my node}=[draw,minimum height=1cm,minimum width=2.5cm]
	\tikzstyle{in node}=[draw,minimum height=1cm,minimum width=5cm]
	\tikzstyle{dashed node}=[draw,minimum height=1cm,minimum width=3cm, dashed]
	
	\node [align=center](input) {\color<2>{emblue} \textbf{Input}: \\ \color<2>{emblue} Program trace $\tau$};
	\node[in node, fill=white](1)[below of=input, yshift=0.25cm]{\color<3>{emblue}Does $\tau$ contain a loop?};
	
	\node[my node,  fill=white](2)[below of=1, xshift=1cm]{\color<5>{emblue} Loop \\ \color<5>{emblue}summarizable?};
	\node[my node, fill=white](3)[below of=2]{Meta-Trace \\ Construction};
	
	\node[my node, fill=white](4)[below of=3]{Meta-Proof \\ Computation};
	
	\node[my node, fill=white](interpolants)[below of=4, xshift=-2.25cm]{Proof \\ Computation};
	\node[my node, fill=white](post)[right of=interpolants, xshift=0.25cm]{Post \\ Processing};
	
	\node[](6)[below of=interpolants, yshift=-0.2cm, xshift=1cm]{\textbf{Output}: \\ Inductive Sequence of \\ State Assertions for $\tau$};
	
	\begin{scope}[on background layer]
		\node[group, line width=0.3mm, draw=emblue, fill=emblue!20, fill opacity=0.2, fit=(1) (interpolants) (3), yscale=1.05, xscale=1.025](FIt1) {};
	\end{scope}
	
	\path (input) edge[] node[]{} (1)
	([xshift=-1.25cm] 1.south) edge[] node[left=0.1cm]{\textbf{no}} ([]interpolants.north)
	;
	\draw [->] ([xshift=1cm]1.south) -- node[right] {\color<4>{emblue} \textbf{yes}} ([xshift=1cm]1.south |-2.north); 
	\draw [->] (2.south) --  node[right] {\textbf{yes}} (2.south |-3.north);
	\draw [->] (3.south) -- (3.south |-4.north); 
	\draw [->] (4.south) -- (4.south |-interpolants.north); 
	\draw [->] ([xshift=1cm]interpolants.south) -- (6.north); 
	;
\end{tikzpicture}}
		\end{column}
		\begin{column}{0.55\textwidth}
			\onslide<2-5>
			\resizebox{0.4\textwidth}{!}{
\begin{tikzpicture}[%
	->,
	>=stealth',
	shorten >=1pt,
	auto,
	node distance=0.75cm,
	scale=0.6,
	transform shape,
	align=center,
	smallnode/.style={inner sep=1.4},
	initial text =,
	anchor=center]
	\node [](t) {\color<2-3>{emblue} $\tau$:};
	\node [](1)[below of=t] {\st{x:=0}};
	\node [](13)[below of=1] {\st{y:=2}};
	\node [](14)[below of=13] {\st{z:= 3}};
	\node [](2)[below of=14] {\color<4-5>{red} \st{x<=20; x<=10}};
	\node [](4)[below of=2]{\color<4-5>{red}\st{z:=x; x:=x+y; y:=y+1}};
	\node [](5)[below of=4]{\color<4-5>{red}\st{x<=20; x<=10}};
	\node [](6)[below of=5] {\color<4-5>{red}\st{z:=x; x:=x+y; y:=y+1}};
	\node [](7)[below of=6] {\color<4-5>{red}\st{x<=20; x>10}};
	\node [](8)[below of=7] {\color<4-5>{red}\st{x:=x+1; y:=y-3}};
	\node [](9)[below of=8] {\color<4-5>{red}\st{x<=20; x>10}};
	\node [](10)[below of=9] {\color<4-5>{red}\st{x:=x+1; y:=y-3}};
	\node [](11)[below of=10] {\st{x>20}};
	\node [](12)[below of=11] {\st{x!=21}};
	
	\onslide<4-5>\draw [-] ([xshift=-1cm]2.north) -- ([xshift=1cm]2.north); 
	\onslide<4-5>\draw [-] ([xshift=-1cm]7.north) -- ([xshift=1cm]7.north); 
	\onslide<4-5>\draw [-] ([xshift=-1cm]11.north) -- ([xshift=1cm]11.north); 
	;
\end{tikzpicture}}
		\end{column}
	\end{columns}
\end{frame}

\begin{frame}[t]
	\frametitle{Accelerated Interpolation using \qvasr}
	\begin{columns}
		\begin{column}{0.3\textwidth}
			\resizebox{0.8\textwidth}{!}{
\begin{tikzpicture}[%
	->,
	>=stealth',
	shorten >=1pt,
	auto,
	node distance=2cm,
	scale=0.75,
	transform shape,
	align=center,
	smallnode/.style={inner sep=1.4},
	initial text =,
	anchor=center]
	
	\tikzstyle{my node}=[draw,minimum height=1cm,minimum width=3cm]
	\tikzstyle{in node}=[draw,minimum height=1cm,minimum width=5cm]
	\tikzstyle{dashed node}=[draw,minimum height=1cm,minimum width=3cm, dashed]
	
	\node [align=center](input) {\textbf{Input}: \\ \ Program trace $\tau$};
	\node[in node, fill=white](1)[below of=input, yshift=0.25cm]{Does $\tau$ contain a loop?};
	
	\node[my node, fill=white](2)[below of=1, xshift=1cm]{\color<1->{emblue}Loop summarizable?};
	\node[my node, fill=white](3)[below of=2]{Meta-Trace \\ Construction};
	
	\node[my node, fill=white](4)[below of=3]{Meta-Interpolant \\ Computation};
	\node[my node, fill=white](interpolants)[below of=4, xshift=-1cm]{Inductive Interpolant  Computation};
	
	\node[](6)[below of=interpolants, yshift=0.25cm]{\textbf{Output}: \\ Inductive Sequence of Interpolants for $\tau$};
	
	\begin{scope}[on background layer]
		\node[group, line width=0.3mm, draw=emblue, fill=emblue!20, fill opacity=0.2, fit=(1) (interpolants) (3), yscale=1.05, xscale=1.025](FIt1) {};
	\end{scope}
	
	\path (input) edge[] node[]{} (1)
	([xshift=-1cm] 1.south) edge[] node[left=0.1cm]{\textbf{no}} ([xshift=-1cm]interpolants.north)
	;
	\draw [->] ([xshift=1cm]1.south) -- node[right] {\textbf{yes}} ([xshift=1cm]1.south |-2.north); 
	\draw [->] (2.south) --  node[right] {\color<4->{emblue}\textbf{yes}} (2.south |-3.north);
	\draw [->] (3.south) -- (3.south |-4.north); 
	\draw [->] (4.south) -- (4.south |-interpolants.north); 
	\draw [->] (interpolants.south) -- (6.north); 
	;
\end{tikzpicture}}
		\end{column}
		\begin{column}{0.55\textwidth}
			Single loop iterations: \vspace*{0.25cm}\\
			\st{x<=20; x<=10} \st{z:=x; x:=x+y; y:=y+1}
			\onslide<2->
			{\small 
			\begin{align*}
			&\rightarrow x \leq 10\ \land\ x' = x + y\ \land\ y' = y + 1\ \land\ z' = x \\
			&\rightarrow \text{Transition formula}\ H 
			\end{align*}
			}%
			\onslide<1-> \\
			\vspace*{1cm}\st{x<=20; x>10} \st{x:=x+1; y:=y-3}
			\onslide<2->
			\begin{align*}
			&\rightarrow	x \leq 21\ \land\ x > 10\ \land\ x' = x + 2\ \land\ y' = y -3 \\
			&\rightarrow \text{Transition formula}\ G
			\end{align*}
			\onslide<3-> \\
			\vspace*{0.5cm}
			$\rightarrow F = G \lor H$ \onslide<4-> with loop summary:
			\begin{align*}
				\psi: \	\exists k_1, k_2.\ &((-x' + y' + z' = 1\ \\ & \lor\ -x' + y' + z' = -x + y + z - 5k_2)\ \\ &\land\ y' = y + k_1 - 3k_2)\ \\ &\lor\ x' = x\ \land\ y' = y\ \land\ z' = z
			\end{align*}
		\end{column}
	\end{columns}
\end{frame}

\begin{frame}[t]
	\frametitle{Accelerated Interpolation using \qvasr}
	\begin{columns}
		\begin{column}{0.35\textwidth}
			\resizebox{0.8\textwidth}{!}{
\begin{tikzpicture}[%
	->,
	>=stealth',
	shorten >=1pt,
	auto,
	node distance=2cm,
	scale=0.75,
	transform shape,
	align=center,
	smallnode/.style={inner sep=1.4},
	initial text =,
	anchor=center]
	
	\tikzstyle{my node}=[draw,minimum height=1cm,minimum width=3cm]
	\tikzstyle{in node}=[draw,minimum height=1cm,minimum width=5cm]
	\tikzstyle{dashed node}=[draw,minimum height=1cm,minimum width=3cm, dashed]
	
	\node [align=center](input) {\textbf{Input}: \\ \ Program trace $\tau$};
	\node[in node, fill=white](1)[below of=input, yshift=0.25cm]{Does $\tau$ contain a loop?};
	
	\node[my node, fill=white](2)[below of=1, xshift=1cm]{Loop summarizable?};
	\node[my node, fill=white](3)[below of=2]{\color<1->{emblue} Meta-Trace \\ \color<1->{emblue} Construction};
	
	\node[my node, fill=white](4)[below of=3]{Meta-Interpolant \\ Computation};
	\node[my node, fill=white](interpolants)[below of=4, xshift=-1cm]{Inductive Interpolant  Computation};
	
	\node[](6)[below of=interpolants, yshift=0.25cm]{\textbf{Output}: \\ Inductive Sequence of Interpolants for $\tau$};
	
	\begin{scope}[on background layer]
		\node[group, line width=0.3mm, draw=emblue, fill=emblue!20, fill opacity=0.2, fit=(1) (interpolants) (3), yscale=1.05, xscale=1.025](FIt1) {};
	\end{scope}
	
	\path (input) edge[] node[]{} (1)
	([xshift=-1cm] 1.south) edge[] node[left=0.1cm]{\textbf{no}} ([xshift=-1cm]interpolants.north)
	;
	\draw [->] ([xshift=1cm]1.south) -- node[right] {\textbf{yes}} ([xshift=1cm]1.south |-2.north); 
	\draw [->] (2.south) --  node[right] {\textbf{yes}} (2.south |-3.north);
	\draw [->] (3.south) -- (3.south |-4.north); 
	\draw [->] (4.south) -- (4.south |-interpolants.north); 
	\draw [->] (interpolants.south) -- (6.north); 
	;
\end{tikzpicture}}
		\end{column}
		\begin{column}{0.55\textwidth}
			\resizebox{0.77\textwidth}{!}{
\begin{tikzpicture}[%
	->,
	>=stealth',
	shorten >=1pt,
	auto,
	node distance=0.75cm,
	scale=0.6,
	transform shape,
	align=center,
	smallnode/.style={inner sep=1.4},
	initial text =,
	anchor=center]
	\node [](t) {$\tau$:};
	\node [](1)[below of=t] {\st{x:=0}};
	\node [](13)[below of=1] {\st{y:=2}};
	\node [](14)[below of=13] {\st{z:=3}};
	\node [](2)[below of=14] {\color<1->{red}\st{x<=20; x<=10}};
	\node [](4)[below of=2]{\color<1->{red}\st{z:=x; x:=x+y; y:=y+1}};
	\node [](5)[below of=4]{\color<1->{red}\st{x<=20; x<=10}};
	\node [](6)[below of=5] {\color<1->{red}\st{z:=x; x:=x+y; y:=y+1}};
	\node [](7)[below of=6] {\color<1->{red}\st{x<=20; x>10}};
	\node [](8)[below of=7] {\color<1->{red}\st{x:=x+1; y:=y-3}};
	\node [](9)[below of=8] {\color<1->{red}\st{x<=20; x>10}};
	\node [](10)[below of=9] {\color<1->{red}\st{x:=x+1; y:=y-3}};
	\node [](11)[below of=10] {\st{x>20}};
	\node [](12)[below of=11] {\st{x!=21}};
	
	% Simple brace
	\draw [-, decorate, 
	decoration = {brace,
		raise=5pt,
		amplitude=5pt}] ([xshift=1.25cm] 2.north) --  node[right, xshift=1cm] {Replace loop by $\psi$} ([xshift=1.25cm] 10.south);
	;
\end{tikzpicture}}
		\end{column}
	\end{columns}
\end{frame}

\begin{frame}[t]
	\frametitle{Accelerated Interpolation using \qvasr}
	\begin{columns}
		\begin{column}{0.35\textwidth}
			\resizebox{0.8\textwidth}{!}{
\begin{tikzpicture}[%
	->,
	>=stealth',
	shorten >=1pt,
	auto,
	node distance=2cm,
	scale=0.75,
	transform shape,
	align=center,
	smallnode/.style={inner sep=1.4},
	initial text =,
	anchor=center]
	
	\tikzstyle{my node}=[draw,minimum height=1cm,minimum width=3cm]
	\tikzstyle{in node}=[draw,minimum height=1cm,minimum width=5cm]
	\tikzstyle{dashed node}=[draw,minimum height=1cm,minimum width=3cm, dashed]
	
	\node [align=center](input) {\textbf{Input}: \\ \ Program trace $\tau$};
	\node[in node, fill=white](1)[below of=input, yshift=0.25cm]{Does $\tau$ contain a loop?};
	
	\node[my node, fill=white](2)[below of=1, xshift=1cm]{Loop summarizable?};
	\node[my node, fill=white](3)[below of=2]{\color<1->{emblue} Meta-Trace \\ \color<1->{emblue} Construction};
	
	\node[my node, fill=white](4)[below of=3]{Meta-Interpolant \\ Computation};
	\node[my node, fill=white](interpolants)[below of=4, xshift=-1cm]{Inductive Interpolant  Computation};
	
	\node[](6)[below of=interpolants, yshift=0.25cm]{\textbf{Output}: \\ Inductive Sequence of Interpolants for $\tau$};
	
	\begin{scope}[on background layer]
		\node[group, line width=0.3mm, draw=emblue, fill=emblue!20, fill opacity=0.2, fit=(1) (interpolants) (3), yscale=1.05, xscale=1.025](FIt1) {};
	\end{scope}
	
	\path (input) edge[] node[]{} (1)
	([xshift=-1cm] 1.south) edge[] node[left=0.1cm]{\textbf{no}} ([xshift=-1cm]interpolants.north)
	;
	\draw [->] ([xshift=1cm]1.south) -- node[right] {\textbf{yes}} ([xshift=1cm]1.south |-2.north); 
	\draw [->] (2.south) --  node[right] {\textbf{yes}} (2.south |-3.north);
	\draw [->] (3.south) -- (3.south |-4.north); 
	\draw [->] (4.south) -- (4.south |-interpolants.north); 
	\draw [->] (interpolants.south) -- (6.north); 
	;
\end{tikzpicture}}
		\end{column}
		\begin{column}{0.55\textwidth}
			\resizebox{0.4\textwidth}{!}{\begin{tikzpicture}[%
	->,
	>=stealth',
	shorten >=1pt,
	auto,
	node distance=0.75cm,
	scale=0.6,
	transform shape,
	align=center,
	smallnode/.style={inner sep=1.4},
	initial text =,
	anchor=center]
	\node [](t) {$\tau$:};
	\node [](1)[below of=t] {\st{x:=0; y:=2; z:=3}};
	\node [](2)[below of=1] {\color<1->{white}x<=20};
	\node [](3)[below of=2]{\color<1->{white} x<=10};
	\node [](4)[below of=3]{\color<1->{white} z:=x; x:=x+y; y:=y+1};
	\node [](5)[below of=4]{\color<1->{white}x<=10};
	\node [](6)[below of=5] {\color<1->{white}z:=x; x:=x+y; y:=y+1};
	\node [](7)[below of=6] {\color<1->{emblue}\st{$\psi$}};
	\node [](8)[below of=7] {\color<1->{white}x:=x+1; y:=y-3};
	\node [](9)[below of=8] {\color<1->{white}x>10};
	\node [](10)[below of=9] {\color<1->{white}x:=x+1; y:=y-3};
	\node [](11)[below of=10] {\color<1->{black}\st{x>20}};
	\node [](12)[below of=11] {\st{x!=22}};
	;
\end{tikzpicture}}
		\end{column}
	\end{columns}
\end{frame}

\begin{frame}[t]
	\frametitle{Accelerated Interpolation using \qvasr}
	\begin{columns}
		\begin{column}{0.3\textwidth}
			\resizebox{0.8\textwidth}{!}{
\begin{tikzpicture}[%
	->,
	>=stealth',
	shorten >=1pt,
	auto,
	node distance=2cm,
	scale=0.75,
	transform shape,
	align=center,
	smallnode/.style={inner sep=1.4},
	initial text =,
	anchor=center]
	
	\tikzstyle{my node}=[draw,minimum height=1cm,minimum width=2.5cm]
	\tikzstyle{in node}=[draw,minimum height=1cm,minimum width=5cm]
	\tikzstyle{dashed node}=[draw,minimum height=1cm,minimum width=3cm, dashed]
	
	\node [align=center](input) {\textbf{Input}: \\ Program trace $\tau$};
	\node[in node, fill=white](1)[below of=input, yshift=0.25cm]{Does $\tau$ contain a loop?};
	
	\node[my node,  fill=white](2)[below of=1, xshift=1cm]{Loop \\ summarizable?};
	\node[my node, fill=white](3)[below of=2]{Meta-Trace \\ Construction};
	
	\node[my node, fill=white](4)[below of=3]{\color<1-9>{emblue}Meta-Proof \\ \color<1-9>{emblue}Computation};
	
	\node[my node, fill=white](interpolants)[below of=4, xshift=-2.25cm]{Proof \\ Computation};
	\node[my node, fill=white](post)[right of=interpolants, xshift=0.25cm]{\color<10->{emblue}Post \\\color<10->{emblue} Processing};
	
	\node[](6)[below of=interpolants, yshift=-0.2cm, xshift=1cm]{\textbf{Output}: \\ Inductive Sequence of \\ State Assertions for $\tau$};
	
	\begin{scope}[on background layer]
		\node[group, line width=0.3mm, draw=emblue, fill=emblue!20, fill opacity=0.2, fit=(1) (interpolants) (3), yscale=1.05, xscale=1.025](FIt1) {};
	\end{scope}
	
	\path (input) edge[] node[]{} (1)
	([xshift=-1.25cm] 1.south) edge[] node[left=0.1cm]{\textbf{no}} ([]interpolants.north)
	;
	\draw [->] ([xshift=1cm]1.south) -- node[right] {\textbf{yes}} ([xshift=1cm]1.south |-2.north); 
	\draw [->] (2.south) --  node[right] {\textbf{yes}} (2.south |-3.north);
	\draw [->] (3.south) -- (3.south |-4.north); 
	\draw [->] (4.south) -- (4.south |-interpolants.north); 
	\draw [->] ([xshift=1cm]interpolants.south) -- (6.north); 
	;
\end{tikzpicture}}
		\end{column}
		\begin{column}{0.35\textwidth}
				\resizebox{\textwidth}{!}{\begin{tikzpicture}[%
	->,
	>=stealth',
	shorten >=1pt,
	auto,
	node distance=0.75cm,
	scale=0.6,
	transform shape,
	align=center,
	smallnode/.style={inner sep=1.4},
	initial text =,
	anchor=center]
	\node [](t) {$\bar{\tau}$:};
	\node [] (1k)[below of=t] {\st{x:=0;}};
	\node [] (13k)[below of=1k] {\st{y:=2; z:=3}};
	\node [] (7k)[below of=13k] {\color<1>{emblue}\st{$\psi$}};
	\node [] (11k)[below of=7k] {\color<1->{black}\st{x>21}};
	\node [] (12k)[below of=11k] {\st{x!=21}};
	\onslide<3->\node [label={[yshift=0cm, xshift=0.7cm, right]:{\itp{\top}}}] (1)[below of=t] {\st{x:=0;}};
	\onslide<4-> \node [label={[yshift=0cm, xshift=0.7cm, right]:{\itp{x = 0}}}] (13)[below of=1] {\st{y:=2; z:=3}};
	\onslide<5-> \node [label={[yshift=0cm, xshift=0.7cm, right]:{\itp{x = 0}}}](7)[below of=13] {\color<1>{emblue}\st{$\psi$}};
	\onslide<6->\node [label={[yshift=0cm, xshift=0.7cm, right]:{\itp{x \leq 22 \land y \leq 6}}}](11)[below of=7] {\color<1->{black}\st{x>21}};
	\onslide<7->\node [label={[yshift=0cm, xshift=0.7cm, right]:{\itp{x = 22}}}](12)[below of=11] {\st{x!=22}};
	\onslide<8-> \node [label={[yshift=0cm, xshift=0.7cm, right]:{\itp{\bot}}}](13)[below of=12] {};
	;
\end{tikzpicture}}
		\end{column}
		\begin{column}{0.3\textwidth}
				\onslide<9->{\resizebox{0.65\textwidth}{!}{
\begin{tikzpicture}[%
	->,
	>=stealth',
	shorten >=1pt,
	auto,
	node distance=0.75cm,
	scale=0.6,
	transform shape,
	align=center,
	smallnode/.style={inner sep=1.4},
	initial text =,
	anchor=center]
	\node [](t) {$\tau$:};
	\node [](1)[below of=t] {\st{x:=0}};
	\node [](13)[below of=1] {\st{y:=2}};
	\node [](14)[below of=13] {\st{z:=3}};
	\node [](2)[below of=14] {\st{x<=21}};
	\node [](3)[below of=2]{\st{x<=10}};
	\node [](4)[below of=3]{\st{z:=x; x:=x+y; y:=y+1}};
	\node [](5)[below of=4]{\st{x<=10}};
	\node [](6)[below of=5] {\st{z:=x; x:=x+y; y:=y+1}};
	\node [](7)[below of=6] {\st{x>10}};
	\node [](8)[below of=7] {\st{x:=x+1; y:=y-3}};
	\node [](9)[below of=8] {\st{x>10}};
	\node [](10)[below of=9] {\st{x:=x+1; y:=y-3}};
	\node [](11)[below of=10] {\st{x>21}};
	\node [](12)[below of=11] {\st{x!=21}};
	\draw [-] ([xshift=-1cm]2.north) -- ([xshift=1cm]2.north); 
	\draw [-] ([xshift=-1cm]5.north) -- ([xshift=1cm]5.north); 
	\draw [-] ([xshift=-1cm]7.north) -- ([xshift=1cm]7.north); 
	\draw [-] ([xshift=-1cm]9.north) -- ([xshift=1cm]9.north); 
	\draw [-] ([xshift=-1cm]11.north) -- ([xshift=1cm]11.north); 
	;
\end{tikzpicture}}}
		\end{column}
	\end{columns}
	\begin{center}
	\onslide<10->{\vspace{-1cm}\color{red}Too few interpolants for $\tau$!}
	\end{center}
\end{frame}

\begin{frame}[t]
	\frametitle{Accelerated Interpolation using \qvasr}
	\begin{center}
		Inductive Interpolant Computation
	\end{center}
	\begin{columns}
		\begin{column}{0.45\textwidth}
			\resizebox{0.8\textwidth}{!}{\begin{tikzpicture}[%
	->,
	>=stealth',
	shorten >=1pt,
	auto,
	node distance=0.75cm,
	scale=0.6,
	transform shape,
	align=center,
	smallnode/.style={inner sep=1.4},
	initial text =,
	anchor=center]
	\node [](t) {$\bar{\tau}$:};
	\node [label={[yshift=0cm, xshift=1.5cm, right]:\itp{\top}}] (1)[below of=t] {\st{x:=0; y:=2; z:=3}};
	\node [label={[yshift=0cm, xshift=1.5cm, right]:\itp{x = 0}}](7)[below of=1] {\st{$\psi$}};
	\node [label={[yshift=0cm, xshift=1.5cm, right]:\itp{x \leq 22 \land x > 11 \land y \leq 6}}](11)[below of=7] {\st{x>21}};
	\node [label={[yshift=0cm, xshift=1.5cm, right]:\itp{x = 22}}](12)[below of=11] {\st{x!=22}};
	\node [label={[yshift=0cm, xshift=1.5cm, right]:\itp{\bot}}](13)[below of=12] {};
	;
\end{tikzpicture}}
		\end{column}
		\begin{column}{0.45\textwidth}
			\begin{center}
				\resizebox{0.8\textwidth}{!}{
\begin{tikzpicture}[%
	->,
	>=stealth',
	shorten >=1pt,
	auto,
	node distance=0.75cm,
	scale=0.6,
	transform shape,
	align=center,
	smallnode/.style={inner sep=1.4},
	initial text =,
	anchor=center]
	\node [](t) {$\tau$:};
	\node [label={[yshift=0cm, xshift=1.5cm, right]:\only<1->{\itp{\top}}}](1)[below of=t] {\st{x:=0;}};
	\node [label={[yshift=0cm, xshift=1.5cm, right]:\only<1->{\itp{x = 0}}}](13)[below of=1] {\st{y:=2; z:=3}};
	\node [label={[yshift=0cm, xshift=1.5cm, right]:\only<1->{\itp{x \leq 22 \land y \leq 6}}}](2)[below of=13] {\st{x<=21; x<=10}};
	\node [label={[yshift=0cm, xshift=1.5cm, right]:\only<1->{\itp{x \leq 10 \land y \leq 6}}}](4)[below of=2]{\st{z:=x; x:=x+y; y:=y+1}};
	\node [label={[yshift=0cm, xshift=1.5cm, right]:\only<1->{\itp{x \leq 16 \land y \leq 7}}}](5)[below of=4]{\st{x<=21; x<=10}};
	\node [label={[yshift=0cm, xshift=1.5cm, right]:\only<1->{\itp{x \leq 10 \land y \leq 7}}}](6)[below of=5] {\st{z:=x; x:=x+y; y:=y+1}};
	\node [label={[yshift=0cm, xshift=1.5cm, right]:\only<1->{\itp{x \leq 16 \land y \leq 7}}}](7)[below of=6] {\st{x<=21; x>10}};
	\node [label={[yshift=0cm, xshift=1.5cm, right]:\only<1->{\itp{x \leq 16 \land y \leq 7}}}](8)[below of=7] {\st{x:=x+1; y:=y-3}};
	\node [label={[yshift=0cm, xshift=1.5cm, right]:\only<1->{\itp{x \leq 17 \land y \leq 4}}}](9)[below of=8] {\st{x<=21; x>10}};
	\node [label={[yshift=0cm, xshift=1.5cm, right]:\only<1->{\itp{x \leq 17 \land y \leq 4}}}](10)[below of=9] {\st{x:=x+1; y:=y-3}};
	\node [label={[yshift=0cm, xshift=1.5cm, right]:\only<1->{\itp{x \leq 18 \land y \leq 1}}}](11)[below of=10] {\st{x>21}};
	\node [label={[yshift=0cm, xshift=1.5cm, right]:\only<1->{\itp{\bot}}}](12)[below of=11] {\st{x!=22}};
	\node [label={[yshift=0cm, xshift=1.5cm, right]:\itp{\bot}}](13)[below of=12] {};
	
	\draw [-] ([xshift=-1cm]2.north) -- ([xshift=1cm]2.north); 
	\draw [-] ([xshift=-1cm]7.north) -- ([xshift=1cm]7.north); 
	\draw [-] ([xshift=-1cm]11.north) -- ([xshift=1cm]11.north); 
	;
\end{tikzpicture}

}
			\end{center}
		\end{column}
	\end{columns}
	\begin{center}
		\only<5>{post(\color{red} $(x = 0 \land y = 2)$, \color{black}\st{$\psi$}) = \color{red} \itp{x \leq y + 9 \land x = 0 \land y = 2}}
		\only<6>{post(\color{red} $x \leq y + 9 \land x = 0 \land y= 2$, \color{black}\st{x<=20; x<=10}) = \color{red}\itp{x \leq y + 9 \land x = 0 \land y = 2}}
		\only<7>{post(\color{red} $x \leq y + 8 \land x = 2 \land y= 3$, \color{black}\st{$\psi$}) = \color{red}\itp{x \leq y + 8 \land x = 2 \land y = 3}}
		\only<8>{post(\color{red} $x \leq y + 8 \land x = 2 \land y = 3$, \color{black}\st{x<=20; x<=10}) = \color{red}\itp{x \leq y + 8 \land x = 2 \land y = 3}}
		\only<9>{post(\color{red} $x \leq y + 7 \land x = 5 \land y = 4$, \color{black}\st{$\psi$}) = \color{red}\itp{x \leq y + 7 \land x = 5 \land y = 4}}
		\only<10>{post(\color{red} $x \leq y + 7 \land x = 5 \land y = 3$, \color{black} \st{x<=20; x>10}) = \color{red}\itp{\bot}}
		\only<11>{post(\color{red} $\bot$, \color{black} \st{$\psi$}) = \color{red}\itp{\bot}}
		\only<12>{post(\color{red} $\bot$, \color{black} \st{x<=20; x>10}) = \color{red}\itp{\bot}}
		\only<13>{post(\color{red} $\bot$, \color{black} \st{x=x+1; y=y-3}) = \color{red}\itp{\bot}}
		\only<14>{post(\color{red} $\bot$, \color{black} \st{x=x+1; y=y-3}) = \color{red}\itp{\bot}}
	\end{center}
\end{frame}