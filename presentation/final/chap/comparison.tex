\begin{frame}[t]
	\frametitle{Accelerated Interpolation vs Control-Flow Transformation}
		Control-Flow Graph Transformation: \\
		\begin{itemize}
			\item Bruteforce $\rightarrow$ summarize all findable loops \\ \vspace*{1cm}
		\end{itemize}
		
		Accelerated Interpolation: \\
		\begin{itemize}
			\item Decide which path is summarizable
			\item Employ heuristics to choose when to summarize
			\item Change meta traces to an underapproximation by chaining summarized loop paths
		\end{itemize}
		\vspace*{1cm}
\resizebox{\textwidth}{!}{
		\begin{tikzpicture}[%
			->,
			>=stealth', shorten >=1pt, auto,
			node distance=2.5cm, scale=1,
			transform shape, align=center,
			smallnode/.style={inner sep=1.4}
			initial text =]
			
			\node[state](1){$\loc{1}$};
			
			\node[state] (2) [right of=1] {$\loc{2}$};
			
			\node[state] (3) [right of=2, xshift=-0.55cm] {$\loc{2}'$};
			
			\node[state] (4) [right of=3, xshift=-0.55cm] {$\loc{2}$};
			
			\node[state] (5) [right of=4, xshift=-0.55cm] {$\loc{2}'$};
			
			\node[state] (6) [right of=5, xshift=0.4cm] {$\loc{7}$};
			
			\node[state] (7) [right of=6, xshift=0.4cm] {$\loc{9}$};
			
			\path (1) edge node {\texttt{x := 0}} (2);
			\path (2) edge node {$\psi^*_{L_{1}}$} (3);
			\path (3) edge node {$\varepsilon$} (4);
			\path (4) edge node {$\psi^*_{L_{2}}$} (5);
			\path (5) edge node {\texttt{!x <= 10}} (6);
			\path (6) edge node {\texttt{x != 11}} (7);
			;
		\end{tikzpicture}
}
\end{frame}