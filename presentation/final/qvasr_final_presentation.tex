
\documentclass[%
9pt,
dvipsnames,
]{beamer}

\usepackage[utf8]{inputenc}
\usepackage{xspace}
\usepackage{tabularx}
\usepackage{hyperref}
\usepackage{algorithm}
\usepackage[noend]{algpseudocode}
\usepackage{cite}
\usepackage{booktabs}
\usepackage{url}
\usepackage{listings}
\usepackage{amsthm}
\usepackage{amsmath}
\usepackage{tikz}
\usetikzlibrary{positioning,shapes.geometric, arrows.meta ,automata, decorations.pathreplacing, calc, fit, backgrounds, quotes, tikzmark}
\usepackage{pgf}
\usepackage{slantsc}
\usepackage{geometry}
\usepackage{amssymb}
\usepackage{subcaption}
\usepackage{float}
\usepackage{pgf}
\usepackage{slashbox}
\usepackage{pgfgantt}
\usepackage{wrapfig}
\usepackage{pdflscape}
\usepackage{xcolor}
\usepackage{xparse}
\usepackage[%disable,%
colorinlistoftodos,%
color=cyan!50!white,%
bordercolor=cyan!50!black]{todonotes}

\usepackage{varwidth}
\usepackage[most]{tcolorbox}% http://ctan.org/pkg/tcolorbox
\tcbuselibrary{skins,breakable}
\usepackage{comment}
\usepackage{makecell}
\usepackage{stmaryrd}

\DeclareSymbolFont{matha}{OML}{txmi}{m}{it}% txfonts
\DeclareMathSymbol{\varv}{\mathord}{matha}{118}

\lstset{ %
	backgroundcolor=\color{white}, 
	basicstyle=\footnotesize,       
	breakatwhitespace=false,        
	breaklines=true,                 
	captionpos=b,                    
	commentstyle=\color{mygreen},   
	escapeinside={\%*}{*)},        
	extendedchars=true,              
	frame=single,                  
	keywordstyle=\color{blue},       
	language=Prolog,                
	numbers=left,                    
	numbersep=5pt,                   
	rulecolor=\color{black},        
	showspaces=false,               
	showstringspaces=false,          
	showtabs=false,                  
	stringstyle=\color{mymauve},   
	tabsize=2,                      
	title=\lstname,                  
	morekeywords={not,\},\{,preconditions,effects },            
	deletekeywords={time}            
}

%%%%%%%%%%%% Colors
%% a somewhat friendly scheme for 5 different colors
\definecolor{g1}		{RGB}{215,25,28} % a kind of red
\definecolor{g2}		{RGB}{253,174,97} % a kind of orange
\definecolor{g3}		{RGB}{255,255,191} % a kind of yellow
\definecolor{g4}		{RGB}{171,217,233} % a kind of light blue
\definecolor{g5}		{RGB}{44,123,182} % a kind of dark blue

\definecolor{gr1}		{RGB}{250, 250, 250}
\definecolor{gr2}		{RGB}{229, 229, 229} % some grey

% color of interpolants
\definecolor{itpGreen}  {rgb}{0,1,0}
\definecolor{grey}      {RGB}{200,200,200}


%color for pictures
\colorlet{outlineblue}		{g5}
\colorlet{fillblue}			{g4}
\colorlet{darkback}			{gr2}
\colorlet{lightback}		{gr1}
\colorlet{stmtcolor}		{gr2} %default statement color
\colorlet{subgraphcolor}	{g3} %default statement color
\colorlet{colexamtitle}     {black} % Example block title color
\colorlet{colexamline}      {g1} % Example block sideline color
\colorlet{itp}              {itpGreen}



%%%%%%%%%%%%% Statements and labels Trace Abstraction Style
\tikzstyle{st} = [%
font=\ttfamily,%
shape=rectangle,%
rounded corners=.5em,%
fill=stmtcolor,%
inner xsep=.3em,%
inner ysep=0em, %
text height=2ex, %
text depth=.6ex,
]


\newcommand{\tikzstmt}[3]{{%
		\tikz[baseline]{%
			\node[st,fill=#2] at (0,.64ex){%
				\hspace{.3em}\texttt{\strut#3#1}\hspace{.3em}\strut};}
}}

\newcommand{\stcol}[2]{\tikzstmt{#1}{#2}{}}
\newcommand{\stsmcol}[2]{\tikzstmt{#1}{#2}{\small}}
\newcommand{\stfootcol}[2]{\tikzstmt{#1}{#2}{\footnotesize}}

\newcommand{\stnorm}[1]{\stcol{#1}{stmtcolor}}
\newcommand{\stsm}[1]{\stsmcol{#1}{stmtcolor}}
\newcommand{\stfoot}[1]{\stfootcol{#1}{stmtcolor}}

\newcommand{\st}[1]{\stfoot{#1}}
\newcommand{\lab}[1]{\stfoot{\ensuremath{#1}}}
\newcommand{\lan}[1]{\stnorm{\ensuremath{#1}}}
\newcommand{\stn}[1]{\stnorm{#1}}

\newcommand{\formula}[2]{\tikz[baseline]{\node[shape=rectangle,line width=1pt,draw=#2,fill=#2!30,inner sep=1pt, align=center] at (0,.64ex){\hspace{.2em}\texttt{\strut#1}\hspace{.1em}\strut};}}
\newcommand{\itp}[1]{\formula{\ensuremath{#1}}{itp}}

\newcommand{\tf}{\ensuremath{\varphi}}
\newcommand{\ctf}{\ensuremath{\widehat{\varphi}}}
\newcommand{\invars}{\ensuremath{In}}
\newcommand{\outvars}{\ensuremath{Out}}
\newcommand{\auxvars}{\ensuremath{Aux}}

\newcommand{\Var}{\ensuremath{\mathit{Var}}}
\newcommand{\stmt}{\ensuremath{\mathit{Stmt}}}
\newcommand{\Loc}{\ensuremath{\mathit{Loc}}}
\newcommand{\err}{\ensuremath{\mathit{err}}}
\newcommand{\init}{\ensuremath{\mathit{init}}}


\newcommand{\pq}[1]{\ensuremath{#1_{p, q}}}
\newcommand{\pqvasr}{\ensuremath{(\pq{S}, \pq{V})}}
\newcommand{\pqformula}{\ensuremath{\pq{\Gamma}}}

%%%%%%%%%%%% Numbered example environment
% \newcounter{example}[section]
% \newenvironment{example}[1][]{\refstepcounter{example}\par\medskip
%    \noindent \textbf{Example~\theexample. #1} \rmfamily}{\medskip}


\usepackage{regexpatch}

\makeatletter
\xpatchcmd{\algorithmic}{\labelsep 0.5em}{\labelsep 1.5em}{\typeout{Success!}}{\typeout{Oh dear!}}
\makeatother

%%%%%%%%%%%% Comments
\newif\iffinal
%\finaltrue % comment out to remove comments

\iffinal
\newcommand\mycom[1]{}
\else
\newcommand\mycom[1]{#1}
\overfullrule=1mm
\fi
\setlength\parindent{0pt}

\newcommand{\jw}[1]{\mycom{\todo[color=blue!40,inline]{\small JW: #1}}}
\newcommand{\dd}[1]{\mycom{\todo[color=orange!40,inline]{\small DD: #1}}}
\newcommand{\ts}[1]{\mycom{\todo[color=green!40,inline]{\small TS: #1}}}


\newcommand{\all}[1]{\mycom{\todo[color=green!40,inline]{\small #1}}}
\newcommand{\meta}[1]{\mycom{\todo[color=blue!10,inline,caption={Beschreibung},nolist]{\setlist{nolistsep}\small #1}}}
\newcommand{\xxx}{\mycom{\stfootcol{Placeholder}{blue!20}}}
\newcommand{\cn}{\mycom{\stfootcol{Cite}{blue!20}}}

\newcommand{\Q}{\ensuremath{\mathbb{Q}}}
\newcommand{\entails}[1]{\vdash_{#1}}

\newcommand{\qvasr}{\ensuremath{\mathit{\mathbb{Q} \text{-}VASR}}}
\newcommand{\qvasrs}{\ensuremath{\mathit{\mathbb{Q} \text{-}VASRS}}}

\newcommand{\tranFormula}{\ensuremath{F(\vec{x}, \vec{x}')}}
\newcommand{\coherent}{\ensuremath{\equiv_{\mathit{V}}}}
\newcommand{\conjunctTF}{\ensuremath{C(\vec{x}, \vec{x}')}}
\newcommand{\GammaTF}{\ensuremath{\Gamma(\vec{x}, \vec{x}')}}


\title[\qvasr]{\textbf{Adapting Loop Summarization \\ Using Rational Vector Addition Systems \\ with Resets for Trace Abstraction}}

% For short you could for example use the last name only, it's optional as is
% the short title
\author[Jonas Werner]{\color{emblue} \textbf{Master's Thesis} \\ Jonas Werner \\ \vspace*{0.5cm} \textbf{Advisers} \\ Dr. Daniel Dietsch, Tanja Schindler}

% Can be set, for students usually not required
%\institute{}

\date{\today}

% removes the navigation symbols
% often they don't work and I find them rather annoying
\beamertemplatenavigationsymbolsempty{}

% You can have a logo to appear on all slides
% \logo{\includegraphics[height=1.5cm]{logo}}

% If you want that at the beginning of each section the table of contents
% is shown, I don't like it for short presentations
\AtBeginSection[]{
	\begin{frame}
		\vfill
		\centering
		\begin{beamercolorbox}[sep=8pt,center,shadow=true,rounded=true]{title}
			\usebeamerfont{title}\insertsectionhead\par%
		\end{beamercolorbox}
		\vfill
	\end{frame}
}


\begin{document}

\begin{frame}
  \titlepage
\end{frame}

%%%%%%%%%%%%%%%%%%%%%%%%%%%%%%%%%%%%%%%%%%%%%%%%%%%%%%%%%%%%%%%%%%%%%%%%%%%%%% motivation

\begin{frame}[t]
	\frametitle{Motivation}
	\begin{center}
		Loop analysis is one of the most challenging parts of software verification \\
		"\color{emblue}Archilles Heel of program verification \color{black}" \cite{DBLP:journals/fmsd/KroeningSTTW13} \pause
			\begin{minipage}{0.35\textwidth}
				\vspace*{0.5cm}
				\resizebox{0.7\textwidth}{!}{\begin{lstlisting}[language=C++, style=withAssert]  % Start your code-block
	
	int x := 0;
	int y := 2;
	int z := 3;
	while x <= 20:
		if x <= 10:
			z := x;
			x := x + y;
			y := y + 1;
		else:
			x := x + 2;
			y := y - 3;
	assert x == 21;
	\end{lstlisting}}
			\end{minipage}
		\pause
		\begin{tabular}{@{}l@{}}
			\tabitem Loop Unwinding \\
			\tabitem Loop Invariants \\
			\tabitem Loop Summarization
		\end{tabular}
	\end{center} 
\end{frame}

\begin{frame}[t]
	\frametitle{Loop Unwinding}
	Todo: Maybe show an endless long trace or something
\end{frame}

\begin{frame}[t]
	\frametitle{Loop Invariants}
	\begin{center}
		\onslide<+->
		Conditions that hold before and after each loop iteration		
		\begin{columns}[c]
		\begin{column}{0.35\textwidth}
			\onslide<+->
			\begin{figure}[h]
				\vspace*{0.5cm}
				\resizebox{0.7\textwidth}{!}{\begin{lstlisting}[language=C++,basicstyle=\ttfamily,keywordstyle=\color{blue}, escapechar=\%]  % Start your code-block
	
	int x := 0;
	int y := 2;
	int z := 3;
%\itp{(x \leq 22 \land x > 11) \lor x \leq y + 9}%
	while x <= 20:
		if x <= 10:
			z := x;
			x := x + y;
			y := y + 1;
		else:
			x := x + 2;
			y := y - 3;
	assert x == 21;
	\end{lstlisting}}
			\end{figure}
		\end{column}
			\onslide<+->
		\begin{column}{0.05\textwidth}
				But:
		\end{column}
		\begin{column}{0.35\textwidth}
			\begin{figure}[h]
				\vspace*{0.5cm}
				\resizebox{0.7\textwidth}{!}{\begin{lstlisting}[language=C++,basicstyle=\ttfamily,keywordstyle=\color{blue}, escapechar=\%]  % Start your code-block
	
	int x := 0;
	int y := 2;
	int z := 3;
	%\itp{Useless Invariant}%
	while x <= 20:
		if x <= 10:
			z := x;
			x := x + y;
			y := y + 1;
		else:
			x := x + 2;
			y := y - 3;
	assert x == 22;
	\end{lstlisting}}
			\end{figure}
		\end{column}
	\end{columns}
	\onslide<+->
	Usefulness is tied to how strong the invariants are! \\
	\onslide<+->
	Finding strong loop invariants "\color{emblue}is an art in itself \color{black}" \cite{DBLP:journals/fmsd/KroeningSTTW13}
	\end{center}
\end{frame}

\begin{frame}[t]
	\frametitle{Loop Summarization}
	\begin{center}
		There are two kinds of loop summarization: \\
		\onslide<+-> Underapproximative and Overapproximative Summarization
		\begin{columns}[t]
			\begin{column}{0.35\textwidth}
				\vspace*{0.5cm} \\
				\onslide<+-> Underapproximation:
				\begin{itemize}
					\onslide<+-> \item Summary forms a subset of actual loop behaviour
					\item Not every counterexample is covered!
				\end{itemize}
			\end{column}
			\begin{column}{0.5\textwidth}
				\vspace*{0.5cm}
				\resizebox{\textwidth}{!}{\begin{tikzpicture}
	\node[] at (0 ,3.25) {\color{emblue}{Actual Loop Behaviour}};
	\draw[emblue, fill=emblue!20, fill opacity=0.2] (0,0) circle (3);
	\node[] at (0,0) {\color{green}{Underapproximation}};
	\draw [rounded corners, green,thick,dashed] (-2,-2) -- (0, -2.75) -- (1.25, -2.5) -- (2.5, -0.5) -- (1.25, 1.5) -- (0.25, 2.8) -- (-2, 2) -- (-2.8, 0) -- cycle;
	\onslide<+-> \node at (3, 4) (head) []{Missed Counterexample};
	\node at (2, 1.5) (dot) [circle,fill,inner sep=1.5pt]{};
	\path (head) edge node []{} (dot);
\end{tikzpicture}}
			\end{column}
		\end{columns}
	\end{center}
\end{frame}

\begin{frame}[t]
	\frametitle{Loop Summarization}
	\begin{center}
		There are two kinds of loop summarization: \\
		\onslide<+-> Underapproximative and Overapproximative Summarization
		\begin{columns}[t]
			\begin{column}{0.35\textwidth}
				\vspace*{0.5cm} \\
				\onslide<+-> Overapproximation:
				\begin{itemize}
					\onslide<+-> \item Summary forms a superset of actual loop behaviour
					\item Covers spurious counterexamples!
				\end{itemize}
			\end{column}
			\begin{column}{0.5\textwidth}
				\vspace*{0.5cm}
				\resizebox{\textwidth}{!}{ \begin{tikzpicture}
	\node[] at (2.7,2) {\color{emblue}{Actual Loop Behaviour}};
	\draw [rounded corners,red,thick,dashed] (0,0) -- (3, -1) -- (5.5,0.5) -- (6, 3.5) -- (4, 5)-- (0.25, 4) node [midway, sloped, above] {Overapproximation} -- cycle;
	\draw[emblue, fill=emblue!20, fill opacity=0.2] (2.7,2) circle (2.5);
	\node at (5, 5.75) (head) []{Spurious Counterexample};
	\node at (5.5, 3.25) (dot) [circle,fill,inner sep=1.5pt]{};
	\path (head) edge node []{} (dot);
\end{tikzpicture}}
			\end{column}
		\end{columns}
	\end{center}
\end{frame}

%%%%%%%%%%%%%%%%%%%%%%%%%%%%%%%%%%%%%%%%%%%%%%%%%%%%%%%%%%%%%%%%%%%%%%%%%%%%%% qvasr introduction

\begin{frame}[t]
	\frametitle{\qvasr}
	\begin{definition}[\qvasr]
		A rational vector addition system with resets (\qvasr)\ $V$ is a set of transformers, that are a pair of vectors $(\vec{r}, \vec{a})$:\\
		\begin{center}
			$ V = 
			\begin{Bmatrix}
				\begin{pmatrix}
					\underbrace{
						\begin{bmatrix}
							r_1 \\
							\vdots \\
							r_n
					\end{bmatrix}}_{\vec{r}\ \in\ \{0,1\}^n},
					\underbrace{
						\begin{bmatrix}
							a_1 \\
							\vdots \\
							a_n
					\end{bmatrix}}_{\vec{a}\ \in\ \mathbb{Q}^n}
				\end{pmatrix}
			\end{Bmatrix}
			$
		\end{center}
	\end{definition}
	\begin{center}
	\onslide<2->
		Given a \qvasr\ $V$ with pair $(\vec{r}, \vec{a}) \in V$ of dimension $n$, $V$ defines a transition system $(Q_V, \rightarrow_V)$ with state space $Q_V =  \mathbb{Q}^n$  and transitions $\vec{x} \rightarrow_V \vec{x}'$ defined as: \\
			\begin{equation*}
				\begin{bmatrix}
					x_1' \\
					\vdots \\
					x_n'
				\end{bmatrix}
				=
				\begin{bmatrix}
					x_1 \\
					\vdots \\
					x_n
				\end{bmatrix}
				*
				\begin{bmatrix}
					r_1 \\
					\vdots \\
					r_n
				\end{bmatrix}
				+
				\begin{bmatrix}
					a_1 \\
					\vdots \\
					a_n
				\end{bmatrix}		
			\end{equation*}
		\end{center}
\end{frame}

%%%%%%%%%%%%%%%%%%%%%%%%%%%%%%%%%%%%%%%%%%%%%%%%%%%%%%%%%%%%%%%%%%%%%%%%%%%%%% qvasr example

\begin{frame}[t]
	\frametitle{\qvasr\ Example}
	\begin{columns}
		\begin{column}{0.4\textwidth}
			\begin{figure}[h]
				\vspace*{0.5cm}
				\resizebox{0.6\textwidth}{!}{\begin{lstlisting}[language=C++, style=withAssert]  % Start your code-block
	
	int x := 0;
	int y := 2;
	int z := 3;
	while x <= 20:
		if x <= 10:
			z := x;
			x := x + y;
			y := y + 1;
		else:
			x := x + 2;
			y := y - 3;
	assert x == 21;
	\end{lstlisting}}
				\vspace{-0.5cm}
			\end{figure}
		\end{column} \pause
		\begin{column}{0.6\textwidth}
			\begin{figure}[h]
				\begin{equation*}
					G: x \leq 20 \land x > 10 \land x' = {\color<4>{red}{x}} + {\color<5>{red}{2}} \land y' = {\color<4>{red}{y}} {\color<5>{red}{-3}}
				\end{equation*}
				\caption*{Transition formula $G$ of the else branch.}
			\end{figure}
		\end{column}
	\end{columns}
	\begin{columns}
		\begin{column}{0.4\textwidth} \uncover<6>{
				\begin{figure}
					$V_G = 
					\begin{Bmatrix}
						\begin{bmatrix}
							x' = x + 2 \\
							y' = y - 3
						\end{bmatrix} 
					\end{Bmatrix}
					$
					\caption*{Intuitive notation of $V_G$.}
				\end{figure}
			}
		\end{column}
		\begin{column}{0.4\textwidth} \uncover<3->{
				\begin{figure}
					$V_G = 
					\begin{Bmatrix}
						\begin{pmatrix}
							\begin{bmatrix}
								{\color<4>{red}{1}} \\
								{\color<4>{red}{1}}
							\end{bmatrix},
							\begin{bmatrix}
								{\color<5>{red}{2}} \\
								{\color<5>{red}{-3}}
							\end{bmatrix}
						\end{pmatrix}
					\end{Bmatrix}
					$
					\caption*{\qvasr\ $V_G$.}
				\end{figure}
			}
		\end{column}
	\end{columns}
\end{frame}

%%%%%%%%%%%%%%%%%%%%%%%%%%%%%%%%%%%%%%%%%%%%%%%%%%%%%%%%%%%%%%%%%%%%%%%%%%%%%% new frame
\begin{frame}[t]
	\frametitle{\qvasr\ Example}
	\begin{columns}
		\begin{column}{0.4\textwidth}
			\begin{figure}[h]
				\vspace*{0.5cm}
				\resizebox{0.6\textwidth}{!}{\begin{lstlisting}[language=C++, style=withAssert]  % Start your code-block
	
	int x := 0;
	int y := 2;
	int z := 3;
	while x <= 20:
		if x <= 10:
			z := x;
			x := x + y;
			y := y + 1;
		else:
			x := x + 2;
			y := y - 3;
	assert x == 21;
	\end{lstlisting}}
				\vspace{-0.5cm}
			\end{figure}
		\end{column} \pause
		\begin{column}{0.6\textwidth}
			\begin{figure}[h]
				\begin{equation*}
					H: x \leq 10 \land x' = x + {\color<3->{red}{y}} \land y' = y + 1 \land z' = {\color<3->{red}{x}}
				\end{equation*}
				\caption*{Transition formula $H$ of the if branch.}
			\end{figure}
		\end{column}
	\end{columns}
	\begin{center} 	\onslide<4-> \vspace*{0.5cm}
			\qvasr\ can model only assignments of constant amounts!
	\end{center}
\end{frame}


%%%%%%%%%%%%%%%%%%%%%%%%%%%%%%%%%%%%%%%%%%%%%%%%%%%%%%%%%%%%%%%%%%%%%%%%%%%%%% qvasr abstraction 


\begin{frame}[c]
	\frametitle{\qvasr-Abstraction Join Example}
	\begin{center}
		\begin{figure}
			% Define block styles
\begin{tikzpicture}[%
	->,
	>=stealth',
	shorten >=1pt,
	auto,
	node distance=5cm,
	scale=0.75,
	transform shape,
	align=center,
	smallnode/.style={inner sep=1.4},
	initial text =,
	anchor=center]
	% Place nodes
	\node [](G) {
		$V_G = 
		\begin{Bmatrix}
			\begin{bmatrix}
				x' = x + 2 \\
				y' = y - 3
			\end{bmatrix}
		\end{Bmatrix}
		$};
	\node [](H)[right of=G, xshift=1cm]  {$V_H = \begin{Bmatrix} \begin{bmatrix} -x' + y' + z' = 1 \\ y' = y + 1 \end{bmatrix} \end{Bmatrix}$};
	\node (best) at ($(G)!0.5!(H)$) [yshift=-3cm] {		    
		$V_F = \begin{Bmatrix}
			\color<3>{red}\begin{bmatrix}
			\color<3-4>{red}\color<6>{red} - x' + y' + z' = 1\\
			\color<3>{red}	y' = \color<7>{red} y + 1
			\color<3>{red}\end{bmatrix} , \\ \\
			\color<3>{red}\begin{bmatrix}
			\color<3>{red}\color<5-6>{red}	-x' + y' + z' = -x + y + z - 5 \\
			\color<3>{red}	y' = \color<8>{red} y - 3
			\color<3>{red}\end{bmatrix}
			\end{Bmatrix}
			$};

	\draw (G) to node[above] {} node[above] {} (best);
	\draw (H) to node[above] {} node[left] {} (best);
	
\end{tikzpicture}
			\caption*{\qvasr-abstraction $V_F$ of $F$}
		\end{figure}
	\end{center}
	\pause
	\begin{center}
	Using $V_F$, we can derive the following overapproximative summarization $\psi$ of $F$:
	\begin{align*}
	 \psi: \	\color<3>{red}\exists k_1, k_2.\ &(\color<4>{red}\color<6>{red}(-x' + y' + z' = 1\ \color<4>{black} \lor \color<5-6>{red}\ -x' + y' + z' = -x + y + z - 5k_2 \color<6>{black})\ \\ &\land\ y' = y + \color<7>{red} k_1 \color<7>{black} - \color<8>{red} 3k_2 \color<8>{black})\
	\end{align*}
	\end{center}
\end{frame}


%%%%%%%%%%%%%%%%%%%%%%%%%%%%%%%%%%%%%%%%%%%%%%%%%%%%%%%%%%%%%%%%%%%%%%%%%%%%%% trace abstraction

\begin{frame}[t]
	\frametitle{Trace Abstraction}
	\begin{center}
		\resizebox{0.8\textwidth}{!}{
\begin{tikzpicture}[%
	->,
	>=stealth',
	shorten >=1pt,
	auto,
	node distance=2.5cm,
	scale=0.9,
	transform shape,
	align=center,
	smallnode/.style={inner sep=1.4},
	initial text =,
	anchor=center]
	
	\tikzstyle{my node}=[draw,minimum height=1cm,minimum width=3cm]
	\node [align=center](tainput) {\textbf{Input}: \\ A program $P$};
	\node[my node] (cfg) [below of=tainput, yshift=1cm] {Control-Flow Graph \\ Builder};
	\node[my node](ta1)[below of=cfg]{$\mathcal{L}(A_P) \subseteq \mathcal{L}(A_D)?$};
	\node[my node] (ta2) [below of=ta1, yshift=-1cm] {$\tau$ feasible?};
	\node[my node] (ta3) [below left of=ta1, xshift=-1cm]{\texttt{generalize($\tau$)}};
	\node [](corr)[right of=ta1, right=0.5cm] {\textbf{return}: $P$ is Safe};
	\node [](incorr)[right of=ta2, right=0.5cm] {\textbf{return}: $P$ is Unsafe};
	
	\path (ta1) edge[bend left] node[align=left, right=0.25cm]{no \\
		$\tau \in \mathcal{L}(A_P) \backslash \mathcal{L}(A_D)$} (ta2)
	(ta3.north) edge[bend left] node[align=right, left=0.25cm] {$A_D := A_D\ \cup$ $\texttt{generalize} (\tau)$} (ta1.west)
	(cfg) edge[] node[align=left, right=0.25cm] {$A_P := \text{control-flow graph as automaton}$ \\ 		$A_D := \emptyset$} (ta1)
	(ta1) edge[] node[above] {yes} (corr)
	(ta2) edge[] node[above] {yes} (incorr)
	(tainput) edge[] node[] {} (cfg);
	
	\draw [->, bend left] (ta2.west) to node[align=right, left=0.5cm] {state assertions \\ as proof} node[align=right, yshift=-0.5cm] {no} (ta3.south); 
	;
\end{tikzpicture}}
	\end{center}
\end{frame}

%%%%%%%%%%%%%%%%%%%%%%%%%%%%%%%%%%%%%%%%%%%%%%%%%%%%%%%%%%%%%%%%%%%%%%%%%%%%%% accelInterpol

\begin{frame}[t]
	\frametitle{Accelerated Interpolation using \qvasr}
	\resizebox{\textwidth}{!}{
\begin{tikzpicture}[%
	->,
	>=stealth',
	shorten >=1pt,
	auto,
	node distance=2.5cm,
	scale=0.9,
	transform shape,
	align=center,
	smallnode/.style={inner sep=1.4},
	initial text =,
	anchor=center]
	
	\tikzstyle{my node}=[draw,minimum height=1cm,minimum width=3cm]
	\begin{scope}
	\node [align=center](tainput) {\textbf{Input}: \\ A program $P$};
	\node[my node] (cfg) [below of=tainput, yshift=1cm] {Control-Flow Graph \\ Builder};
	\node[my node](ta1)[below of=cfg]{$\mathcal{L}(A_P) \subseteq \mathcal{L}(A_D)?$};
	\node[my node] (ta2) [below of=ta1, yshift=-1cm] {$\tau$ feasible?};
	\node[my node] (ta3) [below left of=ta1, xshift=-1cm]{\texttt{generalize($\tau$)}};
	\node [](corr)[right of=ta1, right=0.5cm] {\textbf{return}: $P$ is Safe};
	\node [](incorr)[right of=ta2, right=0.5cm] {\textbf{return}: $P$ is Unsafe};
	
	\path (ta1) edge[bend left] node[align=left, right=0.25cm]{no \\
		$\tau \in \mathcal{L}(A_P) \backslash \mathcal{L}(A_D)$} (ta2)
	(ta3.north) edge[bend left] node[align=right, left=0.25cm] {$A_D := A_D\ \cup$ $\texttt{generalize} (\tau)$} (ta1.west)
	(ta2.west) edge[bend left] node[left=0.25cm] {no} (ta3.south)
	(cfg) edge[] node[align=left, right=0.25cm] {$A_P := \text{control-flow graph as automaton}$ \\ 		$A_D = \emptyset$} (ta1)
	(ta1) edge[] node[above] {yes} (corr)
	(ta2) edge[] node[above] {yes} (incorr)
	(tainput) edge[] node[] {} (cfg)
	;
	\end{scope}


	\begin{scope}[xshift=10cm, node distance=2cm]
	\tikzstyle{in node}=[draw,minimum height=1cm,minimum width=5cm]
	\node [align=center](input) {\textbf{Input}: \\ Program trace $\tau$};
	\node[in node, fill=white](1)[below of=input, yshift=0.25cm]{Does $\tau$ contain a loop?};
	
	\node[my node,  fill=white](2)[below of=1, xshift=1cm]{Loop \\ Summarization};
	\node[my node, fill=white](3)[below of=2]{Meta-Trace \\ Construction};
	
	\node[my node, fill=white](4)[below of=3]{Meta-Interpolant \\ Computation};
	\node[my node, fill=white](interpolants)[below of=4, xshift=-1cm]{Inductive Interpolant  Computation};
	
	\node[](6)[below of=interpolants, yshift=0.25cm]{\textbf{Output}: \\ Inductive Sequence of Interpolants for $\tau$};
	
	\begin{scope}[on background layer]
		\node[group, line width=0.3mm, draw=emblue, fill=emblue!20, fill opacity=0.2, fit=(1) (interpolants) (3), yscale=1.025, xscale=1.025](FIt1) {};
	\end{scope}
	
	\path (input) edge[] node[]{} (1)
	([xshift=-1cm] 1.south) edge[] node[left=0.1cm]{\textbf{no}} ([xshift=-1cm]interpolants.north)
	;
	\draw [->] ([xshift=1cm]1.south) -- node[right] {\textbf{yes}} ([xshift=1cm]1.south |-2.north); 
	\draw [->] (2.south) -- (2.south |-3.north);
	\draw [->] (3.south) -- (3.south |-4.north); 
	\draw [->] (4.south) -- (4.south |-interpolants.north); 
	\draw [->] (interpolants.south) -- (6.north); 
	\end{scope}

	%\draw [Circle - Circle, color=emblue, line width=0.3mm] (ta3) -- (ta3 -| FIt1.west); 
	\draw [-, color=emblue, line width=0.3mm] (ta3.north east) -- ([xshift=0.15cm]FIt1.north west); 
	\draw [-, color=emblue, line width=0.3mm] (ta3.south east) -- ([xshift=0.15cm]FIt1.south west);
	%\draw [-, color=emblue, line width=0.3mm] (ta3.south east) -- (FIt1.south east); 
	%\draw [-, color=emblue, line width=0.3mm] (ta3.south west) -- (FIt1.south west);
\end{tikzpicture}}
\end{frame}

\begin{frame}[t]	\frametitle{Accelerated Interpolation using \qvasr}
	\resizebox{0.5\textwidth}{!}{
\begin{tikzpicture}[%
	->,
	>=stealth',
	shorten >=1pt,
	auto,
	node distance=2cm,
	scale=0.75,
	transform shape,
	align=center,
	smallnode/.style={inner sep=1.4},
	initial text =,
	anchor=center]
	
	\tikzstyle{my node}=[draw,minimum height=1cm,minimum width=2.5cm]
	\tikzstyle{in node}=[draw,minimum height=1cm,minimum width=5cm]
	\tikzstyle{dashed node}=[draw,minimum height=1cm,minimum width=3cm, dashed]
	
	\node [align=center](input) {\textbf{Input}: \\ Program trace $\tau$};
	\node[in node, fill=white](1)[below of=input, yshift=0.25cm]{Does $\tau$ contain a loop?};
	
	\node[my node,  fill=white](2)[below of=1, xshift=1cm]{Loop \\ summarizable?};
	\node[my node, fill=white](3)[below of=2]{Meta-Trace \\ Construction};
	
	\node[my node, fill=white](4)[below of=3]{Meta-Proof \\ Computation};
	
	\node[my node, fill=white](interpolants)[below of=4, xshift=-2.25cm]{Proof \\ Computation};
	\node[my node, fill=white](post)[right of=interpolants, xshift=0.25cm]{Post \\ Processing};
	
	\node[](6)[below of=interpolants, yshift=-0.2cm, xshift=1cm]{\textbf{Output}: \\ Inductive Sequence of \\ State Assertions for $\tau$};
	
	\node[dashed node](7)[right of=2, xshift=2.5cm]{Divide Loop};
	
	\begin{scope}[on background layer]
		\node[group, line width=0.3mm, draw=emblue, fill=emblue!20, fill opacity=0.2, fit=(1) (interpolants) (3), yscale=1.05, xscale=1.025](FIt1) {};
	\end{scope}
	
	\path (input) edge[] node[]{} (1)
	([xshift=-1.25cm] 1.south) edge[] node[left=0.1cm]{\textbf{no}} ([]interpolants.north)
	;
	\draw [->] ([xshift=1cm]1.south) -- node[right] {\textbf{yes}} ([xshift=1cm]1.south |-2.north); 
	\draw [->] (2.south) --  node[right] {\textbf{yes}} (2.south |-3.north);
	\draw [->] (3.south) -- (3.south |-4.north); 
	\draw [->] (4.south) -- (4.south |-interpolants.north); 
	\draw [->] ([xshift=1cm]interpolants.south) -- (6.north); 
	
	\draw[->, dashed] ([yshift=0.2cm]2.east) -- node[above] {\textbf{no}} ([yshift=0.2cm]7.west);
	\draw[->, dashed] ([yshift=-0.2cm]7.west) -- ([yshift=-0.2cm]2.east);
	;
\end{tikzpicture}}
\end{frame}

\begin{frame}[t]
	\frametitle{Accelerated Interpolation using \qvasr}
	\begin{columns}
		\begin{column}{0.35\textwidth}
			\resizebox{0.8\textwidth}{!}{
\begin{tikzpicture}[%
	->,
	>=stealth',
	shorten >=1pt,
	auto,
	node distance=2cm,
	scale=0.75,
	transform shape,
	align=center,
	smallnode/.style={inner sep=1.4},
	initial text =,
	anchor=center]
	
	\tikzstyle{my node}=[draw,minimum height=1cm,minimum width=2.5cm]
	\tikzstyle{in node}=[draw,minimum height=1cm,minimum width=5cm]
	\tikzstyle{dashed node}=[draw,minimum height=1cm,minimum width=3cm, dashed]
	
	\node [align=center](input) {\color<2>{emblue} \textbf{Input}: \\ \color<2>{emblue} Program trace $\tau$};
	\node[in node, fill=white](1)[below of=input, yshift=0.25cm]{\color<3>{emblue}Does $\tau$ contain a loop?};
	
	\node[my node,  fill=white](2)[below of=1, xshift=1cm]{\color<5>{emblue} Loop \\ \color<5>{emblue}summarizable?};
	\node[my node, fill=white](3)[below of=2]{Meta-Trace \\ Construction};
	
	\node[my node, fill=white](4)[below of=3]{Meta-Proof \\ Computation};
	
	\node[my node, fill=white](interpolants)[below of=4, xshift=-2.25cm]{Proof \\ Computation};
	\node[my node, fill=white](post)[right of=interpolants, xshift=0.25cm]{Post \\ Processing};
	
	\node[](6)[below of=interpolants, yshift=-0.2cm, xshift=1cm]{\textbf{Output}: \\ Inductive Sequence of \\ State Assertions for $\tau$};
	
	\begin{scope}[on background layer]
		\node[group, line width=0.3mm, draw=emblue, fill=emblue!20, fill opacity=0.2, fit=(1) (interpolants) (3), yscale=1.05, xscale=1.025](FIt1) {};
	\end{scope}
	
	\path (input) edge[] node[]{} (1)
	([xshift=-1.25cm] 1.south) edge[] node[left=0.1cm]{\textbf{no}} ([]interpolants.north)
	;
	\draw [->] ([xshift=1cm]1.south) -- node[right] {\color<4>{emblue} \textbf{yes}} ([xshift=1cm]1.south |-2.north); 
	\draw [->] (2.south) --  node[right] {\textbf{yes}} (2.south |-3.north);
	\draw [->] (3.south) -- (3.south |-4.north); 
	\draw [->] (4.south) -- (4.south |-interpolants.north); 
	\draw [->] ([xshift=1cm]interpolants.south) -- (6.north); 
	;
\end{tikzpicture}}
		\end{column}
		\begin{column}{0.55\textwidth}
			\onslide<2-5>
			\resizebox{0.4\textwidth}{!}{
\begin{tikzpicture}[%
	->,
	>=stealth',
	shorten >=1pt,
	auto,
	node distance=0.75cm,
	scale=0.6,
	transform shape,
	align=center,
	smallnode/.style={inner sep=1.4},
	initial text =,
	anchor=center]
	\node [](t) {\color<2-3>{emblue} $\tau$:};
	\node [](1)[below of=t] {\st{x:=0}};
	\node [](13)[below of=1] {\st{y:=2}};
	\node [](14)[below of=13] {\st{z:= 3}};
	\node [](2)[below of=14] {\color<4-5>{red} \st{x<=20; x<=10}};
	\node [](4)[below of=2]{\color<4-5>{red}\st{z:=x; x:=x+y; y:=y+1}};
	\node [](5)[below of=4]{\color<4-5>{red}\st{x<=20; x<=10}};
	\node [](6)[below of=5] {\color<4-5>{red}\st{z:=x; x:=x+y; y:=y+1}};
	\node [](7)[below of=6] {\color<4-5>{red}\st{x<=20; x>10}};
	\node [](8)[below of=7] {\color<4-5>{red}\st{x:=x+1; y:=y-3}};
	\node [](9)[below of=8] {\color<4-5>{red}\st{x<=20; x>10}};
	\node [](10)[below of=9] {\color<4-5>{red}\st{x:=x+1; y:=y-3}};
	\node [](11)[below of=10] {\st{x>20}};
	\node [](12)[below of=11] {\st{x!=21}};
	
	\onslide<4-5>\draw [-] ([xshift=-1cm]2.north) -- ([xshift=1cm]2.north); 
	\onslide<4-5>\draw [-] ([xshift=-1cm]7.north) -- ([xshift=1cm]7.north); 
	\onslide<4-5>\draw [-] ([xshift=-1cm]11.north) -- ([xshift=1cm]11.north); 
	;
\end{tikzpicture}}
		\end{column}
	\end{columns}
\end{frame}

\begin{frame}[t]
	\frametitle{Accelerated Interpolation using \qvasr}
	\begin{columns}
		\begin{column}{0.35\textwidth}
			\resizebox{0.8\textwidth}{!}{
\begin{tikzpicture}[%
	->,
	>=stealth',
	shorten >=1pt,
	auto,
	node distance=2cm,
	scale=0.75,
	transform shape,
	align=center,
	smallnode/.style={inner sep=1.4},
	initial text =,
	anchor=center]
	
	\tikzstyle{my node}=[draw,minimum height=1cm,minimum width=3cm]
	\tikzstyle{in node}=[draw,minimum height=1cm,minimum width=5cm]
	\tikzstyle{dashed node}=[draw,minimum height=1cm,minimum width=3cm, dashed]
	
	\node [align=center](input) {\textbf{Input}: \\ \ Program trace $\tau$};
	\node[in node, fill=white](1)[below of=input, yshift=0.25cm]{Does $\tau$ contain a loop?};
	
	\node[my node, fill=white](2)[below of=1, xshift=1cm]{\color<1->{emblue}Loop summarizable?};
	\node[my node, fill=white](3)[below of=2]{Meta-Trace \\ Construction};
	
	\node[my node, fill=white](4)[below of=3]{Meta-Interpolant \\ Computation};
	\node[my node, fill=white](interpolants)[below of=4, xshift=-1cm]{Inductive Interpolant  Computation};
	
	\node[](6)[below of=interpolants, yshift=0.25cm]{\textbf{Output}: \\ Inductive Sequence of Interpolants for $\tau$};
	
	\begin{scope}[on background layer]
		\node[group, line width=0.3mm, draw=emblue, fill=emblue!20, fill opacity=0.2, fit=(1) (interpolants) (3), yscale=1.05, xscale=1.025](FIt1) {};
	\end{scope}
	
	\path (input) edge[] node[]{} (1)
	([xshift=-1cm] 1.south) edge[] node[left=0.1cm]{\textbf{no}} ([xshift=-1cm]interpolants.north)
	;
	\draw [->] ([xshift=1cm]1.south) -- node[right] {\textbf{yes}} ([xshift=1cm]1.south |-2.north); 
	\draw [->] (2.south) --  node[right] {\color<4->{emblue}\textbf{yes}} (2.south |-3.north);
	\draw [->] (3.south) -- (3.south |-4.north); 
	\draw [->] (4.south) -- (4.south |-interpolants.north); 
	\draw [->] (interpolants.south) -- (6.north); 
	;
\end{tikzpicture}}
		\end{column}
		\begin{column}{0.55\textwidth}
			Single loop iterations: \vspace*{0.25cm}\\
			\st{x<=21; x<=10} \st{z:=x; x:=x+y; y:=y+1}
			\onslide<2->
			{\small 
			\begin{align*}
			&\rightarrow x \leq 10\ \land\ x' = x + y\ \land\ y' = y + 1\ \land\ z' = x \\
			&\rightarrow \text{Transition formula}\ H 
			\end{align*}
			}%
			\onslide<1-> \\
			\vspace*{1cm}\st{x<=21; x>10} \st{x:=x+1; y:=y-3}
			\onslide<2->
			\begin{align*}
			&\rightarrow	x \leq 21\ \land\ x > 10\ \land\ x' = x + 2\ \land\ y' = y -3 \\
			&\rightarrow \text{Transition formula}\ G
			\end{align*}
			\onslide<3-> \\
			\vspace*{1cm}
			$\rightarrow F = G \lor H$ \onslide<4-> with loop summary \color<4->{emblue} $\psi$
		\end{column}
	\end{columns}
\end{frame}

\begin{frame}[t]
	\frametitle{Accelerated Interpolation using \qvasr}
	\begin{columns}
		\begin{column}{0.35\textwidth}
			\resizebox{0.8\textwidth}{!}{
\begin{tikzpicture}[%
	->,
	>=stealth',
	shorten >=1pt,
	auto,
	node distance=2cm,
	scale=0.75,
	transform shape,
	align=center,
	smallnode/.style={inner sep=1.4},
	initial text =,
	anchor=center]
	
	\tikzstyle{my node}=[draw,minimum height=1cm,minimum width=3cm]
	\tikzstyle{in node}=[draw,minimum height=1cm,minimum width=5cm]
	\tikzstyle{dashed node}=[draw,minimum height=1cm,minimum width=3cm, dashed]
	
	\node [align=center](input) {\textbf{Input}: \\ \ Program trace $\tau$};
	\node[in node, fill=white](1)[below of=input, yshift=0.25cm]{Does $\tau$ contain a loop?};
	
	\node[my node, fill=white](2)[below of=1, xshift=1cm]{Loop summarizable?};
	\node[my node, fill=white](3)[below of=2]{\color<1->{emblue} Meta-Trace \\ \color<1->{emblue} Construction};
	
	\node[my node, fill=white](4)[below of=3]{Meta-Interpolant \\ Computation};
	\node[my node, fill=white](interpolants)[below of=4, xshift=-1cm]{Inductive Interpolant  Computation};
	
	\node[](6)[below of=interpolants, yshift=0.25cm]{\textbf{Output}: \\ Inductive Sequence of Interpolants for $\tau$};
	
	\begin{scope}[on background layer]
		\node[group, line width=0.3mm, draw=emblue, fill=emblue!20, fill opacity=0.2, fit=(1) (interpolants) (3), yscale=1.05, xscale=1.025](FIt1) {};
	\end{scope}
	
	\path (input) edge[] node[]{} (1)
	([xshift=-1cm] 1.south) edge[] node[left=0.1cm]{\textbf{no}} ([xshift=-1cm]interpolants.north)
	;
	\draw [->] ([xshift=1cm]1.south) -- node[right] {\textbf{yes}} ([xshift=1cm]1.south |-2.north); 
	\draw [->] (2.south) --  node[right] {\textbf{yes}} (2.south |-3.north);
	\draw [->] (3.south) -- (3.south |-4.north); 
	\draw [->] (4.south) -- (4.south |-interpolants.north); 
	\draw [->] (interpolants.south) -- (6.north); 
	;
\end{tikzpicture}}
		\end{column}
		\begin{column}{0.55\textwidth}
			\resizebox{0.77\textwidth}{!}{
\begin{tikzpicture}[%
	->,
	>=stealth',
	shorten >=1pt,
	auto,
	node distance=0.75cm,
	scale=0.6,
	transform shape,
	align=center,
	smallnode/.style={inner sep=1.4},
	initial text =,
	anchor=center]
	\node [](t) {$\tau$:};
	\node [](1)[below of=t] {\st{x:=0}};
	\node [](13)[below of=1] {\st{y:=2}};
	\node [](14)[below of=13] {\st{z:=3}};
	\node [](2)[below of=14] {\color<1->{red}\st{x<=20; x<=10}};
	\node [](4)[below of=2]{\color<1->{red}\st{z:=x; x:=x+y; y:=y+1}};
	\node [](5)[below of=4]{\color<1->{red}\st{x<=20; x<=10}};
	\node [](6)[below of=5] {\color<1->{red}\st{z:=x; x:=x+y; y:=y+1}};
	\node [](7)[below of=6] {\color<1->{red}\st{x<=20; x>10}};
	\node [](8)[below of=7] {\color<1->{red}\st{x:=x+1; y:=y-3}};
	\node [](9)[below of=8] {\color<1->{red}\st{x<=20; x>10}};
	\node [](10)[below of=9] {\color<1->{red}\st{x:=x+1; y:=y-3}};
	\node [](11)[below of=10] {\st{x>20}};
	\node [](12)[below of=11] {\st{x!=21}};
	
	% Simple brace
	\draw [-, decorate, 
	decoration = {brace,
		raise=5pt,
		amplitude=5pt}] ([xshift=1.25cm] 2.north) --  node[right, xshift=1cm] {Replace loop by $\psi$} ([xshift=1.25cm] 10.south);
	;
\end{tikzpicture}}
		\end{column}
	\end{columns}
\end{frame}

\begin{frame}[t]
	\frametitle{Accelerated Interpolation using \qvasr}
	\begin{columns}
		\begin{column}{0.35\textwidth}
			\resizebox{0.8\textwidth}{!}{
\begin{tikzpicture}[%
	->,
	>=stealth',
	shorten >=1pt,
	auto,
	node distance=2cm,
	scale=0.75,
	transform shape,
	align=center,
	smallnode/.style={inner sep=1.4},
	initial text =,
	anchor=center]
	
	\tikzstyle{my node}=[draw,minimum height=1cm,minimum width=3cm]
	\tikzstyle{in node}=[draw,minimum height=1cm,minimum width=5cm]
	\tikzstyle{dashed node}=[draw,minimum height=1cm,minimum width=3cm, dashed]
	
	\node [align=center](input) {\textbf{Input}: \\ \ Program trace $\tau$};
	\node[in node, fill=white](1)[below of=input, yshift=0.25cm]{Does $\tau$ contain a loop?};
	
	\node[my node, fill=white](2)[below of=1, xshift=1cm]{Loop summarizable?};
	\node[my node, fill=white](3)[below of=2]{\color<1->{emblue} Meta-Trace \\ \color<1->{emblue} Construction};
	
	\node[my node, fill=white](4)[below of=3]{Meta-Interpolant \\ Computation};
	\node[my node, fill=white](interpolants)[below of=4, xshift=-1cm]{Inductive Interpolant  Computation};
	
	\node[](6)[below of=interpolants, yshift=0.25cm]{\textbf{Output}: \\ Inductive Sequence of Interpolants for $\tau$};
	
	\begin{scope}[on background layer]
		\node[group, line width=0.3mm, draw=emblue, fill=emblue!20, fill opacity=0.2, fit=(1) (interpolants) (3), yscale=1.05, xscale=1.025](FIt1) {};
	\end{scope}
	
	\path (input) edge[] node[]{} (1)
	([xshift=-1cm] 1.south) edge[] node[left=0.1cm]{\textbf{no}} ([xshift=-1cm]interpolants.north)
	;
	\draw [->] ([xshift=1cm]1.south) -- node[right] {\textbf{yes}} ([xshift=1cm]1.south |-2.north); 
	\draw [->] (2.south) --  node[right] {\textbf{yes}} (2.south |-3.north);
	\draw [->] (3.south) -- (3.south |-4.north); 
	\draw [->] (4.south) -- (4.south |-interpolants.north); 
	\draw [->] (interpolants.south) -- (6.north); 
	;
\end{tikzpicture}}
		\end{column}
		\begin{column}{0.55\textwidth}
			\resizebox{0.4\textwidth}{!}{\begin{tikzpicture}[%
	->,
	>=stealth',
	shorten >=1pt,
	auto,
	node distance=0.75cm,
	scale=0.6,
	transform shape,
	align=center,
	smallnode/.style={inner sep=1.4},
	initial text =,
	anchor=center]
	\node [](t) {$\tau$:};
	\node [](1)[below of=t] {\st{x:=0; y:=2; z:=3}};
	\node [](2)[below of=1] {\color<1->{white}x<=20};
	\node [](3)[below of=2]{\color<1->{white} x<=10};
	\node [](4)[below of=3]{\color<1->{white} z:=x; x:=x+y; y:=y+1};
	\node [](5)[below of=4]{\color<1->{white}x<=10};
	\node [](6)[below of=5] {\color<1->{white}z:=x; x:=x+y; y:=y+1};
	\node [](7)[below of=6] {\color<1->{emblue}\st{$\psi$}};
	\node [](8)[below of=7] {\color<1->{white}x:=x+1; y:=y-3};
	\node [](9)[below of=8] {\color<1->{white}x>10};
	\node [](10)[below of=9] {\color<1->{white}x:=x+1; y:=y-3};
	\node [](11)[below of=10] {\color<1->{black}\st{x>20}};
	\node [](12)[below of=11] {\st{x!=22}};
	;
\end{tikzpicture}}
		\end{column}
	\end{columns}
\end{frame}

\begin{frame}[t]
	\frametitle{Accelerated Interpolation using \qvasr}
	\begin{columns}
		\begin{column}{0.35\textwidth}
			\resizebox{0.8\textwidth}{!}{
\begin{tikzpicture}[%
	->,
	>=stealth',
	shorten >=1pt,
	auto,
	node distance=2cm,
	scale=0.75,
	transform shape,
	align=center,
	smallnode/.style={inner sep=1.4},
	initial text =,
	anchor=center]
	
	\tikzstyle{my node}=[draw,minimum height=1cm,minimum width=2.5cm]
	\tikzstyle{in node}=[draw,minimum height=1cm,minimum width=5cm]
	\tikzstyle{dashed node}=[draw,minimum height=1cm,minimum width=3cm, dashed]
	
	\node [align=center](input) {\textbf{Input}: \\ Program trace $\tau$};
	\node[in node, fill=white](1)[below of=input, yshift=0.25cm]{Does $\tau$ contain a loop?};
	
	\node[my node,  fill=white](2)[below of=1, xshift=1cm]{Loop \\ summarizable?};
	\node[my node, fill=white](3)[below of=2]{Meta-Trace \\ Construction};
	
	\node[my node, fill=white](4)[below of=3]{\color<1-9>{emblue}Meta-Proof \\ \color<1-9>{emblue}Computation};
	
	\node[my node, fill=white](interpolants)[below of=4, xshift=-2.25cm]{Proof \\ Computation};
	\node[my node, fill=white](post)[right of=interpolants, xshift=0.25cm]{\color<10->{emblue}Post \\\color<10->{emblue} Processing};
	
	\node[](6)[below of=interpolants, yshift=-0.2cm, xshift=1cm]{\textbf{Output}: \\ Inductive Sequence of \\ State Assertions for $\tau$};
	
	\begin{scope}[on background layer]
		\node[group, line width=0.3mm, draw=emblue, fill=emblue!20, fill opacity=0.2, fit=(1) (interpolants) (3), yscale=1.05, xscale=1.025](FIt1) {};
	\end{scope}
	
	\path (input) edge[] node[]{} (1)
	([xshift=-1.25cm] 1.south) edge[] node[left=0.1cm]{\textbf{no}} ([]interpolants.north)
	;
	\draw [->] ([xshift=1cm]1.south) -- node[right] {\textbf{yes}} ([xshift=1cm]1.south |-2.north); 
	\draw [->] (2.south) --  node[right] {\textbf{yes}} (2.south |-3.north);
	\draw [->] (3.south) -- (3.south |-4.north); 
	\draw [->] (4.south) -- (4.south |-interpolants.north); 
	\draw [->] ([xshift=1cm]interpolants.south) -- (6.north); 
	;
\end{tikzpicture}}
		\end{column}
		\begin{column}{0.3\textwidth}
				\resizebox{\textwidth}{!}{\begin{tikzpicture}[%
	->,
	>=stealth',
	shorten >=1pt,
	auto,
	node distance=0.75cm,
	scale=0.6,
	transform shape,
	align=center,
	smallnode/.style={inner sep=1.4},
	initial text =,
	anchor=center]
	\node [](t) {$\bar{\tau}$:};
	\node [] (1k)[below of=t] {\st{x:=0;}};
	\node [] (13k)[below of=1k] {\st{y:=2; z:=3}};
	\node [] (7k)[below of=13k] {\color<1>{emblue}\st{$\psi$}};
	\node [] (11k)[below of=7k] {\color<1->{black}\st{x>21}};
	\node [] (12k)[below of=11k] {\st{x!=21}};
	\onslide<3->\node [label={[yshift=0cm, xshift=0.7cm, right]:{\itp{\top}}}] (1)[below of=t] {\st{x:=0;}};
	\onslide<4-> \node [label={[yshift=0cm, xshift=0.7cm, right]:{\itp{x = 0}}}] (13)[below of=1] {\st{y:=2; z:=3}};
	\onslide<5-> \node [label={[yshift=0cm, xshift=0.7cm, right]:{\itp{x = 0}}}](7)[below of=13] {\color<1>{emblue}\st{$\psi$}};
	\onslide<6->\node [label={[yshift=0cm, xshift=0.7cm, right]:{\itp{x \leq 22 \land y \leq 6}}}](11)[below of=7] {\color<1->{black}\st{x>21}};
	\onslide<7->\node [label={[yshift=0cm, xshift=0.7cm, right]:{\itp{x = 22}}}](12)[below of=11] {\st{x!=22}};
	\onslide<8-> \node [label={[yshift=0cm, xshift=0.7cm, right]:{\itp{\bot}}}](13)[below of=12] {};
	;
\end{tikzpicture}}
				\onslide<10->{Too few interpolants for $\tau$!}
		\end{column}
		\begin{column}{0.3\textwidth}
				\onslide<9->{\resizebox{0.7\textwidth}{!}{
\begin{tikzpicture}[%
	->,
	>=stealth',
	shorten >=1pt,
	auto,
	node distance=0.75cm,
	scale=0.6,
	transform shape,
	align=center,
	smallnode/.style={inner sep=1.4},
	initial text =,
	anchor=center]
	\node [](t) {$\tau$:};
	\node [](1)[below of=t] {\st{x:=0}};
	\node [](13)[below of=1] {\st{y:=2}};
	\node [](14)[below of=13] {\st{z:=3}};
	\node [](2)[below of=14] {\st{x<=21}};
	\node [](3)[below of=2]{\st{x<=10}};
	\node [](4)[below of=3]{\st{z:=x; x:=x+y; y:=y+1}};
	\node [](5)[below of=4]{\st{x<=10}};
	\node [](6)[below of=5] {\st{z:=x; x:=x+y; y:=y+1}};
	\node [](7)[below of=6] {\st{x>10}};
	\node [](8)[below of=7] {\st{x:=x+1; y:=y-3}};
	\node [](9)[below of=8] {\st{x>10}};
	\node [](10)[below of=9] {\st{x:=x+1; y:=y-3}};
	\node [](11)[below of=10] {\st{x>21}};
	\node [](12)[below of=11] {\st{x!=21}};
	\draw [-] ([xshift=-1cm]2.north) -- ([xshift=1cm]2.north); 
	\draw [-] ([xshift=-1cm]5.north) -- ([xshift=1cm]5.north); 
	\draw [-] ([xshift=-1cm]7.north) -- ([xshift=1cm]7.north); 
	\draw [-] ([xshift=-1cm]9.north) -- ([xshift=1cm]9.north); 
	\draw [-] ([xshift=-1cm]11.north) -- ([xshift=1cm]11.north); 
	;
\end{tikzpicture}}}
		\end{column}
	\end{columns}
\end{frame}

\begin{frame}[t]
	\frametitle{Accelerated Interpolation using \qvasr}
	\begin{center}
		Inductive Interpolant Computation
	\end{center}
	\begin{columns}
		\begin{column}{0.45\textwidth}
			\resizebox{0.75\textwidth}{!}{\begin{tikzpicture}[%
	->,
	>=stealth',
	shorten >=1pt,
	auto,
	node distance=0.75cm,
	scale=0.6,
	transform shape,
	align=center,
	smallnode/.style={inner sep=1.4},
	initial text =,
	anchor=center]
	\node [](t) {$\bar{\tau}$:};
	\node [label={[yshift=0cm, xshift=1.5cm, right]:\itp{\top}}] (1)[below of=t] {\st{x:=0; y:=2; z:=3}};
	\node [label={[yshift=0cm, xshift=1.5cm, right]:\itp{x = 0}}](7)[below of=1] {\st{$\psi$}};
	\node [label={[yshift=0cm, xshift=1.5cm, right]:\itp{x \leq 22 \land x > 11 \land y \leq 6}}](11)[below of=7] {\st{x>21}};
	\node [label={[yshift=0cm, xshift=1.5cm, right]:\itp{x = 22}}](12)[below of=11] {\st{x!=22}};
	\node [label={[yshift=0cm, xshift=1.5cm, right]:\itp{\bot}}](13)[below of=12] {};
	;
\end{tikzpicture}}
			TODO Post sequence
		\end{column}
		\begin{column}{0.45\textwidth}
			\begin{center}
				\resizebox{0.75\textwidth}{!}{
\begin{tikzpicture}[%
	->,
	>=stealth',
	shorten >=1pt,
	auto,
	node distance=0.75cm,
	scale=0.6,
	transform shape,
	align=center,
	smallnode/.style={inner sep=1.4},
	initial text =,
	anchor=center]
	\node [](t) {$\tau$:};
	\node [label={[yshift=0cm, xshift=1.5cm, right]:\only<1->{\itp{\top}}}](1)[below of=t] {\st{x:=0;}};
	\node [label={[yshift=0cm, xshift=1.5cm, right]:\only<1->{\itp{x = 0}}}](13)[below of=1] {\st{y:=2; z:=3}};
	\node [label={[yshift=0cm, xshift=1.5cm, right]:\only<1->{\itp{x \leq 22 \land y \leq 6}}}](2)[below of=13] {\st{x<=21; x<=10}};
	\node [label={[yshift=0cm, xshift=1.5cm, right]:\only<1->{\itp{x \leq 10 \land y \leq 6}}}](4)[below of=2]{\st{z:=x; x:=x+y; y:=y+1}};
	\node [label={[yshift=0cm, xshift=1.5cm, right]:\only<1->{\itp{x \leq 16 \land y \leq 7}}}](5)[below of=4]{\st{x<=21; x<=10}};
	\node [label={[yshift=0cm, xshift=1.5cm, right]:\only<1->{\itp{x \leq 10 \land y \leq 7}}}](6)[below of=5] {\st{z:=x; x:=x+y; y:=y+1}};
	\node [label={[yshift=0cm, xshift=1.5cm, right]:\only<1->{\itp{x \leq 16 \land y \leq 7}}}](7)[below of=6] {\st{x<=21; x>10}};
	\node [label={[yshift=0cm, xshift=1.5cm, right]:\only<1->{\itp{x \leq 16 \land y \leq 7}}}](8)[below of=7] {\st{x:=x+1; y:=y-3}};
	\node [label={[yshift=0cm, xshift=1.5cm, right]:\only<1->{\itp{x \leq 17 \land y \leq 4}}}](9)[below of=8] {\st{x<=21; x>10}};
	\node [label={[yshift=0cm, xshift=1.5cm, right]:\only<1->{\itp{x \leq 17 \land y \leq 4}}}](10)[below of=9] {\st{x:=x+1; y:=y-3}};
	\node [label={[yshift=0cm, xshift=1.5cm, right]:\only<1->{\itp{x \leq 18 \land y \leq 1}}}](11)[below of=10] {\st{x>21}};
	\node [label={[yshift=0cm, xshift=1.5cm, right]:\only<1->{\itp{\bot}}}](12)[below of=11] {\st{x!=22}};
	\node [label={[yshift=0cm, xshift=1.5cm, right]:\itp{\bot}}](13)[below of=12] {};
	
	\draw [-] ([xshift=-1cm]2.north) -- ([xshift=1cm]2.north); 
	\draw [-] ([xshift=-1cm]7.north) -- ([xshift=1cm]7.north); 
	\draw [-] ([xshift=-1cm]11.north) -- ([xshift=1cm]11.north); 
	;
\end{tikzpicture}

}
			\end{center}
		\end{column}
	\end{columns}
\end{frame}

%%%%%%%%%%%%%%%%%%%%%%%%%%%%%%%%%%%%%%%%%%%%%%%%%%%%%%%%%%%%%%%%%%%%%%%%%%%%%% cfg transformer

\begin{frame}[t]
	\frametitle{Control-Flow Graph Transformation}
	\resizebox{\textwidth}{!}{
\begin{tikzpicture}[%
	->,
	>=stealth',
	shorten >=1pt,
	auto,
	node distance=2.5cm,
	scale=0.9,
	transform shape,
	align=center,
	smallnode/.style={inner sep=1.4},
	initial text =,
	anchor=center]
	
	\tikzstyle{my node}=[draw,minimum height=1cm,minimum width=3cm]
	\begin{scope}
		\node [align=center](tainput) {\textbf{Input}: \\ A program $P$};
		\node[my node] (cfg) [below of=tainput, yshift=1cm] {Control-Flow Graph \\ Builder};
		\node[my node](ta1)[below of=cfg]{$\mathcal{L}(A_P) \subseteq \mathcal{L}(A_D)?$};
		\node[my node] (ta2) [below of=ta1, yshift=-1cm] {$\tau$ feasible?};
		\node[my node] (ta3) [below left of=ta1, xshift=-1cm]{\texttt{generalize($\tau$)}};
		\node [](corr)[right of=ta1, right=0.5cm] {\textbf{return}: $P$ is Safe};
		\node [](incorr)[right of=ta2, right=0.5cm] {\textbf{return}: $P$ is Unsafe};
		
		\path (ta1) edge[bend left] node[align=left, right=0.25cm]{no \\
			$\tau \in \mathcal{L}(A_P) \backslash \mathcal{L}(A_D)$} (ta2)
		(ta3.north) edge[bend left] node[align=right, left=0.25cm] {$A_D := A_D\ \cup$ $\texttt{generalize} (\tau)$} (ta1.west)
		(ta2.west) edge[bend left] node[left=0.25cm] {no} (ta3.south)
		(cfg) edge[] node[align=left, right=0.25cm] {$A_P := \text{control-flow graph as automaton}$ \\ 		$A_D = \emptyset$} (ta1)
		(ta1) edge[] node[above] {yes} (corr)
		(ta2) edge[] node[above] {yes} (incorr)
		(tainput) edge[] node[] {} (cfg)
		;
	\end{scope}
	
	
	\begin{scope}[xshift=10cm, node distance=2cm]
		\tikzstyle{in node}=[draw,minimum height=1cm,minimum width=5cm]
		\node[align=center](input) {\textbf{Input}: \\ Program $P$};
		\node[in node, fill=white](5)[below of=input, yshift=0.25cm]{Construct Control-Flow Graph $G$};
		\node[in node, fill=white](1)[below of=5]{Does $G$ contain a loop?};
		\node[my node, fill=white](2)[below of=1, xshift=1cm]{Loop \\ Summarization};
		\node[my node, fill=white](3)[below of=2]{Loop \\ Replacement};
		% \node[in node, fill=white](4)[below of=3, xshift=-1cm]{Control-Flow Graph \\ Construction};
		\node[](output)[below of=3, xshift=-2cm, yshift=0.25cm]{\textbf{Output}: \\Transformed Control-Flow Graph};
		
		\begin{scope}[on background layer]
			\node[group, line width=0.3mm, draw=emblue, fill=emblue!20, fill opacity=0.2, fit=(5) (3) , xscale=1.25, yscale=1.05](FIt2) {};
		\end{scope}
		
		\path (input) edge[] node[]{} (5)
		([xshift=-1cm] 1.south) edge[] node[left=0.1cm]{\textbf{no}} (output.north)
		;
		\draw [->] (5.south) -- node[right] {} (1.north);
		\draw [->] ([xshift=1cm]1.south) -- node[right] {\textbf{yes}} ([xshift=1cm]1.south |-2.north); 
		\draw [->] (2.south) -- (2.south |-3.north);
		\draw[->] (3) -- node[below right,pos=0.2]{} ++(2,0) |- (1);     
	\end{scope}
	
	%\draw [Circle - Circle, color=emblue, line width=0.3mm] (ta3) -- (ta3 -| FIt1.west); 
	\draw [-, color=emblue, line width=0.3mm] (cfg.north east) -- ([xshift=0.15cm]FIt2.north west); 
	\draw [-, color=emblue, line width=0.3mm] (cfg.south east) -- ([xshift=0.1cm]FIt2.south west);
	%\draw [-, color=emblue, line width=0.3mm] (ta3.south east) -- (FIt1.south east); 
	%\draw [-, color=emblue, line width=0.3mm] (ta3.south west) -- (FIt1.south west);
\end{tikzpicture}}
\end{frame}

\begin{frame}[t]
	\frametitle{Control-Flow Graph Transformation}
	\begin{columns}
		\begin{column}{0.4\textwidth}
			\resizebox{0.75\textwidth}{!}{
\begin{tikzpicture}[%
	->,
	>=stealth',
	shorten >=1pt,
	auto,
	node distance=2cm,
	scale=0.75,
	transform shape,
	align=center,
	smallnode/.style={inner sep=1.4},
	initial text =,
	anchor=center]
	
	\tikzstyle{my node}=[draw,minimum height=1cm,minimum width=3cm]
	\tikzstyle{in node}=[draw,minimum height=1cm,minimum width=5cm]
	\node[align=center](input) {\textbf{Input}: \\ Program $P$};
	\node[in node, fill=white](5)[below of=input, yshift=0.25cm]{Construct Control-Flow Graph $G$};
	\node[in node, fill=white](1)[below of=5]{Does $G$ contain a loop?};
	\node[my node, fill=white](2)[below of=1, xshift=1cm]{Loop \\ Summarization};
	\node[my node, fill=white](3)[below of=2]{Edge \\ Replacement};
	% \node[in node, fill=white](4)[below of=3, xshift=-1cm]{Control-Flow Graph \\ Construction};
	\node[](output)[below of=3, xshift=-1cm, yshift=0.25cm]{\textbf{Output}: \\ Loop-Free Control-Flow Graph};
	
	\begin{scope}[on background layer]
		\node[group, line width=0.3mm, draw=emblue, fill=emblue!20, fill opacity=0.2, fit=(5) (3) (4), xscale=1.025, yscale=1.025](FIt2) {};
	\end{scope}
	
	\path (input) edge[] node[]{} (5)
	([xshift=-1cm] 1.south) edge[] node[left=0.1cm]{\textbf{no}} ([xshift=-1cm]output.north)
	;
	\draw [->] (5.south) -- node[right] {} (1.north);
	\draw [->] ([xshift=1cm]1.south) -- node[right] {\textbf{yes}} ([xshift=1cm]1.south |-2.north); 
	\draw [->] (2.south) -- (2.south |-3.north);
	\draw [->] (3.south) -- (3.south |-output.north); 
\end{tikzpicture}}
		\end{column}
		\begin{column}{0.4\textwidth}
			\onslide<3->
			\resizebox{0.6\textwidth}{!}{\begin{lstlisting}[language=C++, style=withAssert]  % Start your code-block
	
	int x := 0;
	int y := 2;
	int z := 3;
	while x <= 20:
		if x <= 10:
			z := x;
			x := x + y;
			y := y + 1;
		else:
			x := x + 2;
			y := y - 3;
	assert x == 21;
	\end{lstlisting}}
		\end{column}
	\end{columns}
\end{frame}

\begin{frame}[t]
	\frametitle{Control-Flow Graph Transformation}
	\begin{center}
		\begin{columns}[c]
			\begin{column}{0.4\textwidth}
				\resizebox{0.6\textwidth}{!}{\begin{lstlisting}[language=C++, style=withAssert]  % Start your code-block
	
	int x := 0;
	int y := 2;
	int z := 3;
	while x <= 20:
		if x <= 10:
			z := x;
			x := x + y;
			y := y + 1;
		else:
			x := x + 2;
			y := y - 3;
	assert x == 21;
	\end{lstlisting}}
			\end{column}
			\begin{column}{0.03\textwidth}
				$\rightarrow$
			\end{column}
			\begin{column}{0.4\textwidth}
					\resizebox{\textwidth}{!}{
\begin{tikzpicture}[%
    ->,
    >=stealth',
    shorten >=1pt,
    auto,
    node distance=3.25cm,
    scale=0.9,
    transform shape,
    align=center,
    smallnode/.style={inner sep=1.4},
    initial text =,
    anchor=center]

			\node[state, initial above, initial text =](1){$\loc{1}$};
			\node[state] (head) [below of=1] {$\loc{4}$};
			\node[state] (loopEntry)[left of=head] {$\loc{5}$};
			\node[state] (if)[above of=loopEntry] {$\loc{6}$};
			
			\node[state] (else)[below of=loopEntry] {$\loc{10}$};
			\node[state] (loopExit)[right of=head] {$\loc{13}$};
			
			\node[state] (assertTrue)[below left of=loopExit] {$\loc{14}$};
			\node[state, accepting] (assertFalse)[below of=loopExit] {$\loc{err}$};
			
			\path (1) edge node []{\st{x:=0;}\\ \st{y:=2;}\\ \st{z:=3;}} (head)
			(head) edge node []{\st{x<=20}} (loopEntry)
			(head) edge node []{\st{x>20}} (loopExit)
			
			(loopEntry) edge[] node []{\st{x<=10}} (if)
			(if) edge[]node []{\\ \st{z:=x;}\\ \st{x:=x+y;}\\ \st{y:=y+1}} (head)
			
			(loopEntry) edge[]node []{\st{x>10}} (else)
			(else) edge[] node []{\st{x:=x+2;}\\ \st{y:=y-3}} (head)
			
			(loopExit) edge node []{\st{x==22}} (assertTrue)
			(loopExit) edge node []{\st{x!=22}} (assertFalse)
			;
\end{tikzpicture}}
			\end{column}
		\end{columns}
		\end{center}
\end{frame}

\begin{frame}[t]
	\frametitle{Control-Flow Graph Transformation using \qvasr}
	\begin{columns}
		\begin{column}{0.4\textwidth}
			\resizebox{0.75\textwidth}{!}{
\begin{tikzpicture}[%
	->,
	>=stealth',
	shorten >=1pt,
	auto,
	node distance=2cm,
	scale=0.75,
	transform shape,
	align=center,
	smallnode/.style={inner sep=1.4},
	initial text =,
	anchor=center]
	
	\tikzstyle{my node}=[draw,minimum height=1cm,minimum width=3cm]
	\tikzstyle{in node}=[draw,minimum height=1cm,minimum width=5cm]
	\node[align=center](input) {\textbf{Input}: \\ Program $P$};
	\node[in node, fill=white](5)[below of=input, yshift=0.25cm]{Construct Control-Flow Graph $G$};
	\node[in node, fill=white](1)[below of=5]{\color<1-2>{emblue}Does $G$ contain a loop?};
	\node[my node, fill=white](2)[below of=1, xshift=1cm]{\color<3>{emblue} Loop \\ \color<3>{emblue} Summarization};
	\node[my node, fill=white](3)[below of=2]{\color<3>{emblue} Loop \\ \color<3>{emblue} Replacement};
	% \node[in node, fill=white](4)[below of=3, xshift=-1cm]{Control-Flow Graph \\ Construction};
	\node[](output)[below of=3, xshift=-2cm, yshift=0.25cm]{\textbf{Output}: \\ Transformed Control-Flow Graph};
	
	\begin{scope}[on background layer]
		\node[group, line width=0.3mm, draw=emblue, fill=emblue!20, fill opacity=0.2, fit=(5) (3), xscale=1.25, yscale=1.025](FIt2) {};
	\end{scope}
	
	\path (input) edge[] node[]{} (5)
	([xshift=-1cm] 1.south) edge[] node[left=0.1cm]{\textbf{no}} (output.north)
	;
	\draw [->] (5.south) -- node[right] {} (1.north);
	\draw [->] ([xshift=1cm]1.south) -- node[right] {\color<2>{emblue} \textbf{yes}} ([xshift=1cm]1.south |-2.north); 
	\draw [->] (2.south) -- (2.south |-3.north);
	\draw[->] (3) -- node[below right,pos=0.2]{} ++(2,0) |- (1);     
\end{tikzpicture}}
		\end{column}
			\begin{column}{0.4\textwidth}
				\resizebox{\textwidth}{!}{
\begin{tikzpicture}[%
	->,
	>=stealth',
	shorten >=1pt,
	auto,
	node distance=3.25cm,
	scale=0.9,
	transform shape,
	align=center,
	smallnode/.style={inner sep=1.4},
	initial text =,
	anchor=center]
	
	\node[state, initial above, initial text =](1){$\loc{1}$};
	\node[state, red] (head) [below of=1] {$\loc{4}$};
	\node[state, red] (loopEntry)[left of=head] {$\loc{5}$};
	\node[state, red] (if)[above of=loopEntry] {$\loc{6}$};
	
 	\node[state, red] (else)[below of=loopEntry] {$\loc{10}$};
	\node[state] (loopExit)[right of=head] {$\loc{13}$};
	
	\node[state] (assertTrue)[below left of=loopExit] {$\loc{14}$};
	\node[state, accepting] (assertFalse)[below of=loopExit] {$\loc{err}$};
	
	\path (1) edge node []{\st{x:=0}\\ \st{y:=2}\\ \st{z:=3}} (head)
	
	(head) edge[red] node {\color{red}\st{x<=20}} (loopEntry)
	(head) edge node []{\st{x>20}} (loopExit)
	
	(loopEntry) edge[red] node []{\color{red}\st{x<=10}} (if)
	(if) edge[red]node []{\\ \color{red} \st{z:=x}\\ \color{red}\st{x:=x+y}\\\color{red} \st{y:=y+1}} (head)
	
	(loopEntry) edge[red]node []{\color{red}\st{x>10}} (else)
	(else) edge[red] node []{\color{red}\st{x:=x+2;}\\ \color{red}\st{y:=y-3}} (head)
	
	(loopExit) edge node []{\st{x==21}} (assertTrue)
	(loopExit) edge node []{\st{x!=21}} (assertFalse)
	;
\end{tikzpicture}

}
			\end{column}
	\end{columns}
\end{frame}

\begin{frame}[t]
	\frametitle{Control-Flow Graph Transformation}
	\begin{columns}
		\begin{column}{0.4\textwidth}
			\resizebox{0.75\textwidth}{!}{
\begin{tikzpicture}[%
	->,
	>=stealth',
	shorten >=1pt,
	auto,
	node distance=2cm,
	scale=0.75,
	transform shape,
	align=center,
	smallnode/.style={inner sep=1.4},
	initial text =,
	anchor=center]
	
	\tikzstyle{my node}=[draw,minimum height=1cm,minimum width=3cm]
	\tikzstyle{in node}=[draw,minimum height=1cm,minimum width=5cm]
	\node[align=center](input) {\textbf{Input}: \\Program $P$};
	\node[in node, fill=white](5)[below of=input, yshift=0.25cm]{Construct Control-Flow Graph $G$};
	\node[in node, fill=white](1)[below of=5]{Does $G$ contain a loop?};
	\node[my node, fill=white](2)[below of=1, xshift=1cm]{Loop \\ Summarization};
	\node[my node, fill=white](3)[below of=2]{Loop \\ Replacement};
	% \node[in node, fill=white](4)[below of=3, xshift=-1cm]{Control-Flow Graph \\ Construction};
	\node[](output)[below of=3, xshift=-2cm, yshift=0.25cm]{\textbf{Output}: \\ Loop-Free Control-Flow Graph};
	
	\begin{scope}[on background layer]
		\node[group, line width=0.3mm, draw=emblue, fill=emblue!20, fill opacity=0.2, fit=(5) (3), xscale=1.25, yscale=1.025](FIt2) {};
	\end{scope}
	
	\path (input) edge[] node[]{} (5)
	([xshift=-1cm] 1.south) edge[] node[left=0.1cm]{\textbf{no}} (output.north)
	;
	\draw [->] (5.south) -- node[right] {} (1.north);
	\draw [->] ([xshift=1cm]1.south) -- node[right] {\textbf{yes}} ([xshift=1cm]1.south |-2.north); 
	\draw [->] (2.south) -- (2.south |-3.north);
	\draw[->] (3) -- node[below right,pos=0.2]{} ++(2,0) |- (1);     
\end{tikzpicture}}
		\end{column}
		\begin{column}{0.4\textwidth}
			\resizebox{0.8\textwidth}{!}{
\begin{tikzpicture}[%
	->,
	>=stealth',
	shorten >=1pt,
	auto,
	node distance=3.25cm,
	scale=0.9,
	transform shape,
	align=center,
	smallnode/.style={inner sep=1.4},
	initial text =,
	anchor=center]
	
	\node[state, initial above, initial text =](1){$\loc{1}$};
	\node[state] (head) [below of=1] {$\loc{4}$};
	\node[state] (loopExit)[right of=head] {$\loc{13}$};
	
	\node[state] (assertTrue)[below left of=loopExit] {$\loc{14}$};
	\node[state, accepting] (assertFalse)[below of=loopExit] {$\loc{err}$};
	
	\path (1) edge node []{\st{x:=0;}\\ \st{y:=2;}\\ \st{z:=3;}} (head)
	(head) edge node []{\st{x>21}} (loopExit)
	
	(head) edge[loop left] node  {\st{$\psi$}} (head)
	(loopExit) edge node []{\st{x==22}} (assertTrue)
	(loopExit) edge node []{\st{x!=22}} (assertFalse)
	;
\end{tikzpicture}}
		\end{column}
	\end{columns}
\end{frame}

%%%%%%%%%%%%%%%%%%%%%%%%%%%%%%%%%%%%%%%%%%%%%%%%%%%%%%%%%%%%%%%%%%%%%%%%%%%%%% evaluation

\begin{frame}[t]
	\frametitle{Evaluation}
		We tested a subset of SVComp benchmarks containing 859 programs on specifications: \\
		\begin{itemize}
			\item Ultimate version: v0.2.2-dev-fb4f59a \\
			\item Timeout: 900s \\
			\item Memory Limit: 8000 MB\\
			\item CPU core limit: 2
		\end{itemize} \vspace*{0.25cm}
		On a system with specifications:
		\begin{itemize}
			\item OS: Linux-5.11.22-4-pve-x86\_64-with-glibc2.31 \\
			\item CPU: AMD Ryzen Threadripper 3970X 32-Core Processor \\
			\item Cores: 32, frequency: 4549 MHz, Turbo Boost: enabled \\
			\item RAM: 137439 MB
		\end{itemize} \vspace*{0.25cm}
	We compared five tools:
	\begin{itemize}
		\item Automizer Default \\
		\item Automizer with Jordan control-flow transformation (\color{emblue}CFG Jordan) \\
		\item Automizer with \qvasr\ control-flow transformation (\color{emblue}CFG \qvasr)\\
		\item Automizer with Jordan Accelerated Interpolation (\color{emblue}Accel Jordan) \\
		\item Automizer with \qvasr\ Accelerated Interpolation (\color{emblue}Accel \qvasr) \\
	\end{itemize}
\end{frame}

\begin{frame}[t]
	\frametitle{Evaluation}
	Of 859 programs, we archived the following results: \\ \vspace*{1cm}
	\resizebox{\textwidth}{!}{
	\begin{tabular}{c|cccccccc}
		Tool & \#Safe & \#Unsafe &\#Unknown & \#Timeout &\#Exceptions & \shortstack{Out of \\ Memory}& \shortstack{CPUTime \\ Total (s)} & \shortstack{Memory \\ Total (MB)}\\
		\hline
		Default & 372 & 128 & 43 & 273 & 33 & 10 & 297000 & 607000 \\
		CFG Jordan & 375 & 124 & 43 & 242 & 65& 9 & 268000 & 593000 \\
		CFG \qvasr & 372 & 127 & 43 & 267 & 39 & 10 & 292000 & 585000 \\
		Accel Jordan & 257 & 100 & 38 & 312 & 152 & 0 & 331000 & 522000\\
		Accel \qvasr & 319 & 115 & 43 & 346 & 34 & 2 & 365000 & 548000
	\end{tabular}
	}
\end{frame}

%%%%%%%%%%%%%%%%%%%%%%%%%%%%%%%%%%%%%%%%%%%%%%%%%%%%%%%%%%%%%%%%%%%%%%%%%%%%%% conclusion

\begin{frame}[t]
	\frametitle{Conclusion}
	\begin{center}
		\begin{itemize}
			\item Novel loop overapproximating summarization technique based on \qvasr \\
			\onslide<+->$\rightarrow$ extension to \qvasr-abstractions \vspace*{1cm}
			\onslide<+-> \item Used in the trace abstraction scheme:
				\begin{itemize}
				\onslide<+->	\item As control-flow graph preprocessor
				\onslide<+->	\item In proof computation by Accelerated Interpolation
				\end{itemize} \vspace*{1cm}
			\onslide<+->Evaluation shows promising results
		\end{itemize}
	\end{center}
\end{frame}

\begin{frame}
	\label{slide:bibliography}
	\frametitle{Bibliography}
	\fontsize{10}{10}\selectfont
	\bibliographystyle{apalike}
	\nocite{DBLP:conf/cav/SilvermanK19} \nocite{DBLP:conf/cav/SilvermanK19}
	\bibliography{bib/bib}
\end{frame}

\begin{comment}
\begin{frame}{Bibliography}
	\frametitle{References}
	\begin{thebibliography}{99} % Beamer does not support BibTeX so references must be inserted manually as below
	\bibitem[DBLP:conf/rp/HaaseH14]{p1} Christoph Haase and	Simon Halfon
	\newblock Integer Vector Addition Systems with States
	\newblock \href{https://doi.org/10.1007/978-3-319-11439-2\_9}{Lecture Notes in Computer Science, 2014}
	
	\bibitem[DBLP:journals/fmsd/KroeningLW15]{p2} Daniel Kroening and
	Matt Lewis and Georg Weissenbacher
	\newblock Under-approximating Loops in C Programs for Fast Counterexample Detection
	\newblock \href{https://doi.org/10.1007/s10703-015-0228-1}{Reachability Problems - 8th International Workshop, 2014}
	
	\bibitem[DBLP:journals/fmsd/KroeningSTTW13]{p3} Daniel Kroening and
	Natasha Sharygina and Stefano Tonetta and Aliaksei Tsitovich and Christoph M. Wintersteiger
	\newblock Loop Summarization using State and Transition Invariants
	\newblock \href{https://doi.org/10.1007/s10703-012-0176-y}{Formal Methods Syst. Des., 2013}
	
	\bibitem[DBLP:conf/cav/SilvermanK19]{p4} Jake Silverman and Zachary Kincaid
	\newblock Loop Summarization with Rational Vector Addition Systems
	\newblock \href{https://doi.org/10.1007/978-3-030-25543-5\_7}{CAV, 2019}
	
	\bibitem[DBLP:journals/tse/XieCZLLL19]{p5} Xiaofei Xie and
	Bihuan Chen and	Liang Zou and Yang Liu and Wei Le and Xiaohong Li
	\newblock Automatic Loop Summarization via Path Dependency Analysis
	\newblock \href{https://doi.org/10.1109/TSE.2017.2788018}{{IEEE} Trans. Software Eng., 2019}

	\end{thebibliography}

\end{frame}
\end{comment}

\begin{frame}[plain,c]
	%\frametitle{A first slide}
	
	\begin{center}
		\Huge \color{emblue}Appendix
	\end{center}
	
\end{frame}

%%%%%%%%%%%%%%%%%%%%%%%%%%%%%%%%%%%%%%%%%%%%%%%%%%%%%%%%%%%%%%%%%%%%%%%%%%%%%% 
%Todo: Less detailed -> this one can be used as a backup slide

\begin{frame}
	\frametitle{\qvasr-Abstraction}
	\newcommand{\s}{\ensuremath{\begin{bmatrix} s_x & s_y & s_z \end{bmatrix}}}
	\newcommand{\p}{\ensuremath{\begin{bmatrix} x' \\ y' \\ z' \end{bmatrix}}}
	\newcommand{\up}{\ensuremath{\begin{bmatrix} x \\ y \\ z \end{bmatrix}}}
	\begin{definition}
		Given a transition formula $F$, a \qvasr-abstraction for $F$ is a tuple $(S_F, V_F)$ consisting of a matrix $S_F$ and a \qvasr\ $V_F$. The matrix $S_F$ acts as a linear transformation of $F$ to $V_F$. Meaning, for every transition \\
		$
		\begin{bmatrix}
			x_1 \\
			\vdots \\
			x_n
		\end{bmatrix}
		\rightarrow_F
		\begin{bmatrix}
			x_1' \\
			\vdots \\
			x_n'
		\end{bmatrix}
		$
		in $F$, there is: \ \
		$ S_F \cdot
		\begin{bmatrix}
			x_1' \\
			\vdots \\
			x_n'
		\end{bmatrix}
		=
		S_F \cdot
		\begin{bmatrix}
			x_1 \\
			\vdots \\
			x_n
		\end{bmatrix}
		*
		%%%%%%%%%%%%%%%%%%%% RESET
		\only<3-5>{
			\begin{bmatrix}
				\color{red}{0} \\
				\color{red}{\vdots} \\
				\color{red}{0}
			\end{bmatrix}	
		}
		%%%%%%%%%%%%%%%%%%%%% VANILLA
		\only<1-2>{
			\begin{bmatrix}
				r_1 \\
				\vdots \\
				r_n
			\end{bmatrix}
		}
		%%%%%%%%%%%%%%%%%%%%%% NOT RESET
		\only<6->{
			\begin{bmatrix}
				\color{red}{1} \\
				\color{red}{\vdots} \\
				\color{red}{1}
			\end{bmatrix}
		}
		+
		\begin{bmatrix}
			a_1 \\
			\vdots \\
			a_n
		\end{bmatrix}	
		$
		in $V_F$
	\end{definition}
	\pause
	\begin{center}
		$H: x \leq 10 \land x' = x + y \land y' = y + 1 \land z' = x$ \pause
		\only<2-5>{
			\begin{equation*}
				Res_H = \left\{ (\s, a) : H \models \underbrace{\s}_{solve} \cdot \p = \underbrace{a}_{solve} \right\} 	
			\end{equation*} \pause
			$Res_H = \left\{ (\begin{bmatrix} -a & a & a \end{bmatrix}, a) \right\}\ \pause \xrightarrow{\text{Base}}\ Res_H = \left\{ (\begin{bmatrix} -1 & 1 & 1 \end{bmatrix}, 1) \right\}\ $
		}
		\only<7->{
			\resizebox{0.9\textwidth}{!}{
				$Inc_H = \left\{(\s, a) : H \models \s \cdot \p = \s \cdot \up + a\right\}$
			}
			\uncover<8->{
				$Inc_H = \left\{ (\begin{bmatrix} 0 & a & 0 \end{bmatrix}, a) \right\}\ $} 
			\uncover<9->
			{$\xrightarrow{\text{Base}}\ Inc_H = \left\{ (\begin{bmatrix} 0 & 1 & 0 \end{bmatrix}, 1) \right\}\ $}
		}
	\end{center}
\end{frame}

%%%%%%%%%%%%%%%%%%%%%%%%%%%%%%%%%%%%%%%%%%%%%%%%%%%%%%%%%%%%%%%%%%%%%%%%%%%%%%  

\begin{frame}[t]
	\frametitle{\qvasr-Abstraction}
	\begin{center}
		$H: x \leq 10 \land x' = x + y \land y' = y + 1 \land z' = x$ \pause
	\end{center}
	\vspace*{0.5cm}
	\begin{columns}
		\begin{column}{0.5\textwidth}
			\begin{equation*} \only<2-4, 6->{
					Res_H = \left\{ (\begin{bmatrix} {\color<3>{red}{-1}} & {\color<3>{red}{1}} & {\color<3>{red}{1}} \end{bmatrix}, {\color<4>{red}{1}}) \right\}\ 
				}
				\only<5>{
					Res_H = \underbrace{\left\{ (\begin{bmatrix} -1 & 1 & 1 \end{bmatrix}, 1) \right\}\ }_{\color{red}{1}\ reset}
				}
			\end{equation*}
		\end{column}
		\begin{column}{0.5\textwidth}
			\begin{equation*}
				\only<2-5, 7>{
					Inc_H = \left\{ (\begin{bmatrix} {\color<3>{red}{0}} & {\color<3>{red}{1}} & {\color<3>{red}{0}} \end{bmatrix}, {\color<4>{red}{1}}) \right\}\
				}
				\only<6>{
					Inc_H = \underbrace{\left\{ (\begin{bmatrix} 0 & 1 & 0 \end{bmatrix}, 1) \right\}\}}_{\color{red}{1}\ addition}
				}
			\end{equation*}
		\end{column}
	\end{columns}
	\only<2->{
		\begin{figure}
			\begin{equation*}
				S_H = \begin{bmatrix} {\color<3>{red}{-1}} & {\color<3>{red}{1}} & {\color<3>{red}{1}} \\ {\color<3>{red}{0}} & {\color<3>{red}{1}} & {\color<3>{red}{0}} \end{bmatrix} \hspace*{0.5cm}
				V_H = 
				\begin{Bmatrix}
					\begin{pmatrix}
						\begin{bmatrix}
							\color<5>{red}{0} \\
							\color<6>{red}{1}
						\end{bmatrix},
						\begin{bmatrix}
							{\color<4>{red}{1}} \\
							{\color<4>{red}{1}}
						\end{bmatrix}
					\end{pmatrix}
				\end{Bmatrix}
			\end{equation*}
			\caption*{\qvasr-abstraction of $H$}
		\end{figure}
	} 
	\only<7>{
		\begin{figure}
			\begin{equation*}
				V_H = \begin{Bmatrix} \begin{bmatrix} - x' + y' + z' = 1 \\ y' = y + 1 \end{bmatrix} \end{Bmatrix}
			\end{equation*}
			\caption*{Intuitive view of \qvasr-abstraction} 
		\end{figure}
	}
\end{frame}

%%%%%%%%%%%%%%%%%%%%%%%%%%%%%%%%%%%%%%%%%%%%%%%%%%%%%%%%%%%%%%%%%%%%%%%%%%%%%% 

\begin{frame}[t]
	\frametitle{\qvasr-Abstraction Partial Order}
	\begin{columns}
		\begin{column}{0.4\textwidth}
			\begin{figure}[h]
				\vspace*{0.5cm}
				\resizebox{0.6\textwidth}{!}{\begin{lstlisting}[language=C++, style=withAssert]  % Start your code-block
	
	int x := 0;
	int y := 2;
	int z := 3;
	while x <= 20:
		if x <= 10:
			z := x;
			x := x + y;
			y := y + 1;
		else:
			x := x + 2;
			y := y - 3;
	assert x == 21;
	\end{lstlisting}}
				\vspace{-0.5cm}
				\caption*{Program containing a loop.}
			\end{figure}
		\end{column} \pause
		\begin{column}{0.6\textwidth}
			\begin{itemize}
				\item The transition formula of the loop: 
				\begin{equation*}
					F = H \lor G
				\end{equation*}
				\pause
				\item We need a \qvasr-abstraction $(\tilde{S}, \tilde{V})$ that simulates both $(S_H, V_H)$ and $(S_G, V_G)$ 
				\pause
				\begin{equation*}
					\rightarrow T_HS_H = \tilde{S} = T_GS_G 
				\end{equation*}
				\pause
				\begin{equation*}
					\text{Solve:}\ \tilde{S} = \begin{bmatrix} t^{H}_1 & t^{H}_2  \end{bmatrix} S_H  = \begin{bmatrix} t^{G}_1 & t^{G}_2\end{bmatrix} S_G
				\end{equation*}
			\end{itemize}
		\end{column}
	\end{columns}
	\pause
	\vspace{0.5cm}
	\begin{itemize}
		\item Compute new changes on relations as $\tilde{V}$
		\begin{equation*}
			\tilde{V} = \{(T_H \times \vec{r}_H, T\vec{a}_H) \cup (T_G\times \vec{r}_G, T_G\vec{a}_G)\}
		\end{equation*}
		Where $(T \times \vec{r})_i = r_j$ is a translation of $r_i$ to a non-zero $r_j \in T$
	\end{itemize}
\end{frame}


%%%%%%%%%%%%%%%%%%%%%%%%%%%%%%%%%%%%%%%%%%%%%%%%%%%%%%%%%%%%%%%%%%%%%%%%%%%%%% 

\begin{frame}[t]
	\frametitle{\qvasrs}
	To improve precision, we constrain \qvasr\ transitions by using a set of predicates $P$ as states. We introduce a so called \qvasrs
	\begin{definition}
		Given a \qvasr-abstraction $(S,V)$, a \qvasrs\ $\mathcal{V}$ is a nondeterministic finite automaton $(Q, E)$ consisting of a set of states $Q$ and a set of edges $E \subseteq Q \times V \times Q$, \\
		where each transition is labeled by a transition in $V$ \\ \vspace*{0.25cm}
		$\mathcal{V}$ can transition $(q_1, \vec{x}) \rightarrow_\mathcal{V} (q_2, \vec{x}')$ if there is some edge $(q_1, (\vec{r}, \vec{a}), q_2)$
	\end{definition}
	\begin{center}
		We use $P_F = \{ x \leq 10, 10 < x \land x \leq 20\}:$
	\end{center} \vspace{0.5cm}
	
    \centering
    \begin{tikzpicture}[%
    ->,
    >=stealth',
    shorten >=1pt,
    auto,
    node distance=9cm,
    scale=0.9,
    transform shape,
    align=center,
    smallnode/.style={inner sep=1.2},
    initial text =,
    anchor=center]

    	\node[draw, ellipse, initial left, initial text =](1){$x \leq 10$};
    	\node[draw, ellipse](2) [right of=1] {$10 < x \land x \leq 20$};
    	% \node[draw, ellipse, accepting] (3) [right of=2] {$x > 20$};

    	\path (1) edge node [below]{ $
					\begin{bmatrix}
						-x' + y + z = - x + y + z - 5 \\
						y' = y - 3
					\end{bmatrix}
                     $ } (2)
	
	    	 (1) edge[loop above] node { $
					\begin{bmatrix}
						-x' + y' + z' = - x + y + z - 5 \\
					 	y' = y - 3
					\end{bmatrix}
	         $ } (1)
	
	    	 (2) edge[loop above] node { $
	               \begin{bmatrix}
						-x' + y' + z' = 1 \\
						y' = y + 1
					\end{bmatrix}
	                     $ } (2)
	    	;
    \end{tikzpicture}
    %\caption{\qvasrs abstraction $\mathcal{A}$ of program $P$.}
\end{frame}

\begin{frame}
	
    \centering
    \begin{tikzpicture}[%
    ->,
    >=stealth',
    shorten >=1pt,
    auto,
    node distance=9cm,
    scale=0.9,
    transform shape,
    align=center,
    smallnode/.style={inner sep=1.2},
    initial text =,
    anchor=center]

    	\node[draw, ellipse, initial left, initial text =](1){$x \leq 10$};
    	\node[draw, ellipse](2) [right of=1] {$10 < x \land x \leq 20$};
    	% \node[draw, ellipse, accepting] (3) [right of=2] {$x > 20$};

    	\path (1) edge node [below]{ $
					\begin{bmatrix}
						-x' + y + z = - x + y + z - 5 \\
						y' = y - 3
					\end{bmatrix}
                     $ } (2)
	
	    	 (1) edge[loop above] node { $
					\begin{bmatrix}
						-x' + y' + z' = - x + y + z - 5 \\
					 	y' = y - 3
					\end{bmatrix}
	         $ } (1)
	
	    	 (2) edge[loop above] node { $
	               \begin{bmatrix}
						-x' + y' + z' = 1 \\
						y' = y + 1
					\end{bmatrix}
	                     $ } (2)
	    	;
    \end{tikzpicture}
    %\caption{\qvasrs abstraction $\mathcal{A}$ of program $P$.} \\
	Using this \qvasrs, we can now compute a reachability relation $reach(S_F, P_F)$ in polytime \cite{DBLP:conf/rp/HaaseH14}
\end{frame}

%%%%%%%%%%%%%%%%%%%%%%%%%%%%%%%%%%%%%%%%%%%%%%%%%%%%%%%%%%%%%%%%%%%%%%%%%%%%%% 

\begin{frame}[t]
	\frametitle{Approach}
	\resizebox{11cm}{!}{% Define block styles
\tikzstyle{block} = [rectangle, draw, rounded corners, minimum height=5em, minimum width={width("predicate transformer")+15pt},
]
\tikzstyle{group} = [rectangle, draw, rounded corners, minimum height=5em
]
\begin{tikzpicture}[%
    ->,
	>=stealth',
	shorten >=1pt,
	auto,
	node distance=3.25cm and 4.5cm,
	scale=0.9,
	transform shape,
	align=center,
	smallnode/.style={inner sep=1.4},
	initial text =,
	anchor=center]
	% Place nodes
	\node [align=left](input) {\textbf{Input}};
	\node [block, align=left, below=of input](predtransformer) {\texttt{predicate transformer}};
	\node [block, align=center, right=of predtransformer](vasrs) {Are there $p, q \in P$ with no \\ \qvasr-abstraction \\ \pqvasr $\in \mathcal{A}$?};
	\node [block, align=left, right=of vasrs](reach) {\texttt{reach}};
	\node [block, align=left, below=of vasrs, yshift=1cm](H) {\pqformula satisfiable?};
	\node [block, align=center, right=of H](abstr) {\texttt{\qvasr-abstractor}};
	\node [block, align=left, below=of abstr, fill=white, yshift=1cm](pushout) {\texttt{pushout}};
	\node [block, align=left, below=of H, fill=white, yshift=1cm](image) {\texttt{image-builder}};
	\node [align=left, above= of reach](output) {\textbf{Output}};
	\begin{scope}[on background layer]
		\node[group, draw=black,fill=stmtcolor,fit=(pushout) (image), label=below :{\texttt{\qvasr-join}}](FIt1) {};
	\end{scope}
	
	% Draw edges
	\draw (input) to node[above] {} node[above right] {$F :=$ loop's transition formula \\ with $n$ variables} (predtransformer);
	%\draw (predtransformer) to node[right] {} node[right, align=left] {$F$\\ $P$ := set of predicates \\ $\mathcal{V} := \emptyset$} (vasrs);
	\draw (reach) to node[above] {} node[left] {\texttt{reach($\mathcal{V}$)} is \\ loop summary of $F$} (output);
	\draw [bend right]  (vasrs) to node[above left, align=right, yshift=-0.5cm] {\textbf{yes} \\ $\pqformula := p \land F \land q$ \\ $\pqvasr := (\mathit{I_n},\emptyset)$} node[right, align=left] {} (H);
	\draw (H) to node[above, xshift=-1.5cm] {\textbf{yes}} node[below] {$c :=$ cube in $\mathit{DNF(\pqformula)}$} (abstr);
	\draw (abstr) to node[above] {} node[left] {$(S_c, V_c) :=$ \\ \texttt{abstraction(c)}} (pushout);
	\draw (pushout) to node[below] {$\pq{S}$ simulates $(S_c, V_c)$} node[above] {$\pq{S}$ := \texttt{pushout($\pq{S}, S_c$)} } (image);
	
	%\draw (image) to node[above] {} node[left, align=left] {$\pqformula :=$ refine$(\pqformula)$ \\ $\pq{V} :=$ \texttt{image(\pq{V})} \\ $\cup$ \texttt{image$(V_c)$}} (H);
	
		\draw (image) to node[above] {} node[right, align=left] {%
		{\begin{varwidth}{3.5cm}
				\vspace*{-0.5cm}
				\begin{align*}
					\pqformula :=&\ \texttt{refine}(\pqformula) \\ \bigskip
					\pq{V} :=&\ \texttt{image}(\pq{V}) \\
					&\cup \texttt{image}(V_c)
		\end{align*}\end{varwidth}}
	} (H);
	
	
	\draw (predtransformer) to node[above] {} node[above, align=left] {%
		{\begin{varwidth}{3.75cm}
			\begin{align*}
				F \\
				P &:= \text{set of predicates} \\
				\mathcal{A} &:= \emptyset
			\end{align*}\end{varwidth}}
		} (vasrs);

	\draw [bend right]  (H) to node[above] {} node[right, align=left, yshift=-0.25cm] {$\mathcal{A} := \mathcal{A} \cup \pqvasr$ \vspace{0.5cm} \\ \pqvasr is best abstraction\\ \textbf{no}} (vasrs);
	\draw (vasrs) to node[below] {$\mathcal{V} := \qvasrs(\mathcal{A}, P)$} node[above, xshift=-1.5cm] {\textbf{no}} (reach);
\end{tikzpicture}}
\end{frame}

\end{document}
