A master's thesis consists of six months of work. This chapter will give an outline of the approach of how we plan to implement a \qvasr and \qvasrs loop acceleration library in the software-verification framework Ultimate, use the library with accelerated interpolation, evaluate its performance, and writing a thesis, in this given time frame. \\ \par
We divide the project into five distinct milestones:

\begin{itemize}
	\item[1.] \textsl{Implementing a \qvasr library:} \\
               Implement algorithms 1 and 2, \qvasr abstraction computation, \qvasr image calculation, \texttt{pushout}, and needed auxiliary classes for matrix and vector operations into a new loop acceleration library in Ultimate.

			  \textsl{Duration:} 2 months \\
			  \textsl{Outcome:} An implemented library capable of computing \qvasr

	\item[2.] \textsl{Implementing \qvasrs functionality:} \\
               Implementing algorithm 3 and reachability relation computation \texttt{reach} in the \qvasr library.

			  \textsl{Duration:} 1 month \\
			  \textsl{Outcome:} An implemented library capable of computing \qvasrs, \texttt{reach}, and returning the reflexive transitive closure of a loop.

	\item[3.] \textsl{Testing the library:} \\
               Test the library to make sure it works as intended, fix any occurring bugs, and implement its usage in the \texttt{accelerated interpolation} scheme.

			  \textsl{Duration:} 1 months \\
			  \textsl{Outcome:} A tested and bugfree \qvasr library that can be used in the \texttt{accelerated interpolation} scheme.

	\item[4.] \textsl{Evaluating the library's performance:} \\
               Run the library on various benchmarks, log its performance, and compare it to other techniques.

			  \textsl{Duration:} 1 month \\
			  \textsl{Outcome:} Data of the performance of the \qvasr library.

	\item[5.] \textsl{Writing a thesis:} \\
              Writing a thesis on the findings of this project, proofreading it, and printing it.

			  \textsl{Duration:} 2 months \\
			  \textsl{Outcome:}	A master's thesis
\end{itemize}
