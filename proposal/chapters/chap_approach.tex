
The proposed \qvasrs loop summarization library shall be implemented as envisioned in Figure \ref{fig}: As input it is given a transition formula, computed from a loop by other tools, and hands it over to the predicate transformer. The predicate transformer computes a set of predicates $P$ that fulfill the condition of pairwise unsatisfiability and initializes an empty \qvasr-abstraction set $\mathcal{A}$. The predicate transformer can employ different techniques of predicate generation making it possible to test the effectiveness and speed of, for example, strong versus weak predicates. While there are pairs of predicates $p, q \in P$ that do not have a corresponding \qvasr-abstraction $\pqvasr \in \mathcal{A}$, we construct a new formula $\pqformula = p \land F \land q$ and begin to compute \pqvasr. If a model $\mathcal{M}$ satisfies $\pqformula$, we take cube $c$ from the disjunctive normal form of $\pqformula$ that is satisfied by $\mathcal{M}$ and compute a \qvasr-abstraction $(S_c, V_c)$ for it in the \texttt{\qvasr-abstractor}. Because we want $\pqvasr$ to be the least upper bound of $\pqformula$, $(S_c, V_c) \preceq \pqvasr$ has to hold. This is done by \textsl{joining} $\pqvasr$ and $(S_c, V_c)$. In \texttt{pushout} we compute a linear simulation from $S_c$ to $\pq{S}$ and update $\pq{S}$ accordingly. In the \texttt{image-builder} we calculate and incorporate the changes to variables done by $V_c$ into $\pq{V}$. After \texttt{pushout} and \texttt{image} we know that \pqvasr is a least upper bound for $(S_c, V_c)$. Because we do not need another \qvasr-abstraction for $c$, we update $\pqformula$ to exclude the cube from it. If $\pqformula$ is still satisfiable, we begin anew.
If \pqformula is unsatisfiable, then $\pqvasr$ is a least upper bound for \pqformula, we return and add it to $\mathcal{A}$. When every predicate pair $p, q \in P$ has a \qvasr-abstraction $\pqvasr \in \mathcal{A}$, construct edges and transitions building the \qvasrs $\mathcal{V}$. In \texttt{reach} we compute the reachability relation of $\mathcal{V}$ which is then used to form a loop summary that is finally returned to \textsc{Ultimate}.
\begin{figure}[H]
    % Define block styles
\tikzstyle{block} = [rectangle, draw, rounded corners, minimum height=4em, minimum width={width("predicate transformer")+15pt},
]
\tikzstyle{group} = [rectangle, draw, rounded corners, minimum height=5em
]
\begin{tikzpicture}[%
    ->,
	>=stealth',
	shorten >=1pt,
	auto,
	node distance=3.5cm and 5cm,
	scale=1,
	transform shape,
	align=center,
	smallnode/.style={inner sep=1.4},
	initial text =,
	anchor=center]
	% Place nodes
	\node [align=left](input) {\textbf{Input}};
	\node [block, align=left, below=of input, yshift=2cm](predtransformer) {\texttt{predicate transformer}};
	\node [block, align=left, below=of predtransformer, yshift=1.5cm](vasrs) {Are there $p, q \in P$ \\ with $\pqvasr \not\in \mathcal{V}$?};
	\node [block, align=left, right=of vasrs](reach) {\texttt{reach}};
	\node [block, align=left, below=of vasrs, yshift=1cm](H) {\pqformula satisfiable?};
	\node [block, align=center, right=of H](abstr) {\texttt{\qvasr-abstractor}};
	\node [block, align=left, below=of abstr, fill=white, yshift=1cm](pushout) {\texttt{pushout}};
	\node [block, align=left, below=of H, fill=white, yshift=1cm](image) {\texttt{image-builder}};
	\node [align=left, above= of reach](output) {\textbf{Output}};
	\begin{scope}[on background layer]
		\node[group, draw=black,fill=stmtcolor,fit=(pushout) (image), label=below :{\texttt{\qvasr-join}}](FIt1) {};
	\end{scope}
	
	% Draw edges
	\draw (input) to node[above] {} node[right] {$F :=$ loop's transition formula \\ with $n$ variables} (predtransformer);
	%\draw (predtransformer) to node[right] {} node[right, align=left] {$F$\\ $P$ := set of predicates \\ $\mathcal{V} := \emptyset$} (vasrs);
	\draw (reach) to node[above] {} node[left] {\texttt{reach($\mathcal{A}$)} := \\ overapproximation \\ of $F's$ reflexive \\ transitive closure} (output);
	\draw [bend right]  (vasrs) to node[above left, align=right, yshift=-0.5cm] {\textbf{yes} \\ $\pqformula := p \land F \land q$ \\ $\pqvasr := (\mathit{I_n},\emptyset)$} node[right, align=left] {} (H);
	\draw (H) to node[above, xshift=-1.5cm] {\textbf{yes}} node[below] {$c :=$ cube in $\mathit{DNF(\pqformula)}$} (abstr);
	\draw (abstr) to node[above] {} node[left] {$(S_c, V_c) :=$ \\ \texttt{abstraction(c)}} (pushout);
	\draw (pushout) to node[above] {} node[above] {$\pq{S} :=$ \texttt{pushout($\pq{S}, S_c$)}} (image);
	
	%\draw (image) to node[above] {} node[left, align=left] {$\pqformula :=$ refine$(\pqformula)$ \\ $\pq{V} :=$ \texttt{image(\pq{V})} \\ $\cup$ \texttt{image$(V_c)$}} (H);
	
		\draw (image) to node[above] {} node[right, align=left] {%
		{\begin{varwidth}{3.5cm}
				\vspace*{-0.5cm}
				\begin{align*}
					\pqformula :=&\ \texttt{refine}(\pqformula) \\ \bigskip
					\pq{V} :=&\ \texttt{image}(\pq{V}) \\
					&\cup \texttt{image}(V_c)
		\end{align*}\end{varwidth}}
	} (H);
	
	
	\draw (predtransformer) to node[above] {} node[right, align=left] {%
		{\begin{varwidth}{3.5cm}
			\vspace*{-0.5cm}
			\begin{align*}
				F \\
				P &:= \text{set of predicates} \\
				\mathcal{V} &:= \emptyset
			\end{align*}\end{varwidth}}
		} (vasrs);

	\begin{comment}
		\draw[] (vasrs) to node[above] {} node[left, align=right] {% 
	{\begin{varwidth}{4cm}
	\begin{align*}
	\textbf{yes} \\
	\pqformula &:=\ p \land F \land q \\
	\pq{S} &:= I_n \\
	\pq{V} &:= \emptyset
	\end{align*}\end{varwidth}}
	} (H);
	\end{comment}

	\draw [bend right]  (H) to node[above] {} node[below right, align=left] {$\mathcal{V} := \mathcal{V} \cup \pqvasr$ \\ \textbf{no}} (vasrs);
	\draw (vasrs) to node[below] {$\mathcal{A} := \qvasrs(\mathcal{V})$} node[above, xshift=-1.5cm] {\textbf{no}} (reach);
\end{tikzpicture}
    \caption{Proposed implementation of a \qvasrs-based loop summarizer.}
    \label{fig}
\end{figure}

A master's thesis consists of six months of work. In the following we will give an outline of the approach of how we plan to implement a \qvasr and \qvasrs loop acceleration library in the software-verification framework \textsc{Ultimate}, use the library with \texttt{accelerated interpolation}, evaluate its performance, and writing a thesis, in this given time frame. \\ \par
We divide the project into five distinct milestones:

\begin{itemize}
	\item[1.] \textsl{Implementing a \qvasr library:} \\
               Implementation of \qvasr abstraction computation, \qvasr image calculation, \texttt{pushout}, and needed auxiliary classes for matrix and vector operations into a new loop acceleration library in Ultimate.

			  \textsl{Duration:} 2 months \\
			  \textsl{Deliverable:} An implemented library capable of computing \qvasr

	\item[2.] \textsl{Implementing \qvasrs functionality:} \\
               Implementing \qvasrs functionality and reachability relation computation \texttt{reach} in the \qvasr library.

			  \textsl{Duration:} 1 month \\
			  \textsl{Deliverable:} An implemented library capable of computing \qvasrs, \texttt{reach}, and returning the reflexive transitive closure of a loop.

	\item[3.] \textsl{Testing the library:} \\
               Test the library to make sure it works as intended, fix any occurring bugs, and implement its usage in the \texttt{accelerated interpolation} scheme.

			  \textsl{Duration:} 1 months \\
			  \textsl{Deliverable:} A tested and bugfree \qvasr library that can be used in the \texttt{accelerated interpolation} scheme.

	\item[4.] \textsl{Evaluating the library's performance:} \\
               Run the library on various benchmarks, log its performance, and compare it to other techniques.

			  \textsl{Duration:} 1 month \\
			  \textsl{Deliverable:} Data of the performance of the \qvasr library.

	\item[5.] \textsl{Writing a thesis:} \\
              Writing a thesis on the findings of this project, proofreading it, and printing it.

			  \textsl{Duration:} 2 months \\
			  \textsl{Deliverable:}	A master's thesis
\end{itemize}
