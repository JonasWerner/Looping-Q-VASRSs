
The proposed \qvasrs loop summarization library shall be implemented as envisioned in Figure \ref{fig}: The input consists of a loop's transition formula that was computed in advance by other tools. This transition formula is firstly used in the \texttt{predicate transformer} to compute a set of predicates $P$ that fulfill the condition of pairwise unsatisfiability. The \texttt{predicate transformer} will be able to employ different techniques of predicate generation making it possible to test the effectiveness and speed of, for example, strong versus weak predicates. Following the predicate generation, an empty \qvasr-abstraction set $\mathcal{A}$ is initialized. For each pair of predicates $p, q \in P$ we need to compute a \qvasr-abstraction $\pqvasr$. This is done using the formula $\pqformula = p \land F \land q$, which is checked for satisfiability. If a model $\mathcal{M}$ satisfies $\pqformula$, we take cube $c$ from the disjunctive normal form of $\pqformula$ that is satisfied by $\mathcal{M}$ and use the \texttt{\qvasr-abstractor} to compute a \qvasr-abstraction $(S_c, V_c)$ for it. Because we want $\pqvasr$ to be the least upper bound of $\pqformula$, $(S_c, V_c) \preceq \pqvasr$ has to hold. To ensure this, we need to \text{join} the two \qvasr-abstractions $\pqvasr$ and $(S_c, V_c)$. The \texttt{join} consists of two operations, firstly, in \texttt{pushout}, we compute a linear simulation from $S_c$ to $\pq{S}$ and update $\pq{S}$ accordingly. Secondly, in the \texttt{image-builder}, we calculate and incorporate the changes to variables done by $V_c$ into $\pq{V}$. After \texttt{pushout} and \texttt{image} we know that \pqvasr simulates $(S_c, V_c)$ and with that $(S_c, V_c) \preceq \pqvasr$ holds. We do not need another \qvasr-abstraction for $c$, so that we update $\pqformula$ to exclude the cube from it. If $\pqformula$ is still satisfiable, we begin anew.
If \pqformula is unsatisfiable, then $\pqvasr$ is the least upper bound for \pqformula, we return and add it to $\mathcal{A}$. When every predicate pair $p, q \in P$ has a \qvasr-abstraction $\pqvasr \in \mathcal{A}$, construct a \qvasrs $\mathcal{V}$ using the predicates and \qvasr-abstractions in $\mathcal{A}$. In \texttt{reach} we compute the reachability relation of $\mathcal{V}$ which is then used to form a loop summary that is finally returned to \textsc{Ultimate}.
\begin{figure}[H]
    % Define block styles
\tikzstyle{block} = [rectangle, draw, rounded corners, minimum height=4em, minimum width={width("\qvasr-image-builder")+10pt},
]
\tikzstyle{group} = [rectangle, draw, rounded corners, minimum height=5em
]
\tikzstyle{line} = [draw, -latex']

\newcommand{\pq}[1]{\ensuremath{#1_{p, q}}}
\newcommand{\pqvasr}{\ensuremath{(\pq{S}, \pq{V})}}
\newcommand{\pqformula}{\ensuremath{\pq{\Gamma}}}

\begin{tikzpicture}[%
	->,
	>=stealth',
	shorten >=1pt,
	auto,
	node distance=5cm,
	scale=1,
	transform shape,
	align=center,
	smallnode/.style={inner sep=1.4},
	initial text =,
	anchor=center]
	% Place nodes
	\node [align=left, yshift=-1.5cm](input) {\textbf{Input}};
	\node [block, align=left, below of=input, yshift=1.5cm](predtransformer) {Predicate transformer};
	\node [align=left, right of=predtransformer, xshift=3cm](output) {\textbf{Output}: \\
		loop summary};
	\node [block, align=left, below of=predtransformer](vasrs) {Are there $p, q \in P$ \\ with $\pqvasr \not\in \mathcal{A}$?};
	\node [block, align=left, right of=vasrs, xshift=3cm](reach) {\texttt{reach}};
	\node [block, align=left, below of=vasrs, yshift=-1cm](H) {Is \pqformula satisfiable?};
	\node [block, align=center, right of=H, xshift=3cm](abstr) {\texttt{\qvasr-abstractor}};
	\node [block, align=left, below of=abstr, fill=white](pushout) {\texttt{pushout}};
	\node [block, align=left, below of=H, fill=white](image) {\texttt{image-builder}};
	
	\begin{scope}[on background layer]
		\node[group, draw=black,fill=stmtcolor,fit=(pushout) (image), label=above :{\texttt{\qvasr-join}}](FIt1) {};
	\end{scope}
	
	% Draw edges
	\draw [->]  (input) to node[above] {} node[right] {$F :=$ transition formula \\ of a loop with \\ $n$ variables} (predtransformer);
	\draw [->]  (predtransformer) to node[right] {} node[right, align=left] {$F$\\ $P$ := set of predicates \\ $\mathcal{A} := \emptyset$} (vasrs);
	\draw [->]  (reach) to node[above] {} node[above] {} (output);
	\draw [->, bend right]  (vasrs) to node[left] {yes \\ $\pqformula := p \land F \land q$ \\ \\ $\pq{S} := \mathit{I_n}$ \\
	$\pq{V} := \emptyset$} node[right, align=left] {} (H);
	\draw [->]  (H) to node[above] {yes} node[below] {$c :=$ cube in \\ $\mathit{DNF(\pqformula)}$} (abstr);
	\draw [->]  (abstr) to node[above] {} node[left] {$(S_c, V_c) :=$ abstraction(c)} (pushout);
	\draw [->]  (pushout) to node[above] {} node[above] {$\pq{S} :=$ \texttt{pushout($\pq{S}, S_c$)}} (image);
	\draw [->]  (image) to node[above] {} node[left, align=left] {$\pqformula :=$ refine$(\pqformula)$ \\ $\pq{V} :=$ \texttt{image(\pq{V})} $\cup$ \texttt{image$(V_c)$}} (H);
	\draw [->, bend right]  (H) to node[above] {} node[right, align=left] {no \\
	$\pqvasr$ is best \qvasr-abstraction for $\pqformula$ \\ $\mathcal{A} := \mathcal{A} \cup \pqvasr$} (vasrs);
	\draw [->]  (vasrs) to node[above] {} node[above] {yes} (reach);
\end{tikzpicture}
    \caption{Proposed implementation of a \qvasrs-based loop summarizer.}
    \label{fig}
\end{figure}

A master's thesis consists of six months of work. In the following we will give an outline of the approach of how we plan to implement a \qvasr and \qvasrs loop acceleration library in the software-verification framework \textsc{Ultimate}, use the library with \texttt{accelerated interpolation}, evaluate its performance, and writing a thesis, in this given time frame. \\ \par
We divide the project into five distinct milestones:

\begin{itemize}
	\item[1.] \textsl{Implementing a \qvasr library:} \\
               Implementation of \qvasr abstraction computation, \qvasr image calculation, \texttt{pushout}, and needed auxiliary classes for matrix and vector operations into a new loop acceleration library in Ultimate.

			  \textsl{Duration:} 2 months \\
			  \textsl{Deliverable:} An implemented library capable of computing \qvasr

	\item[2.] \textsl{Implementing \qvasrs functionality:} \\
               Implementing \qvasrs functionality and reachability relation computation \texttt{reach} in the \qvasr library.

			  \textsl{Duration:} 1 month \\
			  \textsl{Deliverable:} An implemented library capable of computing \qvasrs, \texttt{reach}, and returning the reflexive transitive closure of a loop.

	\item[3.] \textsl{Testing the library:} \\
               Test the library to make sure it works as intended, fix any occurring bugs, and implement its usage in the \texttt{accelerated interpolation} scheme.

			  \textsl{Duration:} 1 months \\
			  \textsl{Deliverable:} A tested and bugfree \qvasr library that can be used in the \texttt{accelerated interpolation} scheme.

	\item[4.] \textsl{Evaluating the library's performance:} \\
               Run the library on various benchmarks, log its performance, and compare it to other techniques.

			  \textsl{Duration:} 1 month \\
			  \textsl{Deliverable:} Data of the performance of the \qvasr library.

	\item[5.] \textsl{Writing a thesis:} \\
              Writing a thesis on the findings of this project, proofreading it, and printing it.

			  \textsl{Duration:} 2 months \\
			  \textsl{Deliverable:}	A master's thesis
\end{itemize}
