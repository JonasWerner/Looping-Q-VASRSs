
The proposed \qvasrs loop summarization library shall be implemented as envisioned in Figure \ref{fig}: The input consists of a transition formula of a loop, that was computed by other tools. This transition formula is firstly used in the \texttt{predicate transformer} to compute a set of predicates $P$ that fulfill the condition of pairwise unsatisfiability. The \texttt{predicate transformer} will be able to employ different techniques of predicate generation making it possible to test the effectiveness and speed of, for example, stronger versus weaker predicates. Following the predicate generation, an empty \qvasr-abstraction set $\mathcal{A}$ is initialized. We need to compute a best \qvasr-abstraction \pqvasr for each pair of predicates $p, q \in P$. For that we construct the formula $\pqformula = p \land F \land q$, which is checked for satisfiability. If a model $\mathcal{M}$ satisfies $\pqformula$, we search for a cube $c$ in the disjunctive normal form of $\pqformula$ that is satisfied by $\mathcal{M}$ and use the \texttt{\qvasr-abstractor} to compute a \qvasr-abstraction $(S_c, V_c)$ for it. Because we want $\pqvasr$ to be the best \qvasr-abstraction for $\pqformula$, we have to make sure $(S_c, V_c) \preceq \pqvasr$ holds. We have to update \pqvasr in a way that it simulates $(S_c, V_c)$. This is done by a \texttt{join} of the two \qvasr-abstractions. A \texttt{join} consists of two operations, firstly, in \texttt{pushout}, we compute a linear simulation from $S_c$ to $\pq{S}$ and update $\pq{S}$ accordingly. Secondly, in the \texttt{image-builder}, we calculate and incorporate the changes to variables done by $V_c$ into $\pq{V}$. After \texttt{pushout} and \texttt{image} we know that \pqvasr simulates $(S_c, V_c)$ and with that $(S_c, V_c) \preceq \pqvasr$ holds. We do not need another \qvasr-abstraction for $c$, so we update $\pqformula$ to exclude the cube from it. If $\pqformula$ is still satisfiable, we begin anew.
If \pqformula is unsatisfiable, then we know that $\pqvasr$ is the least upper bound and with that the best \qvasr-abstraction for \pqformula. We return and add it to $\mathcal{A}$ and start with the next predicate pair. When every predicate pair has a \qvasr-abstraction, we construct a \qvasrs $\mathcal{V}$. In \texttt{reach} we compute the reachability relation of $\mathcal{V}$ which is then used to form a loop summary that is finally returned to \textsc{Ultimate}.
\begin{figure}[H]
    % Define block styles
\tikzstyle{block} = [rectangle, draw, rounded corners, minimum height=4em, minimum width={width("predicate transformer")+15pt},
]
\tikzstyle{group} = [rectangle, draw, rounded corners, minimum height=5em
]
\begin{tikzpicture}[%
    ->,
	>=stealth',
	shorten >=1pt,
	auto,
	node distance=3.5cm and 5cm,
	scale=1,
	transform shape,
	align=center,
	smallnode/.style={inner sep=1.4},
	initial text =,
	anchor=center]
	% Place nodes
	\node [align=left](input) {\textbf{Input}};
	\node [block, align=left, below=of input, yshift=2cm](predtransformer) {\texttt{predicate transformer}};
	\node [block, align=left, below=of predtransformer, yshift=1.5cm](vasrs) {Are there $p, q \in P$ \\ with $\pqvasr \not\in \mathcal{V}$?};
	\node [block, align=left, right=of vasrs](reach) {\texttt{reach}};
	\node [block, align=left, below=of vasrs, yshift=1cm](H) {\pqformula satisfiable?};
	\node [block, align=center, right=of H](abstr) {\texttt{\qvasr-abstractor}};
	\node [block, align=left, below=of abstr, fill=white, yshift=1cm](pushout) {\texttt{pushout}};
	\node [block, align=left, below=of H, fill=white, yshift=1cm](image) {\texttt{image-builder}};
	\node [align=left, above= of reach](output) {\textbf{Output}};
	\begin{scope}[on background layer]
		\node[group, draw=black,fill=stmtcolor,fit=(pushout) (image), label=below :{\texttt{\qvasr-join}}](FIt1) {};
	\end{scope}
	
	% Draw edges
	\draw (input) to node[above] {} node[right] {$F :=$ loop's transition formula \\ with $n$ variables} (predtransformer);
	%\draw (predtransformer) to node[right] {} node[right, align=left] {$F$\\ $P$ := set of predicates \\ $\mathcal{V} := \emptyset$} (vasrs);
	\draw (reach) to node[above] {} node[left] {\texttt{reach($\mathcal{A}$)} := \\ overapproximation \\ of $F's$ reflexive \\ transitive closure} (output);
	\draw [bend right]  (vasrs) to node[above left, align=right, yshift=-0.5cm] {\textbf{yes} \\ $\pqformula := p \land F \land q$ \\ $\pqvasr := (\mathit{I_n},\emptyset)$} node[right, align=left] {} (H);
	\draw (H) to node[above, xshift=-1.5cm] {\textbf{yes}} node[below] {$c :=$ cube in $\mathit{DNF(\pqformula)}$} (abstr);
	\draw (abstr) to node[above] {} node[left] {$(S_c, V_c) :=$ \\ \texttt{abstraction(c)}} (pushout);
	\draw (pushout) to node[above] {} node[above] {$\pq{S} :=$ \texttt{pushout($\pq{S}, S_c$)}} (image);
	
	%\draw (image) to node[above] {} node[left, align=left] {$\pqformula :=$ refine$(\pqformula)$ \\ $\pq{V} :=$ \texttt{image(\pq{V})} \\ $\cup$ \texttt{image$(V_c)$}} (H);
	
		\draw (image) to node[above] {} node[right, align=left] {%
		{\begin{varwidth}{3.5cm}
				\vspace*{-0.5cm}
				\begin{align*}
					\pqformula :=&\ \texttt{refine}(\pqformula) \\ \bigskip
					\pq{V} :=&\ \texttt{image}(\pq{V}) \\
					&\cup \texttt{image}(V_c)
		\end{align*}\end{varwidth}}
	} (H);
	
	
	\draw (predtransformer) to node[above] {} node[right, align=left] {%
		{\begin{varwidth}{3.5cm}
			\vspace*{-0.5cm}
			\begin{align*}
				F \\
				P &:= \text{set of predicates} \\
				\mathcal{V} &:= \emptyset
			\end{align*}\end{varwidth}}
		} (vasrs);

	\begin{comment}
		\draw[] (vasrs) to node[above] {} node[left, align=right] {% 
	{\begin{varwidth}{4cm}
	\begin{align*}
	\textbf{yes} \\
	\pqformula &:=\ p \land F \land q \\
	\pq{S} &:= I_n \\
	\pq{V} &:= \emptyset
	\end{align*}\end{varwidth}}
	} (H);
	\end{comment}

	\draw [bend right]  (H) to node[above] {} node[below right, align=left] {$\mathcal{V} := \mathcal{V} \cup \pqvasr$ \\ \textbf{no}} (vasrs);
	\draw (vasrs) to node[below] {$\mathcal{A} := \qvasrs(\mathcal{V})$} node[above, xshift=-1.5cm] {\textbf{no}} (reach);
\end{tikzpicture}
    \caption{Proposed implementation of a \qvasrs-based loop summarizer.}
    \label{fig}
\end{figure}

A master's thesis consists of six months of work which translates to 24 work weeks. In the following we will give an outline of the approach of how we plan to implement a \qvasr and \qvasrs loop summarization library in the software-verification framework \textsc{Ultimate}, use the library with \texttt{accelerated interpolation}, evaluate its performance, and writing a thesis, in this given time frame. \\ \par
We divide the project into five distinct milestones:

\begin{itemize}
	\item[1.] \textsl{Implementing a \qvasr library:} \\
               Implementation of \qvasr abstraction computation, \qvasr \texttt{image} calculation, \texttt{pushout}, and needed auxiliary classes for matrix and vector operations into a new loop summarization library in \textsc{Ultimate} with unit tests to verify functionality.

			  \textsl{Duration:} 8 weeks of work \\
			  \textsl{Deliverable:} An implemented library capable of computing the \\ best \qvasr-abstraction with unit tests to verify functionality.

	\item[2.] \textsl{Implementing \qvasrs functionality:} \\
               Implementing \qvasrs functionality and reachability relation computation \texttt{reach} in the \qvasr library.

			  \textsl{Duration:} 4 weeks of work\\
			  \textsl{Deliverable:} An implemented library capable of computing \qvasrs, their reachability relation \texttt{reach}, and returning a loop summary.

	\item[3.] \textsl{Integration into accelerated interpolation} \\
			  Integrating the library into the interpolant generation scheme \\ \texttt{accelerated interpolation}.
	
			  \textsl{Duration:} 1 week of work \\
			  \textsl{Deliverable:} \qvasrs library can be used in the \texttt{accelerated interpolation} scheme.

	\item[4.] \textsl{Testing the library:} \\
               Test the library on a set of tests covering borderline cases, such as nonterminating loops, nondeterministic loops, etc, to make sure it works as intended, fix any occurring bugs, and implement its usage in the \texttt{accelerated interpolation} scheme.

			  \textsl{Duration:} 3 weeks of work \\
			  \textsl{Deliverable:} A tested and bugfree \qvasr library.

	\item[5.] \textsl{Evaluating the library's performance:} \\
               Run the library on a broad suite of benchmarks, such as the set of \texttt{sv-comp} benchmarks \cite{svcomp}, log its performance, compare it to other techniques, and optimize inefficiencies.

			  \textsl{Duration:} 4 weeks of work: \
			  \textsl{Deliverable:} A profiled and optimized library and data of the performance of the \qvasr library.

	\item[6.] \textsl{Writing a thesis:} \\
              Writing a thesis on the findings of this project, proofreading it, and printing it.

			  \textsl{Duration:} 8 weeks of work \\
			  \textsl{Deliverable:}	A master's thesis
\end{itemize}
