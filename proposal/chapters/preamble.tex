
\usepackage[utf8]{inputenc}
\usepackage{xspace}
\usepackage{tabularx}
\usepackage[%
hyperindex,%
plainpages=false,%
pdfusetitle]{hyperref}
\usepackage[all]{hypcap}
\usepackage{algorithm}
\usepackage[noend]{algpseudocode}
\usepackage{cite}
\usepackage{booktabs}
\usepackage{url}
\usepackage{listings}
\usepackage{enumitem}
\usepackage{amsthm}
\usepackage{amsmath}
\usepackage{tikz}
\usetikzlibrary{positioning,shapes.geometric, arrows.meta ,automata, decorations.pathreplacing, calc, fit, backgrounds}
\usepackage{pgf}
\usepackage{slantsc}
\usepackage{geometry}
\usepackage{amssymb}
\usepackage{subcaption}
\usepackage{float}
\usepackage{pgf}
\usepackage{slashbox}
\usepackage{pgfgantt}
\usepackage{wrapfig}
\usepackage{pdflscape}
\usepackage{xcolor}
\usepackage{xparse}
\usepackage[%disable,%
colorinlistoftodos,%
color=cyan!50!white,%
bordercolor=cyan!50!black]{todonotes}

\usepackage{varwidth}
\usepackage[most]{tcolorbox}% http://ctan.org/pkg/tcolorbox
\tcbuselibrary{skins,breakable}
\usepackage{comment}
\usepackage{makecell}
\usepackage{stmaryrd}
\usepackage{nicematrix}

\DeclareSymbolFont{matha}{OML}{txmi}{m}{it}% txfonts
\DeclareMathSymbol{\varv}{\mathord}{matha}{118}


%%%%%%%%%%%% Colors
%% a somewhat friendly scheme for 5 different colors
\definecolor{g1}		{RGB}{215,25,28} % a kind of red
\definecolor{g2}		{RGB}{253,174,97} % a kind of orange
\definecolor{g3}		{RGB}{255,255,191} % a kind of yellow
\definecolor{g4}		{RGB}{171,217,233} % a kind of light blue
\definecolor{g5}		{RGB}{44,123,182} % a kind of dark blue

\definecolor{gr1}		{RGB}{250, 250, 250}
\definecolor{gr2}		{RGB}{229, 229, 229} % some grey

% color of interpolants
\definecolor{itpGreen}  {rgb}{0,1,0}
\definecolor{grey}      {RGB}{200,200,200}


%color for pictures
\colorlet{outlineblue}		{g5}
\colorlet{fillblue}			{g4}
\colorlet{darkback}			{gr2}
\colorlet{lightback}		{gr1}
\colorlet{stmtcolor}		{gr2} %default statement color
\colorlet{subgraphcolor}	{g3} %default statement color
\colorlet{colexamtitle}     {black} % Example block title color
\colorlet{colexamline}      {g1} % Example block sideline color
\colorlet{itp}              {itpGreen}



%%%%%%%%%%%%% Statements and labels Trace Abstraction Style
\tikzstyle{st} = [%
font=\ttfamily,%
shape=rectangle,%
rounded corners=.5em,%
fill=stmtcolor,%
inner xsep=.3em,%
inner ysep=0em, %
text height=2ex, %
text depth=.6ex,
]


\newcommand{\tikzstmt}[3]{{%
		\tikz[baseline]{%
			\node[st,fill=#2] at (0,.64ex){%
				\hspace{.3em}\texttt{\strut#3#1}\hspace{.3em}\strut};}
}}

\newcommand{\stcol}[2]{\tikzstmt{#1}{#2}{}}
\newcommand{\stsmcol}[2]{\tikzstmt{#1}{#2}{\small}}
\newcommand{\stfootcol}[2]{\tikzstmt{#1}{#2}{\footnotesize}}

\newcommand{\stnorm}[1]{\stcol{#1}{stmtcolor}}
\newcommand{\stsm}[1]{\stsmcol{#1}{stmtcolor}}
\newcommand{\stfoot}[1]{\stfootcol{#1}{stmtcolor}}

\newcommand{\st}[1]{\stfoot{#1}}
\newcommand{\lab}[1]{\stfoot{\ensuremath{#1}}}
\newcommand{\lan}[1]{\stnorm{\ensuremath{#1}}}
\newcommand{\stn}[1]{\stnorm{#1}}

\newcommand{\formula}[2]{\tikz[baseline]{\node[shape=rectangle,line width=1pt,draw=#2,fill=#2!30,inner sep=1pt, align=center] at (0,.64ex){\hspace{.2em}\texttt{\strut#1}\hspace{.1em}\strut};}}
\newcommand{\itp}[1]{\formula{\ensuremath{#1}}{itp}}

\newcommand{\tf}{\ensuremath{\varphi}\xspace}
\newcommand{\ctf}{\ensuremath{\widehat{\varphi}}\xspace}
\newcommand{\invars}{\ensuremath{In}\xspace}
\newcommand{\outvars}{\ensuremath{Out}\xspace}
\newcommand{\auxvars}{\ensuremath{Aux}\xspace}

\newcommand{\Var}{\ensuremath{\mathit{Var}}\xspace}
\newcommand{\stmt}{\ensuremath{\mathit{Stmt}}\xspace}
\newcommand{\Loc}{\ensuremath{\mathit{Loc}}\xspace}
\newcommand{\err}{\ensuremath{\mathit{err}}\xspace}
\newcommand{\init}{\ensuremath{\mathit{init}}\xspace}


%The [1] means one parameter, which is then referenced in the #1
\newcommand{\loc}[1]{\ensuremath{\ell_{#1}}\xspace}
\newcommand{\trans}[1]{\ensuremath{\xrightarrow{\st{#1}}}\xspace}
\newcommand{\stateSet}[1]{\ensuremath{\{#1\}}\xspace}
\newcommand{\vocab}[1]{\ensuremath{V_{\mathit{#1}}}\xspace}

\newcommand{\abst}[1]{\ensuremath{\mathit{(S^{#1}, V^{#1})}}}
\algnewcommand\algorithmicforeach{\textbf{for each}}
\algdef{S}[FOR]{ForEach}[1]{\algorithmicforeach\ #1\ \algorithmicdo}

\newcommand{\pq}[1]{\ensuremath{#1_{p, q}}\xspace}
\newcommand{\pqvasr}{\ensuremath{(\pq{S}, \pq{V})}\xspace}
\newcommand{\pqformula}{\ensuremath{\pq{\Gamma}}\xspace}

%%%%%%%%%%%% Numbered example environment
% \newcounter{example}[section]
% \newenvironment{example}[1][]{\refstepcounter{example}\par\medskip
%    \noindent \textbf{Example~\theexample. #1} \rmfamily}{\medskip}

\newcounter{example}[section]
\def\exampletext{Example}
\NewDocumentEnvironment{example}{ O{} }
{
	
	\newtcolorbox[use counter=example]{examplebox}{%
		empty,
		title={\exampletext~\theexample. #1},
		attach boxed title to top left,
		minipage boxed title,
		boxed title style={empty,size=minimal,toprule=0pt,top=4pt,left=3mm,overlay={}},
		coltitle=colexamtitle,
		fonttitle=\bfseries,
		before=\par\medskip\noindent,parbox=false,boxsep=0pt,left=3mm,right=0mm,top=2pt,breakable,pad at break=0mm,
		before upper=\csname @totalleftmargin\endcsname0pt,
		overlay unbroken={\draw[colexamline,line width=.5pt] ([xshift=-0pt]title.north west) -- ([xshift=-0pt]frame.south west); },
		overlay first={\draw[colexamline,line width=.5pt] ([xshift=-0pt]title.north west) -- ([xshift=-0pt]frame.south west); },
		overlay middle={\draw[colexamline,line width=.5pt] ([xshift=-0pt]frame.north west) -- ([xshift=-0pt]frame.south west); },
		overlay last={\draw[colexamline,line width=.5pt] ([xshift=-0pt]frame.north west) -- ([xshift=-0pt]frame.south west); },%
	}
	\begin{examplebox}
	}
	{\end{examplebox}\endlist}


%%%%%%%%%%%% Setup
\newtheorem{name}{Printed output}
\newtheorem{mydef}{Definition}

\hypersetup{
	colorlinks=true,        % false: boxed links; true: colored links
	linkcolor=g1,        % color of internal links
	citecolor=g1,        % color of links to bibliography
	filecolor=g1,        % color of file links
	urlcolor=g1          % color of external links
}


\lstdefinestyle{boogie}{
	belowcaptionskip=1\baselineskip,
	breaklines=true,
	xleftmargin=\parindent,
	showstringspaces=false,
	basicstyle=\footnotesize\ttfamily,
	numbers=left,
	xleftmargin=.6cm
}

\lstset{escapechar=@,style=boogie}

\algrenewcommand\alglinenumber[1]{\footnotesize #1}

\usepackage{regexpatch}

\makeatletter
\xpatchcmd{\algorithmic}{\labelsep 0.5em}{\labelsep 1.5em}{\typeout{Success!}}{\typeout{Oh dear!}}
\makeatother

%%%%%%%%%%%% Comments
\newif\iffinal
%\finaltrue % comment out to remove comments

\iffinal
\newcommand\mycom[1]{}
\else
\newcommand\mycom[1]{#1}
\overfullrule=1mm
\fi
\setlength\parindent{0pt}

\newcommand{\jw}[1]{\mycom{\todo[color=blue!40,inline]{\small JW: #1}}}
\newcommand{\dd}[1]{\mycom{\todo[color=orange!40,inline]{\small DD: #1}}}
\newcommand{\ts}[1]{\mycom{\todo[color=green!40,inline]{\small TS: #1}}}


\newcommand{\all}[1]{\mycom{\todo[color=green!40,inline]{\small #1}}}
\newcommand{\meta}[1]{\mycom{\todo[color=blue!10,inline,caption={Beschreibung},nolist]{\setlist{nolistsep}\small #1}}}
\newcommand{\xxx}{\mycom{\stfootcol{Placeholder}{blue!20}\xspace}}
\newcommand{\cn}{\mycom{\stfootcol{Cite}{blue!20}\xspace}}

\newcommand{\Q}{\ensuremath{\mathbb{Q}}}
\newcommand{\entails}[1]{\vdash_{#1}}

\newcommand{\qvasr}{\ensuremath{\mathit{\mathbb{Q} \text{-}VASR}}\xspace}
\newcommand{\qvasrs}{\ensuremath{\mathit{\mathbb{Q} \text{-}VASRS}}\xspace}

\newcommand{\tranFormula}{\ensuremath{F(\vec{x}, \vec{x}')}\xspace}
\newcommand{\coherent}{\ensuremath{\equiv_{\mathit{V}}}}
\newcommand{\conjunctTF}{\ensuremath{C(\vec{x}, \vec{x}')}}
\newcommand{\GammaTF}{\ensuremath{\Gamma(\vec{x}, \vec{x}')}\xspace}

\usepackage[justification=centering]{caption}