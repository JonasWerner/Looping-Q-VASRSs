
Analyzing program loops is an important part of multiple software engineering tasks, such as bug detection, test case generation, and optimization. Loop analysis is, however, one of the most challenging aspects of program analysis, especially with complex loops that feature multiple paths and nested loops. Loop analysis has been called the "Archilles heel" of program verification\cite{0.1007/s10703-012-0176-y}. \par
There are three prominent techniques of loop analysis: loop unwinding, loop invariant calculation, and loop summarization\cite{0.1007/s10703-012-0176-y, 8241837,10.1007/978-3-030-25543-5_7}. Loop unwinding deals with a loop by unrolling each loop iteration until a certain number. This technique is quite simple but cannot reason about the program behavior beyond the unwinding bound. \\
Loop invariants are properties that hold before and after each loop iteration, modeling its behaviour. The usefulness of loop invariants is however tied to how strong they are, and finding sufficiently strong invariants "is an art"\cite{0.1007/s10703-012-0176-y}. A software-model checker, for example, may not terminate if given loop invariants are insufficiently general. \par
Loop summarization on the other hand provides a more accurate and complete view of the loop. A loop summary models the relationship between inputs and outputs as a set of constraints that can be used to replace the loop in a program. Loop summaries are not guaranteed to be precise, meaning they can over- or underapproximate the loop's behavior. Counterexamples found using underapproximative loop summaries are guaranteed to be feasible counterexamples, they can, however, miss feasible counterexample as they model only a subset of the actual loop behaviour. Counterexamples found by overapproximative loop summaries can, on the other hand, turn out to be spurious. The quality of overapproximative loop summaries depends on how much the summary overapproximates the actual loop behaviour.\par
\jw{Todo: \qvasr and \qvasrs introduction}
Silverman and Kincaid\cite{10.1007/978-3-030-25543-5_7} proposed an overapproximative loop summarization  \\
\begin{minipage}[t]{0.3\linewidth} \centering
	\begin{figure}[H]
		\begin{lstlisting}[language=C++,basicstyle=\ttfamily,keywordstyle=\color{blue}]  % Start your code-block
	
	int x := 0;
	int y := 2;
	int z := 3;
	while x <= 21:
		if x <= 10:
			z := x;
			x := x + y;
			y := y + 1;
		else:
			x := x + 2;
			y := y - 3;
	assert x == 22;
	\end{lstlisting}
		\caption{Program $P$ \\ with \texttt{while} loop.}
	\end{figure}
\end{minipage}
\begin{minipage}[t]{0.35\linewidth} \centering
	\begin{figure}[H]
			\begin{align*}
S_P &= 
\NiceMatrixOptions{code-for-first-row=\scriptstyle}
	\begin{bNiceMatrix}[first-row=1]
		x & y & z \\
		-1 & 1 & 1 \\
		0 & 1 & 0
	\end{bNiceMatrix}, \ \\ \\
    V_P &= \begin{Bmatrix}
        \begin{pmatrix}
              \begin{bmatrix}
                    0 \\
                    1
               \end{bmatrix},
               \begin{bmatrix}
                     1 \\
                     1
               \end{bmatrix}
        \end{pmatrix}, \\ \\
        \begin{pmatrix}
               \begin{bmatrix}
                    1 \\
                    1
               \end{bmatrix},
               \begin{bmatrix}
                    -5 \\
                    -3
              \end{bmatrix}
        \ \end{pmatrix}
    \end{Bmatrix}
\end{align*}
%\caption{\qvasr abstraction $A_p = (S, V)$ of $P$.}
			\caption{\\ \qvasr-abstraction \\ of $P$'s loop.}
	\end{figure}
\end{minipage}
\begin{minipage}[t]{0.3\linewidth} \centering
	\begin{figure}[H]
		
\begin{figure}[H]
    \centering
    \begin{tikzpicture}[%
    ->,
    >=stealth',
    shorten >=1pt,
    auto,
    node distance=5cm,
    scale=1,
    transform shape,
    align=center,
    smallnode/.style={inner sep=1.4},
    initial text =,
    anchor=center]

    	\node[draw, ellipse, initial above, initial text =](1){$x \leq 10$};
    	\node[draw, ellipse](2) [right of=1] {$10 < x \land x \leq 20$};
    	% \node[draw, ellipse, accepting] (3) [right of=2] {$x > 20$};

    	\path (1) edge node [above]{ $
                    \begin{pmatrix}
                      \begin{pmatrix}
                            0 \\
                            1
                       \end{pmatrix},
                       \begin{pmatrix}
                             1 \\
                             1
                       \end{pmatrix}
                     \end{pmatrix}
                     $ } (2)

    	 (1) edge[loop below] node { $
                    \begin{pmatrix}
                      \begin{pmatrix}
                            0 \\
                            1
                       \end{pmatrix},
                       \begin{pmatrix}
                             1 \\
                             1
                       \end{pmatrix}
                     \end{pmatrix}
                     $ } (1)

    	 (2) edge[loop below] node { $
                    \begin{pmatrix}
                      \begin{pmatrix}
                            1 \\
                            1
                       \end{pmatrix},
                       \begin{pmatrix}
                             -5 \\
                             -3
                       \end{pmatrix}
                     \end{pmatrix}
                     $ } (2)
    	;
    \end{tikzpicture}
    %\caption{\qvasrs abstraction $\mathcal{A}$ of program $P$.}
\end{figure}
		\caption{\\ \qvasrs-abstraction of $P$'s loop.}
	\end{figure}
\end{minipage}