\begin{mydef}
A transition system is a pair T = $(S, \rightarrow)$ consisting of a set of states $S$, and a transition relation $\rightarrow \subseteq S \times S$. We define $\rightarrow^*$ as the reflexive transitive closure of $\rightarrow$.
\end{mydef}

\begin{mydef}
A transition formula \tranFormula of dimension $n$ is a first order logic formula ranging over variables $\vec{x} = x_1, \ldots x_n$ and $\vec{x}' = x_1', \ldots, x_n'$, where $\vec{x}$ denotes the state before a transition and $\vec{x}'$ after. \tranFormula defines a transition system $(S_F, \rightarrow_F)$ with state space $S_F = \mathbb{Q}$ and transitions $\vec{u} \rightarrow_F \vec{v}$ if $F(\vec{u}, \vec{v})$ is valid.
\end{mydef}
\begin{example}
    From example program $P$ we can extract the transition formula of the \texttt{while} loop as:
    \begin{align*}
    \tranFormula_\ell =\ \  &x \leq 10 \ \land\ x' = x + y\ \land\ y' = y+1\ \land\ z' = x\ \\ \lor\ & x \leq 20\ \land\ x > 10\ \land\ x' = x + 2\ \land\ y' = y - 3
    \end{align*}
\end{example}

\begin{mydef}
A rational vector addition system with resets (\qvasr) of dimension d is a finite set $V \subseteq \{0, 1\}^d \times \Q^d$, consisting of of tuples $(\vec{r}, \vec{a})$, with binary reset $\vec{r}$ and rational addition vector $\vec{a}$ of dimension d.
\end{mydef}

\begin{mydef}
Given transition systems $A = (\mathbb{Q}^n, \rightarrow_A)$ and $B = (\mathbb{Q}^m, \rightarrow_B)$ of dimensions $n$ and $m$. A linear simulation is a linear transformation represented by matrix $S \in \mathbb{Q}^{m \times n}$ such that for all $\vec{u}, \vec{v} \in \mathbb{Q}^n$ with $\vec{u} \rightarrow_A \vec{v}$, we have $S\vec{u} \rightarrow_B S\vec{v}$. A linear simulation $S$ from $A$ to $B$ is denoted as $A \Vdash_{S} B$
\end{mydef}

\begin{mydef}
A \qvasr abstraction is a tuple $A = (S, V)$ consisting of a rational matrix $S \in \Q^{d \times n}$ and a d-dimensional \qvasr $V$. The dimensions $d$ and $n$ are called concrete and abstract dimensions.
Given a transition formula \tranFormula and a \qvasr abstraction $A$, if $F \entails{S} V$ then we say that $A$ is a \qvasr abstraction of $F$.
\end{mydef}

To compute a \qvasr-abstraction $A = (S, V)$ for a given conjunctive transition formula \conjunctTF, with $C \entails{S} V$, we need to calculate vector space bases of two sets:
\begin{itemize}
 \item The set of linear combinations of variables that are reset in $C$: $Res_C = \{ \langle \vec{s}, a \rangle : C \models \vec{s} \cdot \vec{x}' = a \}$
 \item The set of linear combinations of variables that are incremented in $C$: $Inc_C = \{ \langle \vec{s}, a \rangle : C \models \vec{s} \cdot \vec{x}' = \vec{s} \cdot \vec{x} + a \}$
\end{itemize}
By solving the emerging linear equation systems we get bases for $Res_C: \{\langle \vec{s}_1, a_1 \rangle, \ldots, \langle \vec{s}_m, a_m \rangle\}$ and $Inc_C: \{ \langle \vec{s}_{m+1}, a_{m+1} \rangle , \ldots, \langle \vec{s}_d, a_d\rangle \}$. \\ We define the \qvasr-abstraction of $C$ as $\alpha(C) = (S, \{(\vec{r}, \vec{a})\})$, with \\

\begin{minipage}{0.3\textwidth}
\centering
    $ S = \begin{pmatrix}
            \vec{s}_1 \\
            \vdots \\
            \vec{s}_d
            \end{pmatrix}$
\end{minipage}
\begin{minipage}{0.3\textwidth}
\centering
$\vec{r} = (\underbrace{0 \cdots 0}_{\text{m times}} \ \ \underbrace{1 \cdots 1}_{\text{(d-m times)}})$
\end{minipage}
\begin{minipage}{0.3\textwidth}
\centering
$\vec{a} = \begin{pmatrix}
            \vec{a}_1 \\
            \vdots \\
            \vec{a}_d
            \end{pmatrix}$
\end{minipage}

\begin{example}
Given conjunction $C = x' = x+ y \land y' = y+ 1 \land z' = x$, we can compute a \qvasr-abstraction $\alpha(C)$ as follows: \\

\end{example}

\begin{mydef}
Given a d-dimensional \qvasr $\mathit{V}$,  $i, j \in \{1 ,\ldots, d\}$ are coherent dimensions of $\mathit{V}$ if for all $\vec{r}, \vec{a} \in \mathit{V} r_i = r_j$. Coherency is denoted by $\equiv_{\mathit{V}}$. \\
\coherent forms an equivalence relation on $\{1, \ldots, d\}$. The equivalence classes of $\coherent$ are referred to as the coherence classes of $\mathit{V}$.
\end{mydef}

\begin{mydef}
A \qvasr abstraction $\mathit{A = (S, V)}$ is normal if there is no non-zero vector $\vec{z}$ that is coherent to $\mathit{V}$ such that $\vec{z}S = 0$.
\end{mydef}

\begin{example}
    \begin{align*}
    S_P &= \begin{pmatrix}
        -1 & 1 & 1 \\
        0 & 1 & 0
    \end{pmatrix}, \ \\ \\
    V_P &= \begin{Bmatrix}
        \begin{pmatrix}
              \begin{pmatrix}
                    0 \\
                    1
               \end{pmatrix},
               \begin{pmatrix}
                     1 \\
                     1
               \end{pmatrix}
        \end{pmatrix}, \\ \\
        \begin{pmatrix}
               \begin{pmatrix}
                    1 \\
                    1
               \end{pmatrix},
               \begin{pmatrix}
                    -5 \\
                    -3
              \end{pmatrix}
        \ \end{pmatrix}
    \end{Bmatrix}
\end{align*}
%\caption{\qvasr abstraction $A_p = (S, V)$ of $P$.}
\end{example}