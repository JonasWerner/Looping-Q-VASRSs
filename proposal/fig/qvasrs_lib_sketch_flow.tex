% Define block styles
\tikzstyle{block} = [rectangle, draw, rounded corners, minimum height=4em, minimum width={width("predicate transformer")+15pt},
]
\tikzstyle{group} = [rectangle, draw, rounded corners, minimum height=5em
]
\begin{tikzpicture}[%
    ->,
	>=stealth',
	shorten >=1pt,
	auto,
	node distance=3.5cm and 4.5cm,
	scale=1,
	transform shape,
	align=center,
	smallnode/.style={inner sep=1.4},
	initial text =,
	anchor=center]
	% Place nodes
	\node [align=left](input) {\textbf{Input}};
	\node [block, align=left, below=of input, yshift=2cm](predtransformer) {\texttt{predicate transformer}};
	\node [block, align=center, below=of predtransformer, yshift=1.5cm](vasrs) {Are there $p, q \in P$ with no \\ \qvasr-abstraction \\ \pqvasr $\in \mathcal{A}$?};
	\node [block, align=left, right=of vasrs](reach) {\texttt{reach}};
	\node [block, align=left, below=of vasrs, yshift=1cm](H) {\pqformula satisfiable?};
	\node [block, align=center, right=of H](abstr) {\texttt{\qvasr-abstractor}};
	\node [block, align=left, below=of abstr, fill=white, yshift=1cm](pushout) {\texttt{pushout}};
	\node [block, align=left, below=of H, fill=white, yshift=1cm](image) {\texttt{image-builder}};
	\node [align=left, above= of reach](output) {\textbf{Output}};
	\begin{scope}[on background layer]
		\node[group, draw=black,fill=stmtcolor,fit=(pushout) (image), label=below :{\texttt{\qvasr-join}}](FIt1) {};
	\end{scope}
	
	% Draw edges
	\draw (input) to node[above] {} node[right] {$F :=$ loop's transition formula \\ with $n$ variables} (predtransformer);
	%\draw (predtransformer) to node[right] {} node[right, align=left] {$F$\\ $P$ := set of predicates \\ $\mathcal{V} := \emptyset$} (vasrs);
	\draw (reach) to node[above] {} node[left] {\texttt{reach($\mathcal{V}$)} = \\ loop summary} (output);
	\draw [bend right]  (vasrs) to node[above left, align=right, yshift=-0.5cm] {\textbf{yes} \\ $\pqformula := p \land F \land q$ \\ $\pqvasr := (\mathit{I_n},\emptyset)$} node[right, align=left] {} (H);
	\draw (H) to node[above, xshift=-1.5cm] {\textbf{yes}} node[below] {$c :=$ cube in $\mathit{DNF(\pqformula)}$} (abstr);
	\draw (abstr) to node[above] {} node[left] {$(S_c, V_c) :=$ \\ \texttt{abstraction(c)}} (pushout);
	\draw (pushout) to node[below] {$\pq{S}$ now simulates $(S_c, V_c)$} node[above] {$\pq{S}$ := \texttt{pushout($\pq{S}, S_c$)} } (image);
	
	%\draw (image) to node[above] {} node[left, align=left] {$\pqformula :=$ refine$(\pqformula)$ \\ $\pq{V} :=$ \texttt{image(\pq{V})} \\ $\cup$ \texttt{image$(V_c)$}} (H);
	
		\draw (image) to node[above] {} node[right, align=left] {%
		{\begin{varwidth}{3.5cm}
				\vspace*{-0.5cm}
				\begin{align*}
					\pqformula :=&\ \texttt{refine}(\pqformula) \\ \bigskip
					\pq{V} :=&\ \texttt{image}(\pq{V}) \\
					&\cup \texttt{image}(V_c)
		\end{align*}\end{varwidth}}
	} (H);
	
	
	\draw (predtransformer) to node[above] {} node[right, align=left] {%
		{\begin{varwidth}{3.5cm}
			\vspace*{-0.5cm}
			\begin{align*}
				F \\
				P &= \text{set of predicates} \\
				\mathcal{A} &:= \emptyset
			\end{align*}\end{varwidth}}
		} (vasrs);

	\begin{comment}
		\draw[] (vasrs) to node[above] {} node[left, align=right] {% 
	{\begin{varwidth}{4cm}
	\begin{align*}
	\textbf{yes} \\
	\pqformula &:=\ p \land F \land q \\
	\pq{S} &:= I_n \\
	\pq{V} &:= \emptyset
	\end{align*}\end{varwidth}}
	} (H);
	\end{comment}

	\draw [bend right]  (H) to node[above] {} node[below right, align=left] {$\mathcal{A} := \mathcal{A} \cup \pqvasr$ \\ \textbf{no}} (vasrs);
	\draw (vasrs) to node[below] {$\mathcal{V} := \qvasrs(\mathcal{A}, P)$} node[above, xshift=-1.5cm] {\textbf{no}} (reach);
\end{tikzpicture}