\documentclass[11pt]{article}

\input{chapters/0_preamble.tex}

\begin{document}
\newcommand{\HorizontalLine}{\rule{\linewidth}{0.3mm}}


\begin{titlepage}
\begin{center}


{\Large Undergraduate/Master's Thesis}\\[1.3cm]


% _____________________________________________________________________________
\HorizontalLine \\[0.4cm]
% Write your title in a fancy way like this if you want to customize it, otherwise simply let tex do it for you
% \begin{spacing}{3}
%     {\huge \bfseries The Long, Long } \\
%     {\huge \bfseries Long Long} \\
%     {\huge \bfseries Title}\\
% \end{spacing}
{ \huge \bfseries QVASRS (WIP) }
\HorizontalLine \\[1.5cm]
% _____________________________________________________________________________


{\Huge Jonas Werner } \\[2cm]


\begin{tabular}[hc]{>{\huge}l >{\huge}l}
  Examiner: & Professor Dr. Andreas Podelski \\[0.3cm]
  Advisers: & Dr. Daniel Dietsch, Tanja Schindler \\[1.2cm]
\end{tabular}
\vfill  % move the following text to the bottom

\Large {
    University of Freiburg\\
    Faculty of Engineering\\
    Department of Computer Science\\
    Chair of Software Engineering\\[1cm]

    March 22\textsuperscript{nd}, 2022\\
}
\end{center}
\end{titlepage}

\thispagestyle{empty}
% title page back
\ \vfill \ \\  % at least one space required before vfill
\
\textbf{Writing Period}            \smallskip{} \\
22.\,09.\,2021 -- 22.\,03.\,2022   \bigskip{} \\
\
\textbf{Examiner}                  \smallskip{} \\
Professor Dr. Andreas Podelski                     \bigskip{} \\
\
\textbf{Advisers}                  \smallskip{} \\
Dr. Daniel Dietsch, Tanja Schindler
\pagebreak
\pagebreak

\pagenumbering{roman} 
\input{chapters/0_1-declaration.tex}
\pagebreak

\section*{Abstract}
Analyzing program loops is an important part of multiple software engineering tasks, such as bug detection, test case generation, and optimization. Loop analysis is, however, one of the most challenging aspects of program analysis, especially with complex loops that feature multiple paths and nested loops. A widely used loop analysis technique is computing overapproximative loop summaries that represent the relationship between inputs and outputs as a set of constraints. But overapproximation is prone to imprecision. Kincaid and Silverman proposed a novel overapproximation approach, based on rational vector addition systems with resets (Q-VASR), that guarantees a certain degree of precision. In this thesis we adapted the Q-VASR based loop summarization scheme to the trace abstraction method for proving program safety by implementing a summarization library that is used in two techniques: Directly on traces in the accelerated interpolation framework, which uses loop summaries for creating program state assertions that contain every trace through a loop and as a preprocessor that transforms a given program's control-flow graph by replacing loops with the loop summarization such that trace abstraction considers traces that already contain every trace through a loop.
\pagebreak

\section*{Zusammenfassung}
German version is only needed for an undergraduate thesis.
\jw{Todo when normal abstract is ok}
\pagebreak

\tableofcontents
\pagebreak

\pagenumbering{arabic} 
\section{Introduction}
\label{intro}

\begin{comment}
	\jw{Intuitive introduction to the subject matter, no formulas or definitions. \\
	- Start by firstly introducing trace abstraction: explaining safety of programs \\
	- Introduce program loops, mention what problems they cause $\rightarrow$ nesting, branching etc \\
	- then the 3 possible techniques: unwinding, invariant calculation, summarization with focus on summarization\\
	- related work \\
	- use simple running example \\
	- Continue to accelerated interpolation, intuitively explain how it works and how it integrates loop summaries $\rightarrow$ meta traces \\
	- Say that Ultimate already has a myriad of loop summarization techniques such as FastUPR and Jordan and show their drawbacks, such as Jordan being only to deal with no variable relations and FastUPRs lack of branching support. \\
	- Introduce qvasr as a solution  \\
	IMPORTANT: Only INTUITIVE explanations, no definitions \\ 
	\vspace{1cm}
	3 pages}
\end{comment}


Analyzing program loops is an important part of multiple software engineering tasks, such as bug detection, test case generation, and optimization. Loop analysis is, however, one of the most challenging aspects of program analysis, especially with complex loops that feature multiple paths and nested loops. Loop analysis has been called the `Achilles heel` of program verification\cite{DBLP:journals/fmsd/KroeningSTTW13}. \par
There are three prominent techniques of loop analysis: loop unwinding, loop invariant calculation, and loop summarization\cite{DBLP:journals/fmsd/KroeningSTTW13 , DBLP:conf/cav/SilvermanK19, DBLP:journals/tse/XieCZLLL19}. Loop unwinding deals with a loop by unrolling each loop iteration until a certain bound. This technique is quite simple but cannot reason about the program's behavior beyond the unwinding bound. \\
Loop invariants are properties that hold before and after each loop iteration, defining the loop's behaviour as constraints. The meaningfulness of loop invariants is however tied to their constraints' strength. Finding sufficiently strong invariants "is an art"\cite{DBLP:journals/fmsd/KroeningSTTW13}. A software-model checker, for example, may not terminate if given loop invariants are too general. \par
Loop summarization on the other hand provides a more accurate and complete view of the loop. A loop summary models the relationship between inputs and outputs as a transition formula which can be used to approximate the effect a loop has on program variables. Loop summaries are not guaranteed to be precise, they can over- or underapproximate the loop's effect. Counterexamples found using underapproximative loop summaries are guaranteed to be feasible. But due to the underapproximate nature of the summaries they can miss other feasible counterexample as they model only a subset of the actual loop behavior \cite{DBLP:journals/fmsd/KroeningLW15}. Loop summaries that overapproximate the loop's behaviour on the other hand form a superset of the actual loop behavior, resulting in possible spurious counterexamples. The quality of overapproximative loop summaries depends on the degree of overapproximation of the actual loop behavior. \par
Silverman and Kincaid\cite{DBLP:conf/cav/SilvermanK19} proposed an overapproximative loop summarization that \textsl{guarantees} a certain degree of approximation precision.
This proposed technique makes use of vector addition systems which are a prominent class of infinite-state transition systems with decidable reachability \cite{DBLP:conf/rp/HaaseH14}. A vector addition system consists of a finite number of integer-typed variables. When a transition is taken, these variables are updated by an addition of a constant. Silverman and Kincaid use a variation of vector addition systems that are based on rational-typed variables instead of integers with the notion of reset transitions which set variables to zero. When a transition is taken in a rational vector addition systems (\qvasr) a variable is either incremented by a constant, reset to 0, or reset and incremented.
\qvasr by themselves are a precise method of loop summarization, however, they can only deal with constant variable updates and updates that reference only the updated variable itself, for example \st{x:=x+1} or \st{y:=42}. Most loops contain variable updates, for instance \st{x:=y+x}, which cannot be represented by a \qvasr. To remedy this, Silverman and Kincaid introduce so called \qvasr abstractions, which are \qvasr extended by a, so-called, simulation matrix which projects the loop's transitions to changes between variable relations. For example, for assignments \st{z:=x}, \st{x:=x+y}, \st{y:=y+1}, we would get the relation $-x + y + z = 1$.
This \qvasr abstraction is an overapproximation as it no longer models changes to specific variables but to variable relations.\par
\qvasr abstractions only model changes to variables and do not factor in program assumptions, such as the loop guard or \st{if $\ldots$ else} statements. This leads to a loss of precision as the loop summaries' transitions are taken nondeterministically. This problem lead to the introduction of rational vector addition systems with resets and states (\qvasrs), which expand \qvasr abstractions by making use of states that model the original program's control flow.\par
The goal of this thesis was to adapt a loop summarization technique making use of \qvasr in the trace abstraction\cite{10.1007/978-3-642-03237-0_7} model checking scheme used in the software analysis framework \textsc{Ultimate}\cite{Zitat02}.
\textsc{Ultimate} consists of multiple plugins and libraries for various software verification tasks and already features numerous loop summarization techniques, they, however, are struggling with, loop branching, assignments to variables that are not constant, as for example \st{x:=x+y}, and approximation precision.\par
Our contribution is the adaption of a \qvasr based loop summarization library to the trace abstraction scheme used in \textsc{Ultimate}. We implemented two approaches:
\begin{itemize}
	\item Summarizing loops directly on traces computed by trace abstraction, then abstracting them in the \textsl{Accelerated Interpolation} framework, which then computes state assertions constraining values to program variables.
	\item Using \qvasr summarization as a preprocessor for trace abstraction, which transforms a program's control flow graph, by replacing loops with \qvasr summaries, such that, the traces trace abstraction computes already contain loop summaries.
\end{itemize}

The remainder of this master's thesis is structured as follows, in Chapter \ref{relWork} we will give a brief summary of other kinds of loop summarization techniques each with their strength and weaknesses. In chapter \ref{background} we introduce our notations and definitions of logical formulas, programs, traces, and outline trace abstraction. Further, in chapter \ref{qvasr} we introduce the loop summarization technique using rational vector addition systems with resets (\qvasr), and expand them to \qvasr abstractions. Chapter \ref{qvasrs} serves as an introduction to \qvasrs by further extending \qvasr to make use of states. In Chapter \ref{traceabstractionVasr} we present the \textsl{accelerated interpolation} paradigm, and its usage of \qvasr loop summaries. We also adapted \qvasr to be used as a control-flow graph transformer, which will be outlined in chapter \ref{icfgTransformation}. The new loop summarization library was tested extensively, both in \textsl{Accelerated Interpolation} and as control-flow graph transformer. It was measured against other loop acceleration techniques using the set of \texttt{sv-comp} benchmarks \cite{svcomp}. The results will be shown and discussed in chapter \ref{eval}. After the evaluation we discuss possible improvements in Chapter \ref{futrWork}. Last, but not least, we give a conclusion on this project in chapter \ref{concl}.



\section{Related Work}
\label{relWork}

Next to rational vector addition systems with resets, there are a multitude of other loop summarization techniques, all with their own strengths and weaknesses.

\textsl{Fast Acceleration using Ultimately Periodic Relations}\cite{10.1007/978-3-642-14295-6_23} 

\textsl{Abstracting Path Conditions}\cite{DBLP:conf/issta/StrejcekT12}

\textsl{Under-approximating loops}\cite{DBLP:conf/cav/KroeningLW13}

\textsl{Abstract Acceleration of General Linear Loops}\cite{DBLP:conf/popl/JeannetSS14}
\pagebreak

\section{Preliminaries}
\label{background}

\begin{comment}
	This chapter is mostly focused on trace abstraction $\rightarrow$  It introduces the reader to the concept of trace abstraction. \\
	- Introduce logic, logical variables, terms, formulas, transition formulas with primed and unprimed variables, programs, program states, loops $\rightarrow$  then program-, error traces, feasible and infeasible counterexamples, CFGs, interpolants. \\ - From intuitive to true definitions. \\
	Here the running example from the introduction gets dissected to illustrate the definitions. \\ 
	Further the problems loops can cause are introduced, followed by a definition of loop summaries $\rightarrow$ introduction reflexive transitive closure of a formula 
	15 pages
\end{comment}

This chapter shall introduce our understanding and notation of logic and formulas, programs, control-flow, and other needed background definitions. Furthermore, we will give an overview of the \traceabstraction \cite{10.1007/978-3-642-03237-0_7} counterexample-guided abstraction refinement scheme used in \ultimate.

\subsection{Logical Fundamentals}
We model programs using first order logic, this chapter will introduce the basic notions used in this paper.
\begin{mydef}[Term] 
	Given a vocabulary $V = (\vocab{Var}, \vocab{Const}, \vocab{Fun}, \vocab{pred})$, with countable sets \\ $\vocab{\Var}, \vocab{Const}, \vocab{Fun}, \vocab{pred}$ representing the sets of variables, constant symbols, function symbols, and predicate symbols respectively, we define terms inductively as follows:
	\begin{itemize}
		\item Every $x \in \vocab{Var}$ is a term.
		\item Every $c \in \vocab{Const}$ is a term.
		\item If $t_0, \ldots, t_n$ are terms and $f \in \vocab{Fun}$ being a function symbol with arity $n$, then $f(t_0, \ldots, t_n)$ is a term.
	\end{itemize}
\end{mydef}
Using the definitions of terms, we can introduce first-order logic formulas.

\begin{mydef}[Formula]
	Given vocabulary $V = (\vocab{Var}, \vocab{Const}, \vocab{Fun}, \vocab{pred})$, first-order logic formulas are inductively defined as follows:
	\begin{itemize}
		\item $\bot$ is a formula.
		\item If  $t_0, \ldots, t_n$ are terms, and $p \in \vocab{pred}$ is a predicate symbol with arity $n$, then $p(t_0, \ldots, t_n)$ is a formula.
		\item If $\varphi$ is a formula, then $\neg \varphi$ is a formula.
		\item If $\varphi$ and $\psi$ are formulas, then $\varphi \land \psi$ are formulas.
		\item If $\varphi$ is a formula, and $x \ in \vocab{Var}$ then $\exists x. \varphi$ is a formula.
	\end{itemize}
\end{mydef}

\begin{mydef}[Model]
	Given vocabulary $V = (\vocab{Var}, \vocab{Const}, \vocab{Fun}, \vocab{pred})$, a model $\mathcal{M} = (D, \interpret)$ is a tuple consisting of a nonempty set $D$, called interpretation domain, and an interpretation function \interpret that assigns constants, functions, and predicates over $D$ to symbols in $V$.
\end{mydef}

\begin{mydef}[Assignment of Variables]
	Given vocabulary $V = (\vocab{Var}, \vocab{Const}, \vocab{Fun}, \vocab{pred})$, and domain $D$, an assignment of variable $v \in \vocab{Var}$ is a function $\varv: v \rightarrow D$.
\end{mydef}

\begin{mydef}[Evaluation of Terms]
	Let $V = (\vocab{Var}, \vocab{Const}, \vocab{Fun}, \vocab{pred})$ be a vocabulary, $\mathcal{M} = (D, \interpret)$ a model, and $\varv$ a variable assignment, the evaluation of terms is a function $\eval{\cdot}$ that is inductively defined as:
	\begin{itemize}
		\item For each $x \in \vocab{Var}$, $\eval{x} = \varv(x)$
		\item For each $c \in \vocab{Const}$, $\eval{c} = \interpret(c)$
		\item If $t_0, \ldots, t_n$ are terms, $f \in \vocab{Fun}$, and f has arity $n$ then \\ $\eval{f(t_0, \cdots, t_n)}$ is $\interpret(f)(\eval{t_0}, \ldots, \eval{t_n})$
	\end{itemize}
\end{mydef}

\subsection{Model Checking}

\subsection{Programs}
\begin{figure}[H]
	\centering
	\begin{lstlisting}[language=C++,basicstyle=\ttfamily,keywordstyle=\color{blue}]  % Start your code-block
	
	int x := 0;
	int y := 2;
	int z := 3;
	while x <= 21:
		if x <= 10:
			z := x;
			x := x + y;
			y := y + 1;
		else:
			x := x + 2;
			y := y - 3;
	assert x == 22;
	\end{lstlisting}
	\caption{Program $P$ \\ with \texttt{while} loop.}
	\label{code}
\end{figure}

\begin{figure}[H]
	\centering
	
		\begin{tikzpicture}[%
    ->,
    >=stealth',
    shorten >=1pt,
    auto,
    node distance=3cm,
    scale=1,
    transform shape,
    align=center,
    smallnode/.style={inner sep=1.4},
    initial text =,
    anchor=center]

			\node[state, initial above, initial text =](1){$\loc{1}$};
			\node[state] (head) [below of=1] {$\loc{4}$};
			\node[state] (loopEntry)[left of=head] {$\loc{5}$};
			\node[state] (if)[above left of=loopEntry] {$\loc{6}$};
			
			\node[state] (else)[below left of=loopEntry] {$\loc{10}$};
			\node[state] (loopExit)[right of=head] {$\loc{13}$};
			
			\node[state] (assertTrue)[below left of=loopExit] {$\loc{14}$};
			\node[state, accepting] (assertFalse)[below right of=loopExit] {$\loc{14}$};

			\path (1) edge node []{\st{x:=2}\\ \st{y:=2}\\ \st{z:=3}} (head)
			;
		\end{tikzpicture}
	\caption{Program $P$ \\ with \texttt{while} loop.}
	\label{code}
\end{figure}




\subsection{Trace Abstraction}

\section{Loop Summarization using \qvasr}
\label{qvasr}

\begin{comment}
	\jw{Introduce loop summarization using rational vector addition systems with resets \\
	- What are qvasr? What are qvasr abstractions? what are least upper bounds on abstractions? \\
	- start with qvasr on example that does not need abstraction $\rightarrow$ no relations between variables \\
	- move on to example with relations between variables; show simulation matrix and imaging \\
	- abstractions are an overapproximation \\
	- imprecise because of ignorance of assumptions $\rightarrow$ transfer to next chapter \\
	- usage of running example which is turned to qvasr abstraction \\
	\vspace{1cm}
	20 pages}
\end{comment}

This chapter will introduce the notion of rational vector addition systems (\qvasr) and their extension to \qvasr-abstractions.

\subsection{\qvasr}
A \qvasr of dimension $d$ is a finite set $V \subseteq \{0, 1\}^d \times \mathbb{Q}^d$ of transformers. Each transformer $(\vec{r}, \vec{a}) \in V$ consists of binary \textsl{reset} vector $\vec{r}$, where 0 indicates a reset, and rational \textsl{addition} vector $\vec{a}$. A \qvasr $V$ defines a transition system $(S_V, \rightarrow_V)$, with state space $S_V \subseteq \Q^d$, which can transition between states $\vec{x} \rightarrow_V \vec{x}'$, if $\vec{x}' = \vec{r} * \vec{x} + \vec{a}$ for some transformer $(\vec{r}, \vec{a}) \in V$, with $*$ being the Hadamard product. \par A \qvasr can be used to represent an overapproximation of a given transition formula. A transition formula $F$ is a first-order formula defined over free variables $\vec{x} = x_1, \ldots, x_n$ and $\vec{x}' = x_1', \ldots, x_n'$ that designate the state before and after a transition, where $n$ is called the dimension of the formula. Figure \ref{code} depicts a program containing a \texttt{while} loop from lines 4 - 12 for which we want to compute a loop summary using \qvasr.
\begin{wrapfigure}{L}{0.4\linewidth}
	\vspace{-10pt}
	\centering
	\begin{lstlisting}[language=C++,basicstyle=\ttfamily,keywordstyle=\color{blue}]  % Start your code-block
	
	int x := 0;
	int y := 2;
	int z := 3;
	while x <= 21:
		if x <= 10:
			z := x;
			x := x + y;
			y := y + 1;
		else:
			x := x + 2;
			y := y - 3;
	assert x == 22;
	\end{lstlisting}
	\caption{Program $P$ \\ with \texttt{while} loop.}
	\label{code}
\end{wrapfigure}
To accomplish this we need to compute a \qvasr for every path through the loop. In this example there are two paths created by the \texttt{if else} statement.
Beginning with the \texttt{else} branch, we extract the transition formula:
\begin{equation*}
	G= \ (x \leq 20 \land\ x > 10 \land x' = x + 2 \land y' = y - 3)
\end{equation*}
The variable $x$ is not reset but incremented by 2, variable $y$ is not reset and decremented by 3.
We get reset vector $
\vec{r}_G = \  
\begin{bmatrix}
	1 \\
	1 
\end{bmatrix}
$
and addition vector $
\vec{a}_G = \ 
\begin{bmatrix}
	2 \\
	-3 
\end{bmatrix}$ \\
$G$ can therefore be modeled by the \qvasr
$V_G = 
\begin{Bmatrix}
	\begin{pmatrix}
		\begin{bmatrix}
			1 \\
			1
		\end{bmatrix},
		\begin{bmatrix}
			2 \\
			-3
		\end{bmatrix}
	\end{pmatrix}
\end{Bmatrix}
$ \par \vspace{2pt}
For the remainder of this proposal we will use the following, more intuitive notation of \qvasr:
\begin{equation*}
	V_G = 
	\begin{Bmatrix}
		\begin{bmatrix}
			x \\
			y
		\end{bmatrix} \rightarrow_{V_G}
		\begin{bmatrix}
			x + 2 \\
			y - 3
		\end{bmatrix}
	\end{Bmatrix}
\end{equation*}

From the \texttt{if} branch we extract the following transition formula:
\begin{equation*}
	H= \ (x \leq 10 \land x' = x + y\ \land\ y' = y + 1 \land z' = x)
\end{equation*} In contrast to $G$ there are no constant increments to variables $x$ and $z$. They are changed by an arbitrary amount. A \qvasr cannot represent $H$ by a  \qvasr because they are only able to model transitions with constant resets and increments. 

\subsection{\qvasr-Abstraction}

We can, however, represent an overapproximation of $H$ as a \qvasr. This is done using linear simulations. Given a transition formula $F$ of dimension $n$ and a \qvasr $V$ of dimension $m$, a linear simulation from $F$ to $V$ is a linear transformation matrix: 
$S = 
\begin{bmatrix}
	s_{1 ,1} & \ldots & s_{1, n} \\
	\vdots & \ddots & \vdots \\
	s_{m ,1} & \ldots & s_{m, n} \\
\end{bmatrix}$ 
such that for all transitions $\vec{x} \rightarrow_F \vec{x}'$ we have $S\cdot\vec{x} \rightarrow_V S\cdot\vec{x}'$. Meaning, every transition $\vec{x} \rightarrow_F \vec{x}'$ can be represented in $V$ by a matrix multiplication of $S$ and $\vec{x}$ and $\vec{x}'$. The tuple $(S, V)$ is called a \qvasr-abstraction. \par For $H$ we need to calculate both the linear transformation matrix $S_H$ and the \qvasr $V_H$. We know that a \qvasr consists of pairs of reset and addition vectors, we use the formula $S \cdot \vec{u} \rightarrow_V S \cdot \vec{u}$ to get a transition in $V_H$ as $S\cdot\vec{x}' = \vec{r}*S\cdot\vec{x} + \vec{a}$. \\ To get variables that are reset, we consider $\vec{r}$ as the zero vector $\vec{0}$, which forms a linear set of equations that model a set of resets:
\begin{equation*}
	Res_H = \left\{ (\s, a) : H \models \s \cdot \p = a \right\}	
\end{equation*}
For the set of additions, we consider $\vec{r}$ as the constant one vector $\vec{1}$ which forms the set of additions as:
\begin{equation*}
	Inc_H = \left\{(\s, a) : H \models \s \cdot \p = \s \cdot \up + a\right\}	
\end{equation*}
We use the updates to variables: $x' = x + y, \ y'= y + 1\ z' = x$ , found in $H$, to solve the equations in the sets for \s, resulting in: 
\vspace*{-0.5em}
\begin{center}
	\begin{minipage}{0.5\linewidth}
		\begin{equation*}
			Res_H = \left\{ (\begin{bmatrix} -a & a & a \end{bmatrix}, a) \right\}\
		\end{equation*}
	\end{minipage}
	\begin{minipage}{0.4\linewidth}
		\begin{equation*}
			Inc_H = \left\{ (\begin{bmatrix} 0 & a & 0 \end{bmatrix}, a) \right\}\ 
		\end{equation*}
	\end{minipage}
\end{center}
Observe that these form a vector space. Because we want to construct a linear transformation, we need to compute a basis for each space:
\vspace*{-1em}
\begin{center}
	\begin{minipage}{0.5\linewidth}
		\begin{equation*}
			Res_H = \{(
			\NiceMatrixOptions{code-for-first-row=\scriptstyle}
			\begin{bNiceMatrix}[first-row=1]
				x & y & z \\
				-1 & 1 & 1 
			\end{bNiceMatrix}, 1)\}
		\end{equation*}
	\end{minipage}
	\begin{minipage}{0.4\linewidth}
		\begin{equation*}
			Inc_H = \{(
			\NiceMatrixOptions{code-for-first-row=\scriptstyle}
			\begin{bNiceMatrix}[first-row=1] 
				x & y & z \\
				0 & 1 & 0 
			\end{bNiceMatrix}, 1)\}
		\end{equation*}
	\end{minipage}
\end{center}
Each row of the basis corresponds to a variable in the formula such that we can derive relations between variables. From the reset base we can deduce that the sum $-x + y + z$ is reset and incremented by $1$, meaning that after each transition of $H$ we know that $-x + y + z = 1$ holds. From the basis of additions we derive that $y$ is not reset and incremented by $1$. To form the linear transformation matrix $S_H$ we combine these rows to one coherent matrix. We see that the basis of resets and additions contain only one vector each, resulting in only one reset addition pair for $V_H$. As we have seen $-x + y + z$ is reset and $y$ is not. We get reset vector $\vec{r} = \begin{bmatrix} 0 \\ 1 \end{bmatrix}$ and because $-x + y + z$ is incremented by $1$, same as $y$, we get addition vector $\vec{a} = \begin{bmatrix} 1 \\ 1 \end{bmatrix}$. Resulting in the \qvasr-abstraction depicted in figure \ref{vasr  H}.
\vspace*{-2em}
\begin{figure}[H]
	\begin{center}
		\begin{minipage}{0.3\linewidth}
			\begin{equation*}
				S_H = \begin{bmatrix} -1 & 1 & 1 \\ 0 & 1 & 0 \end{bmatrix}
			\end{equation*}
		\end{minipage}
		\begin{minipage}{0.6\linewidth}
			\begin{equation*}
				V_H = \begin{Bmatrix} \begin{bmatrix} - x + y + z \\ y \end{bmatrix} \rightarrow_{V_H} \begin{bmatrix}	1 \\ y + 1 \end{bmatrix} \end{Bmatrix}
			\end{equation*}
		\end{minipage}
		\caption{\qvasr-abstraction of transition formula $H$.}
		\label{vasr H}
	\end{center}
\end{figure}
\vspace*{-2em}For transition formulas with constant increments, such as $G$, an identity matrix $I$ is used as simulation matrix. \par
The effect on variables by the loop in $P$ can be represented by the following conjunction: 
\begin{equation*}
	x \leq 20 \land (H \lor G)
\end{equation*}
With $x \leq 20$ being the loop guard. We have already computed \qvasr-abstractions for $G$ and $H$. These, however, only model the effect on variables in their respective branch of the \texttt{if else} statement. To get the most precise overapproximation of the whole loop's behavior we need the \textsl{best} \qvasr-abstraction $(\tilde{S}, \tilde{V})$ that simulates every branch in the loop.
We impose a partial order $\preceq$ on \qvasr-abstractions $(S_1, V_1)$ and $(S_2, V_2)$, with $(S_1, V_1) \preceq (S_2, V_2)$ if $(S_2, V_2)$ simulates $(S_1, V_1)$. The best abstraction $(\tilde{S}, \tilde{V})$ is the least upper bound with regard to $\preceq$, meaning $(S, V) \preceq (\tilde{S}, \tilde{V})$ for all \qvasr-abstractions $(S, V)$. The abstraction $(\tilde{S}, \tilde{V})$ computed by iteratively \textsl{joining} abstractions. Joining two \qvasr-abstractions results in a single abstraction simulating both. Figure \ref{vasr} shows the best \qvasr-abstraction of the loop in $P$, which is the result of joining $G$'s and $H$'s abstractions. \\
\begin{center}
	\begin{figure}[H]
		\begin{align*}
S_P &= 
\NiceMatrixOptions{code-for-first-row=\scriptstyle}
	\begin{bNiceMatrix}[first-row=1]
		x & y & z \\
		-1 & 1 & 1 \\
		0 & 1 & 0
	\end{bNiceMatrix}, \ \\ \\
    V_P &= \begin{Bmatrix}
        \begin{pmatrix}
              \begin{bmatrix}
                    0 \\
                    1
               \end{bmatrix},
               \begin{bmatrix}
                     1 \\
                     1
               \end{bmatrix}
        \end{pmatrix}, \\ \\
        \begin{pmatrix}
               \begin{bmatrix}
                    1 \\
                    1
               \end{bmatrix},
               \begin{bmatrix}
                    -5 \\
                    -3
              \end{bmatrix}
        \ \end{pmatrix}
    \end{Bmatrix}
\end{align*}
%\caption{\qvasr abstraction $A_p = (S, V)$ of $P$.}
		\caption{Best \qvasr-abstraction of the loop in $P$.}
		\label{vasr}
	\end{figure}
\end{center}
Using this \qvasr-abstraction we can derive the following transition formula as loop summary:
\begin{align*}
	\exists k_1, k_2.\ &((-x' + y' + z' = 1\ \lor\ -x' + y' + z' = -x + y + z - 5k_2)\ \land\ y' = y + k_1 - 3k_2)\ \\ &\lor\ x' = x\ \land\ y' = y\ \land\ z' = z
\end{align*}


\begin{comment}
	\section{Extension to \qvasrs}
	\label{qvasrs}
	
\jw{Introduction qvasrs $\rightarrow$ summary precision improvement \\
	- What are qvasr? How to compute their reachability relation $\rightarrow$ Parikh image? \\ 
	 How do they improve precision? \\
	- running example to qvasrs \\
		\vspace{1cm}
		15 pages}
\end{comment}

\section{Trace Abstraction with \qvasr}
A loop in a program could potentially introduce infinitely many program traces that the trace abstraction scheme would have to prove infeasible. In this chapter we will introduce methods of utilizing \qvasr-based loop summarization to minimize the number of traces created by loops. We present two approaches, the first being an implementation of the \textsl{abstract} step of trace abstraction as seen in Figure \ref{traceAbstractionScheme} utilizing loop summaries directly in traces, the second being a way of transforming a given program's control-flow graph to replace loops entirely by loop summaries.
\label{qvasrAbstracion}
\subsection{Accelerated Interpolation}

\begin{comment}
	\jw{Accelerated Interpolation as an extension of trace abstraction is introduced $\rightarrow$ Show how to utilize loop summaries within trace abstraction \\
	- Error traces  $\rightarrow$ loop relations  $\rightarrow$ reflexive transitive closures  $\rightarrow$ meta traces  $\rightarrow$ interpolation on meta traces \\
	- "We implemented accelerated interpolation in an earlier project using a myriad of loop summarization techniques... " $\rightarrow$ Introduce various loop summarization techniques (FastUPR, Jordan, Werner) NOT in-depth, only their ideas, but show their pros and cons  $\rightarrow$ The cons build a bridge to the next chapter \\
	- running example will be transformed to a meta trace WITHOUT actual loop summarizations on edges \\
	\vspace{1cm}
	22 pages $\rightarrow$ recycled from earlier}
\end{comment}

Reconsider program $P$ \ref{code} from the previous chapter.
To prove its safety \traceabstraction finds error trace $\tau_1$, as before, we detect a loop with $\loc{3}$ and minimal loop trace $\tau_{L_1}$ from which we construct the loop relation $\psi_{L_1}$.
It is evident that there is only one path through the loop, resulting in a loop without branching.
We consequently compute the loop acceleration $\psi^*_{L_1}$.

The loop acceleration contains every looping trace, meaning it is possible to replace the whole loop in the error trace by that acceleration, modelling a relation consisting of every loop trace.
\dd{Why not explain the idea of meta trace first and then give a definition?}

\begin{mydef}
	Given an error trace $\tau: s_0, s_1, \ldots, s_n$ containing loop $\tau_L: s_i, s_{i+1}, \ldots, s_j$ with loop head $\loc{L}$.
	A meta trace $\bar{\tau}$ is derived from $\tau$ by replacing $\tau_L$ with the loop acceleration.
	Furthermore, the last occurrence of $\loc{L}$ is replaced by a new loop exit $\loc{L}'$.
\end{mydef}
\ts{Now you have a mixture of statements and relations. Do you want to allow this (explanation required) or use transformulas in general?}
\dd{Which loop acceleration? Will you talk about properties of loop accelerations? What is a loop acceleration? Is it "the closure"? Or is it \emph{some} relation over program states that has some properties relative to a closure?}

\begin{comment}
The error trace $\tau_1$ creates the meta trace $\bar{\tau_1}$:
\begin{figure}[H]
\begin{tikzpicture}[%
->,
>=stealth', shorten >=1pt, auto,
node distance=2.5cm, scale=1,
transform shape, align=center,
smallnode/.style={inner sep=1.4}
initial text =]

\node[state](1){$\loc{1}$};

\node[state] (2) [right of=1] {$\loc{2}$};

\node[state] (3) [right of=2] {$\loc{3}$};

\node[state] (4) [right of=3] {$\loc{3}'$};

\node[state] (5) [right of=4, xshift=0.5cm] {$\loc{6}$};

\node[state] (6) [right of=5, xshift=0.5cm] {$\loc{7}$};

\path (1) edge node {\texttt{x := 0}} (2); \\
\path (2) edge node {\texttt{y := 1}} (3); \\
\path (3) edge node {$\psi^*_{L_1}$} (4);\\
\path (4) edge node[] {\texttt{!x <= 50}} (5); \\
\path (5) edge node {\texttt{y != 103}} (6); \\
;
\end{tikzpicture}
\captionof{figure}{Meta trace $\bar{\tau_1}$ generated from $\tau_1$ using $\psi^*_{L_1}$.}
\end{figure}
\end{comment}

\begin{figure}[H]
	\begin{center}
		\begin{tabular}{ccccccccccc}
			\loc{1} & \st{x:=0} & \loc{2} & \st{y:=1} & \loc{3} & \accel{1} & $\loc{3}'$ & \st{x>50} & \loc{6} & \st{y!=103} & \loc{7} \\
		\end{tabular}
	\end{center}
	\captionof{figure}{Meta trace $\bar{\tau_1}$ generated from $\tau_1$ using $\psi^*_{L_1}$.}
\end{figure}
We can now analyze this meta trace for feasibility using an SMT-solver such as SMTInterpol\cite{Zitat03} or z3\cite{z3}. We get the following labelling:

\begin{figure}[H]
	\centering
	\input{fig/fig_metatrace_and_proof.tex}
\end{figure}
\captionof{figure}{Meta trace $\bar{\tau_1}$ generated from $\tau_1$ and $\psi^*_{L_1}$ and its infeasibility proof.}
\label{fig:ex:t0:infproof}
%\dd{Do not capitalize captions. They are just normal sentences.}


\dd{Do you still need the line above?}

We cannot, however, use $I_{\bar{\tau_1}}$ to disprove $\tau_1$ as we need an interpolant for each location in the original trace.
\dd{say which one is missing ;)}
To remedy this, we derive an inductive interpolant sequence $I_{\tau_1}$ by applying the post operator.
\dd{Explain why we need post for the before-location, use the example here}

\newcommand{\accels}[1]{\ensuremath{\psi^{*}_{#1}}}
Given
\begin{itemize}[topsep=0pt,itemsep=-1ex,partopsep=1ex,parsep=1ex]
	\item an error trace $\tau: s_0 s_1 \ldots s_i \ldots s_j \ldots s_n$ where \loc{i} is a loop head for the loop $L$ spanning from $s_i$ to $s_j$,
	\item a loop relation $\psi_L$ for loop $L$,
	\item a corresponding loop acceleration \accels{L},
	\item the meta trace $\bar{\tau}: s_0 s_1 \ldots s_{i-1} \ \accels{L} \ s_{j+1} \ldots s_n$ derived from $\tau$ and \accels{L}, and
	\item the infeasibility proof $I_{\bar{\tau}}: \{\top, I_1, I_2, \ldots , I_i, I_{\psi^*_{L}}, \ldots , I_{n-1}, \bot \}$ for $\bar{\tau}$.
\end{itemize}
\dd{Fix indices s.t. we have the interpolant (sic!) before and after the loop acceleration}

To construct an inductive proof of infeasibility for $\tau$ we need inductive interpolants for the loop statements $s_i, \ldots , s_j$ that were replaced by $\psi^*_{L}$.

Firstly, compute post($I_{\psi^*_L}$, $\psi^*_L$) as the loop entry interpolant $I_{\loc{L}}$.
From there keep applying the post operator with the previous location's interpolant and the following program statement.

We get the inductive interpolant sequence
\begin{equation*}
	I_\tau: \{\top,I_1,I_2, \ldots ,\ \underbrace{post(I_i, \accels{L})}_{I_{i}^*},\ \ \underbrace{post(I_{i}^*, s_i)}_{I_{i+1}^*},\ \ldots ,\ \underbrace{post(I_{j-1}^*, s_j)}_{I_{j}^*},I_{j+1}, \ldots ,I_{n-1}, \bot \}
\end{equation*}
which can now be used by trace abstraction to refine the interpolant automaton.
\ts{Explain why this works and why it is necessary.}

We compute the missing interpolants for example program trace $\tau_1$ as follows:
\dd{Strange wording. This is just the continuaton of the example, right?}
\begin{comment}
\begin{figure}[H]
\centering
\input{fig/fig_iip.tex}
\captionof{figure}{Program trace $\tau_1$ of $P_1$ with inductive infeasibility proof.}
\end{figure}
\end{comment}

\begin{figure}[H]
	\begin{center}
		\input{fig/fig_iip_new.tex}
	\end{center}
\end{figure}
\captionof{figure}{Program trace $\tau_1$ of $P_1$ with inductive infeasibility proof.}
\label{fig:ex:t0:infproof2}

% \ts{The simplification of the interpolant at $\loc{6}$ is wrong.}

\label{accelInterpol}
\subsection{Control-Flow Graph Transformation}

\begin{comment}
	\jw{Introduction to Ultimate, Accelerated Inteprolation in Ultimate, Qvasr and Qvasrs libraries \\
	- Short introduction: what is Ultimate? $\rightarrow$ software verification framework with toolchains, such as trace abstraction \\
	- accelerated interpolation in ultimate $\rightarrow$ How does it work $\rightarrow$ get trace $\rightarrow$ loop detector $\rightarrow$ accelerator $\rightarrow$ meta trace transformer $\rightarrow$  interpolator (maybe as diagram) \\
	- Focus on qvasr $\rightarrow$ how does that library work? $\rightarrow$ same for qvasrs \\
	\vspace{1cm}
	15 pages}
\end{comment}

Another approach of utilizing \qvasr summarization is, instead of applying it to traces, to change trace abstraction's input control-flow graph, such that computed error traces already contain summaries. Given a program $P$ and its control-flow graph $G_P$, we devise a depth-first search algorithm to find loops. We introduce a marking function $\alpha: Loc \rightarrow \{nv, v\}$, over the control-flow graph's locations, with $nv$ meaning not visited, $v$ meaning visited. We initialize a stack $\Gamma$, an empty set of traces $T$, and start the search with the initial location $\ell_{init}$, adding each transition $\pi_i$, where $src(\pi_i) = \ell_{init}$ to $\Gamma$. The location $\ell_{init}$ is now marked $v$ and we repeat the following loop detection scheme until $\Gamma$ is empty:

\begin{enumerate}
	\item Pop new transition $\pi_j$ from $\Gamma$.
	\item Check if $\ell_j = tgt(\pi_j)$ is marked.
	\item If yes, we have found a loop, with $\ell_j$ being the entry point.
		We begin to backtrack by initializing a new stack $\Gamma_j$, pushing every transition $\pi_k$ where $tgt(\pi_k) = src(\pi_j)$ to it and construct the new trace $\tau = \pi_j$ . For finding the loop entry and with that the trace of the loop, we repeat the following steps: While $\Gamma_j$ is not empty, pop new transition $\pi_h$ from $\Gamma_j$ and check if $src(\pi_h) = \ell_j$:
		\begin{enumerate}
			\item If that is the case we found a loop cycle and add $\tau$ to $T$. 
			\item If not, we construct new traces for each $\pi_d$, where $tgt(\pi_d) = src(\pi_h)$, by copying $\tau$ and concatenating $\pi_d$ to it and push each $\pi_d$ to $\Gamma_j$
		\end{enumerate}
	\item If no, add transitions $\pi_l$ with $src(\pi_l) = \ell_j$ to $\Gamma$.
\end{enumerate}
After the search, the set of traces $T$ contains all loops in program $P$. As there can be multiple traces from a loop entry location $\ell_i$, we have to construct a disjunction of the transition formulas beginning in $\ell_i$:
Assume we have two traces $\tau_1 = \pi_{11}, \ldots, \pi_{1n}$ and $\tau_2 = \pi_{21}, \ldots \pi_{2m}$ in $T$, where $src(\pi_{11}) = scr(\pi_{21})$, we construct the formula $\tau_1 \lor \tau_2$ to represent the loop.
This disjunction can now be used in the \qvasr summarization scheme, as defined before.  \par
Recall the program $P$ and its control-flow graph $G_P$ as illustrated in \ref{codeWithAss} and \ref{cfg:P:Ass}. Using $G_P$ and the loop detection scheme, we extract the following loop traces for loop entry $\ell_4$: 
\begin{align*}
	\tau_1 &= \st{x<=20}, \st{x<=10}, \st{z:=x;x:=x+y} \\
	\tau_2 &= \st{x<=20}, \st{x>10}, \st{x:=x+2;y:=y-3}
\end{align*}
Using \qvasr summarization, we get the summary $\psi$ shown before in Figure \ref{loopTF}, which we can now use to replace every transition occurring in either $\tau_1$ and $\tau_2$. We get transformed control-flow graph $\bar{G}_P$:

\begin{figure}[H]
	\centering
	
\begin{tikzpicture}[%
	->,
	>=stealth',
	shorten >=1pt,
	auto,
	node distance=3.25cm,
	scale=0.9,
	transform shape,
	align=center,
	smallnode/.style={inner sep=1.4},
	initial text =,
	anchor=center]
	
	\node[state, initial above, initial text =](1){$\loc{1}$};
	\node[state] (head) [below of=1] {$\loc{4}$};
	\node[state] (loopExit)[right of=head] {$\loc{13}$};
	
	\node[state] (assertTrue)[below left of=loopExit] {$\loc{14}$};
	\node[state, accepting] (assertFalse)[below of=loopExit] {$\loc{err}$};
	
	\path (1) edge node []{\st{x:=0;}\\ \st{y:=2;}\\ \st{z:=3;}} (head)
	(head) edge node []{\st{x>20}} (loopExit)
	(head) edge[loop left] node {\st{$\psi$}} (head)
	
	(loopExit) edge node []{\st{x==22}} (assertTrue)
	(loopExit) edge node []{\st{x!=22}} (assertFalse)
	;
\end{tikzpicture}
	\label{cfg_trans}
	\caption{The transformed control-flow graph $\bar{G}_P$ that is the result of the control-flow graph transformation of $G_P$}
\end{figure}
This transformed control-flow graph can now be put into the trace abstraction scheme, seen in	Figure \ref{traceAbstractionScheme}, and be used to prove safety by proving possible error traces infeasibility.



\label{icfgTransformation}

\section{Evaluation}
\label{eval}

We have implemented a \qvasr based loop summarization library in the software verification framework \ultimate, that uses trace abstraction to prove safety, as an adaption of the previously defined methods. We implemented \qvasr usage as an abstraction method in  accelerated interpolation and as a control-flow graph transformer. To test performance of our \qvasr based loop summarization, we used a set of 1766 example $C$ programs, which are part of the \texttt{sv-comp} \cite{svcomp} program set. Furthermore, we compared it to another summarization scheme, based on transforming transition formulas into a Jordan normal form matrix \cite{DBLP:conf/popl/JeannetSS14}. \par
We used \ultimate Automizer version 0.2.2-d966a43b, with time limit: 900 seconds, memory limit: 8000 MB, using two CPU cores of type AMD Ryzen Threadripper 3970X 32-Core Processor, on Linux-5.11.22-4-pve-x86\_64-with-glibc2.31, with frequency: 4549 MHz and using 137439 MB of RAM. In this Chapter we show the and interpret the results. We begin with the evaluation of trace abstraction using accelerated interpolation, followed by trace abstraction with control-flow graph transformation.

\section{Results of using Accelerated Interpolation}
We present in this Section our results on applying accelerated interpolation using either the Jordan based loop summarization or \qvasr versus trace abstraction without accelerated interpolation. We computed the results as portrayed in Figure \ref{table_acc}
\begin{figure}[H]
	\centering
		\begin{tabular}{cccc}
			\toprule
			& No summarization & Jordan & \qvasr \\
			\cmidrule{1-4}
			Correct Results & 508/859 & 355/859 & 356/859\\
			Wrong Results & 0/859 & 0/859 & 14/859\\
			Timeout & 268/859 & 247/859 & 348/859 \\
			Solve Time &  274000s & 255000s & 349000s
		\end{tabular}
	\caption{Results of benchmarking trace abstraction using accelerated interpolation without loop summarization, with Jordan based loop summarization, and \qvasr loop summarization}
	\label{table_acc}
\end{figure}
Using accelerated interpolation, our \qvasr library was able to solve 20 programs that neither trace abstraction without summarization, nor the Jordan based summarization could solve. These programs are mostly containing variable assigments of the form \st{x:=x+y}. We derive that trace abstraction without summarization is unrolling each loop iteration, leading to infinitely many traces, and that Jordan is not able to compute a matrix needed for the technique, falling back to unrolling as well. Our \qvasr based implementation however can construct a summary in able time. We see, nonetheless, that the \qvasr summarization technique produces a few wrong results, these are caused by $C$ programs whose variables are unsigned, as they produce a formula containing modulo operations, with which our \qvasr implementation struggles in some cases, such as inherit issues with the accelerated interpolation library.

\section{Results of using Control-Flow Graph Transformation}
In contrast to accelerated interpolation, which uses traces for computing Floyd-Hoare annotations, we also implemented the \qvasr summarization library as a preprocessor that transforms a given program's control-flow graph by replacing loops with their summary. The results are depicted in Figure \ref{table_tff}.
\begin{figure}[H]
	\centering
	\begin{tabular}{cccc}
		\toprule
		& No summarization & Jordan & \qvasr \\
		\cmidrule{1-4}
		Correct & 508/859 & 516/859 & 518/859 \\
		Wrong & 0/859 & 0/859 & 6/859 \\
		Timeout & 268/859 & 234/859 & 232/859 \\
		Solve Time &  274000s & 241000s & 242000s
	\end{tabular}
	\caption{Results of benchmarking trace abstraction using control-flow graph transformation without loop summarization, with Jordan based loop summarization, and \qvasr loop summarization}
	\label{table_tff}
\end{figure}
We see that in this test case set, that both loop summarization techniques solved more programs than trace abstraction alone, whereas our \qvasr summarization contributed to solving 16 programs that neither trace abstraction solo nor the Jordan based summarization could solve. Though, there are again wrong solutions, that are caused by the same reasons as before. \par 

\section{Outlook and Future Work}
\label{futrWork}

In this thesis we have shown a technique to compute a loop summary using \qvasr, integrated that technique into trace abstraction, by using it in the accelerated interpolation scheme and as a method to transform a control-flow graph by replacing loops with summaries. There are, however, still some areas that can be improved upon. In this section we introduce possible improvements to \qvasr loop summarization and its usage. \\ \par
\subsection{Using \qvasrs}
In chapter \ref{qvasrs} we have shown that there is an extension to \qvasr that factor in assumptions in a loop's transition formula, such as the expressions of the loop guard and if else statements. It is possible to integrate \qvasrs into our implemented summary utilization schemes, accelerated interpolation and control-flow graph transformation, as follows: \\ \par
For accelerated interpolation, assume we are given a trace $\tau$ with a loop, whose transition formula we extract as $\tau_L$, we compute a \qvasrs $\mathcal{V} = (P, E)$ for $\tau_L$, now because a \qvasrs is a graph structure, we need to compute its reachability relation to summarize the loop. To solve this task, we can utilize Haase and Halfon \cite{DBLP:conf/rp/HaaseH14} proposed polytime procedure that computes a series of formulas from generalized Parikh images that, as conjunction, form a summary of the \qvasrs. \\ \par
For \qvasrs usage in control-flow graph transformation, we adapt our approach by, instead of replacing the whole loop by its computed summary, inserting the \qvasrs instead. We illustrate this approach by transforming the given example program $P$ as seen in Figure \ref{codeWithAss} with its control-flow graph $G_P$ as depicted in Figure \ref{cfg:P:Ass}, to the control-flow graph, seen in Figure \ref{qvasrs_cfg}
\begin{figure}[H]
	
\begin{tikzpicture}[%
	->,
	>=stealth',
	shorten >=1pt,
	auto,
	node distance=3.25cm,
	scale=0.9,
	transform shape,
	align=center,
	smallnode/.style={inner sep=1.4},
	initial text =,
	anchor=center]
	
	\node[state, initial above, initial text =](1){$\loc{1}$};
   	\node[draw, ellipse](vasrs1) [below of=1] {$x \leq 10$};
	\node[draw, ellipse](vasrs2) [left of=vasrs1, xshift=-4cm] {$10 < x \land x \leq 20$};
	\node[state] (loopExit)[right of=head,  xshift=2cm] {$\loc{13}$};
	\node[state] (assertTrue)[below left of=loopExit] {$\loc{14}$};
	\node[state, accepting] (assertFalse)[below of=loopExit] {$\loc{err}$};
	
	\path (1) edge node []{\st{x:=0;}\\ \st{y:=2;}\\ \st{z:=3;}} (head)
	(vasrs1) edge node []{\st{x>20}} (loopExit)
	(vasrs2) edge[bend left] node {\st{-x+y+z:=1;} \\ \st{y:=y+1}} (vasrs1)
	(vasrs1) edge[bend left] node[] {\st{-x+y+z:=-x+y+z-5;}\\\st{y:=y-3}} (vasrs2)
	(loopExit) edge node []{\st{x==22}} (assertTrue)
	(loopExit) edge node []{\st{x!=22}} (assertFalse)
	
	(vasrs1) edge[loop below] node[] {\st{-x+y+z:=-x+y+z-5;}\\ \st{y:=y-3}}(vasrs1)
	(vasrs2) edge[loop above] node[] {\st{-x+y+z:=1;}\\ \st{y:=y+1}}(vasrs2)
	;
	
\end{tikzpicture}
	\label{qvasrs_cfg}
	\caption{Control-flow graph $\bar{G}_P$ created from \ref{cfg:P:Ass} by replacing the loop with a \qvasrs.}
\end{figure}
From this transformed control-flow graph we can now construct state assertions that adhere to assumptions in the program.
\subsection{Heuristics in Accelerated Interpolation}

\begin{comment}
	To compute a loop summary from a \qvasrs, one has to calculate the reachability relation of the automaton. Haase and Halfon \cite{DBLP:conf/rp/HaaseH14} proposed a polytime procedure that computes a series of formulas from computed Parikh images that, as conjunction, form a summary of the system. This procedure can be adapted to work with \qvasrs
\end{comment}


\section{Conclusion}
\label{concl}
In this thesis we have introduced a loop summarization scheme using rational vector addition systems with resets based on the findings by Kincaid et al. \cite{DBLP:conf/cav/SilvermanK19}, adapted it to be used in the automata-theoretic trace abstraction scheme \cite{10.1007/978-3-642-03237-0_7, 10.1007/978-3-642-39799-8_2, 10.1145/1706299.1706353} for proving unreachability of error states in programs. We integrated our \qvasr approach on two distinct levels in trace abstraction:
\begin{itemize}
	\item We use \qvasr summarization directly on traces in the accelerated interpolation library as a method of abstracting traces by finding repeating sequences of transitions and summarizing them.
	\item We implemented a control-flow graph transformer that uses \qvasr summarization to find loops in a given program's control-flow graph and replacing them with the summary. The transformed control-flow graph is then used in trace abstraction.
\end{itemize} 

We evaluated our techniques using the program verification framework \ultimate \cite{Zitat02} and compared it to a loop summarization techniques using Jordan normal form computation and trace abstraction without summarization. The \qvasr library performed admirably, solving 20 programs that neither method could solve in the accelerated interpolation framework, and 16 when used as a control-flow graph transformer.
Furthermore, we presented possible improvements to the \qvasr summarization scheme, such as extending \qvasr to \qvasrs to improve precision of our approximations. We proposed further improvements for accelerated interpolation in the form of possible heuristics that dictate when to summarize a loop and whether it is advantageous to split disjunctive loops, that are loops with branching, into separate summaries.



\pagebreak
\bibliographystyle{ieeetr}
\bibliography{bib/bib}

\end{document}