\documentclass[11pt,
a4paper,
parskip=half, % This is some extra vertical space between paragraphs, the suggestion is 2cm which is really ugly, so we use what koma script gives us
% you can also set it to full for even more space. But there is a bad tex style decision: parskip also changes the spacing between listitems such as
% enumerate and itemize. For this purpose we include the enumitem package and set itemsep=.5em, of course you can change this
BCOR=10mm, % BCOR is binding correction
ngerman, english
% if you'd rather have a one sided thesis, add `oneside' to the documentclass
% oneside,
% ngerman is needed for hyphenation if the thesis contains parts written in German, switch order with english if you write mainly in English.
% Remember to change order in the babel package (below) as well.
% Last language is the preferred one.
]{scrbook}
\usepackage[english]{babel} % If you write mainly in English change order to ngerman, english. Also change that in the documentclass options above.
% Include of titling must happen before \title etc.
% that's why it's not in setup.tex
\usepackage{titling}

\usepackage[utf8]{inputenc}
\usepackage{xspace}
\usepackage{tabularx}
\usepackage[%
hyperindex,%
plainpages=false,%
pdfusetitle]{hyperref}
\usepackage[all]{hypcap}
\usepackage{mathtools}
\usepackage{algorithm}
\usepackage[noend]{algpseudocode}
\usepackage{cite}
\usepackage{booktabs}
\usepackage{url}
\usepackage{listings}
\usepackage{enumitem}
\usepackage{amsthm}
\usepackage{amsmath}
\usepackage{tikz}
\usetikzlibrary{positioning,shapes.geometric, arrows.meta ,automata, decorations.pathreplacing, calc, fit, backgrounds}
\usepackage{pgf}
\usepackage{slantsc}
\usepackage{geometry}
\usepackage{amssymb}
\usepackage{subcaption}
\usepackage{float}
\usepackage{pgf}
\usepackage{slashbox}
\usepackage{pgfgantt}
\usepackage{wrapfig}
\usepackage{pdflscape}
\usepackage{xcolor}
\usepackage{xparse}
\usepackage[%disable,%
colorinlistoftodos,%
color=cyan!50!white,%
bordercolor=cyan!50!black]{todonotes}

\usepackage{varwidth}
\usepackage[most]{tcolorbox}% http://ctan.org/pkg/tcolorbox
\tcbuselibrary{skins,breakable}
\usepackage{comment}
\usepackage{makecell}
\usepackage{stmaryrd}
\usepackage{nicematrix}
\usepackage{wrapfig}
\usepackage[nounderscore]{syntax}

\DeclareSymbolFont{matha}{OML}{txmi}{m}{it}% txfonts
\DeclareMathSymbol{\varv}{\mathord}{matha}{118}


%%%%%%%%%%%% Colors
%% a somewhat friendly scheme for 5 different colors
\definecolor{g1}		{RGB}{215,25,28} % a kind of red
\definecolor{g2}		{RGB}{253,174,97} % a kind of orange
\definecolor{g3}		{RGB}{255,255,191} % a kind of yellow
\definecolor{g4}		{RGB}{171,217,233} % a kind of light blue
\definecolor{g5}		{RGB}{44,123,182} % a kind of dark blue

\definecolor{gr1}		{RGB}{250, 250, 250}
\definecolor{gr2}		{RGB}{229, 229, 229} % some grey

% color of interpolants
\definecolor{itpGreen}  {rgb}{0,1,0}
\definecolor{grey}      {RGB}{200,200,200}


%color for pictures
\colorlet{outlineblue}		{g5}
\colorlet{fillblue}			{g4}
\colorlet{darkback}			{gr2}
\colorlet{lightback}		{gr1}
\colorlet{stmtcolor}		{gr2} %default statement color
\colorlet{subgraphcolor}	{g3} %default statement color
\colorlet{colexamtitle}     {black} % Example block title color
\colorlet{colexamline}      {g1} % Example block sideline color
\colorlet{itp}              {itpGreen}



%%%%%%%%%%%%% Statements and labels Trace Abstraction Style
\tikzstyle{st} = [%
font=\ttfamily,%
shape=rectangle,%
rounded corners=.5em,%
fill=stmtcolor,%
inner xsep=.3em,%
inner ysep=0em, %
text height=2ex, %
text depth=.6ex,
]


\newcommand{\tikzstmt}[3]{{%
		\tikz[baseline]{%
			\node[st,fill=#2] at (0,.64ex){%
				\hspace{.3em}\texttt{\strut#3#1}\hspace{.3em}\strut};}
}}

\newcommand{\stcol}[2]{\tikzstmt{#1}{#2}{}}
\newcommand{\stsmcol}[2]{\tikzstmt{#1}{#2}{\small}}
\newcommand{\stfootcol}[2]{\tikzstmt{#1}{#2}{\footnotesize}}

\newcommand{\stnorm}[1]{\stcol{#1}{stmtcolor}}
\newcommand{\stsm}[1]{\stsmcol{#1}{stmtcolor}}
\newcommand{\stfoot}[1]{\stfootcol{#1}{stmtcolor}}

\newcommand{\st}[1]{\stfoot{#1}}
\newcommand{\lab}[1]{\stfoot{\ensuremath{#1}}}
\newcommand{\lan}[1]{\stnorm{\ensuremath{#1}}}
\newcommand{\stn}[1]{\stnorm{#1}}

\newcommand{\formula}[2]{\tikz[baseline]{\node[shape=rectangle,line width=1pt,draw=#2,fill=#2!30,inner sep=1pt, align=center] at (0,.64ex){\hspace{.2em}\texttt{\strut#1}\hspace{.1em}\strut};}}
\newcommand{\itp}[1]{\formula{\ensuremath{#1}}{itp}}

\newcommand{\tf}{\ensuremath{\varphi}\xspace}
\newcommand{\ctf}{\ensuremath{\widehat{\varphi}}\xspace}
\newcommand{\invars}{\ensuremath{In}\xspace}
\newcommand{\outvars}{\ensuremath{Out}\xspace}
\newcommand{\auxvars}{\ensuremath{Aux}\xspace}

\newcommand{\Var}{\ensuremath{\mathit{Var}}\xspace}
\newcommand{\stmt}{\ensuremath{\mathit{Stmt}}\xspace}
\newcommand{\Loc}{\ensuremath{\mathit{Loc}}\xspace}
\newcommand{\err}{\ensuremath{\mathit{err}}\xspace}
\newcommand{\init}{\ensuremath{\mathit{init}}\xspace}


%The [1] means one parameter, which is then referenced in the #1
\newcommand{\loc}[1]{\ensuremath{\ell_{#1}}\xspace}
\newcommand{\trans}[1]{\ensuremath{\xrightarrow{\st{#1}}}\xspace}
\newcommand{\stateSet}[1]{\ensuremath{\{#1\}}\xspace}
\newcommand{\vocab}[1]{\ensuremath{V_{\mathit{#1}}}\xspace}

\newcommand{\abst}[1]{\ensuremath{\mathit{(S^{#1}, V^{#1})}}}
\algnewcommand\algorithmicforeach{\textbf{for each}}
\algdef{S}[FOR]{ForEach}[1]{\algorithmicforeach\ #1\ \algorithmicdo}

\newcommand{\pq}[1]{\ensuremath{#1_{p, q}}\xspace}
\newcommand{\pqvasr}{\ensuremath{(\pq{S}, \pq{V})}\xspace}
\newcommand{\pqformula}{\ensuremath{\pq{\Gamma}}\xspace}

\newcommand{\qvAbstr}[1]{\ensuremath{(S_{#1}, V_{#1})}\xspace}

\newcommand{\sn}{\ensuremath{\begin{bmatrix} s_1 & \ldots &s_n \end{bmatrix}}}
\newcommand{\pn}{\ensuremath{\begin{bmatrix} x_1' \\ \vdots \\ x_n' \end{bmatrix}}}
\newcommand{\upn}{\ensuremath{\begin{bmatrix} x_1 \\ \vdots \\ x_n \end{bmatrix}}}

\newcommand{\s}{\ensuremath{\begin{bmatrix} s_x & s_y & s_z \end{bmatrix}}}
\newcommand{\p}{\ensuremath{\begin{bmatrix} x' \\ y' \\ z' \end{bmatrix}}}
\newcommand{\up}{\ensuremath{\begin{bmatrix} x \\ y \\ z \end{bmatrix}}}

%%%%%%%%%%%% Numbered example environment
% \newcounter{example}[section]
% \newenvironment{example}[1][]{\refstepcounter{example}\par\medskip
%    \noindent \textbf{Example~\theexample. #1} \rmfamily}{\medskip}

\newcounter{example}[section]
\def\exampletext{Example}
\NewDocumentEnvironment{example}{ O{} }
{
	
	\newtcolorbox[use counter=example]{examplebox}{%
		empty,
		title={\exampletext~\theexample. #1},
		attach boxed title to top left,
		minipage boxed title,
		boxed title style={empty,size=minimal,toprule=0pt,top=4pt,left=3mm,overlay={}},
		coltitle=colexamtitle,
		fonttitle=\bfseries,
		before=\par\medskip\noindent,parbox=false,boxsep=0pt,left=3mm,right=0mm,top=2pt,breakable,pad at break=0mm,
		before upper=\csname @totalleftmargin\endcsname0pt,
		overlay unbroken={\draw[colexamline,line width=.5pt] ([xshift=-0pt]title.north west) -- ([xshift=-0pt]frame.south west); },
		overlay first={\draw[colexamline,line width=.5pt] ([xshift=-0pt]title.north west) -- ([xshift=-0pt]frame.south west); },
		overlay middle={\draw[colexamline,line width=.5pt] ([xshift=-0pt]frame.north west) -- ([xshift=-0pt]frame.south west); },
		overlay last={\draw[colexamline,line width=.5pt] ([xshift=-0pt]frame.north west) -- ([xshift=-0pt]frame.south west); },%
	}
	\begin{examplebox}
	}
	{\end{examplebox}\endlist}


%%%%%%%%%%%% Setup
\newtheorem{name}{Printed output}
\newtheorem{mydef}{Definition}

\hypersetup{
	colorlinks=true,        % false: boxed links; true: colored links
	linkcolor=g1,        % color of internal links
	citecolor=g1,        % color of links to bibliography
	filecolor=g1,        % color of file links
	urlcolor=g1          % color of external links
}


\lstdefinestyle{boogie}{
	belowcaptionskip=1\baselineskip,
	breaklines=true,
	xleftmargin=\parindent,
	showstringspaces=false,
	basicstyle=\footnotesize\ttfamily,
	numbers=left,
	xleftmargin=.6cm
}

\lstset{escapechar=@,style=boogie}

\algrenewcommand\alglinenumber[1]{\footnotesize #1}

%%%%%%%%%%%% Comments
\newif\iffinal
%\finaltrue % comment out to remove comments

\iffinal
\newcommand\mycom[1]{}
\else
\newcommand\mycom[1]{#1}
\overfullrule=1mm
\fi
\setlength\parindent{0pt}

\newcommand{\jw}[1]{\mycom{\todo[color=blue!40,inline]{\small JW: #1}}}
\newcommand{\dd}[1]{\mycom{\todo[color=orange!40,inline]{\small DD: #1}}}
\newcommand{\ts}[1]{\mycom{\todo[color=green!40,inline]{\small TS: #1}}}

\newcommand{\traceabstraction}{\textsl{Trace Abstraction}}
\newcommand{\ultimate}{\textsc{Ultimate}\xspace}

\newcommand{\all}[1]{\mycom{\todo[color=green!40,inline]{\small #1}}}
\newcommand{\meta}[1]{\mycom{\todo[color=blue!10,inline,caption={Beschreibung},nolist]{\setlist{nolistsep}\small #1}}}
\newcommand{\xxx}{\mycom{\stfootcol{Placeholder}{blue!20}\xspace}}
\newcommand{\cn}{\mycom{\stfootcol{Cite}{blue!20}\xspace}}

\newcommand{\Q}{\ensuremath{\mathbb{Q}}}
\newcommand{\entails}[1]{\vdash_{#1}}

\newcommand{\qvasr}{\ensuremath{\mathit{\mathbb{Q} \text{-}VASR}}\xspace}
\newcommand{\qvasrs}{\ensuremath{\mathit{\mathbb{Q} \text{-}VASRS}}\xspace}

\newcommand{\tranFormula}{\ensuremath{F(\vec{x}, \vec{x}')}\xspace}
\newcommand{\coherent}{\ensuremath{\equiv_{\mathit{V}}}}
\newcommand{\conjunctTF}{\ensuremath{C(\vec{x}, \vec{x}')}}
\newcommand{\GammaTF}{\ensuremath{\Gamma(\vec{x}, \vec{x}')}\xspace}

\newcommand{\precon}{\ensuremath{\mathit{\stateSet{\varphi}}\xspace}}
\newcommand{\postcon}{\ensuremath{\mathit{\stateSet{\psi}}}\xspace}
\newcommand{\accel}[1]{\ensuremath{\psi^*_{L_{#1}}}\xspace}
\newcommand{\atm}[1]{\ensuremath{\mathcal{A}_{#1}}\xspace}
\newcommand{\prog}[1]{\ensuremath{P_{#1}}\xspace}

\newcommand{\interpret}{\ensuremath{\mathcal{I}}\xspace}
\newcommand{\eval}[1]{\ensuremath{\llbracket #1 \rrbracket}_{M, \rho}\xspace}

\newcommand{\prg}{\ensuremath{P = (V, \mu, st)}\xspace}
\newcommand{\cfg}{\ensuremath{(Loc, \Pi, \delta, src, tgt, \loc{init}, \loc{err})}\xspace}
\newcommand{\trf}{\ensuremath{\pi}\xspace}

\newcommand*{\transSys}[1]{\ensuremath{T_{#1} = (S_{#1}, \rightarrow_{#1})}}
\newcommand{\cfgAutomat}[1]{\ensuremath{\mathcal{A_{#1}} = (Q, \Sigma, \Delta, Q_{init}, F)}}
\newcommand{\cfgAutomatHoare}[1]{\ensuremath{\mathcal{A_{#1}} = (Q, \Sigma, \Delta, Q_{init}, F, \beta)}}
\newcommand{\valuations}{\ensuremath{ \mathbb{Q} }}

\newcommand{\gramr}[1]{\texttt{#1}}

\usepackage[justification=centering]{caption}

\begin{document}
	\setlength{\abovedisplayskip}{0.25cm}
	\setlength{\belowdisplayskip}{0.25cm}
	\setlength{\abovedisplayshortskip}{0.25cm}
	\setlength{\belowdisplayshortskip}{0.25cm}
	 \raggedbottom
\newcommand{\HorizontalLine}{\rule{\textwidth}{0.4mm}}

\begin{titlepage}
	\begin{centering}
		
		{\scshape\Large Master's Thesis\par}
		
		
		% _____________________________________________________________________________
		\HorizontalLine \\[0.4cm]
		{\huge\bfseries Adapting Loop Summarization Using \\ Rational Vector Addition Systems \\ with Resets for Trace Abstraction \par}
		\HorizontalLine \\[1.5cm]
		% _____________________________________________________________________________
		
		
		{\Large \scshape Jonas Werner}\\[5cm]
		
		
		\begin{tabular}[scshape]{>{\normalsize}l >{\normalsize}l}
			\scshape Examiner: & \scshape Prof. Dr. Andreas Podelski\\[0.3cm]
			\scshape Advisers: & \scshape Dr. Daniel Dietsch, Tanja Schindler  \\[1.2cm]
		\end{tabular}
		\vfill  % move the following text to the bottom
		
		\large { \scshape
			Albert-Ludwigs-University Freiburg\\
			Faculty of Engineering\\
			Department of Computer Science\\
			Chair of Software Engineering \\[1cm]
			
			March 22\textsuperscript{nd}, 2022\\
		}
	\end{centering}
\end{titlepage}

\thispagestyle{empty}
% title page back
\ \vfill \ \\  % at least one space required before vfill
\
\textbf{Writing Period}            \smallskip{} \\
22.\,09.\,2021 -- 22.\,03.\,2022   \bigskip{} \\
\
\textbf{Examiner}                  \smallskip{} \\
Prof. Dr. Andreas Podelski                     \bigskip{} \\
\
\textbf{Second Examiner}                  \smallskip{} \\
Prof. Dr. Peter Thiemann        \bigskip{} \\
\
\textbf{Advisers}                  \smallskip{} \\
Dr. Daniel Dietsch, \\ Tanja Schindler
\pagebreak
\pagebreak

\frontmatter
% Copied from the official declaration from the examination office (typos fixed).
% Please double check the wording on their website and report changes.
% (https://www.tf.uni-freiburg.de/de/studium-lehre/a-bis-z-studium/dokumente/Declarationforthefinalthesis.pdf)

\chapter*{Declaration}

I hereby declare that I am the sole author and composer of my thesis and that no other sources or learning aids, other than those listed, have been used. Furthermore, I declare that I have acknowledged the work of others by providing detailed references of said work.  \newline
I hereby also declare that my Thesis has not been prepared for another examination
or assignment, either wholly or excerpts thereof.
\\[3\normalbaselineskip]
\begin{tabular}{p{\textwidth/2} l}
	\rule{\textwidth/3}{0.4pt}   &   \rule{\textwidth/3}{0.4pt} \\
	Place, Date                  &   Signature
\end{tabular}

\pagebreak

\section*{Abstract}
Analyzing program loops is an important part of multiple software engineering tasks, such as bug detection, test case generation, and optimization. Loop analysis is, however, one of the most challenging aspects of program analysis, especially with complex loops that feature multiple paths and nested loops. A widely used loop analysis technique is computing overapproximative loop summaries that represent the relationship between inputs and outputs as a set of constraints. But overapproximation is prone to imprecision. Kincaid and Silverman proposed a novel overapproximation approach, based on rational vector addition systems with resets (\qvasr), that guarantees a certain degree of precision. In this thesis we adapted the \qvasr based loop summarization scheme to the trace abstraction method for proving program safety by implementing a summarization library that is used in two techniques: Firstly, directly on traces in the accelerated interpolation framework, which uses loop summaries for creating program state assertions that contain every trace through a loop. Secondly, as a preprocessor that transforms a given program's control-flow graph by replacing loops with the loop summarization such that trace abstraction considers traces that already contain every trace through a loop.
\pagebreak

\tableofcontents
\pagebreak

\mainmatter
\chapter{Introduction}
\label{intro}

\begin{comment}
	\jw{Intuitive introduction to the subject matter, no formulas or definitions. \\
	- Start by firstly introducing trace abstraction: explaining safety of programs \\
	- Introduce program loops, mention what problems they cause $\rightarrow$ nesting, branching etc \\
	- then the 3 possible techniques: unwinding, invariant calculation, summarization with focus on summarization\\
	- related work \\
	- use simple running example \\
	- Continue to accelerated interpolation, intuitively explain how it works and how it integrates loop summaries $\rightarrow$ meta traces \\
	- Say that Ultimate already has a myriad of loop summarization techniques such as FastUPR and Jordan and show their drawbacks, such as Jordan being only to deal with no variable relations and FastUPRs lack of branching support. \\
	- Introduce qvasr as a solution  \\
	IMPORTANT: Only INTUITIVE explanations, no definitions \\ 
	\vspace{1cm}
	3 pages}
\end{comment}


Analyzing program loops is an important part of multiple software engineering tasks, such as bug detection, test case generation, and optimization. Loop analysis is, however, one of the most challenging aspects of program analysis, especially with complex loops that feature multiple paths and nested loops. Loop analysis has been called the "Archilles heel" of program verification\cite{DBLP:journals/fmsd/KroeningSTTW13}. \par
There are three prominent techniques of loop analysis: loop unwinding, loop invariant calculation, and loop summarization\cite{DBLP:journals/fmsd/KroeningSTTW13 , DBLP:conf/cav/SilvermanK19, DBLP:journals/tse/XieCZLLL19}. Loop unwinding deals with a loop by unrolling each loop iteration until a certain bound. This technique is quite simple but cannot reason about the program's behavior beyond the unwinding bound. \\
Loop invariants are properties that hold before and after each loop iteration, defining its behaviour as constraints. The meaningfulness of loop invariants is however tied to the constraint's strength. Finding sufficiently strong invariants "is an art"\cite{DBLP:journals/fmsd/KroeningSTTW13}. A software-model checker, for example, may not terminate if given loop invariants are too general. \par
Loop summarization on the other hand provides a more accurate and complete view of the loop. A loop summary models the relationship between inputs and outputs as a set of constraints that can be used to replace the loop in a program. Loop summaries are not guaranteed to be precise, meaning they can over- or underapproximate the loop's behavior. Counterexamples found using underapproximative loop summaries are guaranteed to be feasible. But due to the underapproximate nature of the summaries they can miss other feasible counterexample as they model only a subset of the actual loop behavior \cite{DBLP:journals/fmsd/KroeningLW15}. Loop summaries that overapproximate the loop' behaviour on the other hand form a superset of the actual loop behavior, resulting in possible spurious counterexamples. The quality of overapproximative loop summaries depend on the degree of overapproximation of the actual loop behavior.\\ \par
Silverman and Kincaid\cite{DBLP:conf/cav/SilvermanK19} proposed an overapproximative loop summarization that \textsl{guarantees} a certain degree of approximation precision.\\ \par 

This proposed technique makes use of vector addition systems which are a prominent class of infinite-state transition systems with decidable reachability \cite{DBLP:conf/rp/HaaseH14}. A vector addition system consists of a finite number of integer-typed variables. When a transition is taken, these variables are updated by an addition of a constant. Silverman and Kincaid use a variation of vector addition systems that are based on rational-typed variables instead of integers and extend transitions by employing the notion of resets. When a transition is taken in a rational vector addition systems (\qvasr) a variable is either incremented by a constant, reset to 0, or reset and incremented.\\ \par

\qvasr by themselves are a precise method of loop summarization, however, they can only deal with constant variable updates and updates that reference only the updated variable itself, for example \texttt{x := x + 1} or \text{y := 42}. Most loops contain variable updates, for instance \texttt{x := y + x} which cannot be represented by a \qvasr. To remedy this Silverman and Kincaid introduce so called \qvasr-abstractions, which are \qvasr extended by a, so called, simulation matrix which projects the loop's transitions to changes between variable relations. This \qvasr-abstraction is an overapproximation as it no longer models changes to specific variables but to variable relations.\\ \par

\qvasr-abstractions are only arguing over changes to variables and do not factor in program assumptions, such as the loop guard or \texttt{if else} statements. This leads to a loss of precision as the loop summaries' transitions are taken nondeterministically. This problem lead to the introduction of rational vector addition systems with resets and states (\qvasr), which expand \qvasr-abstractions by making use of states that model the original program's control flow.\\ \par

The goal of this thesis was to implement a loop summarization technique making use of \qvasr-abstractions, and \qvasrs in the software analysis framework \textsc{Ultimate}\cite{Zitat02}.
\textsc{Ultimate} consists of multiple plugins and libraries for various software verification tasks and already features numerous loop summarization techniques, they, however, have problems with nested loops, loop branching, variable assignments, and approximation precision.\\  \par
Our contribution is the adaption of a \qvasr-abstraction and \qvasrs based loop summarization library to \textsc{Ultimate}'s \textsl{Accelerated Interpolation} interpolant computation technique, that is embedded in the \textsl{Trace Abstraction} \cite{10.1007/978-3-642-03237-0_7} counterexample-guided abstraction refinement scheme. \textsl{Accelerated Interpolation} uses program traces, computed by \textsl{trace abstraction}, to calculate state assertions for programs utilizing loop summaries. Additionally, we integrated \qvasr-abstractions and \qvasrs to transform a program's control flow graph, by replacing loops with loop summaries, such that, \textsl{Trace Abstraction} computes program traces already containing loop summaries.\\ \par

The remainder of this master's thesis is structured as follows, in \



\chapter{Related Work}
\label{relWork}

Next to rational vector addition systems with resets, there are a multitude of other loop summarization techniques, all with their own strengths and weaknesses.

\textsl{Fast Acceleration using Ultimately Periodic Relations}\cite{10.1007/978-3-642-14295-6_23} 

\textsl{Abstracting Path Conditions}\cite{DBLP:conf/issta/StrejcekT12}

\textsl{Under-approximating loops}\cite{DBLP:conf/cav/KroeningLW13}

\textsl{Abstract Acceleration of General Linear Loops}\cite{DBLP:conf/popl/JeannetSS14}
\pagebreak

\chapter{Preliminaries}
\label{background}

\begin{comment}
	This chapter is mostly focused on trace abstraction $\rightarrow$  It introduces the reader to the concept of trace abstraction. \\
	- Introduce logic, logical variables, terms, formulas, transition formulas with primed and unprimed variables, programs, program states, loops $\rightarrow$  then program-, error traces, feasible and infeasible counterexamples, CFGs, interpolants. \\ - From intuitive to true definitions. \\
	Here the running example from the introduction gets dissected to illustrate the definitions. \\ 
	Further the problems loops can cause are introduced, followed by a definition of loop summaries $\rightarrow$ introduction reflexive transitive closure of a formula 
	15 pages
\end{comment}

This chapter shall introduce our understanding and notation of logic and formulas, programs, control-flow, and other needed background definitions. Furthermore, we will give an overview of the \traceabstraction \cite{10.1007/978-3-642-03237-0_7} counterexample-guided abstraction refinement scheme used in \ultimate.

\subsection{Logical Background}
We use first-order logic to model programs, this chapter will introduce our definitions and notations used throughout this thesis.
\begin{mydef}[Term] 
	Given a vocabulary $V = (\vocab{Var}, \vocab{Const}, \vocab{Fun}, \vocab{pred})$, with countable sets $\vocab{\Var}, \vocab{Const}, \vocab{Fun}, \vocab{pred}$ representing the sets of variables, constant symbols, function symbols, and predicate symbols respectively, we define terms inductively as follows:
	\begin{itemize}
		\item Every $x \in \vocab{Var}$ is a term.
		\item Every $c \in \vocab{Const}$ is a term.
		\item If $t_0, \ldots, t_n$ are terms and $f \in \vocab{Fun}$ being a function symbol with arity $n$, then $f(t_0, \ldots, t_n)$ is a term.
	\end{itemize}
\end{mydef} \vspace*{1cm} \par
Using the definitions of terms, we can introduce first-order logic formulas.

\begin{mydef}[Formula]
	Given vocabulary $V = (\vocab{Var}, \vocab{Const}, \vocab{Fun}, \vocab{pred})$, first-order logic formulas are inductively defined as follows:
	\begin{itemize}
		\item $\bot$ (\textsl{false}) is a formula.
		\item If  $t_0, \ldots, t_n$ are terms, and $p \in \vocab{pred}$ is a predicate symbol with arity $n$, \\ then $p(t_0, \ldots, t_n)$ is a formula.
		\item If $\varphi$ is a formula, then $\neg \varphi$ is a formula.
		\item If $\varphi$ and $\psi$ are formulas, then $\varphi \land \psi$ are formulas.
		\item If $\varphi$ is a formula, and $x \in \vocab{Var}$ then $\exists x. \varphi$ is a formula.
	\end{itemize}
\end{mydef} \vspace*{1cm} \par

\begin{mydef}[Model]
	Given vocabulary $V = (\vocab{Var}, \vocab{Const}, \vocab{Fun}, \vocab{pred})$, a model $\mathcal{M} = (D, \interpret)$ is a tuple consisting of a nonempty set $D$, called interpretation domain, and an interpretation function \interpret that assigns constants, functions, and predicates over $D$ to symbols in $V$.
\end{mydef}

\begin{mydef}[Assignment of Variables]
	Given vocabulary $V = (\vocab{Var}, \vocab{Const}, \vocab{Fun}, \vocab{pred})$, and domain $D$, an assignment of variable $v \in \vocab{Var}$ is a function $\varv: v \rightarrow D$.
\end{mydef}

\begin{mydef}[Evaluation of Terms]
	Let $V = (\vocab{Var}, \vocab{Const}, \vocab{Fun}, \vocab{pred})$ be a vocabulary, $\mathcal{M} = (D, \interpret)$ a model, and $\varv$ a variable assignment, the evaluation of terms is a function $\eval{\cdot}$ that is inductively defined as:
	\begin{itemize}
		\item For each $x \in \vocab{Var}$, $\eval{x} = \varv(x)$
		\item For each $c \in \vocab{Const}$, $\eval{c} = \interpret(c)$
		\item If $t_0, \ldots, t_n$ are terms, $f \in \vocab{Fun}$, and f has arity $n$ then \\ $\eval{f(t_0, \cdots, t_n)}$ is $\interpret(f)(\eval{t_0}, \ldots, \eval{t_n})$
	\end{itemize}
\end{mydef}

\subsection{Programs}
Assume we are given a program as seen in \ref{code}. We consider each line of code a so called program statement. These statements use the following context-free grammar that is a derived and simplified version of the grammar of the intermediate verification language Boogie\cite{Boogie}.
\setlength{\grammarparsep}{20pt plus 1pt minus 1pt} % increase separation between rules
\setlength{\grammarindent}{12em} % increase separation between LHS/RHS 
\begin{grammar}
	<Stmt> ::= `assume' <Expr> ;
	\alt $Var_{id}$ `:=' <Expr> ;
	\alt `havoc' $Var_{id}$ ;
	\alt `assert' <Expr> ;
	\alt `while' ( <WildcardExpr> ) <Stmt>* ;
	\alt <IfStmt>;
	
	<IfStmt> ::= `if' ( <WildcardExpr> ) <Stmt>* `else' <Stmt>*

	<Expr> ::= <Expr> <BinOp> <Expr>
	\alt <UnOp> <Expr>
	\alt `True'
	\alt `False'
		
	<WildCardExpr> ::= <Expr> | `*'
	
	<BinOp> ::= `+' | `-' | `*' | `/' | `\%'
	\alt `\&\&' | `||' | `==' | `!='
	\alt `<' | `<=' | `>' | `>='
	
	<UnOp> ::= `-' | `!'
\end{grammar}
Where $Var_{id}$ represents any variable declared in the program and $*$ corresponds to a nondetermnistic choice.

\begin{mydef}[Control-Flow Graph]
	Given a finite set of program statements \stmt. A control-flow graph is a labelled graph $G_P = (\Loc, \delta, \loc{\init}, \loc{\err})$, with
	\Loc being a finite set of locations,
	a set of edges between two locations labelled with a statement $\delta \subseteq \Loc \times \stmt \times \Loc$,
	an initial location $\loc{init} \in \Loc$, and
	an error location $\loc{err} \in \Loc$.
\end{mydef}
In this paper we will use control-flow graphs to represent programs.
% \dd{It is useful to wrap all the function symbols in Latex macros}
% \ts{And to use mathit for identifiers consisting of several letters (like $\mathit{Var}$)}
Program variables are typed, e.g. integer or boolean variables, there exists \\
an interpretation domain $D$ defining the set of all possible variable values.
\jw{TODO}
Assigning every program variable a valuation creates a program state.

%the n may be wrong, because traces can be infinite in theory? no, because a program can only have finitely many variable declarations. The source code is finite.

\begin{mydef}[Variable Valuation]
	Assume a program $P$ is defined over $n$ variables, a program state $\sigma$ is a function assigning each variable $v_i \in \Var$, \ $0 \leq i \leq n$ a variable valuation $\rho_i$. The set $S$ denotes the set of all program states.
\end{mydef}

\begin{center}
	\begin{minipage}[b]{0.4\linewidth}
			\begin{figure}[H]
			\centering
			\begin{lstlisting}[language=C++, style=withAssert]  % Start your code-block
	
	int x := 0;
	int y := 2;
	int z := 3;
	while x <= 20:
		if x <= 10:
			z := x;
			x := x + y;
			y := y + 1;
		else:
			x := x + 2;
			y := y - 3;
	assert x == 21;
	\end{lstlisting}
			\caption{Program $P$ \\ with \texttt{while} loop.}
			\label{code}
		\end{figure}
	\end{minipage}
	\hfill
	\begin{minipage}[b]{0.59\linewidth}
		\begin{figure}[H]
			\centering
			
\begin{tikzpicture}[%
    ->,
    >=stealth',
    shorten >=1pt,
    auto,
    node distance=3.25cm,
    scale=0.9,
    transform shape,
    align=center,
    smallnode/.style={inner sep=1.4},
    initial text =,
    anchor=center]

			\node[state, initial above, initial text =](1){$\loc{1}$};
			\node[state] (head) [below of=1] {$\loc{4}$};
			\node[state] (loopEntry)[left of=head] {$\loc{5}$};
			\node[state] (if)[above of=loopEntry] {$\loc{6}$};
			
			\node[state] (else)[below of=loopEntry] {$\loc{10}$};
			\node[state] (loopExit)[right of=head] {$\loc{13}$};
			
			\node[state] (assertTrue)[below left of=loopExit] {$\loc{14}$};
			\node[state, accepting] (assertFalse)[below of=loopExit] {$\loc{err}$};
			
			\path (1) edge node []{\st{x:=0;}\\ \st{y:=2;}\\ \st{z:=3;}} (head)
			(head) edge node []{\st{x<=20}} (loopEntry)
			(head) edge node []{\st{x>20}} (loopExit)
			
			(loopEntry) edge[] node []{\st{x<=10}} (if)
			(if) edge[]node []{\\ \st{z:=x;}\\ \st{x:=x+y;}\\ \st{y:=y+1}} (head)
			
			(loopEntry) edge[]node []{\st{x>10}} (else)
			(else) edge[] node []{\st{x:=x+2;}\\ \st{y:=y-3}} (head)
			
			(loopExit) edge node []{\st{x==22}} (assertTrue)
			(loopExit) edge node []{\st{x!=22}} (assertFalse)
			;
\end{tikzpicture}
			\caption{Control-flow graph for program $P$.}
			\label{code}
		\end{figure}
	\end{minipage}
\end{center}


\subsection{Model Checking}


\subsection{Trace Abstraction}

\chapter{Loop Summarization using \qvasr}
\label{qvasr}

\begin{comment}
	\jw{Introduce loop summarization using rational vector addition systems with resets \\
	- What are qvasr? What are qvasr abstractions? what are least upper bounds on abstractions? \\
	- start with qvasr on example that does not need abstraction $\rightarrow$ no relations between variables \\
	- move on to example with relations between variables; show simulation matrix and imaging \\
	- abstractions are an overapproximation \\
	- imprecise because of ignorance of assumptions $\rightarrow$ transfer to next chapter \\
	- usage of running example which is turned to qvasr abstraction \\
	\vspace{1cm}
	20 pages}
\end{comment}

This chapter will firstly give an overview of the challenges program loops introduce, then we will define our understanding of the loop summarization technique based on rational vector addition systems with resets (\qvasr), which will then be extended to \qvasr-abstractions, and lastly we introduce rational vector addition systems with resets and states (\qvasrs). \\ \par
A program loop, created by, for example, a \texttt{while} statement, can introduce up to infinitely many program traces. Let us, for instance, assume we have a non deterministic loop:
\begin{equation*}
	\texttt{while (*); x:=2*x}
\end{equation*}
and an error location outside of the loop asserting \texttt{x\%2==0}. To check safety we would have to prove, in the worst case, infinitely many program traces as infeasible because there are theoretically infinitely many traces through the loop.
.\begin{center}
	\begin{minipage}[t]{0.4\linewidth}
		\begin{figure}[H]
			\centering
			\begin{lstlisting}[]  % Start your code-block
	
int x := 1;
while (*) {
    x:=2*x;
}
assert x%2 == 0;
\end{lstlisting}
			\caption{Program $P_1$ with nondeterministically terminating loop.}
			\label{codeExMotAss}
		\end{figure}
	\end{minipage}
	\hfill
	\begin{minipage}[t]{0.59\linewidth}
		\begin{figure}[H]
			\centering
			\begin{tikzpicture}[%
	->,
	>=stealth',
	shorten >=1pt,
	auto,
	node distance=3cm,
	scale=0.9,
	transform shape,
	align=center,
	smallnode/.style={inner sep=1.4},
	initial text =,
	anchor=center]
	
	\node[state, initial above, initial text =](1){$\loc{1}$};
	\node[state] (head) [below of=1] {$\loc{2}$};
	\node[state] (nondet) [left of=head] {$\loc{3}$};
	\node[state] (asstrue) [below of=head] {$\loc{5}$};
	\node[state, accepting] (assfalse) [below right of=head] {$\loc{err}$};
	
	\path (1) edge node []{\st{x:=1}} (head)
	(head) edge[bend right] node {\st{*}} (nondet)
	(nondet) edge[bend right] node []{\st{x:=2*x}} (head)
	(head) edge node []{\st{x\%2==0}} (asstrue)
	(head) edge node []{\st{x\%2!=0}} (assfalse)
	;
\end{tikzpicture}
			\caption{Control-flow graph $G_{P_1}$ for program $P_1$.}
			\label{cfgExMotAss}
		\end{figure}
	\end{minipage}
\end{center}
Instead, we can compute a loop summarization that characterizes changes to variables as a first-order logic formula. Loop summaries are not guaranteed to be precise, meaning they can over- or underapproximate the loop's behavior. Feasible error traces, found using underapproximative loop summaries, are guaranteed to be feasible. But due to the underapproximate nature of the summaries they can miss other feasible counterexample as they model only a subset of the actual loop behavior \cite{DBLP:journals/fmsd/KroeningLW15}. Loop summaries that overapproximate the loop's behaviour, on the other hand, form a superset of the actual loop behavior, resulting in possible spurious feasible error traces. The quality of overapproximative loop summaries depend on the degree of overapproximation of the actual loop behavior.
Kincaid et al. \cite{DBLP:conf/cav/SilvermanK19} devised an overapproximative loop summarization method that guarantees a certain degree of precision by utilizing rational vector addition systems with resets. In the following we present our adaption of that summarization technique.
\subsection{\qvasr}
To summarize a loop, we make use vector addition systems. Vector additions systems are a class of infinite-state transition systems.

\begin{mydef}[Transition System]
	A transition system \transSys{} is a tuple, consisting of a finite or infinite set of states $S$ and a transition relation $\rightarrow \subseteq S \times S$.
\end{mydef}

A concretization of a transition system is a vector addition system that is defined over rational values.

\begin{mydef}[Rational Vector Addition System]
	Given two vectors $\vec{u}$, and $\vec{v}$ consisting of $d$ rational numbers, a rational vector addition system $V$ of dimension $d$ is a finite set $V \subseteq \mathbb{Q}^d$ of transformers. A transformer $\vec{a}$ is a vector of length $d$, called addition vector, containing rational numbers. Rational vector addition systems define a transition system \transSys{V}, with state space $S_V = \mathbb{Q}$ and transition relation $\vec{u} \rightarrow_V \vec{v}$ if and only if $\vec{v} = \vec{u} + \vec{a}$ for some $\vec{a} \in V$. 
\end{mydef}
We want to model changes to variables in a loop using rational vector addition systems, however, additions alone do not cover all possible transitions. A variable can be \textsl{reset} to an arbitrary constant. For example, the transition \st{x:=0} sets variable \texttt{x} to zero no matter what value $x$ had before the transition. We extend rational vector addition systems to rational vector addition systems with resets (\qvasr).

\begin{mydef}[Rational Vector Addition Systems with Resets]
		Given two vectors $\vec{u}$, and $\vec{v}$ consisting of $d$ rational numbers, a rational vector addition system with resets (\qvasr) $V$ of dimension $d$ is a rational vector addition system $V \subseteq \{0, 1\}^d \times \mathbb{Q}^d$, where each transformer consists of a addition vector $\vec{a} \subseteq \mathbb{Q}^d$ and a binary reset vector $\vec{r} \subseteq \{0, 1\}^d$. A \qvasr defines a transition system \transSys{V}, with state space $S_V = \mathbb{Q}$ and transition relation $\vec{u} \rightarrow_V \vec{v}$ if and only if $\vec{v} = \vec{r} * \vec{u} + \vec{a}$ for some $(\vec{r}, \vec{a}) \in V$, with $*$ being the Hadamard product.
\end{mydef}
We see, that there are three distinct changes to variables modelled in \qvasr: A variable is incremented, for example \st{x:=x+1}, a variable is reset to zero, as in \st{x:=0}, or a variable is reset and then incremented to a constant, \st{x:=3}. \par

A \qvasr can be used to represent an overapproximation of a given transition formula. A transition formula $\trf_1$ over variables $\vec{x} = x_1, \ldots, x_n$ and $\vec{x}' = x_1', \ldots, x_n'$ designate the program state before and after the transition, $n$ is called the dimension of the formula. Transition formula $\trf_1$ defines a transition system $(S_{\trf_1}, \rightarrow_{\trf_1})$, where $S_{\trf_1}$ is defined over the domains of $\vec{x}$, whereas two program states $\sigma_1$, $\sigma_2$ can transition $\sigma_1 \rightarrow_{\trf_1}$ if and only if the variable valuations of $\sigma_1$ as $\vec{x}$ and valuations of $\sigma_2$ as $\vec{x}'$ is valid. \\ \par
Figure \ref{codeWithAss} depicts a program containing a \texttt{while} loop from lines 4 - 12 for which we want to compute a loop summary using \qvasr. \par
To get a loop summary, approximating the entire loop, we need to compute a \qvasr for every path through the loop. In this example there are two paths created by the \texttt{if else} statement.
Beginning with the \texttt{else} branch, from line 9 to 12, we extract the transition formula:
\begin{equation*}
	G= \ (x \leq 20 \land\ x > 10 \land x' = x + 2 \land y' = y - 3)
\end{equation*}
We see, that variable $x$ is not reset but incremented by 2 and variable $y$ is not reset and decremented by 3.
Knowing this, we get reset vector $
\vec{r}_G = \  
\begin{bmatrix}
	1 \\
	1 
\end{bmatrix}
$
and addition vector $
\vec{a}_G = \ 
\begin{bmatrix}
	2 \\
	-3 
\end{bmatrix}$ \\
$G$ can therefore be modeled by the \qvasr
$V_G = 
\begin{Bmatrix}
	\begin{pmatrix}
		\begin{bmatrix}
			1 \\
			1
		\end{bmatrix},
		\begin{bmatrix}
			2 \\
			-3
		\end{bmatrix}
	\end{pmatrix}
\end{Bmatrix}
$ \par \vspace{2pt}
For the remainder of this thesis we will use the following, more intuitive notation that shows relations between variables more clearly:
\begin{equation*}
	V_G = 
	\begin{Bmatrix}
		\begin{bmatrix}
			x' = x + 2 \\
			y' = y - 3
		\end{bmatrix}
	\end{Bmatrix}
\end{equation*}
\subsection{\qvasr Summarization}
\qvasr can only model the aforementioned three transitions: increment, reset, and reset and increment. Increments, and resets and increments, however, are tied to constant variable changes. Updates that reference, for example, other variables, cannot be modeled as a \qvasr. When we look at the example program \ref{cfg:P:Ass}, we can see that in the \texttt{if} branch there are increments that reference other variables, such as \st{z:=x} and \st{x:=x+y}. \\

\subsubsection{\qvasr-Abstraction}
We extract the transition formula $H$ of the if branch:
\begin{equation*}
	H= \ (x \leq 10 \land x' = x + y\ \land\ y' = y + 1 \land z' = x)
\end{equation*}
It is possible to overapproximate $H$ using a \qvasr using linear simulations.

\begin{mydef}[Linear Simulation]
	Given two transition systems $T_1 = (\valuations^n, \rightarrow_1)$ and $T_2 = (\valuations^m, \rightarrow_2)$, operating over the vector space containing rational numbers, a linear simulation from $T_1$ to $T_2$ is a linear transformation $S: \valuations^{m \times n}$ such that for all vectors $\vec{u}$, $\vec{v} \in \valuations^n$ with $\vec{u} \rightarrow_1 \vec{v}$, we have $S\vec{u} \rightarrow_2 S\vec{v}$. We denote the linear simulation from $T_1$ to $T_2$ as $T_1 \vdash_S T_2$
\end{mydef}
We want to synthesize a \qvasr $V_H$ and linear simulation $S_H$, such that $H \vdash_S V$. The tuple $(S_H, V_H)$ is called a \qvasr-abstraction.
\begin{mydef}[\qvasr-Abstraction]
	Given a transition formula \trf of dimension $n$ and a \qvasr $V$ of dimension $m$, a linear simulation from \trf to $V$ is a linear transformation matrix: 
	$S = 
	\begin{bmatrix}
		s_{1 ,1} & \ldots & s_{1, n} \\
		\vdots & \ddots & \vdots \\
		s_{m ,1} & \ldots & s_{m, n} \\
	\end{bmatrix}$ 
	such that for all transitions $\vec{x} \rightarrow_\trf \vec{x}'$ we have $S\vec{x} \rightarrow_V S\vec{x}'$. Meaning, every transition $\vec{x} \rightarrow_\trf \vec{x}'$ can be represented in $V$ by a matrix multiplication of $S$ and $\vec{x}$ and $\vec{x}'$. We call $n$ the abstraction's concrete, and $d$ the abstraction's abstract dimension. 
\end{mydef}
To abstract a transition formula $\trf$ with $n$ variables $\vec{x} = x_1, \ldots, x_n$, we need to calculate both the linear transformation matrix $S$ and the \qvasr $V$. We know that a \qvasr consists of pairs of transformers, which contain reset and addition vectors. Using the formula $S \vec{u} \rightarrow_V S\vec{u}$, we get a transition in $V$ as $S\vec{x}' = \vec{r}*S\vec{x} + \vec{a}$. \\ To get variables that are reset, we consider $\vec{r}$ as the zero vector $\vec{0}$, and get the formula $S\vec{x}' = \vec{a}$ which forms a linear set of equations that models the set of resets:
\begin{equation*}
	Res_\trf = \left\{ (\sn, a)\ |\ \sn \cdot \pn = a \right\}	
\end{equation*}
For the set of additions, we consider $\vec{r}$ as the constant one vector $\vec{1}$, and get formula $S\vec{x}' = S\vec{x} + \vec{a}$ which forms the set of additions as:
\begin{equation*}
	Inc_\trf = \left\{(\sn, a)\ |\ \sn \cdot \pn = \sn \cdot \upn + a\right\}	
\end{equation*}
\begin{comment}
	\begin{equation*}
	Res_H = \left\{ (\s, a) | H \models \s \cdot \p = a \right\}	
	\end{equation*}
	
	\begin{equation*}
	Inc_H = \left\{(\s, a) | H \models \s \cdot \p = \s \cdot \up + a\right\}	
	\end{equation*}
\end{comment}
By solving the following equations for $s_1, \ldots, s_n$: \\
\begin{align*}
	Res\trf &= \sn \cdot \pn = a \\
 			&= s_1x_1 + \ldots s_nx_n = a
\end{align*}

\begin{align*}
	Inc_\trf &= \sn \cdot \pn = \sn \cdot \upn + a  \\ \vspace{0.5cm}
			 &= s_1x_1' + \ldots s_nx_n' = s_1x_1 + \ldots + s_nx_n + a
\end{align*}
We get two sets that represent linear combinations of variables that are reset and incremented across \trf. We observe that the two form a vector space. These spaces can be represented using vector space basis vectors, meaning, we have to compute a basis. Assume we have basis for the set of resets:
$Res_\trf = \{(\vec{s}_1, a_1), \ldots, (\vec{s}_m, a_m)\}$ and basis for the set of additions as: $Inc_\trf = \{(\vec{s}_{m+1}, a_{m+1}), \ldots, (\vec{s}_d, a_d)\}$. \\ We construct the \qvasr-abstraction $(S, V)$ as: \\
\begin{center}
\begin{minipage}{0.3\linewidth}
	\begin{equation*}
		S = \begin{bmatrix} \vec{s_1} & \ldots & \vec{s_d} \end{bmatrix}
	\end{equation*}
\end{minipage}
\begin{minipage}{0.3\linewidth}
	\begin{equation*}
		\vec{r} = [ \underbrace{0 \ldots 0}_{m\text{ times}} \overbrace{1 \ldots 1}^{d - m \text{ times}} ]
	\end{equation*}
\end{minipage}
\begin{minipage}{0.3\linewidth}
	\begin{equation*}
		\vec{a} = \begin{bmatrix} a_1 \\ \vdots \\ a_d \end{bmatrix}
	\end{equation*}
\end{minipage}
\end{center}

For transition formula $H$, we need to calculate the matrix $S_H$ and the \qvasr $V_H$.
We use the assignments to variables: $x' = x + y, \ y'= y + 1\ z' = x$ , found in $H$, to solve the following equations for $s_1, s_2,  s_3$:
\begin{align*}
	Res_H:& \hspace*{1cm} s_1(x + y) + s_2(y + 1) + s_3x = a \\
	Inc_H:& \hspace*{1cm} s_1(x + y) + s_2(y + 1) + s_3x = s_1x + s_2y + s_3z + a
\end{align*}
resulting in the following sets:
\vspace*{-0.5em}
\begin{center}
	\begin{minipage}{0.5\linewidth}
		\begin{equation*}
			Res_H = \left\{ (\begin{bmatrix} -a & a & a \end{bmatrix}, a) \right\}\
		\end{equation*}
	\end{minipage}
	\begin{minipage}{0.4\linewidth}
		\begin{equation*}
			Inc_H = \left\{ (\begin{bmatrix} 0 & a & 0 \end{bmatrix}, a) \right\}\ 
		\end{equation*}
	\end{minipage}
\end{center}
$Res_H$ and $Inc_H$ form a vector space, and, because we want to construct a linear combination of reset and incremented variables along $H$, we need to compute a basis for each space:
\vspace*{-1em}
\begin{center}
	\begin{minipage}{0.5\linewidth}
		\begin{equation*}
			Res_H = \{(
			\NiceMatrixOptions{code-for-first-row=\scriptstyle}
			\begin{bNiceMatrix}[first-row=1]
				x & y & z \\
				-1 & 1 & 1 
			\end{bNiceMatrix}, 1)\}
		\end{equation*}
	\end{minipage}
	\begin{minipage}{0.4\linewidth}
		\begin{equation*}
			Inc_H = \{(
			\NiceMatrixOptions{code-for-first-row=\scriptstyle}
			\begin{bNiceMatrix}[first-row=1] 
				x & y & z \\
				0 & 1 & 0 
			\end{bNiceMatrix}, 1)\}
		\end{equation*}
	\end{minipage}
\end{center}
Each column of the basis corresponds to a variable in the formula such that we can derive relations between variables. From the reset base we can deduce that the sum $-x + y + z$ is reset and incremented by $1$, meaning that after each transition of $H$ we know that $-x + y + z = 1$ holds. From the basis of additions we derive that $y$ is not reset and incremented by $1$. To form the linear transformation matrix $S_H$ we combine these rows to one coherent matrix. We see that the basis of resets and additions contain only one vector each, resulting in only one reset addition pair for $V_H$. As we have seen $-x + y + z$ is reset and $y$ is not. We get reset vector $\vec{r} = \begin{bmatrix} 0 \\ 1 \end{bmatrix}$ and because $-x + y + z$ is incremented by $1$, same as $y$, we get addition vector $\vec{a} = \begin{bmatrix} 1 \\ 1 \end{bmatrix}$. Resulting in the \qvasr-abstraction depicted in figure \ref{vasr  H}.
\vspace*{-2em}
\begin{figure}[H]
	\begin{center}
		\begin{minipage}{0.3\linewidth}
			\begin{equation*}
				S_H = \begin{bmatrix} -1 & 1 & 1 \\ 0 & 1 & 0 \end{bmatrix}
			\end{equation*}
		\end{minipage}
		\begin{minipage}{0.6\linewidth}
			\begin{equation*}
				V_H = \begin{Bmatrix} \begin{bmatrix} -x' + y' + z' = 1\\ y' = y + 1 \end{bmatrix} \end{Bmatrix}
			\end{equation*}
		\end{minipage}
		\caption{\qvasr-abstraction of transition formula $H$.}
		\label{vasr H}
	\end{center}
\end{figure}
\vspace*{-2em}
Given a transition formula \trf with variables $\vec{x} = x_1, \ldots, x_n$ and $\vec{x}' = x_1', \ldots, x_n'$ a \qvasr-abstraction $(S, V)$ that simulates \trf, with transformer $(\vec{r}_1, \vec{a}_1), \ldots, (\vec{r}_m, \vec{a}_m) \in V$ where each $\vec{r}_i, \vec{a}_i$ has $d$ entries, we compute a transition formula overapproximating \trf by constructing the following disjunction:
\begin{equation*}
	\bigvee\limits_{i=1}^m \bigwedge\limits_{j=0}^d S\vec{x}' = S\vec{x} * \vec{r}_j + \vec{a}_j
\end{equation*}
The \qvasr-abstractions are overapproximations because they do not model single program variable transitions but, rather, transitions of relations of variables. These relations lead to an overhead of actual program behaviour, as a transition is possible, as long as the conditions to the relations, as described by the \qvasr-abstraction, hold. \\
For example, the \qvasr-abstraction  $(S_H, V_H)$ of $H$ is an overapproximation, because, for instance, the transition: $\begin{bNiceMatrix}[first-col=1]  x & 1\\ y & 2 \\ z & 3 \end{bNiceMatrix} \rightarrow \begin{bNiceMatrix}[last-col]  -20 & x'\\ 3 & y' \\ 18 & z' \end{bNiceMatrix}$ is possible because $-x' + y' + z' = 1$ and $y' = y + 1$ holds. In $H$ this transition is not possible, because of $z' = x$ and $x' = x + y$. \\ \par
For transition formulas with constant increments, such as $G$, the identity matrix $I$ is used as simulation matrix. These \qvasr-abstractions are precise. \par

\subsubsection{\qvasr-Abstraction Join}
The changes of program variable valuations by the loop in $P$ can be represented by the transition formula: 
\begin{equation*}
	L = x \leq 20 \land (H \lor G)
\end{equation*}
With $x \leq 20$ being the loop guard. We have already computed \qvasr-abstractions for $G$ and $H$. These, however, only model the effect on variables in their respective branch of the \texttt{if else} statement. To get the approximation of the whole loop's behavior we need to compute the \qvasr-abstraction $(\tilde{S}, \tilde{V})$ that simulates $L$. \\ \par We impose a partial order to model simulations between \qvasr-abstractions.
\begin{mydef}[\qvasr-Abstraction Preorder]
	Given two \qvasr-abstractions  \qvAbstr{1} and \qvAbstr{2}, with abstract dimensions $e$ and $d$ respectively, we define the preorder \\ of \qvasr-abstractions as \qvAbstr{1} $\preceq$ \qvAbstr{2} if and only if there exists a linear transformation $T \in \valuations^{e \times d}$ where $V_1 \vdash_S V_2$ and $TS_1 = S_2$
\end{mydef}
The \qvasr-abstraction $(\tilde{S}_L, \tilde{V}_L)$ for the whole loop has to simulate both $(S_G, V_G)$ and $(S_H, V_H)$, consequently $(S_G, V_G) \preceq (\tilde{S}, \tilde{V})$ and $(S_H, V_H) \preceq (\tilde{S}_L, \tilde{V}_L)$ has to hold. Additionally, we want a precise as possible \qvasr-abstraction, we have to define $(\tilde{S}, \tilde{V})$ as the least upper bound with regard to the preorder. The \qvasr-abstraction that represents the least upper bound is called the \textsl{best} \qvasr-abstraction. \\ \par

Given two \qvasr-abstractions $(S_1, V_1)$ and $(S_2, V_2)$ to compute a best \qvasr-abstraction $(\tilde{S}, \tilde{V})$ such that $(S_1, V_1) \preceq (\tilde{S}, \tilde{V})$ and $(S_2, V_2) \preceq (\tilde{S}, \tilde{V})$ we use the definition of the preorder, to describe that if $(\tilde{S}, \tilde{V})$ is an upper bound of $(S_1, V_1)$ and $(S_2, V_2)$ then there exists matrices $T_1$ and $T_2$ such that $T_1S_1 = \tilde{S} = T_2S_2$ with $V_1 \vdash_{T_1} \tilde{V}$ and $V_2 \vdash_{T_2} \tilde{V}$. To compute the unknown $T_1$ and $T_2$ we convert the equation $T_1S_1 = \tilde{S} = T_2S_2$ to $T_1S_1 = T_2S_2$...



The abstraction $(\tilde{S}, \tilde{V})$ computed by iteratively \textsl{joining} abstractions. Joining two \qvasr-abstractions results in a single abstraction simulating both. Figure \ref{vasr} shows the best \qvasr-abstraction of the loop in $P$, which is the result of joining $G$'s and $H$'s abstractions. \\
\begin{center}
	\begin{figure}[H]
		\begin{align*}
    S_P &= \begin{pmatrix}
        -1 & 1 & 1 \\
        0 & 1 & 0
    \end{pmatrix}, \ \\ \\
    V_P &= \begin{Bmatrix}
        \begin{pmatrix}
              \begin{pmatrix}
                    0 \\
                    1
               \end{pmatrix},
               \begin{pmatrix}
                     1 \\
                     1
               \end{pmatrix}
        \end{pmatrix}, \\ \\
        \begin{pmatrix}
               \begin{pmatrix}
                    1 \\
                    1
               \end{pmatrix},
               \begin{pmatrix}
                    -5 \\
                    -3
              \end{pmatrix}
        \ \end{pmatrix}
    \end{Bmatrix}
\end{align*}
%\caption{\qvasr abstraction $A_p = (S, V)$ of $P$.}
		\caption{Best \qvasr-abstraction of the loop in $P$.}
		\label{vasr}
	\end{figure}
\end{center}
Using this \qvasr-abstraction we can derive the following transition formula as loop summary:
\begin{align*}
	\exists k_1, k_2.\ &((-x' + y' + z' = 1\ \lor\ -x' + y' + z' = -x + y + z - 5k_2)\ \land\ y' = y + k_1 - 3k_2)\ \\ &\lor\ x' = x\ \land\ y' = y\ \land\ z' = z
\end{align*}


\begin{comment}
	\section{Extension to \qvasrs}
	\label{qvasrs}
	
\begin{comment}
	\jw{Introduction qvasrs $\rightarrow$ summary precision improvement \\
	- What are qvasr? How to compute their reachability relation $\rightarrow$ Parikh image? \\ 
	How do they improve precision? \\
	- running example to qvasrs \\
	\vspace{1cm}
	15 pages}
\end{comment}
In the previous chapter, we have seen how to compute a summary for a given transition formula using \qvasr-abstractions. However, these summaries are representing transitions nondeterministically.
We see in the summary, depicted in Figure \ref{loopSummary}, that neither the loop guard \texttt{x <= 20} nor the expressions of the \texttt{if else} statement, \texttt{x <= 10} and \texttt{x > 10}, are considered in the \qvasr-abstraction $(\tilde{S}_L, \tilde{V}_L)$, leading to the inclusion of transitions that would violate these conditions in the actual loop and would therefore not exist. \\
To exclude such transitions, and with that improve precision of the summary, we constrain \qvasr: We only allow transitions that satisfy a set of constraints $C$ modeled by first-order formulas. Using the constraints in $C$ as control states, we can construct a nondeterministic finite automaton that represents all allowed transitions. This automaton is called a \qvasr with states (\qvasrs), where each transition is labeled by a reset, addition vector pair $(\vec{r}, \vec{a})$. The predicates are pairwise unsatisfiable, minimizing the automaton's possible successor state space. A vector $\vec{x}$ can only transition to $\vec{x}'$ if there is an edge $(p, (\vec{r}, \vec{a}), q)$, with $\vec{x}' = \vec{r} * \vec{x} + \vec{a}$ and $p, q \in P$.
Figure \ref{vasrs} shows the \qvasrs of $P$'s loop, it was constructed using $(\tilde{S}_L, \tilde{V}_L)$, seen in Figure \ref{vasr}, and the predicates $x \leq 10$ and $10 < x \land x \leq 20$ as states.
\begin{figure}[H]
	
    \centering
    \begin{tikzpicture}[%
    ->,
    >=stealth',
    shorten >=1pt,
    auto,
    node distance=9cm,
    scale=0.9,
    transform shape,
    align=center,
    smallnode/.style={inner sep=1.2},
    initial text =,
    anchor=center]

    	\node[draw, ellipse, initial left, initial text =](1){$x \leq 10$};
    	\node[draw, ellipse](2) [right of=1] {$10 < x \land x \leq 20$};
    	% \node[draw, ellipse, accepting] (3) [right of=2] {$x > 20$};

    	\path (1) edge node [below]{ $
					\begin{bmatrix}
						-x' + y + z = - x + y + z - 5 \\
						y' = y - 3
					\end{bmatrix}
                     $ } (2)
	
	    	 (1) edge[loop above] node { $
					\begin{bmatrix}
						-x' + y' + z' = - x + y + z - 5 \\
					 	y' = y - 3
					\end{bmatrix}
	         $ } (1)
	
	    	 (2) edge[loop above] node { $
	               \begin{bmatrix}
						-x' + y' + z' = 1 \\
						y' = y + 1
					\end{bmatrix}
	                     $ } (2)
	    	;
    \end{tikzpicture}
    %\caption{\qvasrs abstraction $\mathcal{A}$ of program $P$.}
	\caption{\\ \qvasrs of the loop of $P$.}
	\label{vasrs}
\end{figure}
We see that there are no transitions that violate the \texttt{if else} conditions. \par
\end{comment}

\chapter{Trace Abstraction with \qvasr}
A loop in a program could potentially introduce infinitely many program traces that the trace abstraction scheme would have to prove infeasible. In this chapter we will introduce methods of utilizing \qvasr-based loop summarization to minimize the number of traces created by loops. We present two approaches, the first being an implementation of the \textsl{abstract} step of trace abstraction as seen in Figure \ref{traceAbstractionScheme} utilizing loop summaries directly in traces, the second being a way of transforming a given program's control-flow graph to replace loops entirely by loop summaries.
\label{qvasrAbstracion}
\section{Accelerated Interpolation}

\begin{comment}
	\jw{Accelerated Interpolation as an extension of trace abstraction is introduced $\rightarrow$ Show how to utilize loop summaries within trace abstraction \\
	- Error traces  $\rightarrow$ loop relations  $\rightarrow$ reflexive transitive closures  $\rightarrow$ meta traces  $\rightarrow$ interpolation on meta traces \\
	- "We implemented accelerated interpolation in an earlier project using a myriad of loop summarization techniques... " $\rightarrow$ Introduce various loop summarization techniques (FastUPR, Jordan, Werner) NOT in-depth, only their ideas, but show their pros and cons  $\rightarrow$ The cons build a bridge to the next chapter \\
	- running example will be transformed to a meta trace WITHOUT actual loop summarizations on edges \\
	\vspace{1cm}
	22 pages $\rightarrow$ recycled from earlier}
\end{comment}

Reconsider program $P$ \ref{codeWithAss} from the previous chapter.
To prove its safety \traceabstraction finds error trace $\tau_1$, as before, we detect a loop with $\loc{3}$ and minimal loop trace $\tau_{L_1}$ from which we construct the loop relation $\psi_{L_1}$.
It is evident that there is only one path through the loop, resulting in a loop without branching.
We consequently compute the loop acceleration $\psi^*_{L_1}$.

The loop acceleration contains every looping trace, meaning it is possible to replace the whole loop in the error trace by that acceleration, modelling a relation consisting of every loop trace.
\dd{Why not explain the idea of meta trace first and then give a definition?}

\begin{mydef}
	Given an error trace $\tau: s_0, s_1, \ldots, s_n$ containing loop $\tau_L: s_i, s_{i+1}, \ldots, s_j$ with loop head $\loc{L}$.
	A meta trace $\bar{\tau}$ is derived from $\tau$ by replacing $\tau_L$ with the loop acceleration.
	Furthermore, the last occurrence of $\loc{L}$ is replaced by a new loop exit $\loc{L}'$.
\end{mydef}
\ts{Now you have a mixture of statements and relations. Do you want to allow this (explanation required) or use transformulas in general?}
\dd{Which loop acceleration? Will you talk about properties of loop accelerations? What is a loop acceleration? Is it "the closure"? Or is it \emph{some} relation over program states that has some properties relative to a closure?}

\begin{comment}
The error trace $\tau_1$ creates the meta trace $\bar{\tau_1}$:
\begin{figure}[H]
\begin{tikzpicture}[%
->,
>=stealth', shorten >=1pt, auto,
node distance=2.5cm, scale=1,
transform shape, align=center,
smallnode/.style={inner sep=1.4}
initial text =]

\node[state](1){$\loc{1}$};

\node[state] (2) [right of=1] {$\loc{2}$};

\node[state] (3) [right of=2] {$\loc{3}$};

\node[state] (4) [right of=3] {$\loc{3}'$};

\node[state] (5) [right of=4, xshift=0.5cm] {$\loc{6}$};

\node[state] (6) [right of=5, xshift=0.5cm] {$\loc{7}$};

\path (1) edge node {\texttt{x := 0}} (2); \\
\path (2) edge node {\texttt{y := 1}} (3); \\
\path (3) edge node {$\psi^*_{L_1}$} (4);\\
\path (4) edge node[] {\texttt{!x <= 50}} (5); \\
\path (5) edge node {\texttt{y != 103}} (6); \\
;
\end{tikzpicture}
\captionof{figure}{Meta trace $\bar{\tau_1}$ generated from $\tau_1$ using $\psi^*_{L_1}$.}
\end{figure}
\end{comment}

\begin{figure}[H]
	\begin{center}
		\begin{tabular}{ccccccccccc}
			\loc{1} & \st{x:=0} & \loc{2} & \st{y:=1} & \loc{3} & \accel{1} & $\loc{3}'$ & \st{x>50} & \loc{6} & \st{y!=103} & \loc{7} \\
		\end{tabular}
	\end{center}
	\captionof{figure}{Meta trace $\bar{\tau_1}$ generated from $\tau_1$ using $\psi^*_{L_1}$.}
\end{figure}
We can now analyze this meta trace for feasibility using an SMT-solver such as SMTInterpol\cite{Zitat03} or z3\cite{z3}. We get the following labelling:

\begin{figure}[H]
	\centering
	\begin{tabular}{ccc}
    & $I_{\bar{\tau_1}}$ & \\
    \hline
                & \itp{\top}              & \loc{1} \\
    \st{x:=0}   &                         &         \\
                & \itp{x=0}               & \loc{2} \\
    \st{y:=1}   &                         &         \\
                & \itp{y = 1 \land x = 0} & \loc{3} \\
    \accel{1}   &                         &         \\
                & \itp{                             %
        \begin{array}{rl}
                 & x = 0                               \\
            \lor & ( y \leq 103 \; \land \;  2x + 1 \leq y) \\
            \lor & y = 103
        \end{array}
    }           & \loc{3'}                          \\
    \st{x>50}   &                         &         \\
                & \itp{y = 103}           & \loc{6} \\
    \st{y!=103} &                         &         \\
                & \itp{\bot}              & \loc{7} \\
\end{tabular}
\end{figure}
\captionof{figure}{Meta trace $\bar{\tau_1}$ generated from $\tau_1$ and $\psi^*_{L_1}$ and its infeasibility proof.}
\label{fig:ex:t0:infproof}
%\dd{Do not capitalize captions. They are just normal sentences.}


\dd{Do you still need the line above?}

We cannot, however, use $I_{\bar{\tau_1}}$ to disprove $\tau_1$ as we need an interpolant for each location in the original trace.
\dd{say which one is missing ;)}
To remedy this, we derive an inductive interpolant sequence $I_{\tau_1}$ by applying the post operator.
\dd{Explain why we need post for the before-location, use the example here}

\newcommand{\accels}[1]{\ensuremath{\psi^{*}_{#1}}}
Given
\begin{itemize}[topsep=0pt,itemsep=-1ex,partopsep=1ex,parsep=1ex]
	\item an error trace $\tau: s_0 s_1 \ldots s_i \ldots s_j \ldots s_n$ where \loc{i} is a loop head for the loop $L$ spanning from $s_i$ to $s_j$,
	\item a loop relation $\psi_L$ for loop $L$,
	\item a corresponding loop acceleration \accels{L},
	\item the meta trace $\bar{\tau}: s_0 s_1 \ldots s_{i-1} \ \accels{L} \ s_{j+1} \ldots s_n$ derived from $\tau$ and \accels{L}, and
	\item the infeasibility proof $I_{\bar{\tau}}: \{\top, I_1, I_2, \ldots , I_i, I_{\psi^*_{L}}, \ldots , I_{n-1}, \bot \}$ for $\bar{\tau}$.
\end{itemize}
\dd{Fix indices s.t. we have the interpolant (sic!) before and after the loop acceleration}

To construct an inductive proof of infeasibility for $\tau$ we need inductive interpolants for the loop statements $s_i, \ldots , s_j$ that were replaced by $\psi^*_{L}$.

Firstly, compute post($I_{\psi^*_L}$, $\psi^*_L$) as the loop entry interpolant $I_{\loc{L}}$.
From there keep applying the post operator with the previous location's interpolant and the following program statement.

We get the inductive interpolant sequence
\begin{equation*}
	I_\tau: \{\top,I_1,I_2, \ldots ,\ \underbrace{post(I_i, \accels{L})}_{I_{i}^*},\ \ \underbrace{post(I_{i}^*, s_i)}_{I_{i+1}^*},\ \ldots ,\ \underbrace{post(I_{j-1}^*, s_j)}_{I_{j}^*},I_{j+1}, \ldots ,I_{n-1}, \bot \}
\end{equation*}
which can now be used by trace abstraction to refine the interpolant automaton.
\ts{Explain why this works and why it is necessary.}

We compute the missing interpolants for example program trace $\tau_1$ as follows:
\dd{Strange wording. This is just the continuaton of the example, right?}
\begin{comment}
\begin{figure}[H]
\centering
\begin{tikzpicture}[%
    ->,
    >=stealth', shorten >=1pt, auto,
    node distance=3.25cm, scale=1,
    transform shape, align=center,
    smallnode/.style={inner sep=2}
    initial text =]

\node[state, label=above:{$\top$}](1){$\ell_1$};
\node[state, label=above:{$x = 0$}] (2) [right of=1] {$\ell_2$};
\node[state, label=above:{$\begin{aligned}
         & y = 2 \cdot x + 1 \\&\land x \leq 51
    \end{aligned}$}] (3) [right of=2] {$\ell_3$};
\node[state, label=above:{$\begin{aligned}
         & y = 2 \cdot x + 1 \\&\land x \leq 50
    \end{aligned}$}] (4) [right of=3] {$\ell_4$};
\node[state, label=above:{$\begin{aligned}
         & y + 1 = 2 \cdot x \\&\land x \leq 51
    \end{aligned}$}] (5) [right of=4] {$\ell_5$};
\node[state, label=above:{$\begin{aligned}
         & y = 2 \cdot x + 1 \\&\land x \leq 51
    \end{aligned}$}] (6) [below of=1] {$\ell_3$};
\node[state, label=below:{$\begin{aligned}
         & y = 2 \cdot x + 1 \\&\land x \leq 50
    \end{aligned}$}] (7) [right of=6] {$\ell_4$};
\node[state, label={[xshift=0.7cm]above:{$\begin{aligned}
         & y + 1 = 2 \cdot x \\&\land x \leq 51
    \end{aligned}$}}] (8) [right of=7] {$\ell_5$};
\node[state, label={[xshift=0cm]above:{$\begin{aligned}
         & y = 2 \cdot x + 1 \\&\land x \leq 51
    \end{aligned}$}}] (9) [right of=8] {$\ell_3$};
\node[state, label={[xshift=0cm]above:{$\begin{aligned}
         & y = 2 \cdot x + 1 \\&\land x \leq 50
    \end{aligned}$}}] (10) [right of=9] {$\ell_4$};
\node[state, label=above:{$\begin{aligned}
         & y + 1 = 2 \cdot x \\&\land x \leq 51
    \end{aligned}$}] (11) [below of=6] {$\ell_5$};
\node[state, label={[xshift=0cm]below:{$\begin{aligned}
         & y = 2 \cdot x + 1 \\&\land x \leq 51
    \end{aligned}$}}] (12) [right of=11] {$\ell_3$};
\node[state, label={[xshift=0cm]below:{$\begin{aligned}
         & y = 2 \cdot x + 1 \\&\land x \leq 51\ \land\ x > 50 \\
         & \equiv y = 103
    \end{aligned}$}}] (13) [right of=12] {$\ell_6$};
\node[state, label=above:{$\bot$}] (14) [right of=13] {$\ell_7$};


\path (1) edge node {\texttt{x := 0}} (2);
\path (2) edge node {\texttt{y := 1}} (3);
\path (3) edge node {\texttt{x <= 50}} (4);
\path (4) edge node {\texttt{x := x + 1}} (5);
\path (5) edge node[left, xshift=-.7cm] {\texttt{y := y + 2}} (6);
\path (6) edge node[below] {\texttt{x <= 50}} (7);
\path (7) edge node[below] {\texttt{x := x + 1}} (8);
\path (8) edge node[below] {\texttt{y := y + 2}} (9);
\path (9) edge node {\texttt{x <= 50}} (10);
\path (10) edge node {\texttt{x := x + 1}} (11);
\path (11) edge node[below] {\texttt{y := y + 2}} (12);
\path (12) edge node[below] {\texttt{!x <= 50}} (13);
\path (13) edge node[below] {\texttt{y != 103}} (14);
;
\end{tikzpicture}
\captionof{figure}{Program trace $\tau_1$ of $P_1$ with inductive infeasibility proof.}
\end{figure}
\end{comment}

\begin{figure}[H]
	\begin{center}
		\begin{tabular}{ccc}
    & $I_{\tau_1}$ & \\
    \hline
                & \itp{\top}                                  & \loc{1} \\
    \st{x:=0}   &                                             &         \\
                & \itp{x=0}                                   & \loc{2} \\
    \st{y:=1}   &                                             &         \\
    \midrule
                & \itp{y = 2x + 1 \land x \leq 51}            & \loc{3} \\
    \st{x<=50}  &                                             &         \\
                & \makecell{\itp{y = 2x + 1 \land x \leq 50}} & \loc{4} \\
    \st{x:=x+1} &                                             &         \\
                & \itp{y + 1 = 2x}                            & \loc{5} \\
    \st{y:=y+2} &                                             &         \\
    \midrule
                & \itp{y = 2x + 1 \land x \leq 51}            & \loc{3} \\
    \st{x<=50}  &                                             &         \\
                & \makecell{\itp{y = 2x + 1 \land x \leq 50}} & \loc{4} \\
    \st{x:=x+1} &                                             &         \\
                & \itp{y + 1 = 2x}                            & \loc{5} \\
    \st{y:=y+2} &                                             &         \\
    \midrule
                & \itp{y = 2x + 1 \land x \leq 51}            & \loc{3} \\
    \st{x<=50}  &                                             &         \\
                & \makecell{\itp{y = 2x + 1 \land x \leq 50}} & \loc{4} \\
    \st{x:=x+1} &                                             &         \\
                & \itp{y + 1 = 2x}                            & \loc{5} \\
    \st{y:=y+2} &                                             &         \\
    \midrule
                & \itp{y = 2x + 1 \land x \leq 51}            & \loc{3} \\
    \st{x>50}   &                                             &         \\
                & \itp{\begin{array}{ll} & y = 2x + 1 \land x \leq 51 \land x > 50 \\[-1mm] \equiv & y = 103 \land x = 51 \end{array}}             & \loc{6} \\
    \st{y!=103} &                                             &         \\
                & \itp{\bot}                                  & \loc{7}
\end{tabular}
	\end{center}
\end{figure}
\captionof{figure}{Program trace $\tau_1$ of $P_1$ with inductive infeasibility proof.}
\label{fig:ex:t0:infproof2}

% \ts{The simplification of the interpolant at $\loc{6}$ is wrong.}

\label{accelInterpol}

\section{Control-Flow Graph Transformation}

\begin{comment}
	\jw{Introduction to Ultimate, Accelerated Inteprolation in Ultimate, Qvasr and Qvasrs libraries \\
	- Short introduction: what is Ultimate? $\rightarrow$ software verification framework with toolchains, such as trace abstraction \\
	- accelerated interpolation in ultimate $\rightarrow$ How does it work $\rightarrow$ get trace $\rightarrow$ loop detector $\rightarrow$ accelerator $\rightarrow$ meta trace transformer $\rightarrow$  interpolator (maybe as diagram) \\
	- Focus on qvasr $\rightarrow$ how does that library work? $\rightarrow$ same for qvasrs \\
	\vspace{1cm}
	15 pages}
\end{comment}

Another approach of utilizing \qvasr summarization is, instead applying it to traces, to change trace abstraction's input control-flow graph, such that computed error traces already contain summaries. Given a program $P$ and its control-flow graph $G_P$, we devise a depth-first search algorithm to find loops. We introduce a marking function $\alpha: Loc \rightarrow \{nv, v\}$, where over the control-flow graphs's locations, with $nv$ meaning not visited, $v$ meaning visited. We initialize a stack $\Gamma$, an empty set of traces $T$, and start the search with $\ell_{init}$, we add each transition $\pi_i$, where $src(\pi_i) = \ell_{init}$ to $\Gamma$. The location $\ell_{init}$ is now marked $v$ and we repeat the following loop detection scheme until $\Gamma$ is empty:

\begin{enumerate}
	\item Pop new transition $\pi_j$ from $\Gamma$
	\item Check if $\ell_j = tgt(\pi_j)$ is marked.
	\item If yes, we have found a loop, with $\ell_j$ being the entry point.
		We begin to backtrack by initializing a new stack $\Gamma_j$, pushing every transition $\pi_k$ where $tgt(\pi_k) = src(\pi_j)$ to it and construct the new trace $\tau = \pi_j$ . For finding the loop entry and with that the trace of the loop, we repeat the following steps: While $\Gamma_j$ is not empty, pop new transition $\pi_h$ from $\Gamma_j$ and check if $src(\pi_h) = \ell_j$:
		\begin{enumerate}
			\item If that is the case we found a loop cycle and add $\tau$ to $T$. 
			\item If not, we construct new traces for each $\pi_d$, where $tgt(\pi_d) = src(\pi_h)$, by copying $\tau$ and concatenating $\pi_d$ to it and push each $\pi_d$ to $\Gamma_j$
		\end{enumerate}
	\item If no, add transitions $\pi_l$ with $src(\pi_l) = \ell_j$ to $\Gamma$.
\end{enumerate}
After the search, the set of traces $T$ contains all loops in program $P$. As there can be multiple traces from a loop entry location $\ell_i$, we have to construct a disjunction of the transition formulas beginning in $\ell_i$:
Assume we have two traces $\tau_1 = \pi_{11}, \ldots, \pi_{1n}$ and $\tau_2 = \pi_{21}, \ldots \pi_{2m}$ in $T$, where $src(\pi_{11}) = scr(\pi_{21})$, we construct the formula $\tau_1 \lor \tau_2$ to represent the loop.
This disjunction can now be used in the \qvasr summarization scheme, as defined before.  \\ \par
Recall the program $P$ and its control-flow graph $G_P$ as illustrated in \ref{codeWithAss} and \ref{cfg:P:Ass}, using $G_P$ and the loop detection scheme, we extract the following loop traces for loop entry $\ell_4$: 
\begin{align*}
	\tau_1 &= \st{x<=20}, \st{x<=10}, \st{z:=x;x:=x+y} \\
	\tau_2 &= \st{x<=20}, \st{x>10}, \st{x:=x+2;y:=y-3}
\end{align*}
Using \qvasr summarization, we get the summary $\psi$ shown before in Figure \ref{loopTF}, which we can now use to replace the every transition occuring in either $\tau_1$ and $\tau_2$. We get transformed control-flow graph $\bar{G}_P$:

\begin{figure}[H]
	\centering
	
\begin{tikzpicture}[%
	->,
	>=stealth',
	shorten >=1pt,
	auto,
	node distance=3.25cm,
	scale=0.9,
	transform shape,
	align=center,
	smallnode/.style={inner sep=1.4},
	initial text =,
	anchor=center]
	
	\node[state, initial above, initial text =](1){$\loc{1}$};
	\node[state] (head) [below of=1] {$\loc{4}$};
	\node[state] (loopExit)[right of=head] {$\loc{13}$};
	
	\node[state] (assertTrue)[below left of=loopExit] {$\loc{14}$};
	\node[state, accepting] (assertFalse)[below of=loopExit] {$\loc{err}$};
	
	\path (1) edge node []{\st{x:=0;}\\ \st{y:=2;}\\ \st{z:=3;}} (head)
	(head) edge node []{\st{x>20}} (loopExit)
	(head) edge[loop left] node {\st{$\psi$}} (head)
	
	(loopExit) edge node []{\st{x==22}} (assertTrue)
	(loopExit) edge node []{\st{x!=22}} (assertFalse)
	;
\end{tikzpicture}
	\label{cfg_trans}
	\caption{The transformed control-flow graph $\bar{G}_P$ that is the result of the control-flow graph transformation of $G_P$}
\end{figure}





\label{icfgTransformation}

\chapter{Evaluation}
\label{eval}

We have implemented a \qvasr based loop summarization library in the software verification framework \ultimate, that uses trace abstraction to prove safety, as an adaption of the previously defined methods. We implemented \qvasr usage as an abstraction method in  accelerated interpolation and as a control-flow graph transformer. To test performance of our \qvasr based loop summarization, we used a set of 1766 example $C$ programs, which are part of the \texttt{sv-comp} \cite{svcomp} program set. Furthermore, we compared it to another summarization scheme, based on transforming transition formulas into a Jordan normal form matrix \cite{DBLP:conf/popl/JeannetSS14}. \par
We used \ultimate Automizer version 0.2.2-d966a43b, with time limit: 900 seconds, memory limit: 8000 MB, using two CPU cores of type AMD Ryzen Threadripper 3970X 32-Core Processor, on Linux-5.11.22-4-pve-x86\_64-with-glibc2.31, with frequency: 4549 MHz and using 137439 MB of RAM. In this Chapter we show the and interpret the results. We begin with the evaluation of trace abstraction using accelerated interpolation, followed by trace abstraction with control-flow graph transformation.

\section{Results of using Accelerated Interpolation}
We present in this Section our results on applying accelerated interpolation using either the Jordan based loop summarization or \qvasr versus trace abstraction without accelerated interpolation. We computed the results as portrayed in Figure \ref{table_acc}
\begin{figure}[H]
	\centering
		\begin{tabular}{cccc}
			\toprule
			& No summarization & Jordan & \qvasr \\
			\cmidrule{1-4}
			Correct Results & 508/859 & 355/859 & 356/859\\
			Wrong Results & 0/859 & 0/859 & 14/859\\
			Timeout & 268/859 & 247/859 & 348/859 \\
			Solve Time &  274000s & 255000s & 349000s
		\end{tabular}
	\caption{Results of benchmarking trace abstraction using accelerated interpolation without loop summarization, with Jordan based loop summarization, and \qvasr loop summarization}
	\label{table_acc}
\end{figure}
Using accelerated interpolation, our \qvasr library was able to solve 20 programs that neither trace abstraction without summarization, nor the Jordan based summarization could solve. These programs are mostly containing variable assigments of the form \st{x:=x+y}. We derive that trace abstraction without summarization is unrolling each loop iteration, leading to infinitely many traces, and that Jordan is not able to compute a matrix needed for the technique, falling back to unrolling as well. Our \qvasr based implementation however can construct a summary in able time. We see, nonetheless, that the \qvasr summarization technique produces a few wrong results, these are caused by $C$ programs whose variables are unsigned, as they produce a formula containing modulo operations, with which our \qvasr implementation struggles in some cases, such as inherit issues with the accelerated interpolation library.

\section{Results of using Control-Flow Graph Transformation}
In contrast to accelerated interpolation, which uses traces for computing Floyd-Hoare annotations, we also implemented the \qvasr summarization library as a preprocessor that transforms a given program's control-flow graph by replacing loops with their summary. The results are depicted in Figure \ref{table_tff}.
\begin{figure}[H]
	\centering
	\begin{tabular}{cccc}
		\toprule
		& No summarization & Jordan & \qvasr \\
		\cmidrule{1-4}
		Correct & 508/859 & 516/859 & 518/859 \\
		Wrong & 0/859 & 0/859 & 6/859 \\
		Timeout & 268/859 & 234/859 & 232/859 \\
		Solve Time &  274000s & 241000s & 242000s
	\end{tabular}
	\caption{Results of benchmarking trace abstraction using control-flow graph transformation without loop summarization, with Jordan based loop summarization, and \qvasr loop summarization}
	\label{table_tff}
\end{figure}
We see that in this test case set, that both loop summarization techniques solved more programs than trace abstraction alone, whereas our \qvasr summarization contributed to solving 16 programs that neither trace abstraction solo nor the Jordan based summarization could solve. Though, there are again wrong solutions, that are caused by the same reasons as before. \par 

\chapter{Future Work}
\label{futrWork}

In this thesis we have shown a technique to compute a loop summary using \qvasr, integrated that technique into trace abstraction, by using it in the accelerated interpolation scheme and as a method to transform a control-flow graph by replacing loops with summaries. There are, however, still some areas that can be improved upon. In this section we introduce possible improvements to \qvasr loop summarization and its usage. \\ \par
\subsection{Using \qvasrs}
In chapter \ref{qvasrs} we have shown that there is an extension to \qvasr that factor in assumptions in a loop's transition formula, such as the expressions of the loop guard and if else statements. It is possible to integrate \qvasrs into our implemented summary utilization schemes, accelerated interpolation and control-flow graph transformation, as follows: \\ \par
For accelerated interpolation, assume we are given a trace $\tau$ with a loop, whose transition formula we extract as $\tau_L$, we compute a \qvasrs $\mathcal{V} = (P, E)$ for $\tau_L$, now because a \qvasrs is a graph structure, we need to compute its reachability relation to summarize the loop. To solve this task, we can utilize Haase and Halfon \cite{DBLP:conf/rp/HaaseH14} proposed polytime procedure that computes a series of formulas from generalized Parikh images that, as conjunction, form a summary of the \qvasrs. \\ \par
For \qvasrs usage in control-flow graph transformation, we adapt our approach by, instead of replacing the whole loop by its computed summary, inserting the \qvasrs instead. We illustrate this approach by transforming the given example program $P$ as seen in Figure \ref{codeWithAss} with its control-flow graph $G_P$ as depicted in Figure \ref{cfg:P:Ass}, to the control-flow graph, seen in Figure \ref{qvasrs_cfg}
\begin{figure}[H]
	
\begin{tikzpicture}[%
	->,
	>=stealth',
	shorten >=1pt,
	auto,
	node distance=3.25cm,
	scale=0.9,
	transform shape,
	align=center,
	smallnode/.style={inner sep=1.4},
	initial text =,
	anchor=center]
	
	\node[state, initial above, initial text =](1){$\loc{1}$};
   	\node[draw, ellipse](vasrs1) [below of=1] {$x \leq 10$};
	\node[draw, ellipse](vasrs2) [left of=vasrs1, xshift=-4cm] {$10 < x \land x \leq 20$};
	\node[state] (loopExit)[right of=head,  xshift=2cm] {$\loc{13}$};
	\node[state] (assertTrue)[below left of=loopExit] {$\loc{14}$};
	\node[state, accepting] (assertFalse)[below of=loopExit] {$\loc{err}$};
	
	\path (1) edge node []{\st{x:=0;}\\ \st{y:=2;}\\ \st{z:=3;}} (head)
	(vasrs1) edge node []{\st{x>20}} (loopExit)
	(vasrs2) edge[bend left] node {\st{-x+y+z:=1;} \\ \st{y:=y+1}} (vasrs1)
	(vasrs1) edge[bend left] node[] {\st{-x+y+z:=-x+y+z-5;}\\\st{y:=y-3}} (vasrs2)
	(loopExit) edge node []{\st{x==22}} (assertTrue)
	(loopExit) edge node []{\st{x!=22}} (assertFalse)
	
	(vasrs1) edge[loop below] node[] {\st{-x+y+z:=-x+y+z-5;}\\ \st{y:=y-3}}(vasrs1)
	(vasrs2) edge[loop above] node[] {\st{-x+y+z:=1;}\\ \st{y:=y+1}}(vasrs2)
	;
	
\end{tikzpicture}
	\label{qvasrs_cfg}
	\caption{Control-flow graph $\bar{G}_P$ created from \ref{cfg:P:Ass} by replacing the loop with a \qvasrs.}
\end{figure}
From this transformed control-flow graph we can now construct state assertions that adhere to assumptions in the program.
\subsection{Heuristics in Accelerated Interpolation}

\begin{comment}
	To compute a loop summary from a \qvasrs, one has to calculate the reachability relation of the automaton. Haase and Halfon \cite{DBLP:conf/rp/HaaseH14} proposed a polytime procedure that computes a series of formulas from computed Parikh images that, as conjunction, form a summary of the system. This procedure can be adapted to work with \qvasrs
\end{comment}


	\setlength{\abovedisplayskip}{0.25cm}
\setlength{\belowdisplayskip}{0.25cm}
\setlength{\abovedisplayshortskip}{0.25cm}
\setlength{\belowdisplayshortskip}{0.25cm}
\chapter{Conclusion}
\label{concl}

\jw{What have we done in this project? What have we learned? What can there still be done to boost performance/fix errors/etc \\
	Give an outlook on the future \\
	\vspace{1cm} 
	2 pages}


\pagebreak
\bibliographystyle{ieeetr}
\bibliography{bib/bib}

\listoffigures

\end{document}