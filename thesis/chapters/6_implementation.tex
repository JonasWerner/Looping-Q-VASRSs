
\begin{comment}
	\jw{Introduction to Ultimate, Accelerated Inteprolation in Ultimate, Qvasr and Qvasrs libraries \\
	- Short introduction: what is Ultimate? $\rightarrow$ software verification framework with toolchains, such as trace abstraction \\
	- accelerated interpolation in ultimate $\rightarrow$ How does it work $\rightarrow$ get trace $\rightarrow$ loop detector $\rightarrow$ accelerator $\rightarrow$ meta trace transformer $\rightarrow$  interpolator (maybe as diagram) \\
	- Focus on qvasr $\rightarrow$ how does that library work? $\rightarrow$ same for qvasrs \\
	\vspace{1cm}
	15 pages}
\end{comment}

Another approach of utilizing \qvasr summarization is, instead applying it to traces, to change trace abstraction's input control-flow graph, such that computed error traces already contain summaries. Given a program $P$ and its control-flow graph $G_P$, we devise a depth-first search algorithm to find loops. We introduce a marking function $\alpha: Loc \rightarrow \{nv, v\}$, where over the control-flow graphs's locations, with $nv$ meaning not visited, $v$ meaning visited. We initialize a stack $\Gamma$, an empty set of traces $T$, and start the search with $\ell_{init}$, we add each transition $\pi_i$, where $src(\pi_i) = \ell_{init}$ to $\Gamma$. The location $\ell_{init}$ is now marked $v$ and we repeat the following loop detection scheme until $\Gamma$ is empty:

\begin{enumerate}
	\item Pop new transition $\pi_j$ from $\Gamma$
	\item Check if $\ell_j = tgt(\pi_j)$ is marked.
	\item If yes, we have found a loop, with $\ell_j$ being the entry point.
		We begin to backtrack by initializing a new stack $\Gamma_j$, pushing every transition $\pi_k$ where $tgt(\pi_k) = src(\pi_j)$ to it and construct the new trace $\tau = \pi_j$ . For finding the loop entry and with that the trace of the loop, we repeat the following steps: While $\Gamma_j$ is not empty, pop new transition $\pi_h$ from $\Gamma_j$ and check if $src(\pi_h) = \ell_j$:
		\begin{enumerate}
			\item If that is the case we found a loop cycle and add $\tau$ to $T$. 
			\item If not, we construct new traces for each $\pi_d$, where $tgt(\pi_d) = src(\pi_h)$, by copying $\tau$ and concatenating $\pi_d$ to it and push each $\pi_d$ to $\Gamma_j$
		\end{enumerate}
	\item If no, add transitions $\pi_l$ with $src(\pi_l) = \ell_j$ to $\Gamma$.
\end{enumerate}
After the search, the set of traces $T$ contains all loops in program $P$. As there can be multiple traces from a loop entry location $\ell_i$, we have to construct a disjunction of the transition formulas beginning in $\ell_i$:
Assume we have two traces $\tau_1 = \pi_{11}, \ldots, \pi_{1n}$ and $\tau_2 = \pi_{21}, \ldots \pi_{2m}$ in $T$, where $src(\pi_{11}) = scr(\pi_{21})$, we construct the formula $\tau_1 \lor \tau_2$ to represent the loop.
This disjunction can now be used in the \qvasr summarization scheme, as defined before.  \par
Recall the program $P$ and its control-flow graph $G_P$ as illustrated in \ref{codeWithAss} and \ref{cfg:P:Ass}, using $G_P$ and the loop detection scheme, we extract the following loop traces for loop entry $\ell_4$: 
\begin{align*}
	\tau_1 &= \st{x<=20}, \st{x<=10}, \st{z:=x;x:=x+y} \\
	\tau_2 &= \st{x<=20}, \st{x>10}, \st{x:=x+2;y:=y-3}
\end{align*}
Using \qvasr summarization, we get the summary $\psi$ shown before in Figure \ref{loopTF}, which we can now use to replace every transition occurring in either $\tau_1$ and $\tau_2$. We get transformed control-flow graph $\bar{G}_P$:

\begin{figure}[H]
	\centering
	
\begin{tikzpicture}[%
	->,
	>=stealth',
	shorten >=1pt,
	auto,
	node distance=3.25cm,
	scale=0.9,
	transform shape,
	align=center,
	smallnode/.style={inner sep=1.4},
	initial text =,
	anchor=center]
	
	\node[state, initial above, initial text =](1){$\loc{1}$};
	\node[state] (head) [below of=1] {$\loc{4}$};
	\node[state] (loopExit)[right of=head] {$\loc{13}$};
	
	\node[state] (assertTrue)[below left of=loopExit] {$\loc{14}$};
	\node[state, accepting] (assertFalse)[below of=loopExit] {$\loc{err}$};
	
	\path (1) edge node []{\st{x:=0;}\\ \st{y:=2;}\\ \st{z:=3;}} (head)
	(head) edge node []{\st{x>20}} (loopExit)
	(head) edge[loop left] node {\st{$\psi$}} (head)
	
	(loopExit) edge node []{\st{x==22}} (assertTrue)
	(loopExit) edge node []{\st{x!=22}} (assertFalse)
	;
\end{tikzpicture}
	\label{cfg_trans}
	\caption{The transformed control-flow graph $\bar{G}_P$ that is the result of the control-flow graph transformation of $G_P$}
\end{figure}
This transformed control-flow graph can now be put into the trace abstraction scheme, seen in	Figure \ref{traceAbstractionScheme}, and be used to prove safety.


