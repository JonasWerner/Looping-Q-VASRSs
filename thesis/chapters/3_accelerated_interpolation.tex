
We implemented the \qvasr based loop summarization into a framework that uses loop summaries as an abstraction in trace abstraction, called accelerated interpolation. This scheme is based on the findings by \cite{10.1007/978-3-642-33386-6_16}, in the following we will introduce the idea behind it. \par Given a program $P$ and its control-flow graph $G_P = $ \cfg, we have seen earlier that to determine safety of $P$ we need to prove infeasibility of every trace leading to an error-location. We translate $P's$ control-flow graph into a non-deterministic finite automata $\cfgAutomat{P}$, and try to compute a Floyd-Hoare automaton $\cfgAutomatHoare{D}$, whose language are the infeasible traces. If $\mathcal{L(A_P)} \subseteq \mathcal{L(A_D)}$ we know that $P$ is safe, if there exists a feasible error trace it is unsafe. \\ To construct the Floyd-Hoare annotation of $\mathcal{L(A_D)}$, we have to iteratively prove traces infeasible, and then, to exclude as many error traces as possible, compute an abstraction. In the example from earlier, given trace $\tau = \pi_1, \ldots, \pi_n$, we utilized the abstraction method of adding traces from $\Sigma$, such that the condition $sp(\varphi_i, \pi_i) \subseteq \{\varphi\}$, with $\varphi_i$ being the annotation of state $q_i$ and $pi_i \in \tau$, holds. \\
In this section we introduce an abstraction technique that utilizes loop summaries to refute whole loops with one trace abstraction iteration. \par
Assume we are give a program $P$ and its control-flow graph \\ $G_P$ = \cfg, and assume we employ trace abstraction that computes the program trace $\tau$:
\begin{equation*}
	\tau = \pi_1, \ldots, \underbrace{\pi_i, \ldots, \pi_j, \ldots, \pi_i, \ldots, \pi_j, \ldots, \pi_i, \ldots, \pi_j}_{m \text{ times}}, \ldots, \pi_n
\end{equation*}
We observe that the sequence of transitions $\pi_i, \ldots, \pi_j, \ldots, \pi_i, \ldots, \pi_j$ is repeated $m$ times, indicating that the sequence is a loop in the program which is iterated repeatedly. We construct a transition formula $\bar{\pi} = \pi_i; \ldots; \pi_j$ representing one loop iteration by concatenating formulas $\pi_i, \ldots, \pi_j$. \\
Using $\bar{\pi}$ we compute a \qvasr loop summary $\psi$ following the earlier presented summarization paradigm. The loop summary $\psi$ now represents an overapproximation of all changes to program variables by the loop. We can now transform $\tau$ by replacing the sequence $\pi_i \ldots \pi_j$ with $\psi$, yielding a so called \textsl{meta trace}.

\begin{mydef}[Meta Trace]
	Given a trace $\tau$:
	\begin{equation*}
		\tau = \pi_1, \ldots, \pi_i, \ldots, \pi_j, \ldots, \pi_i, \ldots, \pi_j, \ldots, \pi_i, \ldots, \pi_j, \ldots, \pi_n
	\end{equation*}
	Where $\tau$ contains a repeating sequence of transition formulas $\pi_i, \ldots, \pi_j$ that can be represented by the concatenation of all $\pi_l \{\pi_i, \ldots, \pi_j\}$: $\bar{\pi} = \pi_i; \ldots; \pi_j$. Assume further that we are given a \qvasr summary for formula $\bar{\pi}$, we call the trace  $\bar{\tau} = \pi_1, \ldots, \psi, \ldots, \pi_n$ the meta trace of $\tau$, which is constructed by replacing the sequence $\pi_1, \ldots, \pi_j$ by $\psi$.
\end{mydef}
\begin{comment}
	\begin{equation*}
	\bar{\tau} = \pi_1, \ldots, \psi, \ldots, \pi_n
	\end{equation*}
	The meta trace $\bar{\tau}$ now contains every trace that begins with $\pi, \ldots, \pi_{i-1}$ and ends with $\pi_{j+1}$ going through the looping sequence $\pi_i, \ldots, \pi_j$. \par
\end{comment}
Following this definition, we get the following meta trace for $\tau$:


For example, consider again the program depicted in Figure \ref{codeWithAss} and its control-flow graph in Figure \ref{cfg:P:Ass}, and assume trace abstraction computed the following program trace: 
\begin{align*}
	\tau_1= &\st{x:=0;y:=2;z:=3}, \\ &\st{x<=20}, \st{x<=10}, \st{z:=x;x:=x+y;y:=y+1}, \\ &\st{x<=20}, \st{x<=10}, \st{z:=x;x:=x+y;y:=y+1}, \\ & \st{x<=20}, \st{x>10}, \st{x:=x+2;y:=y-3}, \\ &\st{x<=20}, \st{x>10}, \st{x:=x+2;y:=y-3}, \\ &\st{x>20}, \st{x!=22}
\end{align*}
We see that the following sequence of transitions occur in the trace multiple times:
\begin{align*}
	G: &\st{x<=20}, \st{x>10}, \st{x:=x+2;y:=y-3} \\
	H: &\st{x<=20}, \st{x<=10}, \st{z:=x;x:=x+y;y:=y+1}
\end{align*}
We compute the transition formula $L$ as $G \lor H$, which coincides with the transition formula portrayed earlier in Figure \ref{loopTF}:
\begin{equation*}
	L: x \leq 20 \land ((x \leq 10 \land z' = x \land x' = x + y \land y' = y + 1) \lor (x > 10 \land x' = x + 2 \land y' = y - 3)
\end{equation*}
As we have already computed a \qvasr summary $\psi_L$ for transition formula $L$, depicted in Figure \ref{loopSummary}, we replace sequences $G$ and $H$ with $\psi_L$ to get meta trace $\bar{\tau}_1:$
\begin{align*}
	\bar{\tau}_1: \st{x:=0;y:=2;z:=3}, \psi_L, \st{x>20}, \st{x!=22}
\end{align*}
This meta trace now contains every trace through the loop, from which we can now construct an approximation.