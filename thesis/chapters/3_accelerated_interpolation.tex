
\begin{comment}
	\jw{Accelerated Interpolation as an extension of trace abstraction is introduced $\rightarrow$ Show how to utilize loop summaries within trace abstraction \\
	- Error traces  $\rightarrow$ loop relations  $\rightarrow$ reflexive transitive closures  $\rightarrow$ meta traces  $\rightarrow$ interpolation on meta traces \\
	- "We implemented accelerated interpolation in an earlier project using a myriad of loop summarization techniques... " $\rightarrow$ Introduce various loop summarization techniques (FastUPR, Jordan, Werner) NOT in-depth, only their ideas, but show their pros and cons  $\rightarrow$ The cons build a bridge to the next chapter \\
	- running example will be transformed to a meta trace WITHOUT actual loop summarizations on edges \\
	\vspace{1cm}
	22 pages $\rightarrow$ recycled from earlier}
\end{comment}

Reconsider program $P$ \ref{code} from the previous chapter.
To prove its safety \traceabstraction finds error trace $\tau_1$, as before, we detect a loop with $\loc{3}$ and minimal loop trace $\tau_{L_1}$ from which we construct the loop relation $\psi_{L_1}$.
It is evident that there is only one path through the loop, resulting in a loop without branching.
We consequently compute the loop acceleration $\psi^*_{L_1}$.

The loop acceleration contains every looping trace, meaning it is possible to replace the whole loop in the error trace by that acceleration, modelling a relation consisting of every loop trace.
\dd{Why not explain the idea of meta trace first and then give a definition?}

\begin{mydef}
	Given an error trace $\tau: s_0, s_1, \ldots, s_n$ containing loop $\tau_L: s_i, s_{i+1}, \ldots, s_j$ with loop head $\loc{L}$.
	A meta trace $\bar{\tau}$ is derived from $\tau$ by replacing $\tau_L$ with the loop acceleration.
	Furthermore, the last occurrence of $\loc{L}$ is replaced by a new loop exit $\loc{L}'$.
\end{mydef}
\ts{Now you have a mixture of statements and relations. Do you want to allow this (explanation required) or use transformulas in general?}
\dd{Which loop acceleration? Will you talk about properties of loop accelerations? What is a loop acceleration? Is it "the closure"? Or is it \emph{some} relation over program states that has some properties relative to a closure?}

\begin{comment}
The error trace $\tau_1$ creates the meta trace $\bar{\tau_1}$:
\begin{figure}[H]
\begin{tikzpicture}[%
->,
>=stealth', shorten >=1pt, auto,
node distance=2.5cm, scale=1,
transform shape, align=center,
smallnode/.style={inner sep=1.4}
initial text =]

\node[state](1){$\loc{1}$};

\node[state] (2) [right of=1] {$\loc{2}$};

\node[state] (3) [right of=2] {$\loc{3}$};

\node[state] (4) [right of=3] {$\loc{3}'$};

\node[state] (5) [right of=4, xshift=0.5cm] {$\loc{6}$};

\node[state] (6) [right of=5, xshift=0.5cm] {$\loc{7}$};

\path (1) edge node {\texttt{x := 0}} (2); \\
\path (2) edge node {\texttt{y := 1}} (3); \\
\path (3) edge node {$\psi^*_{L_1}$} (4);\\
\path (4) edge node[] {\texttt{!x <= 50}} (5); \\
\path (5) edge node {\texttt{y != 103}} (6); \\
;
\end{tikzpicture}
\captionof{figure}{Meta trace $\bar{\tau_1}$ generated from $\tau_1$ using $\psi^*_{L_1}$.}
\end{figure}
\end{comment}

\begin{figure}[H]
	\begin{center}
		\begin{tabular}{ccccccccccc}
			\loc{1} & \st{x:=0} & \loc{2} & \st{y:=1} & \loc{3} & \accel{1} & $\loc{3}'$ & \st{x>50} & \loc{6} & \st{y!=103} & \loc{7} \\
		\end{tabular}
	\end{center}
	\captionof{figure}{Meta trace $\bar{\tau_1}$ generated from $\tau_1$ using $\psi^*_{L_1}$.}
\end{figure}
We can now analyse this meta trace for feasibility using an SMT-solver such as SMTInterpol\cite{Zitat03} or z3\cite{z3}. We get the following labelling:

\begin{figure}[H]
	\centering
	\begin{tabular}{ccc}
    & $I_{\bar{\tau_1}}$ & \\
    \hline
                & \itp{\top}              & \loc{1} \\
    \st{x:=0}   &                         &         \\
                & \itp{x=0}               & \loc{2} \\
    \st{y:=1}   &                         &         \\
                & \itp{y = 1 \land x = 0} & \loc{3} \\
    \accel{1}   &                         &         \\
                & \itp{                             %
        \begin{array}{rl}
                 & x = 0                               \\
            \lor & ( y \leq 103 \; \land \;  2x + 1 \leq y) \\
            \lor & y = 103
        \end{array}
    }           & \loc{3'}                          \\
    \st{x>50}   &                         &         \\
                & \itp{y = 103}           & \loc{6} \\
    \st{y!=103} &                         &         \\
                & \itp{\bot}              & \loc{7} \\
\end{tabular}
\end{figure}
\captionof{figure}{Meta trace $\bar{\tau_1}$ generated from $\tau_1$ and $\psi^*_{L_1}$ and its infeasibility proof.}
\label{fig:ex:t0:infproof}
%\dd{Do not capitalize captions. They are just normal sentences.}


\dd{Do you still need the line above?}

We cannot, however, use $I_{\bar{\tau_1}}$ to disprove $\tau_1$ as we need an interpolant for each location in the original trace.
\dd{say which one is missing ;)}
To remedy this, we derive an inductive interpolant sequence $I_{\tau_1}$ by applying the post operator.
\dd{Explain why we need post for the before-location, use the example here}

\newcommand{\accels}[1]{\ensuremath{\psi^{*}_{#1}}}
Given
\begin{itemize}[topsep=0pt,itemsep=-1ex,partopsep=1ex,parsep=1ex]
	\item an error trace $\tau: s_0 s_1 \ldots s_i \ldots s_j \ldots s_n$ where \loc{i} is a loop head for the loop $L$ spanning from $s_i$ to $s_j$,
	\item a loop relation $\psi_L$ for loop $L$,
	\item a corresponding loop acceleration \accels{L},
	\item the meta trace $\bar{\tau}: s_0 s_1 \ldots s_{i-1} \ \accels{L} \ s_{j+1} \ldots s_n$ derived from $\tau$ and \accels{L}, and
	\item the infeasibility proof $I_{\bar{\tau}}: \{\top, I_1, I_2, \ldots , I_i, I_{\psi^*_{L}}, \ldots , I_{n-1}, \bot \}$ for $\bar{\tau}$.
\end{itemize}
\dd{Fix indices s.t. we have the interpolant (sic!) before and after the loop acceleration}

To construct an inductive proof of infeasibility for $\tau$ we need inductive interpolants for the loop statements $s_i, \ldots , s_j$ that were replaced by $\psi^*_{L}$.

Firstly, compute post($I_{\psi^*_L}$, $\psi^*_L$) as the loop entry interpolant $I_{\loc{L}}$.
From there keep applying the post operator with the previous location's interpolant and the following program statement.

We get the inductive interpolant sequence
\begin{equation*}
	I_\tau: \{\top,I_1,I_2, \ldots ,\ \underbrace{post(I_i, \accels{L})}_{I_{i}^*},\ \ \underbrace{post(I_{i}^*, s_i)}_{I_{i+1}^*},\ \ldots ,\ \underbrace{post(I_{j-1}^*, s_j)}_{I_{j}^*},I_{j+1}, \ldots ,I_{n-1}, \bot \}
\end{equation*}
which can now be used by trace abstraction to refine the interpolant automaton.
\ts{Explain why this works and why it is necessary.}

We compute the missing interpolants for example program trace $\tau_1$ as follows:
\dd{Strange wording. This is just the continuaton of the example, right?}
\begin{comment}
\begin{figure}[H]
\centering
\begin{tikzpicture}[%
    ->,
    >=stealth', shorten >=1pt, auto,
    node distance=3.25cm, scale=1,
    transform shape, align=center,
    smallnode/.style={inner sep=2}
    initial text =]

\node[state, label=above:{$\top$}](1){$\ell_1$};
\node[state, label=above:{$x = 0$}] (2) [right of=1] {$\ell_2$};
\node[state, label=above:{$\begin{aligned}
         & y = 2 \cdot x + 1 \\&\land x \leq 51
    \end{aligned}$}] (3) [right of=2] {$\ell_3$};
\node[state, label=above:{$\begin{aligned}
         & y = 2 \cdot x + 1 \\&\land x \leq 50
    \end{aligned}$}] (4) [right of=3] {$\ell_4$};
\node[state, label=above:{$\begin{aligned}
         & y + 1 = 2 \cdot x \\&\land x \leq 51
    \end{aligned}$}] (5) [right of=4] {$\ell_5$};
\node[state, label=above:{$\begin{aligned}
         & y = 2 \cdot x + 1 \\&\land x \leq 51
    \end{aligned}$}] (6) [below of=1] {$\ell_3$};
\node[state, label=below:{$\begin{aligned}
         & y = 2 \cdot x + 1 \\&\land x \leq 50
    \end{aligned}$}] (7) [right of=6] {$\ell_4$};
\node[state, label={[xshift=0.7cm]above:{$\begin{aligned}
         & y + 1 = 2 \cdot x \\&\land x \leq 51
    \end{aligned}$}}] (8) [right of=7] {$\ell_5$};
\node[state, label={[xshift=0cm]above:{$\begin{aligned}
         & y = 2 \cdot x + 1 \\&\land x \leq 51
    \end{aligned}$}}] (9) [right of=8] {$\ell_3$};
\node[state, label={[xshift=0cm]above:{$\begin{aligned}
         & y = 2 \cdot x + 1 \\&\land x \leq 50
    \end{aligned}$}}] (10) [right of=9] {$\ell_4$};
\node[state, label=above:{$\begin{aligned}
         & y + 1 = 2 \cdot x \\&\land x \leq 51
    \end{aligned}$}] (11) [below of=6] {$\ell_5$};
\node[state, label={[xshift=0cm]below:{$\begin{aligned}
         & y = 2 \cdot x + 1 \\&\land x \leq 51
    \end{aligned}$}}] (12) [right of=11] {$\ell_3$};
\node[state, label={[xshift=0cm]below:{$\begin{aligned}
         & y = 2 \cdot x + 1 \\&\land x \leq 51\ \land\ x > 50 \\
         & \equiv y = 103
    \end{aligned}$}}] (13) [right of=12] {$\ell_6$};
\node[state, label=above:{$\bot$}] (14) [right of=13] {$\ell_7$};


\path (1) edge node {\texttt{x := 0}} (2);
\path (2) edge node {\texttt{y := 1}} (3);
\path (3) edge node {\texttt{x <= 50}} (4);
\path (4) edge node {\texttt{x := x + 1}} (5);
\path (5) edge node[left, xshift=-.7cm] {\texttt{y := y + 2}} (6);
\path (6) edge node[below] {\texttt{x <= 50}} (7);
\path (7) edge node[below] {\texttt{x := x + 1}} (8);
\path (8) edge node[below] {\texttt{y := y + 2}} (9);
\path (9) edge node {\texttt{x <= 50}} (10);
\path (10) edge node {\texttt{x := x + 1}} (11);
\path (11) edge node[below] {\texttt{y := y + 2}} (12);
\path (12) edge node[below] {\texttt{!x <= 50}} (13);
\path (13) edge node[below] {\texttt{y != 103}} (14);
;
\end{tikzpicture}
\captionof{figure}{Program trace $\tau_1$ of $P_1$ with inductive infeasibility proof.}
\end{figure}
\end{comment}

\begin{figure}[H]
	\begin{center}
		\begin{tabular}{ccc}
    & $I_{\tau_1}$ & \\
    \hline
                & \itp{\top}                                  & \loc{1} \\
    \st{x:=0}   &                                             &         \\
                & \itp{x=0}                                   & \loc{2} \\
    \st{y:=1}   &                                             &         \\
    \midrule
                & \itp{y = 2x + 1 \land x \leq 51}            & \loc{3} \\
    \st{x<=50}  &                                             &         \\
                & \makecell{\itp{y = 2x + 1 \land x \leq 50}} & \loc{4} \\
    \st{x:=x+1} &                                             &         \\
                & \itp{y + 1 = 2x}                            & \loc{5} \\
    \st{y:=y+2} &                                             &         \\
    \midrule
                & \itp{y = 2x + 1 \land x \leq 51}            & \loc{3} \\
    \st{x<=50}  &                                             &         \\
                & \makecell{\itp{y = 2x + 1 \land x \leq 50}} & \loc{4} \\
    \st{x:=x+1} &                                             &         \\
                & \itp{y + 1 = 2x}                            & \loc{5} \\
    \st{y:=y+2} &                                             &         \\
    \midrule
                & \itp{y = 2x + 1 \land x \leq 51}            & \loc{3} \\
    \st{x<=50}  &                                             &         \\
                & \makecell{\itp{y = 2x + 1 \land x \leq 50}} & \loc{4} \\
    \st{x:=x+1} &                                             &         \\
                & \itp{y + 1 = 2x}                            & \loc{5} \\
    \st{y:=y+2} &                                             &         \\
    \midrule
                & \itp{y = 2x + 1 \land x \leq 51}            & \loc{3} \\
    \st{x>50}   &                                             &         \\
                & \itp{\begin{array}{ll} & y = 2x + 1 \land x \leq 51 \land x > 50 \\[-1mm] \equiv & y = 103 \land x = 51 \end{array}}             & \loc{6} \\
    \st{y!=103} &                                             &         \\
                & \itp{\bot}                                  & \loc{7}
\end{tabular}
	\end{center}
\end{figure}
\captionof{figure}{Program trace $\tau_1$ of $P_1$ with inductive infeasibility proof.}
\label{fig:ex:t0:infproof}

% \ts{The simplification of the interpolant at $\loc{6}$ is wrong.}
