
In this thesis we have shown a technique to compute a loop summary using \qvasr, integrated that technique into trace abstraction, by using it in the accelerated interpolation scheme and as a method to transform a control-flow graph by replacing loops with summaries. There are, however, still some areas that can be improved upon. In this section we introduce possible improvements to \qvasr loop summarization and its usage. \\ \par
\subsection{Using \qvasrs}
In chapter \ref{qvasrs} we have shown that there is an extension to \qvasr that factor in assumptions in a loop's transition formula, such as the expressions of the loop guard and if else statements. It is possible to integrate \qvasrs into our implemented summary utilization schemes, accelerated interpolation and control-flow graph transformation, as follows: \\ \par
For accelerated interpolation, assume we are given a trace $\tau$ with a loop, whose transition formula we extract as $\tau_L$, we compute a \qvasrs $\mathcal{V} = (P, E)$ for $\tau_L$, now because a \qvasrs is a graph structure, we need to compute its reachability relation to summarize the loop. To solve this task, we can utilize Haase and Halfon \cite{DBLP:conf/rp/HaaseH14} proposed polytime procedure that computes a series of formulas from generalized Parikh images that, as conjunction, form a summary of the \qvasrs. \\ \par
For \qvasrs usage in control-flow graph transformation, we adapt our approach by, instead of replacing the whole loop by its computed summary, inserting the \qvasrs instead. We illustrate this approach by transforming the given example program $P$ as seen in Figure \ref{codeWithAss} with its control-flow graph $G_P$ as depicted in Figure \ref{cfg:P:Ass}, to the control-flow graph, seen in Figure \ref{qvasrs_cfg}
\begin{figure}[H]
	
\begin{tikzpicture}[%
	->,
	>=stealth',
	shorten >=1pt,
	auto,
	node distance=3.25cm,
	scale=0.9,
	transform shape,
	align=center,
	smallnode/.style={inner sep=1.4},
	initial text =,
	anchor=center]
	
	\node[state, initial above, initial text =](1){$\loc{1}$};
   	\node[draw, ellipse](vasrs1) [below of=1] {$x \leq 10$};
	\node[draw, ellipse](vasrs2) [left of=vasrs1, xshift=-4cm] {$10 < x \land x \leq 20$};
	\node[state] (loopExit)[right of=head,  xshift=2cm] {$\loc{13}$};
	\node[state] (assertTrue)[below left of=loopExit] {$\loc{14}$};
	\node[state, accepting] (assertFalse)[below of=loopExit] {$\loc{err}$};
	
	\path (1) edge node []{\st{x:=0;}\\ \st{y:=2;}\\ \st{z:=3;}} (head)
	(vasrs1) edge node []{\st{x>20}} (loopExit)
	(vasrs2) edge[bend left] node {\st{-x+y+z:=1;} \\ \st{y:=y+1}} (vasrs1)
	(vasrs1) edge[bend left] node[] {\st{-x+y+z:=-x+y+z-5;}\\\st{y:=y-3}} (vasrs2)
	(loopExit) edge node []{\st{x==22}} (assertTrue)
	(loopExit) edge node []{\st{x!=22}} (assertFalse)
	
	(vasrs1) edge[loop below] node[] {\st{-x+y+z:=-x+y+z-5;}\\ \st{y:=y-3}}(vasrs1)
	(vasrs2) edge[loop above] node[] {\st{-x+y+z:=1;}\\ \st{y:=y+1}}(vasrs2)
	;
	
\end{tikzpicture}
	\label{qvasrs_cfg}
	\caption{Control-flow graph $\bar{G}_P$ created from \ref{cfg:P:Ass} by replacing the loop with a \qvasrs.}
\end{figure}
From this transformed control-flow graph we can now construct state assertions that adhere to assumptions in the program.
\subsection{Heuristics in Accelerated Interpolation}

\begin{comment}
	To compute a loop summary from a \qvasrs, one has to calculate the reachability relation of the automaton. Haase and Halfon \cite{DBLP:conf/rp/HaaseH14} proposed a polytime procedure that computes a series of formulas from computed Parikh images that, as conjunction, form a summary of the system. This procedure can be adapted to work with \qvasrs
\end{comment}
