
Next to rational vector addition systems with resets, there are a multitude of other loop summarization techniques:

\begin{itemize}
	\item \textsl{Fast Acceleration using Ultimately Periodic Relations}\cite{10.1007/978-3-642-14295-6_23}\\ Bozga et al. propose an overapproximative loop summarization scheme that tries to compute an ultimately periodic relation from a given loop's transition relation. A transition relation is ultimately periodic if and only if there exists a bound from which on previous transitions keep getting repeated in the same manner. There are three classes of relations that are proven to be ultimately periodic: Difference bounds, octagonal and finite monoid affine relations. The paper introduces techniques of verifying whether a given loop relation is ultimately periodic and assigning it to one of the classes. The drawback, however, is that this loop summarization scheme can only be applied to loop relations belonging to one of these three classes. Other, non periodic loops, for instance, a loop with a non deterministic branching, created for example by \texttt{if (*)}, cannot be summarized, whereas \qvasr can deal with that.
	
	\item \textsl{Abstracting Path Conditions}\cite{DBLP:conf/issta/StrejcekT12} \\
	In their work, Strejček and Trtík, propose a loop summarization technique based on symbolic execution. Their scheme extracts non-cyclic paths through a loop, called backbones, simulate one iteration of each, and then abstract a formula representing an arbitrary number of iterations, called an iterated memory. From this iterated memory a condition for each backbone is derived that serves to overapproximate the number of times a backbone can be iterated upon. 
	
	\item \textsl{Under-approximating loops}\cite{DBLP:conf/cav/KroeningLW13}
	
	\item \textsl{Abstract Acceleration of General Linear Loops}\cite{DBLP:conf/popl/JeannetSS14}
\end{itemize}