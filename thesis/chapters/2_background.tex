
\jw{This chapter is mostly focused on trace abstraction $\rightarrow$  It introduces the reader to the concept of trace abstraction. \\
	- Introduce logic, logical variables, terms, formulas, transition formulas with primed and unprimed variables, programs, program states, loops $\rightarrow$  then program-, error traces, feasible and infeasible counterexamples, CFGs, interpolants. \\ - From intuitive to true definitions. \\
	Here the running example from the introduction gets dissected to illustrate the definitions. \\ 
	Further the problems loops can cause are introduced, followed by a definition of loop summaries $\rightarrow$ introduction reflexive transitive closure of a formula \\
	\vspace{1cm}
	15 pages}