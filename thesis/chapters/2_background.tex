
\begin{comment}
	This chapter is mostly focused on trace abstraction $\rightarrow$  It introduces the reader to the concept of trace abstraction. \\
	- Introduce logic, logical variables, terms, formulas, transition formulas with primed and unprimed variables, programs, program states, loops $\rightarrow$  then program-, error traces, feasible and infeasible counterexamples, CFGs, interpolants. \\ - From intuitive to true definitions. \\
	Here the running example from the introduction gets dissected to illustrate the definitions. \\ 
	Further the problems loops can cause are introduced, followed by a definition of loop summaries $\rightarrow$ introduction reflexive transitive closure of a formula 
	15 pages
\end{comment}

This chapter shall introduce our understanding and notation of logic and formulas, programs, control-flow, and other needed background definitions. Furthermore, we will give an overview of the \traceabstraction \cite{10.1007/978-3-642-03237-0_7} counterexample-guided abstraction refinement scheme used in \ultimate.

\subsection{Logical Background}
To represent programs formally, we make use of first-order logic. This chapter will introduce our definitions and notations used throughout this thesis.

\subsubsection{Notation}
We utilize the standard logical notations. We represent boolean values: $\bot$ meaning \textsl{false} and $\top$ meaning \textsl{true}. \\ Logical connectives are defined as:
\begin{itemize}
	\item Negation (\textsl{not}) denoted by: $\neg$
	\item Conjunction (\textsl{and}) denoted by: $\land$
	\item Disjunction (\textsl{or}) denoted by: $\lor$
	\item Implication (\textsl{if $\ldots$ then}) denoted by: $\rightarrow$
	\item Biconditional (\textsl{if and only if}) denoted by: $\leftrightarrow$
\end{itemize}
Formulas can be quantified, we define quantifiers as:
\begin{itemize}
	\item Existential quantification (\textsl{there exists}) denoted by: $\exists$
	\item Universal quantification (\textsl{for all}) denoted by: $\forall$
\end{itemize}

\subsubsection{Syntax}
We firstly introduce our first-order logic syntax.
Let $V = (\vocab{Var}, \vocab{Const}, \vocab{Fun}, \vocab{pred})$ be a vocabulary consisting of the countable sets:
\begin{itemize}
	\item $\vocab{Var}$ containing all so called \textsl{variables}
	\item $\vocab{Const}$ containing all so called \textsl{constant symbols}
	\item $\vocab{Fun} $ containing all so called \textsl{function symbols}. Each symbol $f \in \vocab{Fun}$ has a natural number $\geq 1$ called arity of $f$
 	\item $\vocab{Pred}$ containing all so called \textsl{predicate symbols}. Each symbol $p \in \vocab{Pred}$ has a natural number $\geq 0$ called arity of $p$
\end{itemize}
Assume we are given such a vocabulary, we can construct first-order logic terms using the symbols in the vocabulary.
\begin{mydef}[Term] 
	Given a vocabulary $V = (\vocab{Var}, \vocab{Const}, \vocab{Fun}, \vocab{pred})$, we define terms inductively as follows:
	\begin{itemize}
		\item Every $x \in \vocab{Var}$ is a term.
		\item Every $c \in \vocab{Const}$ is a term.
		\item If $t_0, \ldots, t_n$ are terms and $f \in \vocab{Fun}$ being a function symbol with arity $n$, then $f(t_0, \ldots, t_n)$ is a term.
	\end{itemize}
\end{mydef}
Using first-order logic terms, we can introduce first-order logic formulas.
\begin{mydef}[Formula]
	Given vocabulary $V = (\vocab{Var}, \vocab{Const}, \vocab{Fun}, \vocab{pred})$, first-order logic formulas are inductively defined as follows:
	\begin{itemize}
		\item $\bot$ is a formula.
		\item If  $t_0, \ldots, t_n$ are terms, and $p \in \vocab{pred}$ is a predicate symbol with arity $n$, \\ then $p(t_0, \ldots, t_n)$ is a formula.
		\item If $\varphi$ is a formula, then $\neg \varphi$ is a formula.
		\item If $\varphi$ and $\psi$ are formulas, then $\varphi \land \psi$ are formulas.
		\item If $\varphi$ is a formula, and $x \in \vocab{Var}$ then $\exists x. \varphi$ is a formula.
	\end{itemize}
\end{mydef}
To give variables, constants, functions, and predicates concrete values, we can assign them a model.
\begin{mydef}[Model]
	Given vocabulary $V = (\vocab{Var}, \vocab{Const}, \vocab{Fun}, \vocab{pred})$, a model $\mathcal{M} = (D, \interpret)$ is a tuple consisting of a nonempty set $D$, called interpretation domain, and an interpretation function \interpret that assigns constants, functions, and predicates over $D$ to symbols in $V$. $M$ has the following characteristics:
	\begin{itemize}
		\item The domain of \interpret is $\vocab{Const} \cup \vocab{Fun} \cup \vocab{Pred}$
		\item \interpret maps every constant symbol $c \in \vocab{Const}$ to an element in $D$
		\item \interpret maps every function symbol $f \in \vocab{Fun}$, with arity $n$, to a corresponding n-ary function with domain $D^n$ and range $D$
		\item \interpret maps every predicate symbol $p \in \vocab{Pred}$ with arity $n$ to an n-ary relation over the domain $D$
	\end{itemize}
\end{mydef}
Using a model, we can now assign concrete values to variables.
\begin{mydef}[Assignment of Variables]
	Given vocabulary $V = (\vocab{Var}, \vocab{Const}, \vocab{Fun}, \vocab{pred})$, and model $\mathcal{M} = (D, \interpret)$, an assignment of variable $v \in \vocab{Var}$ is a function $\rho: v \rightarrow D$. Mapping each variable a value in domain $D$.
\end{mydef}
Assume $f \in \vocab{Fun}$ is a function defined as $f: X \rightarrow Y$ with some domain $X$ and range $Y$. Let $x \in X$ and $y \in Y$, we use $f[x \rightarrow y]$  to denote the function that maps all $\bar{x} \in X \backslash \{ x \}$ to $f(\bar{x})$ and $x$ to $y$.

\subsubsection{Semantics}
Assume we are given a vocabulary $V = (\vocab{Var}, \vocab{Const}, \vocab{Fun}, \vocab{pred})$, we know how to assign values using models and variable assignments. The task now is to understand how to interpret them. This section serves to introduce semantics of fist-order logic.
\begin{mydef}[Evaluation of Terms]
	Let $V = (\vocab{Var}, \vocab{Const}, \vocab{Fun}, \vocab{pred})$ be a vocabulary, $\mathcal{M} = (D, \interpret)$ a model, and $\rho$ a variable assignment. The evaluation of terms is a function $\eval{\cdot}$ that is inductively defined as:
	\begin{itemize}
		\item For each $v \in \vocab{Var}$, $\eval{v} = \rho(v)$
		\item For each $c \in \vocab{Const}$, $\eval{c} = \interpret(c)$
		\item If $t_0, \ldots, t_n$ are terms, $f \in \vocab{Fun}$, with f having arity $n$ then \\ $\eval{f(t_0, \cdots, t_n)}$ is $\interpret(f)(\eval{t_0}, \ldots, \eval{t_n})$
	\end{itemize}
\end{mydef}
From the evaluation of terms we can derive the evaluation of formulas, which decides whether a formula is \textsl{true} or \textsl{false}.

\begin{mydef}[Evaluation of Formulas]
		Let $V = (\vocab{Var}, \vocab{Const}, \vocab{Fun}, \vocab{pred})$ be a vocabulary, $\mathcal{M} = (D, \interpret)$ a model, a variable assignment $\rho$, and $\varphi_0, \varphi_1$ being first-order logic formulas. The evaluation of formulas is a function $\eval{\cdot}$ that is inductively defined as: \\
		\begin{itemize}
			\item {\makebox[3cm]{$\eval{\bot} \hfill$}} \textbf{false}
			\item {\makebox[3cm]{$ \eval{p(t_0, \ldots, t_n)} \hfill$}} 
				$
				\begin{cases}
					\textbf{true}, & \text{for } (\eval{t_0}, \ldots, \eval{t_n}) \in \interpret(p)\\
					\textbf{false}, & \text{otherwise}
				\end{cases}
				$
			\item {\makebox[3cm]{$\eval{\neg \varphi_0} \hfill$}}
				$
				\begin{cases}
					\textbf{true}, & \text{for } \eval{\varphi_0} \text{ \textbf{false}}\\
					\textbf{false}, & \text{for } \eval{\varphi_0} \text{ \textbf{true}}\\
				\end{cases}
				$
			\item {\makebox[3cm]{$\eval{\varphi_0 \land \varphi_1} \hfill$}}
				$
				\begin{cases}
					\textbf{true}, & \text{for } \eval{\varphi_0} \text{ \textbf{true} and } \eval{\varphi_1} \text{ \textbf{true}} \\
					\textbf{false}, & \text{otherwise} \\
				\end{cases}
				$
			\item {\makebox[3cm]{$\eval{\exists v. \varphi_0} \hfill$}}
				$
				\begin{cases}
					\textbf{true}, & \text{if there exists } x \in D \text{ where } \eval{\varphi_0 [v \rightarrow x]} \text{ \textbf{true}} \\
					\textbf{false}, & \text{otherwise} \\
				\end{cases}
				$
		\end{itemize}
\end{mydef}

\subsection{Programs}
Assume we are given a program as seen in \ref{code}. We consider each line of code a so called program statement. These statements use the following context-free grammar that is a derived and simplified version of the grammar of the intermediate verification language Boogie\cite{Boogie}.
\setlength{\grammarparsep}{20pt plus 1pt minus 1pt} % increase separation between rules
\setlength{\grammarindent}{12em} % increase separation between LHS/RHS 
\begin{grammar}
	<Stmt> ::= `assume' <Expr> ;
	\alt $Var_{id}$ `:=' <Expr> ;
	\alt `havoc' $Var_{id}$ ;
	\alt `assert' <Expr> ;
	\alt `while' ( <WildcardExpr> ) <Stmt>* ;
	\alt <IfStmt>;
	
	<IfStmt> ::= `if' ( <WildcardExpr> ) <Stmt>* `else' <Stmt>*

	<Expr> ::= <Expr> <BinOp> <Expr>
	\alt <UnOp> <Expr>
	\alt `True'
	\alt `False'
		
	<WildCardExpr> ::= <Expr> | `*'
	
	<BinOp> ::= `+' | `-' | `*' | `/' | `\%'
	\alt `\&\&' | `||' | `==' | `!='
	\alt `<' | `<=' | `>' | `>='
	
	<UnOp> ::= `-' | `!'
\end{grammar}
Where $Var_{id}$ represents any variable declared in the program and $*$ corresponds to a nondetermnistic expression.

\begin{mydef}[Control-Flow Graph]
	Given a finite set of program statements \stmt. A control-flow graph is a labelled graph $G_P = (\Loc, \delta, \loc{\init}, \loc{\err})$, with
	\Loc being a finite set of locations,
	a set of edges between two locations labelled with a statement $\delta \subseteq \Loc \times \stmt \times \Loc$,
	an initial location $\loc{init} \in \Loc$, and
	an error location $\loc{err} \in \Loc$.
\end{mydef}
In this paper we will use control-flow graphs to represent programs.
% \dd{It is useful to wrap all the function symbols in Latex macros}
% \ts{And to use mathit for identifiers consisting of several letters (like $\mathit{Var}$)}
Program variables are typed, e.g. integer or boolean variables, there exists \\
an interpretation domain $D$ defining the set of all possible variable values.
\jw{TODO}
Assigning every program variable a valuation creates a program state.

%the n may be wrong, because traces can be infinite in theory? no, because a program can only have finitely many variable declarations. The source code is finite.

\begin{mydef}[Variable Valuation]
	Assume a program $P$ is defined over $n$ variables, a program state $\sigma$ is a function assigning each variable $v_i \in \Var$, \ $0 \leq i \leq n$ a variable valuation $\rho_i$. The set $S$ denotes the set of all program states.
\end{mydef}

\begin{center}
	\begin{minipage}[b]{0.4\linewidth}
			\begin{figure}[H]
			\centering
			\begin{lstlisting}[language=C++, style=withAssert]  % Start your code-block
	
	int x := 0;
	int y := 2;
	int z := 3;
	while x <= 20:
		if x <= 10:
			z := x;
			x := x + y;
			y := y + 1;
		else:
			x := x + 2;
			y := y - 3;
	assert x == 21;
	\end{lstlisting}
			\caption{Program $P$ \\ with \texttt{while} loop.}
			\label{code}
		\end{figure}
	\end{minipage}
	\hfill
	\begin{minipage}[b]{0.59\linewidth}
		\begin{figure}[H]
			\centering
			
\begin{tikzpicture}[%
    ->,
    >=stealth',
    shorten >=1pt,
    auto,
    node distance=3.25cm,
    scale=0.9,
    transform shape,
    align=center,
    smallnode/.style={inner sep=1.4},
    initial text =,
    anchor=center]

			\node[state, initial above, initial text =](1){$\loc{1}$};
			\node[state] (head) [below of=1] {$\loc{4}$};
			\node[state] (loopEntry)[left of=head] {$\loc{5}$};
			\node[state] (if)[above of=loopEntry] {$\loc{6}$};
			
			\node[state] (else)[below of=loopEntry] {$\loc{10}$};
			\node[state] (loopExit)[right of=head] {$\loc{13}$};
			
			\node[state] (assertTrue)[below left of=loopExit] {$\loc{14}$};
			\node[state, accepting] (assertFalse)[below of=loopExit] {$\loc{err}$};
			
			\path (1) edge node []{\st{x:=0;}\\ \st{y:=2;}\\ \st{z:=3;}} (head)
			(head) edge node []{\st{x<=20}} (loopEntry)
			(head) edge node []{\st{x>20}} (loopExit)
			
			(loopEntry) edge[] node []{\st{x<=10}} (if)
			(if) edge[]node []{\\ \st{z:=x;}\\ \st{x:=x+y;}\\ \st{y:=y+1}} (head)
			
			(loopEntry) edge[]node []{\st{x>10}} (else)
			(else) edge[] node []{\st{x:=x+2;}\\ \st{y:=y-3}} (head)
			
			(loopExit) edge node []{\st{x==22}} (assertTrue)
			(loopExit) edge node []{\st{x!=22}} (assertFalse)
			;
\end{tikzpicture}
			\caption{Control-flow graph for program $P$.}
			\label{code}
		\end{figure}
	\end{minipage}
\end{center}


\subsection{Model Checking}


\subsection{Trace Abstraction}