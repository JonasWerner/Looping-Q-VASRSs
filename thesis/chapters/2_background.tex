
\begin{comment}
	This chapter is mostly focused on trace abstraction $\rightarrow$  It introduces the reader to the concept of trace abstraction. \\
	- Introduce logic, logical variables, terms, formulas, transition formulas with primed and unprimed variables, programs, program states, loops $\rightarrow$  then program-, error traces, feasible and infeasible counterexamples, CFGs, interpolants. \\ - From intuitive to true definitions. \\
	Here the running example from the introduction gets dissected to illustrate the definitions. \\ 
	Further the problems loops can cause are introduced, followed by a definition of loop summaries $\rightarrow$ introduction reflexive transitive closure of a formula 
	15 pages
\end{comment}

This chapter shall introduce our understanding and notation of logic and formulas, programs, control-flow, and other needed background definitions. Furthermore, we will give an overview of the \traceabstraction \cite{10.1007/978-3-642-03237-0_7} counterexample-guided abstraction refinement scheme used in \ultimate.

\subsection{Logical Fundamentals}
We model programs using first order logic, this chapter will introduce the basic notions used in this paper.
\begin{mydef}[Term] 
	Given a vocabulary $V = (\vocab{Var}, \vocab{Const}, \vocab{Fun}, \vocab{pred})$, with countable sets \\ $\vocab{\Var}, \vocab{Const}, \vocab{Fun}, \vocab{pred}$ representing the sets of variables, constant symbols, function symbols, and predicate symbols respectively, we define terms inductively as follows:
	\begin{itemize}
		\item Every $x \in \vocab{Var}$ is a term.
		\item Every $c \in \vocab{Const}$ is a term.
		\item If $t_0, \ldots, t_n$ are terms and $f \in \vocab{Fun}$ being a function symbol with arity $n$, then $f(t_0, \ldots, t_n)$ is a term.
	\end{itemize}
\end{mydef}
Using the definitions of terms, we can introduce first-order logic formulas.

\begin{mydef}[Formula]
	Given vocabulary $V = (\vocab{Var}, \vocab{Const}, \vocab{Fun}, \vocab{pred})$, first-order logic formulas are inductively defined as follows:
	\begin{itemize}
		\item $\bot$ is a formula.
		\item If  $t_0, \ldots, t_n$ are terms, and $p \in \vocab{pred}$ is a predicate symbol with arity $n$, then $p(t_0, \ldots, t_n)$ is a formula.
		\item If $\varphi$ is a formula, then $\neg \varphi$ is a formula.
		\item If $\varphi$ and $\psi$ are formulas, then $\varphi \land \psi$ are formulas.
		\item If $\varphi$ is a formula, and $x \ in \vocab{Var}$ then $\exists x. \varphi$ is a formula.
	\end{itemize}
\end{mydef}

\begin{mydef}[Model]
	Given vocabulary $V = (\vocab{Var}, \vocab{Const}, \vocab{Fun}, \vocab{pred})$, a model $\mathcal{M} = (D, \interpret)$ is a tuple consisting of a nonempty set $D$, called interpretation domain, and an interpretation function \interpret that assigns constants, functions, and predicates over $D$ to symbols in $V$.
\end{mydef}

\begin{mydef}[Assignment of Variables]
	Given vocabulary $V = (\vocab{Var}, \vocab{Const}, \vocab{Fun}, \vocab{pred})$, and domain $D$, an assignment of variable $v \in \vocab{Var}$ is a function $\varv: v \rightarrow D$.
\end{mydef}

\begin{mydef}[Evaluation of Terms]
	Let $V = (\vocab{Var}, \vocab{Const}, \vocab{Fun}, \vocab{pred})$ be a vocabulary, $\mathcal{M} = (D, \interpret)$ a model, and $\varv$ a variable assignment, the evaluation of terms is a function $\eval{\cdot}$ that is inductively defined as:
	\begin{itemize}
		\item For each $x \in \vocab{Var}$, $\eval{x} = \varv(x)$
		\item For each $c \in \vocab{Const}$, $\eval{c} = \interpret(c)$
		\item If $t_0, \ldots, t_n$ are terms, $f \in \vocab{Fun}$, and f has arity $n$ then \\ $\eval{f(t_0, \cdots, t_n)}$ is $\interpret(f)(\eval{t_0}, \ldots, \eval{t_n})$
	\end{itemize}
\end{mydef}

\subsection{Model Checking}

\subsection{Programs}
\begin{figure}[H]
	\centering
	\begin{lstlisting}[language=C++,basicstyle=\ttfamily,keywordstyle=\color{blue}]  % Start your code-block
	
	int x := 0;
	int y := 2;
	int z := 3;
	while x <= 21:
		if x <= 10:
			z := x;
			x := x + y;
			y := y + 1;
		else:
			x := x + 2;
			y := y - 3;
	assert x == 22;
	\end{lstlisting}
	\caption{Program $P$ \\ with \texttt{while} loop.}
	\label{code}
\end{figure}

\begin{figure}[H]
	\centering
	
		\begin{tikzpicture}[%
    ->,
    >=stealth',
    shorten >=1pt,
    auto,
    node distance=3cm,
    scale=1,
    transform shape,
    align=center,
    smallnode/.style={inner sep=1.4},
    initial text =,
    anchor=center]

			\node[state, initial above, initial text =](1){$\loc{1}$};
			\node[state] (head) [below of=1] {$\loc{4}$};
			\node[state] (loopEntry)[left of=head] {$\loc{5}$};
			\node[state] (if)[above left of=loopEntry] {$\loc{6}$};
			
			\node[state] (else)[below left of=loopEntry] {$\loc{10}$};
			\node[state] (loopExit)[right of=head] {$\loc{13}$};
			
			\node[state] (assertTrue)[below left of=loopExit] {$\loc{14}$};
			\node[state, accepting] (assertFalse)[below right of=loopExit] {$\loc{14}$};

			\path (1) edge node []{\st{x:=2}\\ \st{y:=2}\\ \st{z:=3}} (head)
			;
		\end{tikzpicture}
	\caption{Program $P$ \\ with \texttt{while} loop.}
	\label{code}
\end{figure}




\subsection{Trace Abstraction}