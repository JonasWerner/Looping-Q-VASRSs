
\begin{comment}
	\jw{Introduce loop summarization using rational vector addition systems with resets \\
	- What are qvasr? What are qvasr abstractions? what are least upper bounds on abstractions? \\
	- start with qvasr on example that does not need abstraction $\rightarrow$ no relations between variables \\
	- move on to example with relations between variables; show simulation matrix and imaging \\
	- abstractions are an overapproximation \\
	- imprecise because of ignorance of assumptions $\rightarrow$ transfer to next chapter \\
	- usage of running example which is turned to qvasr abstraction \\
	\vspace{1cm}
	20 pages}
\end{comment}

\section{\qvasr}
A \qvasr of dimension $d$ is a finite set $V \subseteq \{0, 1\}^d \times \mathbb{Q}^d$ of transformers. Each transformer $(\vec{r}, \vec{a}) \in V$ consists of binary \textsl{reset} vector $\vec{r}$, where 0 indicates a reset, and rational \textsl{addition} vector $\vec{a}$. A \qvasr $V$ defines a transition system $(S_V, \rightarrow_V)$, with state space $S_V \subseteq \Q^d$, which can transition between states $\vec{x} \rightarrow_V \vec{x}'$, if $\vec{x}' = \vec{r} * \vec{x} + \vec{a}$ for some transformer $(\vec{r}, \vec{a}) \in V$, with $*$ being the Hadamard product. \par A \qvasr can be used to represent an overapproximation of a given transition formula. A transition formula $F$ is a first-order formula defined over free variables $\vec{x} = x_1, \ldots, x_n$ and $\vec{x}' = x_1', \ldots, x_n'$ that designate the state before and after a transition, where $n$ is called the dimension of the formula. Figure \ref{code} depicts a program containing a \texttt{while} loop from lines 4 - 12 for which we want to compute a loop summary using \qvasr.
\begin{wrapfigure}{L}{0.4\linewidth}
	\vspace{-10pt}
	\centering
	\begin{lstlisting}[language=C++, style=withAssert]  % Start your code-block
	
	int x := 0;
	int y := 2;
	int z := 3;
	while x <= 20:
		if x <= 10:
			z := x;
			x := x + y;
			y := y + 1;
		else:
			x := x + 2;
			y := y - 3;
	assert x == 21;
	\end{lstlisting}
	\caption{Program $P$ \\ with \texttt{while} loop.}
	\label{code}
\end{wrapfigure}
To accomplish this we need to compute a \qvasr for every path through the loop. In this example there are two paths created by the \texttt{if else} statement.
Beginning with the \texttt{else} branch, we extract the transition formula:
\begin{equation*}
	G= \ (x \leq 20 \land\ x > 10 \land x' = x + 2 \land y' = y - 3)
\end{equation*}
The variable $x$ is not reset but incremented by 2, variable $y$ is not reset and decremented by 3.
We get reset vector $
\vec{r}_G = \  
\begin{bmatrix}
	1 \\
	1 
\end{bmatrix}
$
and addition vector $
\vec{a}_G = \ 
\begin{bmatrix}
	2 \\
	-3 
\end{bmatrix}$ \\
$G$ can therefore be modeled by the \qvasr
$V_G = 
\begin{Bmatrix}
	\begin{pmatrix}
		\begin{bmatrix}
			1 \\
			1
		\end{bmatrix},
		\begin{bmatrix}
			2 \\
			-3
		\end{bmatrix}
	\end{pmatrix}
\end{Bmatrix}
$ \par \vspace{2pt}
For the remainder of this proposal we will use the following, more intuitive notation of \qvasr:
\begin{equation*}
	V_G = 
	\begin{Bmatrix}
		\begin{bmatrix}
			x \\
			y
		\end{bmatrix} \rightarrow_{V_G}
		\begin{bmatrix}
			x + 2 \\
			y - 3
		\end{bmatrix}
	\end{Bmatrix}
\end{equation*}

From the \texttt{if} branch we extract the following transition formula:
\begin{equation*}
	H= \ (x \leq 10 \land x' = x + y\ \land\ y' = y + 1 \land z' = x)
\end{equation*} In contrast to $G$ there are no constant increments to variables $x$ and $z$. They are changed by an arbitrary amount. A \qvasr cannot represent $H$ by a  \qvasr because they are only able to model transitions with constant resets and increments. 

\subsection{\qvasr-Abstraction}

We can, however, represent an overapproximation of $H$ as a \qvasr. This is done using linear simulations. Given a transition formula $F$ of dimension $n$ and a \qvasr $V$ of dimension $m$, a linear simulation from $F$ to $V$ is a linear transformation matrix: 
$S = 
\begin{bmatrix}
	s_{1 ,1} & \ldots & s_{1, n} \\
	\vdots & \ddots & \vdots \\
	s_{m ,1} & \ldots & s_{m, n} \\
\end{bmatrix}$ 
such that for all transitions $\vec{x} \rightarrow_F \vec{x}'$ we have $S\cdot\vec{x} \rightarrow_V S\cdot\vec{x}'$. Meaning, every transition $\vec{x} \rightarrow_F \vec{x}'$ can be represented in $V$ by a matrix multiplication of $S$ and $\vec{x}$ and $\vec{x}'$. The tuple $(S, V)$ is called a \qvasr-abstraction. \par For $H$ we need to calculate both the linear transformation matrix $S_H$ and the \qvasr $V_H$. We know that a \qvasr consists of pairs of reset and addition vectors, we use the formula $S \cdot \vec{u} \rightarrow_V S \cdot \vec{u}$ to get a transition in $V_H$ as $S\cdot\vec{x}' = \vec{r}*S\cdot\vec{x} + \vec{a}$. \\ To get variables that are reset, we consider $\vec{r}$ as the zero vector $\vec{0}$, which forms a linear set of equations that model a set of resets:
\begin{equation*}
	Res_H = \left\{ (\s, a) : H \models \s \cdot \p = a \right\}	
\end{equation*}
For the set of additions, we consider $\vec{r}$ as the constant one vector $\vec{1}$ which forms the set of additions as:
\begin{equation*}
	Inc_H = \left\{(\s, a) : H \models \s \cdot \p = \s \cdot \up + a\right\}	
\end{equation*}
We use the updates to variables: $x' = x + y, \ y'= y + 1\ z' = x$ , found in $H$, to solve the equations in the sets for \s, resulting in: 
\vspace*{-0.5em}
\begin{center}
	\begin{minipage}{0.5\linewidth}
		\begin{equation*}
			Res_H = \left\{ (\begin{bmatrix} -a & a & a \end{bmatrix}, a) \right\}\
		\end{equation*}
	\end{minipage}
	\begin{minipage}{0.4\linewidth}
		\begin{equation*}
			Inc_H = \left\{ (\begin{bmatrix} 0 & a & 0 \end{bmatrix}, a) \right\}\ 
		\end{equation*}
	\end{minipage}
\end{center}
Observe that these form a vector space. Because we want to construct a linear transformation, we need to compute a basis for each space:
\vspace*{-1em}
\begin{center}
	\begin{minipage}{0.5\linewidth}
		\begin{equation*}
			Res_H = \{(
			\NiceMatrixOptions{code-for-first-row=\scriptstyle}
			\begin{bNiceMatrix}[first-row=1]
				x & y & z \\
				-1 & 1 & 1 
			\end{bNiceMatrix}, 1)\}
		\end{equation*}
	\end{minipage}
	\begin{minipage}{0.4\linewidth}
		\begin{equation*}
			Inc_H = \{(
			\NiceMatrixOptions{code-for-first-row=\scriptstyle}
			\begin{bNiceMatrix}[first-row=1] 
				x & y & z \\
				0 & 1 & 0 
			\end{bNiceMatrix}, 1)\}
		\end{equation*}
	\end{minipage}
\end{center}
Each row of the basis corresponds to a variable in the formula such that we can derive relations between variables. From the reset base we can deduce that the sum $-x + y + z$ is reset and incremented by $1$, meaning that after each transition of $H$ we know that $-x + y + z = 1$ holds. From the basis of additions we derive that $y$ is not reset and incremented by $1$. To form the linear transformation matrix $S_H$ we combine these rows to one coherent matrix. We see that the basis of resets and additions contain only one vector each, resulting in only one reset addition pair for $V_H$. As we have seen $-x + y + z$ is reset and $y$ is not. We get reset vector $\vec{r} = \begin{bmatrix} 0 \\ 1 \end{bmatrix}$ and because $-x + y + z$ is incremented by $1$, same as $y$, we get addition vector $\vec{a} = \begin{bmatrix} 1 \\ 1 \end{bmatrix}$. Resulting in the \qvasr-abstraction depicted in figure \ref{vasr  H}.
\vspace*{-2em}
\begin{figure}[H]
	\begin{center}
		\begin{minipage}{0.3\linewidth}
			\begin{equation*}
				S_H = \begin{bmatrix} -1 & 1 & 1 \\ 0 & 1 & 0 \end{bmatrix}
			\end{equation*}
		\end{minipage}
		\begin{minipage}{0.6\linewidth}
			\begin{equation*}
				V_H = \begin{Bmatrix} \begin{bmatrix} - x + y + z \\ y \end{bmatrix} \rightarrow_{V_H} \begin{bmatrix}	1 \\ y + 1 \end{bmatrix} \end{Bmatrix}
			\end{equation*}
		\end{minipage}
		\caption{\qvasr-abstraction of transition formula $H$.}
		\label{vasr H}
	\end{center}
\end{figure}
\vspace*{-2em}For transition formulas with constant increments, such as $G$, an identity matrix $I$ is used as simulation matrix. \par
The effect on variables by the loop in $P$ can be represented by the following conjunction: 
\begin{equation*}
	x \leq 20 \land (H \lor G)
\end{equation*}
With $x \leq 20$ being the loop guard. We have already computed \qvasr-abstractions for $G$ and $H$. These, however, only model the effect on variables in their respective branch of the \texttt{if else} statement. To get the most precise overapproximation of the whole loop's behavior we need the \textsl{best} \qvasr-abstraction $(\tilde{S}, \tilde{V})$ that simulates every branch in the loop.
We impose a partial order $\preceq$ on \qvasr-abstractions $(S_1, V_1)$ and $(S_2, V_2)$, with $(S_1, V_1) \preceq (S_2, V_2)$ if $(S_2, V_2)$ simulates $(S_1, V_1)$. The best abstraction $(\tilde{S}, \tilde{V})$ is the least upper bound with regard to $\preceq$, meaning $(S, V) \preceq (\tilde{S}, \tilde{V})$ for all \qvasr-abstractions $(S, V)$. The abstraction $(\tilde{S}, \tilde{V})$ computed by iteratively \textsl{joining} abstractions. Joining two \qvasr-abstractions results in a single abstraction simulating both. Figure \ref{vasr} shows the best \qvasr-abstraction of the loop in $P$, which is the result of joining $G$'s and $H$'s abstractions. \\
\begin{center}
	\begin{figure}[H]
		\begin{align*}
    S_P &= \begin{pmatrix}
        -1 & 1 & 1 \\
        0 & 1 & 0
    \end{pmatrix}, \ \\ \\
    V_P &= \begin{Bmatrix}
        \begin{pmatrix}
              \begin{pmatrix}
                    0 \\
                    1
               \end{pmatrix},
               \begin{pmatrix}
                     1 \\
                     1
               \end{pmatrix}
        \end{pmatrix}, \\ \\
        \begin{pmatrix}
               \begin{pmatrix}
                    1 \\
                    1
               \end{pmatrix},
               \begin{pmatrix}
                    -5 \\
                    -3
              \end{pmatrix}
        \ \end{pmatrix}
    \end{Bmatrix}
\end{align*}
%\caption{\qvasr abstraction $A_p = (S, V)$ of $P$.}
		\caption{Best \qvasr-abstraction of the loop in $P$.}
		\label{vasr}
	\end{figure}
\end{center}
Using this \qvasr-abstraction we can derive the following transition formula as loop summary:
\begin{align*}
	\exists k_1, k_2.\ &((-x' + y' + z' = 1\ \lor\ -x' + y' + z' = -x + y + z - 5k_2)\ \land\ y' = y + k_1 - 3k_2)\ \\ &\lor\ x' = x\ \land\ y' = y\ \land\ z' = z
\end{align*}
