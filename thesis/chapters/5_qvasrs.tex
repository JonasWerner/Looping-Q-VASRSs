
\begin{comment}
	\jw{Introduction qvasrs $\rightarrow$ summary precision improvement \\
	- What are qvasr? How to compute their reachability relation $\rightarrow$ Parikh image? \\ 
	How do they improve precision? \\
	- running example to qvasrs \\
	\vspace{1cm}
	15 pages}
\end{comment}

We see in the example that neither the loop guard \texttt{x <= 20} nor the conditions of the \texttt{if else} statement, \texttt{x <= 10} and \texttt{x > 10}, are not considered in the \qvasr-abstraction $(S_P, V_P)$, leading to the inclusion of transitions that would violate these conditions in the actual loop and would therefore not exist. \\
To exclude such transitions, and by that improve precision of the summary, we constrain \qvasr. We only allow transitions that satisfy a set of predicates $P$. Using predicates in $P$ as control states, we can construct a nondeterministic finite automaton that represents all allowed transitions. This automaton is called a \qvasr with states (\qvasrs), where each transition is labeled by a reset, addition vector pair $(\vec{r}, \vec{a})$. The predicates are pairwise unsatisfiable, minimizing the automaton's possible successor state space. A vector $\vec{x}$ can only transition to $\vec{x}'$ if there is an edge $(p, (\vec{r}, \vec{a}), q)$, with $\vec{x}' = \vec{r} * \vec{x} + \vec{a}$ and $p, q \in P$.
Figure \ref{vasrs} shows the \qvasrs of $P$'s loop, it was constructed using $(S_P, V_P)$, seen in Figure \ref{vasr}, and the predicates $x \leq 10$ and $10 < x \land x \leq 20$ as states.
\begin{figure}[H]
	
    \centering
    \begin{tikzpicture}[%
    ->,
    >=stealth',
    shorten >=1pt,
    auto,
    node distance=9cm,
    scale=0.9,
    transform shape,
    align=center,
    smallnode/.style={inner sep=1.2},
    initial text =,
    anchor=center]

    	\node[draw, ellipse, initial left, initial text =](1){$x \leq 10$};
    	\node[draw, ellipse](2) [right of=1] {$10 < x \land x \leq 20$};
    	% \node[draw, ellipse, accepting] (3) [right of=2] {$x > 20$};

    	\path (1) edge node [below]{ $
					\begin{bmatrix}
						-x' + y + z = - x + y + z - 5 \\
						y' = y - 3
					\end{bmatrix}
                     $ } (2)
	
	    	 (1) edge[loop above] node { $
					\begin{bmatrix}
						-x' + y' + z' = - x + y + z - 5 \\
					 	y' = y - 3
					\end{bmatrix}
	         $ } (1)
	
	    	 (2) edge[loop above] node { $
	               \begin{bmatrix}
						-x' + y' + z' = 1 \\
						y' = y + 1
					\end{bmatrix}
	                     $ } (2)
	    	;
    \end{tikzpicture}
    %\caption{\qvasrs abstraction $\mathcal{A}$ of program $P$.}
	\caption{\\ \qvasrs of the loop of $P$.}
	\label{vasrs}
\end{figure}
We see that there are no transitions that violate the \texttt{if else} conditions. \par
To compute a loop summary from a \qvasrs, one has to calculate the reachability relation of the automaton. Haase and Halfon \cite{DBLP:conf/rp/HaaseH14} proposed a polytime procedure that computes a series of formulas from computed Parikh images that, as conjunction, form a summary of the system. This procedure can be adapted to work with \qvasrs. \\ \par